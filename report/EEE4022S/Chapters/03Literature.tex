% ----------------------------------------------------
% ----------------------------------------------------
% Literature Review
% ----------------------------------------------------
\documentclass[class=report,11pt,crop=false]{standalone}
% Page geometry
\usepackage[a4paper,margin=20mm,top=25mm,bottom=25mm]{geometry}

% Font choice
\usepackage{lmodern}

\usepackage{lipsum}

% Use IEEE bibliography style
\bibliographystyle{IEEEtran}

% Line spacing
\usepackage{setspace}
\setstretch{1.2}

% Ensure UTF8 encoding
\usepackage[utf8]{inputenc}

% Language standard (not too important)
\usepackage[english]{babel}

% Skip a line in between paragraphs
\usepackage{parskip}

% For the creation of dummy text
\usepackage{blindtext}

% Math
\usepackage{amsmath}

% Header & Footer stuff
\usepackage{fancyhdr}
\pagestyle{fancy}
\fancyhead{}
\fancyhead[R]{\nouppercase{\rightmark}}
\fancyfoot{}
\fancyfoot[C]{\thepage}
\renewcommand{\headrulewidth}{0.0pt}
\renewcommand{\footrulewidth}{0.0pt}
\setlength{\headheight}{13.6pt}

% Epigraphs
\usepackage{epigraph}
\setlength\epigraphrule{0pt}
\setlength{\epigraphwidth}{0.65\textwidth}

% Colour
\usepackage{color}
\usepackage[usenames,dvipsnames]{xcolor}

% Hyperlinks & References
\usepackage{hyperref}
\definecolor{linkColour}{RGB}{77,71,179}
\definecolor{urlColour}{RGB}{255, 179, 102}

\hypersetup{
    colorlinks=true,
    linkcolor=linkColour,
    filecolor=linkColour,
    urlcolor=urlColour,
    citecolor=linkColour,
}
\urlstyle{same}

% Automatically correct front-side quotes
\usepackage[autostyle=false, style=ukenglish]{csquotes}
\MakeOuterQuote{"}

% Graphics
\usepackage{graphicx}
\graphicspath{{Figures/}{../Figures/}}
\usepackage{makecell}
\usepackage{transparent}
\usepackage{pgfplots}
\pgfplotsset{compat=newest}
%% the following commands are needed for some matlab2tikz features
\usetikzlibrary{plotmarks}
\usetikzlibrary{arrows.meta}
\usepgfplotslibrary{patchplots}
\usepackage{float}

% Tables
\usepackage{multirow} 
\usepackage{colortbl}

% SI units
\usepackage{siunitx}

% Microtype goodness
\usepackage{microtype}

% Listings
\usepackage[T1]{fontenc}
\usepackage{listings}
\usepackage[scaled=0.8]{DejaVuSansMono}

% Custom colours for listings
\definecolor{backgroundColour}{RGB}{250,250,250}
\definecolor{commentColour}{RGB}{73, 175, 102}
\definecolor{identifierColour}{RGB}{196, 19, 66}
\definecolor{stringColour}{RGB}{252, 156, 30}
\definecolor{keywordColour}{RGB}{50, 38, 224}
\definecolor{lineNumbersColour}{RGB}{127,127,127}
\lstset{
  language=Matlab,
  captionpos=b,
  aboveskip=15pt,belowskip=10pt,
  backgroundcolor=\color{backgroundColour},
  basicstyle=\ttfamily,%\footnotesize,        % the size of the fonts that are used for the code
  breakatwhitespace=false,         % sets if automatic breaks should only happen at whitespace
  breaklines=true,                 % sets automatic line breaking
  postbreak=\mbox{\textcolor{red}{$\hookrightarrow$}\space},
  commentstyle=\color{commentColour},    % comment style
  identifierstyle=\color{identifierColour},
  stringstyle=\color{stringColour},
   keywordstyle=\color{keywordColour},       % keyword style
  %escapeinside={\%*}{*)},          % if you want to add LaTeX within your code
  extendedchars=true,              % lets you use non-ASCII characters; for 8-bits encodings only, does not work with UTF-8
  frame=single,	                   % adds a frame around the code
  keepspaces=true,                 % keeps spaces in text, useful for keeping indentation of code (possibly needs columns=flexible)
  morekeywords={*,...},            % if you want to add more keywords to the set
  numbers=left,                    % where to put the line-numbers; possible values are (none, left, right)
  numbersep=5pt,                   % how far the line-numbers are from the code
  numberstyle=\tiny\color{lineNumbersColour}, % the style that is used for the line-numbers
  rulecolor=\color{black},         % if not set, the frame-color may be changed on line-breaks within not-black text (e.g. comments (green here))
  showspaces=false,                % show spaces everywhere adding particular underscores; it overrides 'showstringspaces'
  showstringspaces=false,          % underline spaces within strings only
  showtabs=false,                  % show tabs within strings adding particular underscores
  stepnumber=1,                    % the step between two line-numbers. If it's 1, each line will be numbered
  tabsize=2,	                   % sets default tabsize to 2 spaces
  %title=\lstname                   % show the filename of files included with \lstinputlisting; also try caption instead of title
}

% Caption stuff
\usepackage[hypcap=true, justification=centering]{caption}
\usepackage{subcaption}

% Glossary package
% \usepackage[acronym]{glossaries}
\usepackage{glossaries-extra}
\setabbreviationstyle[acronym]{long-short}

% For Proofs & Theorems
\usepackage{amsthm}

% Maths symbols
\usepackage{amssymb}
\usepackage{mathrsfs}
\usepackage{mathtools}

% For algorithms
\usepackage[]{algorithm2e}

% Spacing stuff
\setlength{\abovecaptionskip}{5pt plus 3pt minus 2pt}
\setlength{\belowcaptionskip}{5pt plus 3pt minus 2pt}
\setlength{\textfloatsep}{10pt plus 3pt minus 2pt}
\setlength{\intextsep}{15pt plus 3pt minus 2pt}

% For aligning footnotes at bottom of page, instead of hugging text
\usepackage[bottom]{footmisc}

% Add LoF, Bib, etc. to ToC
\usepackage[nottoc]{tocbibind}

% SI
\usepackage{siunitx}

% For removing some whitespace in Chapter headings etc
\usepackage{etoolbox}
\makeatletter
\patchcmd{\@makechapterhead}{\vspace*{50\p@}}{\vspace*{-10pt}}{}{}%
\patchcmd{\@makeschapterhead}{\vspace*{50\p@}}{\vspace*{-10pt}}{}{}%
\makeatother

% Wrap figure
\usepackage{wrapfig}
\makenoidxglossaries

\newacronym{af}{AF}{Autofocus}
\newacronym{cli}{CLI}{Command-line Interface}
\newacronym{cpi}{CPI}{Coherent Processing Interval}
\newacronym{cptwl}{CPTWL}{Coherent Processing Time Window Length}
\newacronym{cw}{CW}{Continuous Waveform}
\newacronym{ds}{DS}{Dominant Scatterer}
\newacronym{dsa}{DSA}{Dominant Scatterer Algorithm}
\newacronym{sdsaf}{SDSAF}{Single Dominant Scatterer Autofocus}
\newacronym{fft}{FFT}{Fast Fourier Transform}
\newacronym{fmcw}{FMCW}{Frequency Modulated Continuous Waveform} % Not sure
\newacronym{hrr}{HRR}{High Resolution Range}
\newacronym{hrrp}{HRRP}{High Resolution Range Profile}
\newacronym{ic}{IC}{Image Contrast}
\newacronym{isar}{ISAR}{Inverse Synthetic Aperture Radar}
\newacronym{jtf}{JTF}{Joint Time-Frequency}
\newacronym{pri}{PRI}{Pulse Repetition Interval}
\newacronym{prf}{PRF}{Pulse Repetition Frequency}
\newacronym{qlp}{QLP}{Quick-look Processor}
\newacronym{ra}{RA}{Range Alignment}
\newacronym{rlos}{RLOS}{Radar Line of Sight}
\newacronym{rmc}{RMC}{Rotational Motion Compensation}
\newacronym{sfw}{SFW}{Stepped Frequency Waveform}
\newacronym{sf}{SF}{Scaling Factor for Haywood Autofocus}
\newacronym{sar}{SAR}{Synthetic Aperture Radar}
\newacronym{snr}{SNR}{Signal-to-Noise Ratio}
\newacronym{sir}{SIR}{Signal-to-Interference Ratio}
\newacronym{tmc}{TMC}{Translational Motion Compensation}


\begin{document}
\ifstandalone
\tableofcontents
\fi
% ----------------------------------------------------
\chapter{Literature Review \label{ch:literature}}
\epigraph{If you wish to make an apple pie from scratch, you must first invent the universe.}%
    {\emph{---Carl Sagan}}
\vspace{0.5cm}
% ----------------------------------------------------
% Characteristics of the literature review
% •	There is a strong relevance linking literature to problem statement
% •	Comprehensive coverage of the important literature relating to the topic and the problem
% •	Critical engagement with the literature.
    % o	The original contributions of the literature are identified, and results are summarised.
    % o	Strengths of the literature are identified.
    % o	Weakness/limitations of the literature are also identified in the context of the work done.
    % o	Work that has not been sufficiently addressed in the literature was identified. This ‘gap’ relates to the problem in your work
    % o	Clarify how your work fits in the existing literature
    % o	Show a link between the literature and your work by explaining how the literature has influenced your decision making in the work
    
% Ensure the structure/flow of your literature review corresponds to the following
% •	First discuss papers that are related to the broad research area
% •	Then, discuss literature that is more focused on the problem statement and/or objectives. Provide a critical engagement with the literature.
% •	Then, back out again to big picture where you summarise the key technical points discovered and how these apply to your project. This may feel repetitive, but you are leading the reader through a story and its helpful for them to be reminded again of why the deep technical concepts you’re discussing is aligned with your work

% Important elements of a literature survey
% •	Include graphs/tables/photos from the literature to enhance understanding of results found in the literature 
% •	Discussion on literature is broken down into themes in the research field, instead of discussing one paper after the next.  By commenting on multiple papers at a time, the important elements of the papers that are applicable to your research are highlighted, and you can then move onto critically commenting on the work in the literature and identifying gaps or challenges that have not been sufficiently addressed. 
\section{Current radar systems and the purpose of radar}
% 1 Real World Radar Systems
% 1.	What radar systems do you think we have in Cape Town or in South Africa? What functions are performed by these radar systems?
% Weather Radar for weather tracking and measurement.
% Air traffic control Radar (3GHz) at airports and air traffic protected sites (like SKA).
% Surveillance Radar for military and port control.
% Instrumentation Radar like at the SKA.
% Moving target detection radar – used for poacher detection in the Kruger.
% Correct. There is scope to transform the SKA system into a radar by incorporating a transmitter.  


\section{Radar Imaging}
% Carl Wiley of the Goodyear Aircraft Corporation was the first person who stated, in June 1951, that the Doppler frequency in the backscattered signal from a tar- get could be used to obtain fine cross range resolution for radar imaging [5]. He argued that if a relative speed exists between the radar and target, each compo- nent of the target would have a slightly different speed relative to the antenna. Precise analysis of the Doppler frequency (velocity) of the reflections through the use of the Fourier Transform, modern spectral estimation or time-frequency representations [6] will allow for the construction of a detailed image of the target. This is the fundamental principle of coherent radar imaging, specifically Synthetic Aperture Radar (SAR) and Inverse SAR (ISAR), which provide sig- nificant benefits over non-coherent imaging such as amplitude-only tomography. Therefore, the effective aperture size, or more importantly, the cross range resolution is a function of the change in angle, or rotation rate of the target % Meng or Yunis

\section{ISAR}

\section{ISAR for sea-vessels}
% small vs large


\section{Range Alignment Algorithms}
% We are specifically doing correlation-based approaches
% READ YUAN's PAPER AND THE SOURCES RECOMMENDED IN THERE

% Simple correlation technique: Martorella ch4.1.1 limitation is integer bin shifts
% Haywood allows fractional shifts through bins.
% Other techniques discussed in Martorella and Chen


\section{Autofocus Algorithms}
% We are specifically doing dominant-scatterer based approaches
% READ YUAN's PAPER AND THE SOURCES RECOMMENDED IN THERE
% Haywood AF
\section{Quick-look Processors}



% ----------------------------------------------------
\ifstandalone
\bibliography{../Bibliography/References.bib}
\printnoidxglossary[type=\acronymtype,nonumberlist]
\fi
\end{document}
% ----------------------------------------------------
% ----------------------------------------------------
% Appendix A
% ----------------------------------------------------
\documentclass[class=report,11pt,crop=false]{standalone}
% Page geometry
\usepackage[a4paper,margin=20mm,top=25mm,bottom=25mm]{geometry}

% Font choice
\usepackage{lmodern}

\usepackage{lipsum}

% Use IEEE bibliography style
\bibliographystyle{IEEEtran}

% Line spacing
\usepackage{setspace}
\setstretch{1.2}

% Ensure UTF8 encoding
\usepackage[utf8]{inputenc}

% Language standard (not too important)
\usepackage[english]{babel}

% Skip a line in between paragraphs
\usepackage{parskip}

% For the creation of dummy text
\usepackage{blindtext}

% Math
\usepackage{amsmath}

% Header & Footer stuff
\usepackage{fancyhdr}
\pagestyle{fancy}
\fancyhead{}
\fancyhead[R]{\nouppercase{\rightmark}}
\fancyfoot{}
\fancyfoot[C]{\thepage}
\renewcommand{\headrulewidth}{0.0pt}
\renewcommand{\footrulewidth}{0.0pt}
\setlength{\headheight}{13.6pt}

% Epigraphs
\usepackage{epigraph}
\setlength\epigraphrule{0pt}
\setlength{\epigraphwidth}{0.65\textwidth}

% Colour
\usepackage{color}
\usepackage[usenames,dvipsnames]{xcolor}

% Hyperlinks & References
\usepackage{hyperref}
\definecolor{linkColour}{RGB}{77,71,179}
\definecolor{urlColour}{RGB}{255, 179, 102}

\hypersetup{
    colorlinks=true,
    linkcolor=linkColour,
    filecolor=linkColour,
    urlcolor=urlColour,
    citecolor=linkColour,
}
\urlstyle{same}

% Automatically correct front-side quotes
\usepackage[autostyle=false, style=ukenglish]{csquotes}
\MakeOuterQuote{"}

% Graphics
\usepackage{graphicx}
\graphicspath{{Figures/}{../Figures/}}
\usepackage{makecell}
\usepackage{transparent}
\usepackage{pgfplots}
\pgfplotsset{compat=newest}
%% the following commands are needed for some matlab2tikz features
\usetikzlibrary{plotmarks}
\usetikzlibrary{arrows.meta}
\usepgfplotslibrary{patchplots}
\usepackage{float}

% Tables
\usepackage{multirow} 
\usepackage{colortbl}

% SI units
\usepackage{siunitx}

% Microtype goodness
\usepackage{microtype}

% Listings
\usepackage[T1]{fontenc}
\usepackage{listings}
\usepackage[scaled=0.8]{DejaVuSansMono}

% Custom colours for listings
\definecolor{backgroundColour}{RGB}{250,250,250}
\definecolor{commentColour}{RGB}{73, 175, 102}
\definecolor{identifierColour}{RGB}{196, 19, 66}
\definecolor{stringColour}{RGB}{252, 156, 30}
\definecolor{keywordColour}{RGB}{50, 38, 224}
\definecolor{lineNumbersColour}{RGB}{127,127,127}
\lstset{
  language=Matlab,
  captionpos=b,
  aboveskip=15pt,belowskip=10pt,
  backgroundcolor=\color{backgroundColour},
  basicstyle=\ttfamily,%\footnotesize,        % the size of the fonts that are used for the code
  breakatwhitespace=false,         % sets if automatic breaks should only happen at whitespace
  breaklines=true,                 % sets automatic line breaking
  postbreak=\mbox{\textcolor{red}{$\hookrightarrow$}\space},
  commentstyle=\color{commentColour},    % comment style
  identifierstyle=\color{identifierColour},
  stringstyle=\color{stringColour},
   keywordstyle=\color{keywordColour},       % keyword style
  %escapeinside={\%*}{*)},          % if you want to add LaTeX within your code
  extendedchars=true,              % lets you use non-ASCII characters; for 8-bits encodings only, does not work with UTF-8
  frame=single,	                   % adds a frame around the code
  keepspaces=true,                 % keeps spaces in text, useful for keeping indentation of code (possibly needs columns=flexible)
  morekeywords={*,...},            % if you want to add more keywords to the set
  numbers=left,                    % where to put the line-numbers; possible values are (none, left, right)
  numbersep=5pt,                   % how far the line-numbers are from the code
  numberstyle=\tiny\color{lineNumbersColour}, % the style that is used for the line-numbers
  rulecolor=\color{black},         % if not set, the frame-color may be changed on line-breaks within not-black text (e.g. comments (green here))
  showspaces=false,                % show spaces everywhere adding particular underscores; it overrides 'showstringspaces'
  showstringspaces=false,          % underline spaces within strings only
  showtabs=false,                  % show tabs within strings adding particular underscores
  stepnumber=1,                    % the step between two line-numbers. If it's 1, each line will be numbered
  tabsize=2,	                   % sets default tabsize to 2 spaces
  %title=\lstname                   % show the filename of files included with \lstinputlisting; also try caption instead of title
}

% Caption stuff
\usepackage[hypcap=true, justification=centering]{caption}
\usepackage{subcaption}

% Glossary package
% \usepackage[acronym]{glossaries}
\usepackage{glossaries-extra}
\setabbreviationstyle[acronym]{long-short}

% For Proofs & Theorems
\usepackage{amsthm}

% Maths symbols
\usepackage{amssymb}
\usepackage{mathrsfs}
\usepackage{mathtools}

% For algorithms
\usepackage[]{algorithm2e}

% Spacing stuff
\setlength{\abovecaptionskip}{5pt plus 3pt minus 2pt}
\setlength{\belowcaptionskip}{5pt plus 3pt minus 2pt}
\setlength{\textfloatsep}{10pt plus 3pt minus 2pt}
\setlength{\intextsep}{15pt plus 3pt minus 2pt}

% For aligning footnotes at bottom of page, instead of hugging text
\usepackage[bottom]{footmisc}

% Add LoF, Bib, etc. to ToC
\usepackage[nottoc]{tocbibind}

% SI
\usepackage{siunitx}

% For removing some whitespace in Chapter headings etc
\usepackage{etoolbox}
\makeatletter
\patchcmd{\@makechapterhead}{\vspace*{50\p@}}{\vspace*{-10pt}}{}{}%
\patchcmd{\@makeschapterhead}{\vspace*{50\p@}}{\vspace*{-10pt}}{}{}%
\makeatother

% Wrap figure
\usepackage{wrapfig}
\makenoidxglossaries

\newacronym{af}{AF}{Autofocus}
\newacronym{cli}{CLI}{Command-line Interface}
\newacronym{cpi}{CPI}{Coherent Processing Interval}
\newacronym{cptwl}{CPTWL}{Coherent Processing Time Window Length}
\newacronym{cw}{CW}{Continuous Waveform}
\newacronym{ds}{DS}{Dominant Scatterer}
\newacronym{dsa}{DSA}{Dominant Scatterer Algorithm}
\newacronym{sdsaf}{SDSAF}{Single Dominant Scatterer Autofocus}
\newacronym{fft}{FFT}{Fast Fourier Transform}
\newacronym{fmcw}{FMCW}{Frequency Modulated Continuous Waveform} % Not sure
\newacronym{hrr}{HRR}{High Resolution Range}
\newacronym{hrrp}{HRRP}{High Resolution Range Profile}
\newacronym{ic}{IC}{Image Contrast}
\newacronym{isar}{ISAR}{Inverse Synthetic Aperture Radar}
\newacronym{jtf}{JTF}{Joint Time-Frequency}
\newacronym{pri}{PRI}{Pulse Repetition Interval}
\newacronym{prf}{PRF}{Pulse Repetition Frequency}
\newacronym{qlp}{QLP}{Quick-look Processor}
\newacronym{ra}{RA}{Range Alignment}
\newacronym{rlos}{RLOS}{Radar Line of Sight}
\newacronym{rmc}{RMC}{Rotational Motion Compensation}
\newacronym{sfw}{SFW}{Stepped Frequency Waveform}
\newacronym{sf}{SF}{Scaling Factor for Haywood Autofocus}
\newacronym{sar}{SAR}{Synthetic Aperture Radar}
\newacronym{snr}{SNR}{Signal-to-Noise Ratio}
\newacronym{sir}{SIR}{Signal-to-Interference Ratio}
\newacronym{tmc}{TMC}{Translational Motion Compensation}


\begin{document}
% ----------------------------------------------------
\appendix
\chapter{Additional Algorithm Verification and Validation \label{apndxA}}
% ----------------------------------------------------
\section{\gls{sdsaf} Additional Figures}
This section contains additional plots of results discussed in \autoref{subsec:hayAF}
    %***************************************************************************************%    
    \subsection{Algorithm Verification}
    The scatterer power and variance plots in \autoref{subsec:hayAF_verification} are zoomed in to make the \gls{ds} selection clearer. The plots in this section contain both the original and zoomed in plots for context. 

    \begin{figure}
        \centering
        \begin{subfigure}{0.4\linewidth}
                \centering
                \resizebox{\linewidth}{!}{% This file was created by matlab2tikz.
%
%The latest updates can be retrieved from
%  http://www.mathworks.com/matlabcentral/fileexchange/22022-matlab2tikz-matlab2tikz
%where you can also make suggestions and rate matlab2tikz.
%
\definecolor{mycolor1}{rgb}{0.00000,0.44700,0.74100}%
%
\begin{tikzpicture}

\begin{axis}[%
width=6.028in,
height=4.754in,
at={(1.011in,0.642in)},
scale only axis,
xmin=0,
xmax=256,
xlabel style={font=\fontsize{25}{20}\selectfont\color{black}, yshift = -10},
xlabel={Range Bin Number},
ymin=0,
ymax=25000000,
ylabel style={font=\fontsize{25}{20}\selectfont\color{black}, yshift=10pt},
ylabel={Power},
axis background/.style={fill=white},
tick label style={font=\fontsize{20}{11}\selectfont\color{black}},
xtick distance = 50,
yticklabel={\ifdim\tick pt=0pt\else\pgfmathprintnumber{\tick}\fi}, 
scaled y ticks=base 10:-6,
legend style={legend cell align=left, align = left, draw=white!15!black, font=\fontsize{12}{11}\selectfont\color{black}}
]
\addplot [color=mycolor1, mark=asterisk, mark options={solid, mycolor1}]
  table[row sep=crcr]{%
1	260.459583212987\\
2	255.853316684552\\
3	253.849369048871\\
4	262.902589895056\\
5	269.395431769183\\
6	257.942118684153\\
7	264.777673790391\\
8	267.345173590508\\
9	267.442550084895\\
10	274.763167201476\\
11	269.657038331506\\
12	273.52865970639\\
13	270.838241173277\\
14	270.848862745988\\
15	268.994507249835\\
16	278.092807479352\\
17	275.678440062448\\
18	280.764024183879\\
19	280.264596895383\\
20	280.844834122902\\
21	284.609662384031\\
22	289.349950042811\\
23	289.63453778992\\
24	297.216247874669\\
25	304.044956124828\\
26	299.728539743741\\
27	304.570239574428\\
28	312.586614746517\\
29	312.578394352122\\
30	313.03569904371\\
31	318.676695752314\\
32	334.429012992535\\
33	326.05568970326\\
34	329.794039972701\\
35	347.471923711035\\
36	357.356896491005\\
37	359.280067262786\\
38	370.172276306485\\
39	367.252690733492\\
40	384.036353269705\\
41	403.514451969125\\
42	409.987699077831\\
43	422.515008557612\\
44	437.076708129586\\
45	455.336146360901\\
46	474.987020352395\\
47	506.63892953046\\
48	524.698547335733\\
49	585.497580378801\\
50	627.110711153952\\
51	693.524309995736\\
52	784.287092174285\\
53	935.526405704739\\
54	1173.71124130302\\
55	1600.67007465833\\
56	2583.29049019285\\
57	5800.48900106346\\
58	40442.0449329261\\
59	234600.510606336\\
60	16827.8819816654\\
61	4770.1478970912\\
62	3240.29772132079\\
63	3634.40328263001\\
64	7268.77499224631\\
65	76531.2093170108\\
66	213667.240436973\\
67	10052.7936091226\\
68	4507.02792533953\\
69	3559.31518627866\\
70	4106.8388665825\\
71	8159.23968924983\\
72	127074.927266568\\
73	167160.607608639\\
74	9606.39694763152\\
75	4990.0602038816\\
76	4292.92605082353\\
77	5229.47347252321\\
78	10356.742292667\\
79	182309.818486979\\
80	108097.426019492\\
81	8627.78437379697\\
82	4318.19931958007\\
83	3854.49429303449\\
84	5199.41183792736\\
85	14022.0503602094\\
86	227432.741146917\\
87	63225.9569214638\\
88	8170.50934399543\\
89	4150.73895156004\\
90	3475.38764420246\\
91	4876.64326862932\\
92	22725.5309440892\\
93	244399.775686268\\
94	37495.2781672515\\
95	10573.4177170359\\
96	7100.60019107358\\
97	6753.16861846795\\
98	9242.40740232017\\
99	51530.6884280494\\
100	239715.038608182\\
101	21021.5521803837\\
102	12367.4161492386\\
103	12919.2292195377\\
104	18560.6270392406\\
105	50352.3007433402\\
106	491915.041237475\\
107	722546.673736452\\
108	42839.0513423732\\
109	18946.8184049635\\
110	13294.4378465799\\
111	11888.3265603265\\
112	14228.1063884036\\
113	161392.80534233\\
114	207528.804911758\\
115	32227.6875301607\\
116	26620.7506624016\\
117	28153.2588950145\\
118	33451.3666831277\\
119	47362.6527416718\\
120	243035.473131084\\
121	113402.734843283\\
122	67680.325702816\\
123	101247.331523687\\
124	175595.773473503\\
125	381658.413684424\\
126	1567657.49064522\\
127	23225002.3847125\\
128	4866592.85612614\\
129	624148.878839017\\
130	238968.489337471\\
131	125467.57654359\\
132	78185.2090712392\\
133	73030.3601940519\\
134	271320.771835579\\
135	66016.2557543734\\
136	35644.2191386429\\
137	27660.6982729765\\
138	24525.1417075442\\
139	27005.715637359\\
140	93273.0398557962\\
141	280507.640693694\\
142	15266.9861055596\\
143	11660.2329098454\\
144	12774.0609258221\\
145	18308.3665983194\\
146	54196.6326667642\\
147	559399.5054032\\
148	675605.960268304\\
149	46832.2981207944\\
150	20414.8653773866\\
151	13596.8722164049\\
152	11704.4608140175\\
153	14834.5483454693\\
154	165878.818809875\\
155	129155.452493415\\
156	12845.7472234616\\
157	8425.93245283305\\
158	7924.50240121107\\
159	9630.82839402855\\
160	18167.8674919882\\
161	221464.952930091\\
162	66296.2165083529\\
163	6282.75073094234\\
164	3550.82893427832\\
165	3693.84173148895\\
166	5793.15980013346\\
167	21194.8549443591\\
168	249988.987910509\\
169	37348.5151127678\\
170	6707.77320205252\\
171	3834.62559434609\\
172	3666.58727613934\\
173	6179.43754018277\\
174	39621.937991042\\
175	244027.580562077\\
176	19175.2805279218\\
177	6528.63349151219\\
178	4772.4713207705\\
179	5128.36765511789\\
180	9037.87893829267\\
181	79769.6559031287\\
182	213548.500351052\\
183	9189.08795174902\\
184	4363.70260986501\\
185	3712.07025077464\\
186	4602.33188561278\\
187	9220.50288249593\\
188	129112.974710105\\
189	162501.742729066\\
190	7574.41472023649\\
191	3473.96314063804\\
192	2883.8804389084\\
193	3771.98710657173\\
194	8603.85021353972\\
195	174816.050956961\\
196	107424.122922777\\
197	8158.46008299097\\
198	3188.81296756292\\
199	1874.3706597991\\
200	1319.41764201677\\
201	1034.95594151075\\
202	869.247784578133\\
203	742.863481776727\\
204	658.01083274331\\
205	610.719153796497\\
206	558.676138785167\\
207	527.727140126541\\
208	493.143467884876\\
209	472.767945849341\\
210	450.923013010194\\
211	429.591125867796\\
212	412.554276876183\\
213	404.874610605797\\
214	387.775528276528\\
215	391.405099058641\\
216	365.717226442089\\
217	364.330352475694\\
218	356.484671896599\\
219	358.687392545226\\
220	345.13318693796\\
221	338.830857316187\\
222	344.279776728209\\
223	335.852598946057\\
224	321.208276027795\\
225	320.221419869568\\
226	316.106449388835\\
227	306.342738798994\\
228	301.826156405654\\
229	300.26529353038\\
230	290.634496667669\\
231	295.148021859695\\
232	294.361094581251\\
233	291.018501216427\\
234	290.531983794715\\
235	293.620323539453\\
236	282.415873342337\\
237	281.56353725174\\
238	276.975587713\\
239	285.539867461153\\
240	274.106466494687\\
241	273.037195880877\\
242	282.942195034159\\
243	271.428841093763\\
244	265.861359523173\\
245	264.640464669466\\
246	269.787791563485\\
247	269.628633305419\\
248	259.668277498303\\
249	270.659371561747\\
250	265.10876919209\\
251	260.083470473997\\
252	261.220926812129\\
253	265.88563906247\\
254	257.030182737016\\
255	266.64222467541\\
256	259.75335716483\\
};
\addlegendentry{Scatterer power}

\addplot [color=orange]
  table[row sep=crcr]{%
0	158841.769294874\\
300	158841.769294874\\
};
\addlegendentry{Average power of all scatterers}

\addplot [color=black, only marks, mark size=4.0pt, mark=o, mark options={solid, black}]
  table[row sep=crcr]{%
59	234600.510606336\\
66	213667.240436973\\
73	167160.607608639\\
79	182309.818486979\\
86	227432.741146917\\
93	244399.775686268\\
100	239715.038608182\\
106	491915.041237475\\
107	722546.673736452\\
113	161392.80534233\\
114	207528.804911758\\
120	243035.473131084\\
124	175595.773473503\\
125	381658.413684424\\
126	1567657.49064522\\
127	23225002.3847125\\
128	4866592.85612614\\
129	624148.878839017\\
130	238968.489337471\\
134	271320.771835579\\
141	280507.640693694\\
147	559399.5054032\\
148	675605.960268304\\
154	165878.818809875\\
161	221464.952930091\\
168	249988.987910509\\
175	244027.580562077\\
182	213548.500351052\\
189	162501.742729066\\
195	174816.050956961\\
};
\addlegendentry{Candidate scatterers}

\addplot [color=red, only marks, mark size=7.5pt, mark=o, mark options={solid, red}]
  table[row sep=crcr]
                \caption{Scatterer power.\label{subfig:apndxA_hayAF_SCRA_sim_power}}
        \end{subfigure}
         \hspace{1cm}
        \begin{subfigure}{0.4\linewidth}
                \centering
                \resizebox{\linewidth}{!}{% This file was created by matlab2tikz.
%
%The latest updates can be retrieved from
%  http://www.mathworks.com/matlabcentral/fileexchange/22022-matlab2tikz-matlab2tikz
%where you can also make suggestions and rate matlab2tikz.
%
\definecolor{mycolor1}{rgb}{0.00000,0.44700,0.74100}%
%
\begin{tikzpicture}

\begin{axis}[%
width=6.028in,
height=4.754in,
at={(1.011in,0.642in)},
scale only axis,
xmin=0,
xmax=256,
xlabel style={font=\fontsize{25}{20}\selectfont\color{black}, yshift = -10},
xlabel={Range Bin Number},
ymin=0,
ymax=0.1e7,
ylabel style={font=\fontsize{25}{20}\selectfont\color{black}, yshift=10pt},
ylabel={Power},
axis background/.style={fill=white},
tick label style={font=\fontsize{20}{11}\selectfont\color{black}},
xtick distance = 50,
yticklabel={\ifdim\tick pt=0pt\else\pgfmathprintnumber{\tick}\fi}, 
legend style={legend cell align=left, align = left, draw=white!15!black, font=\fontsize{12}{11}\selectfont\color{black}}
]
\addplot [color=mycolor1, mark=asterisk, mark options={solid, mycolor1}]
  table[row sep=crcr]{%
1	260.459583212987\\
2	255.853316684552\\
3	253.849369048871\\
4	262.902589895056\\
5	269.395431769183\\
6	257.942118684153\\
7	264.777673790391\\
8	267.345173590508\\
9	267.442550084895\\
10	274.763167201476\\
11	269.657038331506\\
12	273.52865970639\\
13	270.838241173277\\
14	270.848862745988\\
15	268.994507249835\\
16	278.092807479352\\
17	275.678440062448\\
18	280.764024183879\\
19	280.264596895383\\
20	280.844834122902\\
21	284.609662384031\\
22	289.349950042811\\
23	289.63453778992\\
24	297.216247874669\\
25	304.044956124828\\
26	299.728539743741\\
27	304.570239574428\\
28	312.586614746517\\
29	312.578394352122\\
30	313.03569904371\\
31	318.676695752314\\
32	334.429012992535\\
33	326.05568970326\\
34	329.794039972701\\
35	347.471923711035\\
36	357.356896491005\\
37	359.280067262786\\
38	370.172276306485\\
39	367.252690733492\\
40	384.036353269705\\
41	403.514451969125\\
42	409.987699077831\\
43	422.515008557612\\
44	437.076708129586\\
45	455.336146360901\\
46	474.987020352395\\
47	506.63892953046\\
48	524.698547335733\\
49	585.497580378801\\
50	627.110711153952\\
51	693.524309995736\\
52	784.287092174285\\
53	935.526405704739\\
54	1173.71124130302\\
55	1600.67007465833\\
56	2583.29049019285\\
57	5800.48900106346\\
58	40442.0449329261\\
59	234600.510606336\\
60	16827.8819816654\\
61	4770.1478970912\\
62	3240.29772132079\\
63	3634.40328263001\\
64	7268.77499224631\\
65	76531.2093170108\\
66	213667.240436973\\
67	10052.7936091226\\
68	4507.02792533953\\
69	3559.31518627866\\
70	4106.8388665825\\
71	8159.23968924983\\
72	127074.927266568\\
73	167160.607608639\\
74	9606.39694763152\\
75	4990.0602038816\\
76	4292.92605082353\\
77	5229.47347252321\\
78	10356.742292667\\
79	182309.818486979\\
80	108097.426019492\\
81	8627.78437379697\\
82	4318.19931958007\\
83	3854.49429303449\\
84	5199.41183792736\\
85	14022.0503602094\\
86	227432.741146917\\
87	63225.9569214638\\
88	8170.50934399543\\
89	4150.73895156004\\
90	3475.38764420246\\
91	4876.64326862932\\
92	22725.5309440892\\
93	244399.775686268\\
94	37495.2781672515\\
95	10573.4177170359\\
96	7100.60019107358\\
97	6753.16861846795\\
98	9242.40740232017\\
99	51530.6884280494\\
100	239715.038608182\\
101	21021.5521803837\\
102	12367.4161492386\\
103	12919.2292195377\\
104	18560.6270392406\\
105	50352.3007433402\\
106	491915.041237475\\
107	722546.673736452\\
108	42839.0513423732\\
109	18946.8184049635\\
110	13294.4378465799\\
111	11888.3265603265\\
112	14228.1063884036\\
113	161392.80534233\\
114	207528.804911758\\
115	32227.6875301607\\
116	26620.7506624016\\
117	28153.2588950145\\
118	33451.3666831277\\
119	47362.6527416718\\
120	243035.473131084\\
121	113402.734843283\\
122	67680.325702816\\
123	101247.331523687\\
124	175595.773473503\\
125	381658.413684424\\
126	1567657.49064522\\
127	23225002.3847125\\
128	4866592.85612614\\
129	624148.878839017\\
130	238968.489337471\\
131	125467.57654359\\
132	78185.2090712392\\
133	73030.3601940519\\
134	271320.771835579\\
135	66016.2557543734\\
136	35644.2191386429\\
137	27660.6982729765\\
138	24525.1417075442\\
139	27005.715637359\\
140	93273.0398557962\\
141	280507.640693694\\
142	15266.9861055596\\
143	11660.2329098454\\
144	12774.0609258221\\
145	18308.3665983194\\
146	54196.6326667642\\
147	559399.5054032\\
148	675605.960268304\\
149	46832.2981207944\\
150	20414.8653773866\\
151	13596.8722164049\\
152	11704.4608140175\\
153	14834.5483454693\\
154	165878.818809875\\
155	129155.452493415\\
156	12845.7472234616\\
157	8425.93245283305\\
158	7924.50240121107\\
159	9630.82839402855\\
160	18167.8674919882\\
161	221464.952930091\\
162	66296.2165083529\\
163	6282.75073094234\\
164	3550.82893427832\\
165	3693.84173148895\\
166	5793.15980013346\\
167	21194.8549443591\\
168	249988.987910509\\
169	37348.5151127678\\
170	6707.77320205252\\
171	3834.62559434609\\
172	3666.58727613934\\
173	6179.43754018277\\
174	39621.937991042\\
175	244027.580562077\\
176	19175.2805279218\\
177	6528.63349151219\\
178	4772.4713207705\\
179	5128.36765511789\\
180	9037.87893829267\\
181	79769.6559031287\\
182	213548.500351052\\
183	9189.08795174902\\
184	4363.70260986501\\
185	3712.07025077464\\
186	4602.33188561278\\
187	9220.50288249593\\
188	129112.974710105\\
189	162501.742729066\\
190	7574.41472023649\\
191	3473.96314063804\\
192	2883.8804389084\\
193	3771.98710657173\\
194	8603.85021353972\\
195	174816.050956961\\
196	107424.122922777\\
197	8158.46008299097\\
198	3188.81296756292\\
199	1874.3706597991\\
200	1319.41764201677\\
201	1034.95594151075\\
202	869.247784578133\\
203	742.863481776727\\
204	658.01083274331\\
205	610.719153796497\\
206	558.676138785167\\
207	527.727140126541\\
208	493.143467884876\\
209	472.767945849341\\
210	450.923013010194\\
211	429.591125867796\\
212	412.554276876183\\
213	404.874610605797\\
214	387.775528276528\\
215	391.405099058641\\
216	365.717226442089\\
217	364.330352475694\\
218	356.484671896599\\
219	358.687392545226\\
220	345.13318693796\\
221	338.830857316187\\
222	344.279776728209\\
223	335.852598946057\\
224	321.208276027795\\
225	320.221419869568\\
226	316.106449388835\\
227	306.342738798994\\
228	301.826156405654\\
229	300.26529353038\\
230	290.634496667669\\
231	295.148021859695\\
232	294.361094581251\\
233	291.018501216427\\
234	290.531983794715\\
235	293.620323539453\\
236	282.415873342337\\
237	281.56353725174\\
238	276.975587713\\
239	285.539867461153\\
240	274.106466494687\\
241	273.037195880877\\
242	282.942195034159\\
243	271.428841093763\\
244	265.861359523173\\
245	264.640464669466\\
246	269.787791563485\\
247	269.628633305419\\
248	259.668277498303\\
249	270.659371561747\\
250	265.10876919209\\
251	260.083470473997\\
252	261.220926812129\\
253	265.88563906247\\
254	257.030182737016\\
255	266.64222467541\\
256	259.75335716483\\
};
\addlegendentry{Scatterer power}

\addplot [color=orange]
  table[row sep=crcr]{%
0	158841.769294874\\
300	158841.769294874\\
};
\addlegendentry{Average power of all scatterers}

\addplot [color=black, only marks, mark size=4.0pt, mark=o, mark options={solid, black}]
  table[row sep=crcr]{%
59	234600.510606336\\
66	213667.240436973\\
73	167160.607608639\\
79	182309.818486979\\
86	227432.741146917\\
93	244399.775686268\\
100	239715.038608182\\
106	491915.041237475\\
107	722546.673736452\\
113	161392.80534233\\
114	207528.804911758\\
120	243035.473131084\\
124	175595.773473503\\
125	381658.413684424\\
126	1567657.49064522\\
127	23225002.3847125\\
128	4866592.85612614\\
129	624148.878839017\\
130	238968.489337471\\
134	271320.771835579\\
141	280507.640693694\\
147	559399.5054032\\
148	675605.960268304\\
154	165878.818809875\\
161	221464.952930091\\
168	249988.987910509\\
175	244027.580562077\\
182	213548.500351052\\
189	162501.742729066\\
195	174816.050956961\\
};
\addlegendentry{Candidate scatterers}

\addplot [color=red, only marks, mark size=7.5pt, mark=o, mark options={solid, red}]
  table[row sep=crcr]
                \caption{Scatterer power (zoomed-in).\label{subfig:apndxA_hayAF_SCRA_sim_power_zoomed}}
        \end{subfigure}
        \begin{subfigure}{0.4\linewidth}
                \centering
                \resizebox{\linewidth}{!}{% This file was created by matlab2tikz.
%
%The latest updates can be retrieved from
%  http://www.mathworks.com/matlabcentral/fileexchange/22022-matlab2tikz-matlab2tikz
%where you can also make suggestions and rate matlab2tikz.
%
\definecolor{mycolor1}{rgb}{0.00000,0.44700,0.74100}%
%
\begin{tikzpicture}

\begin{axis}[%
width=6.028in,
height=4.754in,
at={(1.011in,0.642in)},
scale only axis,
xmin=0,
xmax=256,
xlabel style={font=\fontsize{25}{20}\selectfont\color{black}, yshift = -10},
xlabel={Range Bin Number},
ymin=0,
ymax=60000,
ylabel style={font=\fontsize{25}{20}\selectfont\color{black}, yshift=10pt},
ylabel={Amplitude Variance},
axis background/.style={fill=white},
tick label style={font=\fontsize{20}{11}\selectfont\color{black}},
xtick distance = 50,
scaled y ticks=base 10:-3,
yticklabel={\ifdim\tick pt=0pt\else\pgfmathprintnumber{\tick}\fi}, 
legend style={legend cell align=left, align=left, draw=white!15!black, font=\fontsize{12}{11}\selectfont\color{black}},
grid=both,  % Enables major grids
minor tick num=10,  % Sets the number of minor grid lines between major lines
grid style={dashed,gray!50},  % Style for major grids
minor grid style={dotted,gray!50} % Style for minor grids
]
\addplot [color=mycolor1, mark=asterisk, mark options={solid, mycolor1}]
  table[row sep=crcr]{%
1	2.05737908304292\\
2	1.85575449052375\\
3	1.94419916622086\\
4	1.97731558212904\\
5	1.94093380004352\\
6	1.9358937287336\\
7	1.93953862201703\\
8	1.9771528727676\\
9	1.96834637986995\\
10	2.12660136918566\\
11	2.03673430039263\\
12	2.01268686658586\\
13	2.01698766709781\\
14	1.84279497260678\\
15	2.00164902278947\\
16	2.09460771999247\\
17	2.04257816341812\\
18	2.12812067993302\\
19	2.12138163189761\\
20	2.0625414274544\\
21	2.06221226740546\\
22	2.08858806772501\\
23	2.1085576416507\\
24	2.11695168130969\\
25	2.11884352293369\\
26	2.27845217791114\\
27	2.13188273293057\\
28	2.2958260078983\\
29	2.36145825292486\\
30	2.35296425246966\\
31	2.40675658285026\\
32	2.40474114323719\\
33	2.31045758149772\\
34	2.34457118447211\\
35	2.4153213714403\\
36	2.62643897627921\\
37	2.54331345642505\\
38	2.5631609465458\\
39	2.66570288041185\\
40	2.73083388981557\\
41	2.79542599543643\\
42	2.9032614798364\\
43	2.97833691890515\\
44	3.02154550379851\\
45	3.14889199955147\\
46	3.22320968920339\\
47	3.68837266262699\\
48	3.62315673508141\\
49	3.95134251699233\\
50	4.12410179466886\\
51	4.48042444893653\\
52	4.75538497619942\\
53	5.49179425478158\\
54	6.59567935877715\\
55	8.8620418620446\\
56	15.0805575299783\\
57	40.0918635305862\\
58	467.366777344471\\
59	212.860517690975\\
60	120.635250334277\\
61	23.8443247966236\\
62	15.32203908096\\
63	18.8754409707475\\
64	46.960564364429\\
65	921.409134695644\\
66	624.956756008045\\
67	44.3462878157575\\
68	15.8434588046976\\
69	11.0313705900366\\
70	14.7148791080368\\
71	41.2129123274094\\
72	1165.78339202722\\
73	1083.47517174755\\
74	61.9353118582872\\
75	30.6317777939068\\
76	26.2331153776291\\
77	32.0589292149724\\
78	61.208296977669\\
79	807.193952491206\\
80	1054.00562008634\\
81	51.3086401021939\\
82	25.0552367271964\\
83	24.645435876197\\
84	34.3880585361111\\
85	89.2616429763252\\
86	412.022171953476\\
87	707.385505854269\\
88	70.4909740208722\\
89	37.3708696489367\\
90	28.1872662770878\\
91	31.8181059014351\\
92	214.803956656843\\
93	167.035155026629\\
94	352.24109571606\\
95	74.9937503692081\\
96	47.4843272285187\\
97	41.2377883806217\\
98	53.1785040986701\\
99	554.177762709179\\
100	273.400370982818\\
101	62.4294210083709\\
102	24.0867058952232\\
103	29.0626937054166\\
104	60.8987997357927\\
105	306.705302279755\\
106	5031.43973780062\\
107	10058.7117990739\\
108	486.334706443482\\
109	171.388272475032\\
110	76.8102436538542\\
111	41.5496908073745\\
112	34.7807147450841\\
113	1432.7499803385\\
114	1675.52928762012\\
115	173.681500952215\\
116	129.429083547151\\
117	131.33153604024\\
118	146.04394046422\\
119	181.461838494977\\
120	219.699661675122\\
121	634.57507614067\\
122	257.072279734164\\
123	419.825796876926\\
124	763.967800150026\\
125	1759.99290585144\\
126	10117.5535871621\\
127	22350.9565937575\\
128	52005.592228381\\
129	4429.99410666212\\
130	1517.5166733067\\
131	760.384400265963\\
132	440.85626253539\\
133	326.167164976106\\
134	84.2369082187049\\
135	403.975442296336\\
136	189.002195910455\\
137	152.035304484769\\
138	139.704510429657\\
139	154.469781784144\\
140	928.719587239615\\
141	974.893785657169\\
142	60.5969765410862\\
143	48.8125082224575\\
144	62.2348167674982\\
145	102.870139666629\\
146	393.548183623953\\
147	5781.49829683031\\
148	9600.11309481653\\
149	646.908659456387\\
150	251.700343749591\\
151	142.601347529567\\
152	97.7909012341403\\
153	88.1273955882369\\
154	778.168464203035\\
155	1035.20520987721\\
156	74.4928058637182\\
157	47.7002388054471\\
158	44.7046391902101\\
159	52.6983762010651\\
160	87.1865984504365\\
161	369.965474095604\\
162	712.37171516924\\
163	43.5249895781135\\
164	27.6563750594264\\
165	30.7407061806614\\
166	41.4463707434985\\
167	169.057497144586\\
168	207.224769574435\\
169	393.275327237745\\
170	52.5608321701357\\
171	26.5607340504985\\
172	24.3550232952744\\
173	44.8069531340425\\
174	473.552490683966\\
175	219.636189360627\\
176	112.430916983961\\
177	27.2388390380248\\
178	20.1657757051255\\
179	21.6440464932856\\
180	49.6055785788647\\
181	904.160241436579\\
182	629.458968056839\\
183	44.4994350956159\\
184	20.2320153874046\\
185	17.2463729802717\\
186	24.5568335642233\\
187	57.9769881936061\\
188	1170.87795976595\\
189	1099.66918779979\\
190	44.4768291335985\\
191	17.8383389635699\\
192	14.2504589958134\\
193	18.6065341225249\\
194	39.8488773241925\\
195	781.857147753294\\
196	1052.95949881977\\
197	54.6145182595696\\
198	20.7228409375793\\
199	12.3911543835641\\
200	9.45644508502595\\
201	7.58770076056527\\
202	6.70432818996595\\
203	6.23214022376627\\
204	5.66638646341624\\
205	5.42401685490908\\
206	5.16946424795573\\
207	5.05526728453367\\
208	4.79270036541067\\
209	4.34485583684973\\
210	4.19585936898847\\
211	4.00205564088481\\
212	3.89409316323992\\
213	3.78251711220201\\
214	3.5544127265486\\
215	3.56839797454136\\
216	3.33975364055965\\
217	3.1982732388029\\
218	3.1706795359667\\
219	3.02504835985895\\
220	2.91675093067297\\
221	2.93859595723472\\
222	2.89170407386251\\
223	2.81003807626215\\
224	2.62368908556373\\
225	2.6936235341362\\
226	2.57290357536041\\
227	2.41203824206536\\
228	2.43818227906128\\
229	2.48520498880339\\
230	2.2298127822684\\
231	2.40054548431245\\
232	2.27833759145899\\
233	2.30355762862229\\
234	2.23527485425152\\
235	2.35724671848064\\
236	2.23265253940036\\
237	2.21914065307578\\
238	2.08545587217413\\
239	2.20622021835369\\
240	2.17663890274453\\
241	2.15226712266348\\
242	2.26672380614512\\
243	2.00511541234779\\
244	1.98795345957498\\
245	1.97291109691645\\
246	2.08455595601723\\
247	2.0180567846452\\
248	2.04170701359294\\
249	2.11865871826218\\
250	1.96113633416688\\
251	1.92609804526956\\
252	1.95535187999296\\
253	1.9345429363074\\
254	1.90359892233789\\
255	2.02345059762138\\
256	1.92279491529172\\
};
\addlegendentry{Scatterer variance}

\addplot [color=black, only marks, mark size=4.0pt, mark=o, mark options={solid, black}]
  table[row sep=crcr]{%
59	212.860517690975\\
66	624.956756008045\\
73	1083.47517174755\\
79	807.193952491206\\
86	412.022171953476\\
93	167.035155026629\\
100	273.400370982818\\
106	5031.43973780062\\
107	10058.7117990739\\
113	1432.7499803385\\
114	1675.52928762012\\
120	219.699661675122\\
124	763.967800150026\\
125	1759.99290585144\\
126	10117.5535871621\\
127	22350.9565937575\\
128	52005.592228381\\
129	4429.99410666212\\
130	1517.5166733067\\
134	84.2369082187049\\
141	974.893785657169\\
147	5781.49829683031\\
148	9600.11309481653\\
154	778.168464203035\\
161	369.965474095604\\
168	207.224769574435\\
175	219.636189360627\\
182	629.458968056839\\
189	1099.66918779979\\
195	781.857147753294\\
};
\addlegendentry{Candidate scatterers}

\addplot [color=red]
  table[row sep=crcr]{%
0	84.2369082187049\\
300	84.2369082187049\\
};
\addlegendentry{Minimum variance}

\addplot [color=red, only marks, mark size=4.0pt, mark=*, mark options={solid, fill=red, red}]
  table[row sep=crcr]
                \caption{Scatterer variance.\label{subfig:apndxA_hayAF_SCRA_sim_var}}
        \end{subfigure}
        \hspace{1cm}
        \begin{subfigure}{0.4\linewidth}
                \centering
                \resizebox{\linewidth}{!}{% This file was created by matlab2tikz.
%
%The latest updates can be retrieved from
%  http://www.mathworks.com/matlabcentral/fileexchange/22022-matlab2tikz-matlab2tikz
%where you can also make suggestions and rate matlab2tikz.
%
\definecolor{mycolor1}{rgb}{0.00000,0.44700,0.74100}%
%
\begin{tikzpicture}

\begin{axis}[%
width=6.028in,
height=4.754in,
at={(1.011in,0.642in)},
scale only axis,
xmin=0,
xmax=256,
xlabel style={font=\fontsize{25}{20}\selectfont\color{black}, yshift = -10},
xlabel={Range Bin Number},
ymin=0,
ymax=0.2e4,
ylabel style={font=\fontsize{25}{20}\selectfont\color{black}, yshift=10pt},
ylabel={Amplitude Variance},
axis background/.style={fill=white},
tick label style={font=\fontsize{20}{11}\selectfont\color{black}},
xtick distance = 50,
ytick distance=2e2,
yticklabel={\ifdim\tick pt=0pt\else\pgfmathprintnumber{\tick}\fi}, 
scaled y ticks=base 10:-3,
legend style={legend cell align=left, align=left, draw=white!15!black, font=\fontsize{12}{11}\selectfont\color{black}}
]
\addplot [color=mycolor1, mark=asterisk, mark options={solid, mycolor1}]
  table[row sep=crcr]{%
1	2.05737908304292\\
2	1.85575449052375\\
3	1.94419916622086\\
4	1.97731558212904\\
5	1.94093380004352\\
6	1.9358937287336\\
7	1.93953862201703\\
8	1.9771528727676\\
9	1.96834637986995\\
10	2.12660136918566\\
11	2.03673430039263\\
12	2.01268686658586\\
13	2.01698766709781\\
14	1.84279497260678\\
15	2.00164902278947\\
16	2.09460771999247\\
17	2.04257816341812\\
18	2.12812067993302\\
19	2.12138163189761\\
20	2.0625414274544\\
21	2.06221226740546\\
22	2.08858806772501\\
23	2.1085576416507\\
24	2.11695168130969\\
25	2.11884352293369\\
26	2.27845217791114\\
27	2.13188273293057\\
28	2.2958260078983\\
29	2.36145825292486\\
30	2.35296425246966\\
31	2.40675658285026\\
32	2.40474114323719\\
33	2.31045758149772\\
34	2.34457118447211\\
35	2.4153213714403\\
36	2.62643897627921\\
37	2.54331345642505\\
38	2.5631609465458\\
39	2.66570288041185\\
40	2.73083388981557\\
41	2.79542599543643\\
42	2.9032614798364\\
43	2.97833691890515\\
44	3.02154550379851\\
45	3.14889199955147\\
46	3.22320968920339\\
47	3.68837266262699\\
48	3.62315673508141\\
49	3.95134251699233\\
50	4.12410179466886\\
51	4.48042444893653\\
52	4.75538497619942\\
53	5.49179425478158\\
54	6.59567935877715\\
55	8.8620418620446\\
56	15.0805575299783\\
57	40.0918635305862\\
58	467.366777344471\\
59	212.860517690975\\
60	120.635250334277\\
61	23.8443247966236\\
62	15.32203908096\\
63	18.8754409707475\\
64	46.960564364429\\
65	921.409134695644\\
66	624.956756008045\\
67	44.3462878157575\\
68	15.8434588046976\\
69	11.0313705900366\\
70	14.7148791080368\\
71	41.2129123274094\\
72	1165.78339202722\\
73	1083.47517174755\\
74	61.9353118582872\\
75	30.6317777939068\\
76	26.2331153776291\\
77	32.0589292149724\\
78	61.208296977669\\
79	807.193952491206\\
80	1054.00562008634\\
81	51.3086401021939\\
82	25.0552367271964\\
83	24.645435876197\\
84	34.3880585361111\\
85	89.2616429763252\\
86	412.022171953476\\
87	707.385505854269\\
88	70.4909740208722\\
89	37.3708696489367\\
90	28.1872662770878\\
91	31.8181059014351\\
92	214.803956656843\\
93	167.035155026629\\
94	352.24109571606\\
95	74.9937503692081\\
96	47.4843272285187\\
97	41.2377883806217\\
98	53.1785040986701\\
99	554.177762709179\\
100	273.400370982818\\
101	62.4294210083709\\
102	24.0867058952232\\
103	29.0626937054166\\
104	60.8987997357927\\
105	306.705302279755\\
106	5031.43973780062\\
107	10058.7117990739\\
108	486.334706443482\\
109	171.388272475032\\
110	76.8102436538542\\
111	41.5496908073745\\
112	34.7807147450841\\
113	1432.7499803385\\
114	1675.52928762012\\
115	173.681500952215\\
116	129.429083547151\\
117	131.33153604024\\
118	146.04394046422\\
119	181.461838494977\\
120	219.699661675122\\
121	634.57507614067\\
122	257.072279734164\\
123	419.825796876926\\
124	763.967800150026\\
125	1759.99290585144\\
126	10117.5535871621\\
127	22350.9565937575\\
128	52005.592228381\\
129	4429.99410666212\\
130	1517.5166733067\\
131	760.384400265963\\
132	440.85626253539\\
133	326.167164976106\\
134	84.2369082187049\\
135	403.975442296336\\
136	189.002195910455\\
137	152.035304484769\\
138	139.704510429657\\
139	154.469781784144\\
140	928.719587239615\\
141	974.893785657169\\
142	60.5969765410862\\
143	48.8125082224575\\
144	62.2348167674982\\
145	102.870139666629\\
146	393.548183623953\\
147	5781.49829683031\\
148	9600.11309481653\\
149	646.908659456387\\
150	251.700343749591\\
151	142.601347529567\\
152	97.7909012341403\\
153	88.1273955882369\\
154	778.168464203035\\
155	1035.20520987721\\
156	74.4928058637182\\
157	47.7002388054471\\
158	44.7046391902101\\
159	52.6983762010651\\
160	87.1865984504365\\
161	369.965474095604\\
162	712.37171516924\\
163	43.5249895781135\\
164	27.6563750594264\\
165	30.7407061806614\\
166	41.4463707434985\\
167	169.057497144586\\
168	207.224769574435\\
169	393.275327237745\\
170	52.5608321701357\\
171	26.5607340504985\\
172	24.3550232952744\\
173	44.8069531340425\\
174	473.552490683966\\
175	219.636189360627\\
176	112.430916983961\\
177	27.2388390380248\\
178	20.1657757051255\\
179	21.6440464932856\\
180	49.6055785788647\\
181	904.160241436579\\
182	629.458968056839\\
183	44.4994350956159\\
184	20.2320153874046\\
185	17.2463729802717\\
186	24.5568335642233\\
187	57.9769881936061\\
188	1170.87795976595\\
189	1099.66918779979\\
190	44.4768291335985\\
191	17.8383389635699\\
192	14.2504589958134\\
193	18.6065341225249\\
194	39.8488773241925\\
195	781.857147753294\\
196	1052.95949881977\\
197	54.6145182595696\\
198	20.7228409375793\\
199	12.3911543835641\\
200	9.45644508502595\\
201	7.58770076056527\\
202	6.70432818996595\\
203	6.23214022376627\\
204	5.66638646341624\\
205	5.42401685490908\\
206	5.16946424795573\\
207	5.05526728453367\\
208	4.79270036541067\\
209	4.34485583684973\\
210	4.19585936898847\\
211	4.00205564088481\\
212	3.89409316323992\\
213	3.78251711220201\\
214	3.5544127265486\\
215	3.56839797454136\\
216	3.33975364055965\\
217	3.1982732388029\\
218	3.1706795359667\\
219	3.02504835985895\\
220	2.91675093067297\\
221	2.93859595723472\\
222	2.89170407386251\\
223	2.81003807626215\\
224	2.62368908556373\\
225	2.6936235341362\\
226	2.57290357536041\\
227	2.41203824206536\\
228	2.43818227906128\\
229	2.48520498880339\\
230	2.2298127822684\\
231	2.40054548431245\\
232	2.27833759145899\\
233	2.30355762862229\\
234	2.23527485425152\\
235	2.35724671848064\\
236	2.23265253940036\\
237	2.21914065307578\\
238	2.08545587217413\\
239	2.20622021835369\\
240	2.17663890274453\\
241	2.15226712266348\\
242	2.26672380614512\\
243	2.00511541234779\\
244	1.98795345957498\\
245	1.97291109691645\\
246	2.08455595601723\\
247	2.0180567846452\\
248	2.04170701359294\\
249	2.11865871826218\\
250	1.96113633416688\\
251	1.92609804526956\\
252	1.95535187999296\\
253	1.9345429363074\\
254	1.90359892233789\\
255	2.02345059762138\\
256	1.92279491529172\\
};
\addlegendentry{Scatterer variance}

\addplot [color=black, only marks, mark size=4.0pt, mark=o, mark options={solid, black}]
  table[row sep=crcr]{%
59	212.860517690975\\
66	624.956756008045\\
73	1083.47517174755\\
79	807.193952491206\\
86	412.022171953476\\
93	167.035155026629\\
100	273.400370982818\\
106	5031.43973780062\\
107	10058.7117990739\\
113	1432.7499803385\\
114	1675.52928762012\\
120	219.699661675122\\
124	763.967800150026\\
125	1759.99290585144\\
126	10117.5535871621\\
127	22350.9565937575\\
128	52005.592228381\\
129	4429.99410666212\\
130	1517.5166733067\\
134	84.2369082187049\\
141	974.893785657169\\
147	5781.49829683031\\
148	9600.11309481653\\
154	778.168464203035\\
161	369.965474095604\\
168	207.224769574435\\
175	219.636189360627\\
182	629.458968056839\\
189	1099.66918779979\\
195	781.857147753294\\
};
\addlegendentry{Candidate scatterers}

\addplot [color=red]
  table[row sep=crcr]{%
0	84.2369082187049\\
300	84.2369082187049\\
};
\addlegendentry{Minimum variance}

\addplot [color=red, only marks, mark size=4.0pt, mark=*, mark options={solid, fill=red, red}]
  table[row sep=crcr]
                \caption{Scatterer amplitude variance (zoomed-in). \label{subfig:apndxA_hayAF_SCRA_sim_var_zoomed}}
        \end{subfigure}
        \caption{\gls{sdsaf} \gls{ds} selection plots for Simple Correlation \gls{ra} \gls{hrr} profiles.\label{fig:apndxA_hayAF_SCRA_sim}}
    \end{figure}

    %***************************************************************************************%    
    
%%%%%%%%%%%%%%%%%%%%%%%%%%%%%%%%%%%%%%%%%%%%%%%%%%%%%%%%%%%%%%%%%%%%%%%%%%%%%%%%%%%%%%%%%%%%%%%%%%%%
\section{Scaling Factor for Dominant Scatterer Selection \label{apndxA:scale_factor_effect}}
In the Haywood \gls{af} algorithm, given in \autoref{alg:haywood_AF}, phase compensation is calculated based on the phase history of the \gls{ds} over all profiles. A set of selection criteria is used to determine the \gls{ds}:
\begin{itemize}
    \item Criterion 1: The set of candidate dominant scatterers includes all scatterers with a power greater than the average scatterer power (the threshold value) across all range bins.
    \item Criterion 2: The \gls{ds} is the candidate scatterer with the minimum amplitude variance.
\end{itemize}
%initially choosing candidate scatterers, from which the one with the minimum variance is selected as the \gls{ds}. The candidate scatterers are those with a power greater than the average scatterer power (the threshold value). 
As discussed in \autoref{theory:noise} of the theory chapter, external interference sources can influence the phase of the received signal. Consequently, selecting a scatterer severely affected by noise will impair the performance of the \gls{af}, resulting in a less focused \gls{isar} image compared to choosing a scatterer with less noise interference. Therefore, filtering is essential to enhance the performance of this algorithm. To this effect, an average power scaling factor (sf) was introduced to select higher power scatterers as candidates. 
%This is advantageous because higher power scatterers tend to have a phase history that is less affected by noise. 
The implementation of the sf variable is shown in \autoref{code:scale_factor}.
% !!!!!!!!!!!!!! DO NOT INDENT IT MESSES UP THE LATEX !!!!!!!!!!!!!!!!!!!!!!!!!!!!!!!!!
\begin{lstlisting}[caption={\textsc{Matlab} code for selecting candidate dominant scatterers.},label={code:scale_factor}]
candidate_scatterers_idx = find(power_scatterers>scaling_factor*average_power_scatterers);
\end{lstlisting}
% !!!!!!!!!!!!!!!!!!!!!!!!!!!!!!!!!!!!!!!!!!!!!!!!!!!!!!!!!!!!!!!!!!!!!!!!!!!!!!!!!!!!!!!!!!!!!!!!!!!
\gls{af} is applied to both the Simple Correlation and Haywood range-aligned profiles. The following subsections discuss the effect of the sf value on the selection of the \gls{ds} and subsequently the \gls{isar} image for both sets of range-aligned profiles.

    %***************************************************************************************%    
    \subsection{Using Simple Correlation Range-aligned Profiles} \label{subsec:sf_corrRA}
    The Simple correlation range-aligned \gls{isar} image, as shown in \autoref{fig:sf_corrRA_isar}, is included for comparison to the autofocused \gls{isar} images, \autoref{fig:sf_corrRA}(a)-(c). Additionally, the \gls{isar} image generated when the object had no translational motion, as shown in \autoref{subfig:sf_focused_isar}, is a reference for the expected appearance of a focused image.
    
    \begin{figure}[H]
        \centering
        \begin{minipage}{0.4\linewidth}
            \begin{subfigure}{\linewidth}
                \centering
                \resizebox{\linewidth}{!}{% This file was created by matlab2tikz.
%
%The latest updates can be retrieved from
%  http://www.mathworks.com/matlabcentral/fileexchange/22022-matlab2tikz-matlab2tikz
%where you can also make suggestions and rate matlab2tikz.
%
\begin{tikzpicture}

\begin{axis}[%
width=5.554in,
height=4.754in,
at={(0.932in,0.642in)},
scale only axis,
point meta min=-35,
point meta max=0,
axis on top,
xmin=-0.0732421875,
xmax=37.4267578125,
xlabel style={font=\color{white!15!black}},
xlabel={Range (m)},
ymin=-32.2265625,
ymax=30.2734375,
ylabel style={font=\color{white!15!black}},
ylabel={Doppler frquency (Hz)},
axis background/.style={fill=white},
title style={font=\bfseries},
title={Range-aligned ISAR image},
colormap/jet,
colorbar
]
\addplot [forget plot] graphics [xmin=-0.0732421875, xmax=37.4267578125, ymin=-32.2265625, ymax=30.2734375] {SCRA_Sim_ISAR_1mps_6deg-1.png};
\end{axis}

\begin{axis}[%
width=7.778in,
height=5.833in,
at={(0in,0in)},
scale only axis,
point meta min=0,
point meta max=1,
xmin=0,
xmax=1,
ymin=0,
ymax=1,
axis line style={draw=none},
ticks=none,
axis x line*=bottom,
axis y line*=left
]
\end{axis}
\end{tikzpicture}%}
            \end{subfigure}
            \caption{Range-aligned image.\label{subfig:sf_corrRA_isar}}   
        \end{minipage}
        \hspace{1cm}
        \begin{minipage}{0.4\linewidth}
            \begin{subfigure}{\linewidth}
                \centering
                \resizebox{\linewidth}{!}{% This file was created by matlab2tikz.
%
%The latest updates can be retrieved from
%  http://www.mathworks.com/matlabcentral/fileexchange/22022-matlab2tikz-matlab2tikz
%where you can also make suggestions and rate matlab2tikz.
%
\begin{tikzpicture}

    \begin{axis}[%
    width=5.554in,
    height=4.754in,
    at={(0.932in,0.642in)},
    scale only axis,
    point meta min=-40,
    point meta max=0,
    axis on top,
    xmin=-0.0732421875,
    xmax=37.4267578125,
    xlabel style={font=\color{white!15!black}},
    xlabel={Range (m)},
    ymin=-32.2265625,
    ymax=30.2734375,
    ylabel style={font=\color{white!15!black}},
    ylabel={Doppler frquency (Hz)},
    axis background/.style={fill=white},
    colormap/jet,
    colorbar
    ]
    \addplot [forget plot] graphics [xmin=-0.0732421875, xmax=37.4267578125, ymin=-32.2265625, ymax=30.2734375] {Sim_ISAR_0mps_6deg.png};
    \end{axis}
    
    \begin{axis}[%
    width=7.778in,
    height=5.833in,
    at={(0in,0in)},
    scale only axis,
    point meta min=0,
    point meta max=1,
    xmin=0,
    xmax=1,
    ymin=0,
    ymax=1,
    axis line style={draw=none},
    ticks=none,
    axis x line*=bottom,
    axis y line*=left
    ]
    \end{axis}
    \end{tikzpicture}%}
            \end{subfigure} 
            \caption{Reference focused image.\label{subfig:sf_focused_isar}}
        \end{minipage}
    \end{figure}
    
    \autoref{fig:sf_corrRA} shows autofocused ISAR images and plots demonstrating the fulfillment of both selection criteria for various \gls{sf} values. When comparing \autoref{fig:sf_corrRA}(d)-(f), it becomes evident that as the \gls{sf} increases, the number of candidate scatterers decreases. Consequently, the number of scatterers to which criterion 2 is applied also decreases, leading to the selection of scatterers with higher amplitude variance, as demonstrated in \autoref{fig:sf_corrRA}(g)-(i). Overall, higher power scatterers are chosen as the \gls{ds}, as shown by the larger circle in \autoref{fig:sf_corrRA}(d)-(f).

    Before comparing the autofocused \gls{isar} images, it is important to recall that sidelobes formed when using the \gls{fft} can result in Doppler spreading in \gls{isar} images, as discussed in \autoref{sec:theory_isar_complications}.  Although windowing was applied to minimise these sidelobes, it was not completely effective, as is evident in the images throughout this section. The spectral spreading caused by these sidelobes is present in the range-aligned image, and persists in the autofocused images.

    The impact of choosing a higher power \gls{ds} is reflected in \autoref{fig:sf_corrRA}(a) and (c), where, with increasing \gls{sf}, the \gls{isar} image becomes more focused. This is evident by comparing the \gls{ds} (brightest scatterer) in both of the images; in (c), the \gls{ds} power resides mostly at zero-Doppler, which is not the case in (a). This is consistent across most scatterers, where the power mostly resides in one place in (c) but not in (a). While \autoref{fig:sf_corrRA}(c) still exhibits effects of spectral spreading, which is indicated by the presence of 'shadows' in the image, it more closely resembles \autoref{subfig:sf_focused_isar} than (a). This is because the center of the image in (c) is distinctly positioned at zero-Doppler, and the Doppler spread is more evidently attributed to sidelobes.

    In the case of \autoref{fig:sf_corrRA}(b), the spectral spreading results in a notably defocused image. Unlike in \autoref{fig:sf_corrRA}(a), the spreading in (b) could be mistaken for another object, especially without prior knowledge that it is caused by sidelobes. The spreading is likely due to the selection of a \gls{ds} with a phase, which, when applied to all profiles and passed through the \gls{fft}, increases integrated sidelobes. Importantly, this doesn't imply that a \gls{sf} of 5 is inherently less desirable than a \gls{sf} of 1. Rather, it is a consequence of thee characteristics of this Simple Correlation dataset's characteristics and could occur regardless of the chosen \gls{sf}. This emphasises the importance of considering and processing multiple frames of \gls{hrr} profiles in \gls{isar} imaging to obtain a high-quality image of the object.

    The image focus can be quantitatively measured using the \gls{ic} value calculated as in \autoref{eq:image_contrast}. The values are \textbf{34.08}, \textbf{19.58}, and \textbf{38.42} for \autoref{fig:sf_corrRA}(a)-(c), respectively. These values indicate that (c) has the highest \gls{ic} value, confirming the previous analysis that the image is most focused for a \gls{sf} of 10. Therefore, by comparing \autoref{fig:sf_corrRA}(a) and (c) and their respective \gls{ic} values, it is evident that introducing a \gls{sf} can lead to a more focused \gls{isar} image. Specifically, for these Simple Correlation range-aligned profiles, a \gls{sf} of 10 produces the most focused image.
     
    % Grid of the HRRP and ISAR images SCRA
    \begin{figure}[H]
    \vspace*{\baselineskip}
    \centering
    \begin{minipage}{0.98\linewidth}
        \begin{tabular}{@{}ccc@{}}
            \begin{subfigure}{0.33\linewidth}
                \centering
                \resizebox{\linewidth}{!}{% This file was created by matlab2tikz.
%
%The latest updates can be retrieved from
%  http://www.mathworks.com/matlabcentral/fileexchange/22022-matlab2tikz-matlab2tikz
%where you can also make suggestions and rate matlab2tikz.
%
\begin{tikzpicture}
\begin{axis}[%
width=5.554in,
height=4.754in,
at={(0.932in,0.642in)},
scale only axis,
point meta min=-40,
point meta max=0,
axis on top,
xmin=-0.0732421875,
xmax=37.4267578125,
xlabel style={font=\fontsize{25}{14}\selectfont\color{black}, yshift=-10pt},
xlabel={Range (m)},
ymin=-32.2265625,
ymax=30.2734375,
ylabel style={font=\fontsize{25}{14}\selectfont\color{black}},
ylabel={Doppler frquency (Hz)},
axis background/.style={fill=white},
tick label style={font=\fontsize{20}{11}\selectfont\color{black}},
xtick distance= 4,             % Set the spacing between x-axis ticks
ytick distance = 10,
colormap/jet,
colorbar
]
\addplot [forget plot] graphics [xmin=-0.0732421875, xmax=37.4267578125, ymin=-32.2265625, ymax=30.2734375] {Figures/09Appendix/ScalingFactor/CorrRA/HayAF_SCRA_Sim_ISAR_sf1.png};
\end{axis}

\end{tikzpicture}%}
                \caption{Autofocused \gls{isar} image, \\ sf = 1.\label{subfig:sf1_corrRA_isar}}
            \end{subfigure}
            &
            \begin{subfigure}{0.33\linewidth}
                \centering
                \resizebox{\linewidth}{!}{% This file was created by matlab2tikz.
%
%The latest updates can be retrieved from
%  http://www.mathworks.com/matlabcentral/fileexchange/22022-matlab2tikz-matlab2tikz
%where you can also make suggestions and rate matlab2tikz.
%
\begin{tikzpicture}
\begin{axis}[%
width=5.554in,
height=4.754in,
at={(0.932in,0.642in)},
scale only axis,
point meta min=-40,
point meta max=0,
axis on top,
xmin=-0.0732421875,
xmax=37.4267578125,
xlabel style={font=\fontsize{25}{14}\selectfont\color{black}, yshift=-10pt},
xlabel={Range (m)},
ymin=-32.2265625,
ymax=30.2734375,
ylabel style={font=\fontsize{25}{14}\selectfont\color{black}},
ylabel={Doppler frquency (Hz)},
axis background/.style={fill=white},
tick label style={font=\fontsize{20}{11}\selectfont\color{black}},
xtick distance= 4,             % Set the spacing between x-axis ticks
ytick distance = 10,
colormap/jet,
colorbar
]
\addplot [forget plot] graphics [xmin=-0.0732421875, xmax=37.4267578125, ymin=-32.2265625, ymax=30.2734375] {Figures/09Appendix/ScalingFactor/CorrRA/HayAF_SCRA_Sim_ISAR_sf5.png};
\end{axis}

\end{tikzpicture}%}
                \caption{Autofocused \gls{isar} image, \\ sf = 5.\label{subfig:sf5_corrRA_isar}}
            \end{subfigure}
             &
            \begin{subfigure}{0.33\linewidth}
                \centering
                \resizebox{\linewidth}{!}{% This file was created by matlab2tikz.
%
%The latest updates can be retrieved from
%  http://www.mathworks.com/matlabcentral/fileexchange/22022-matlab2tikz-matlab2tikz
%where you can also make suggestions and rate matlab2tikz.
%
\begin{tikzpicture}
\begin{axis}[%
width=5.554in,
height=4.754in,
at={(0.932in,0.642in)},
scale only axis,
point meta min=-40,
point meta max=0,
axis on top,
xmin=-0.0732421875,
xmax=37.4267578125,
xlabel style={font=\fontsize{25}{14}\selectfont\color{black}, yshift=-10pt},
xlabel={Range (m)},
ymin=-32.2265625,
ymax=30.2734375,
ylabel style={font=\fontsize{25}{14}\selectfont\color{black}},
ylabel={Doppler frquency (Hz)},
axis background/.style={fill=white},
tick label style={font=\fontsize{20}{11}\selectfont\color{black}},
xtick distance= 4,             % Set the spacing between x-axis ticks
ytick distance = 10,
colormap/jet,
colorbar
]
\addplot [forget plot] graphics [xmin=-0.0732421875, xmax=37.4267578125, ymin=-32.2265625, ymax=30.2734375] {Figures/09Appendix/ScalingFactor/CorrRA/ISAR/HayAF_SCRA_Sim_ISAR_sf10.png};
\end{axis}

\end{tikzpicture}%}
                \caption{Autofocused \gls{isar} image, \\ sf = 10.\label{subfig:sf10_corrRA_isar}}
            \end{subfigure}
            \\
            \begin{subfigure}{0.33\linewidth}
                \centering
                \resizebox{\linewidth}{!}{% This file was created by matlab2tikz.
%
%The latest updates can be retrieved from
%  http://www.mathworks.com/matlabcentral/fileexchange/22022-matlab2tikz-matlab2tikz
%where you can also make suggestions and rate matlab2tikz.
%
\definecolor{mycolor1}{rgb}{0.00000,0.44700,0.74100}%
%
\begin{tikzpicture}

\begin{axis}[%
width=6.028in,
height=4.754in,
at={(1.011in,0.642in)},
scale only axis,
xmin=0,
xmax=256,
xlabel style={font=\fontsize{25}{20}\selectfont\color{black}, yshift = -10},
xlabel={Range Bin},
ymin=0,
ymax=0.1e7,
ylabel style={font=\fontsize{25}{20}\selectfont\color{black}, yshift=10pt},
ylabel={Power},
axis background/.style={fill=white},
tick label style={font=\fontsize{20}{11}\selectfont\color{black}},
xtick distance = 50,
ytick distance= 0.1e6,
yticklabel={\ifdim\tick pt=0pt\else\pgfmathprintnumber{\tick}\fi}, 
scaled y ticks=base 10:-6,
legend style={legend cell align=left, align = left, draw=white!15!black, font=\fontsize{12}{11}\selectfont\color{black}}
]
\addplot [color=mycolor1, mark=asterisk, mark options={solid, mycolor1}]
  table[row sep=crcr]{%
1	262.577633405177\\
2	262.223997203794\\
3	260.587746143257\\
4	262.165948835539\\
5	266.370740045406\\
6	269.751691129423\\
7	268.372577835589\\
8	264.328240273671\\
9	267.210920185527\\
10	267.159343588493\\
11	272.471964953293\\
12	271.733768437709\\
13	273.204252219756\\
14	273.401297413845\\
15	279.939014967128\\
16	271.317790363962\\
17	276.66694833236\\
18	287.453857022635\\
19	282.498378926959\\
20	282.424615908933\\
21	282.237995184674\\
22	287.182445609935\\
23	296.88246320039\\
24	293.376809736088\\
25	299.544038232535\\
26	307.945899499856\\
27	308.34916592009\\
28	302.46767851089\\
29	313.545932424688\\
30	319.693901409099\\
31	327.838402778644\\
32	325.412275000734\\
33	329.962225473231\\
34	339.780823710529\\
35	345.065837305361\\
36	351.620452666554\\
37	364.37120330017\\
38	373.063512866931\\
39	374.996429565808\\
40	388.481018840225\\
41	402.028509784367\\
42	410.253994002813\\
43	415.872942397272\\
44	433.048131010869\\
45	459.915080945587\\
46	473.409570465863\\
47	509.273698648161\\
48	534.084651023651\\
49	571.863528488545\\
50	632.011666961169\\
51	686.10388722041\\
52	800.909763589265\\
53	926.970635820219\\
54	1169.47290501226\\
55	1606.85543326294\\
56	2579.98910836732\\
57	5803.63039172138\\
58	40476.8926273612\\
59	234509.09387101\\
60	16812.5905675379\\
61	4748.44031343692\\
62	3258.56766572553\\
63	3609.19091675377\\
64	7271.89793873345\\
65	76462.8638749144\\
66	213763.19883202\\
67	10080.9121387595\\
68	4505.24246037311\\
69	3559.60602970705\\
70	4097.00870601672\\
71	8129.95852185237\\
72	127110.777437309\\
73	167004.96702928\\
74	9571.29347875923\\
75	4973.42213117302\\
76	4303.78311248336\\
77	5199.58233305912\\
78	10381.4623302359\\
79	182254.013181938\\
80	108107.786003549\\
81	8587.05940745672\\
82	4386.38218972955\\
83	3888.88544639908\\
84	5201.55122335099\\
85	13990.6637254485\\
86	227467.353622077\\
87	63195.4072386991\\
88	8193.22529846067\\
89	4118.72807301388\\
90	3448.91074307703\\
91	4833.85184580242\\
92	22694.9097035337\\
93	244827.045465842\\
94	37643.2398942935\\
95	10525.6936569466\\
96	7119.48875276094\\
97	6687.54158992564\\
98	9259.70452372373\\
99	51581.5898604238\\
100	239626.484632749\\
101	21081.8937461551\\
102	12390.466664605\\
103	12969.8628577967\\
104	18553.0224931073\\
105	50425.0459051908\\
106	492159.951355135\\
107	722466.661470341\\
108	42845.4679587927\\
109	18904.8966167844\\
110	13311.8882371492\\
111	11892.6969105579\\
112	14231.0511261405\\
113	161310.022127177\\
114	207383.060829013\\
115	32230.4864489048\\
116	26689.6559112108\\
117	28029.5540798509\\
118	33541.5756317694\\
119	47288.5364117497\\
120	243049.420993879\\
121	113356.462295404\\
122	67951.6373400222\\
123	101249.279715054\\
124	175647.487522582\\
125	382031.425633381\\
126	1567528.09945156\\
127	23222434.9618517\\
128	4866763.94966091\\
129	624247.624633293\\
130	239143.421720153\\
131	125461.785185967\\
132	78179.4257393306\\
133	73077.6537215823\\
134	271201.515747773\\
135	65931.0424830517\\
136	35672.0138791419\\
137	27654.2722732861\\
138	24583.9369850773\\
139	26999.5479715209\\
140	93379.6124259705\\
141	280300.088807601\\
142	15205.4776885098\\
143	11661.1310430251\\
144	12769.6894139469\\
145	18402.2829947879\\
146	54225.2198477609\\
147	558976.693699555\\
148	675665.481591376\\
149	46962.7983886691\\
150	20393.9719870215\\
151	13535.4930230347\\
152	11761.055490961\\
153	14768.0813924932\\
154	165989.325372009\\
155	129169.53691609\\
156	12868.1621171875\\
157	8411.35286185695\\
158	7933.21185777849\\
159	9588.44659244235\\
160	18160.9178834827\\
161	221597.321922832\\
162	66308.3324321176\\
163	6306.92369286591\\
164	3520.35951295018\\
165	3676.63483627014\\
166	5820.45767786015\\
167	21232.7602733964\\
168	249945.230252575\\
169	37403.2425101297\\
170	6704.96902736564\\
171	3832.53243857378\\
172	3700.63923155125\\
173	6150.75623269814\\
174	39604.4771244479\\
175	244069.97144358\\
176	19163.6650471279\\
177	6543.53768190825\\
178	4763.80584744868\\
179	5143.26651622724\\
180	8981.76443566913\\
181	79701.9991589925\\
182	213463.509099999\\
183	9205.80202458775\\
184	4369.7249562291\\
185	3773.22980782136\\
186	4606.80345066959\\
187	9211.49891750507\\
188	128870.792246111\\
189	162669.293467407\\
190	7576.16282016939\\
191	3496.35460457662\\
192	2909.90245100321\\
193	3774.65158520432\\
194	8582.92339736164\\
195	174796.753846002\\
196	107450.969942292\\
197	8153.14181718801\\
198	3237.84037236851\\
199	1875.56915152923\\
200	1335.84461730023\\
201	1033.91058588523\\
202	856.439259758838\\
203	762.653045285108\\
204	668.930743746777\\
205	625.901072207345\\
206	556.486706127849\\
207	530.503276413855\\
208	496.045662997232\\
209	476.96061148105\\
210	460.039553501425\\
211	440.427277040158\\
212	420.744070083467\\
213	406.26548994886\\
214	391.074160852518\\
215	379.015338937019\\
216	381.002373420812\\
217	370.718395878786\\
218	346.158491498091\\
219	357.849584994842\\
220	352.449619069298\\
221	340.339510167936\\
222	345.10518776573\\
223	327.335679051124\\
224	323.783159711637\\
225	317.325257908808\\
226	304.192683315749\\
227	295.199951649738\\
228	305.86238727585\\
229	296.591571815832\\
230	298.662236201926\\
231	298.492937094633\\
232	290.683540518951\\
233	290.366845415873\\
234	289.732973521913\\
235	290.309428094255\\
236	280.995516102322\\
237	278.531696332776\\
238	276.730525062284\\
239	274.924875519189\\
240	271.13797603103\\
241	270.958320234206\\
242	272.70345874414\\
243	276.476328776733\\
244	273.23394172863\\
245	264.938748656207\\
246	266.485536484425\\
247	265.766858599058\\
248	265.207047749107\\
249	258.427321219398\\
250	268.381198158333\\
251	261.430204439809\\
252	264.739520910823\\
253	262.567571330099\\
254	266.722422494786\\
255	261.47945048988\\
256	265.524771370158\\
};
\addlegendentry{Scatterer power}

\addplot [color=orange]
  table[row sep=crcr]{%
0	158835.729579474\\
300	158835.729579474\\
};
\addlegendentry{Average power of all scatterers}

\addplot [color=black, only marks, mark size=4.0pt, mark=o, mark options={solid, black}]
  table[row sep=crcr]{%
59	234509.09387101\\
66	213763.19883202\\
73	167004.96702928\\
79	182254.013181938\\
86	227467.353622077\\
93	244827.045465842\\
100	239626.484632749\\
106	492159.951355135\\
107	722466.661470341\\
113	161310.022127177\\
114	207383.060829013\\
120	243049.420993879\\
124	175647.487522582\\
125	382031.425633381\\
126	1567528.09945156\\
127	23222434.9618517\\
128	4866763.94966091\\
129	624247.624633293\\
130	239143.421720153\\
134	271201.515747773\\
141	280300.088807601\\
147	558976.693699555\\
148	675665.481591376\\
154	165989.325372009\\
161	221597.321922832\\
168	249945.230252575\\
175	244069.97144358\\
182	213463.509099999\\
189	162669.293467407\\
195	174796.753846002\\
};
\addlegendentry{Candidate scatterers}

\addplot [color=red, only marks, mark size=7.5pt, mark=o, mark options={solid, red}]
  table[row sep=crcr]
                \caption{Scatterer power, \\ sf = 1.\label{subfig:sf1_corrRA_power}}
            \end{subfigure}
             &
            \begin{subfigure}{0.33\linewidth}
                \centering
                \resizebox{\linewidth}{!}{% This file was created by matlab2tikz.
%
%The latest updates can be retrieved from
%  http://www.mathworks.com/matlabcentral/fileexchange/22022-matlab2tikz-matlab2tikz
%where you can also make suggestions and rate matlab2tikz.
%
\definecolor{mycolor1}{rgb}{0.00000,0.44700,0.74100}%
%
\begin{tikzpicture}

\begin{axis}[%
width=6.028in,
height=4.754in,
at={(1.011in,0.642in)},
scale only axis,
xmin=0,
xmax=256,
xlabel style={font=\fontsize{25}{20}\selectfont\color{black}, yshift = -10},
xlabel={Range Bin},
ymin=0,
ymax=0.2e7,
ylabel style={font=\fontsize{25}{20}\selectfont\color{black}, yshift=10pt},
ylabel={Power},
axis background/.style={fill=white},
tick label style={font=\fontsize{20}{11}\selectfont\color{black}},
xtick distance = 50,
ytick distance= 0.2e6,
yticklabel={\ifdim\tick pt=0pt\else\pgfmathprintnumber{\tick}\fi}, 
scaled y ticks=base 10:-6,
legend style={legend cell align=left, align=left, draw=white!15!black, font=\fontsize{12}{11}\selectfont\color{black}}
]
\addplot [color=mycolor1, mark=asterisk, mark options={solid, mycolor1}]
  table[row sep=crcr]{%
1	267.590403019806\\
2	255.400968720464\\
3	264.999912312277\\
4	262.539753606501\\
5	261.818417045418\\
6	259.15658133819\\
7	264.817711251638\\
8	262.058780690642\\
9	262.840089084763\\
10	266.144147892013\\
11	267.54053374168\\
12	268.812123087204\\
13	266.485018620937\\
14	266.321566167241\\
15	270.424770217735\\
16	282.047638635658\\
17	286.386800530431\\
18	283.397256119929\\
19	277.299315120883\\
20	286.016840486917\\
21	290.983367844274\\
22	285.858487721665\\
23	293.132533075198\\
24	281.135394758485\\
25	298.253256486322\\
26	297.893202974783\\
27	303.787988123915\\
28	298.598345463695\\
29	317.925516046478\\
30	316.977325686178\\
31	320.098331654412\\
32	331.157223148045\\
33	345.514345303823\\
34	333.794692741485\\
35	338.767343531348\\
36	340.100850905873\\
37	365.176234348452\\
38	372.220240788854\\
39	371.123466580276\\
40	395.974354115365\\
41	394.878451441685\\
42	415.752888381649\\
43	426.594244085755\\
44	433.003089511907\\
45	453.923958176231\\
46	476.476159925392\\
47	503.550671601751\\
48	541.64219679519\\
49	567.793341857371\\
50	642.477554924576\\
51	692.489020420194\\
52	794.003581796397\\
53	928.187586114199\\
54	1173.74462908604\\
55	1605.5889473773\\
56	2585.31945511748\\
57	5820.11088797017\\
58	40512.0024360843\\
59	234637.484959815\\
60	16841.0078036831\\
61	4789.1001667563\\
62	3235.72345396025\\
63	3621.671124788\\
64	7239.74525142419\\
65	76426.8531261069\\
66	213431.257296552\\
67	10058.9398378074\\
68	4515.78881683889\\
69	3533.05011180871\\
70	4122.91917015136\\
71	8118.33604195623\\
72	127267.481225726\\
73	167263.308179538\\
74	9616.46683114758\\
75	4967.89555999995\\
76	4307.28710110246\\
77	5197.775364982\\
78	10335.3220378342\\
79	182284.419488649\\
80	108159.966044409\\
81	8621.12783580651\\
82	4345.11299173339\\
83	3870.44077563474\\
84	5222.66415707551\\
85	14004.491689984\\
86	227352.392122548\\
87	63160.7997367143\\
88	8154.67252546312\\
89	4139.70776610491\\
90	3438.24643259268\\
91	4860.24406921386\\
92	22698.2058656717\\
93	244742.461098225\\
94	37575.5521866883\\
95	10451.6123049568\\
96	7133.19454891044\\
97	6681.78274838573\\
98	9278.44061855741\\
99	51591.9260422492\\
100	239784.107740901\\
101	20964.6344712149\\
102	12396.3888609053\\
103	12941.7533465924\\
104	18625.6476907128\\
105	50418.5631415139\\
106	491958.821642342\\
107	722240.762895506\\
108	42885.4646331668\\
109	18959.5021376887\\
110	13278.4529976445\\
111	11908.5376459887\\
112	14192.120205667\\
113	161433.670830332\\
114	207629.120295204\\
115	32180.8378020127\\
116	26657.8956947516\\
117	27986.7983238225\\
118	33437.4799171163\\
119	47362.1644074165\\
120	242716.673323717\\
121	113430.417500742\\
122	67857.6266078146\\
123	101318.162526693\\
124	175545.020154429\\
125	381883.532548747\\
126	1567961.42840303\\
127	23221213.0950726\\
128	4865888.48976607\\
129	624544.880595286\\
130	238973.040538831\\
131	125587.29946538\\
132	78308.7898539806\\
133	73037.3974013629\\
134	271150.038317276\\
135	65995.0943438357\\
136	35703.7422212397\\
137	27633.7808084823\\
138	24599.5035856174\\
139	27019.9191448833\\
140	93374.2459183023\\
141	280533.979110995\\
142	15226.2813621478\\
143	11640.2633507886\\
144	12759.534727283\\
145	18334.5773873386\\
146	54271.533584491\\
147	558933.217235893\\
148	675903.495221173\\
149	46977.2541009707\\
150	20387.1812894993\\
151	13564.6968715487\\
152	11740.3807702939\\
153	14834.6826789188\\
154	166262.022447405\\
155	129036.06070688\\
156	12895.1728444693\\
157	8328.85836700699\\
158	7889.91099433692\\
159	9630.51999193638\\
160	18107.3298460024\\
161	221470.143124542\\
162	66275.4401033131\\
163	6299.37000443946\\
164	3539.31390591582\\
165	3685.73411017274\\
166	5833.57351459431\\
167	21241.5801918213\\
168	249982.579358422\\
169	37267.5206214353\\
170	6662.50837550657\\
171	3815.86398452993\\
172	3656.65401558687\\
173	6170.72388839779\\
174	39691.5323123567\\
175	244179.195121135\\
176	19182.3434080757\\
177	6519.73352353916\\
178	4790.89753149547\\
179	5131.21312940206\\
180	8976.73409770214\\
181	79722.5873118596\\
182	213262.97070559\\
183	9170.79786938848\\
184	4350.72299991406\\
185	3741.46915392263\\
186	4617.38028479086\\
187	9303.87082312299\\
188	128787.013352456\\
189	162816.2429628\\
190	7590.51648705448\\
191	3494.63609691419\\
192	2887.46816905788\\
193	3794.70083757787\\
194	8580.58562399651\\
195	174660.466107146\\
196	107476.021551741\\
197	8153.64261900702\\
198	3227.70448374539\\
199	1896.24067264913\\
200	1317.72083921525\\
201	1026.6086873185\\
202	855.916133552541\\
203	751.700182898775\\
204	664.170504494067\\
205	613.657513818042\\
206	559.85004035324\\
207	535.728893655505\\
208	504.963024589523\\
209	468.145632839188\\
210	451.131468064014\\
211	434.298406952323\\
212	424.340171971602\\
213	416.483041566607\\
214	393.8252565115\\
215	390.151872640734\\
216	376.590918924518\\
217	361.839384165296\\
218	357.693505138926\\
219	352.112203810827\\
220	347.181768094572\\
221	342.338390012586\\
222	329.359845621858\\
223	321.603250396006\\
224	320.483416025978\\
225	315.446100007911\\
226	310.357583630252\\
227	317.482771819105\\
228	298.762186832923\\
229	303.209356418795\\
230	302.292389452814\\
231	298.112076351372\\
232	293.947221471244\\
233	283.999343889006\\
234	298.431349883647\\
235	272.52168599285\\
236	280.402929072728\\
237	290.74558337388\\
238	276.068587248429\\
239	274.299548595029\\
240	274.127617228544\\
241	277.252237791542\\
242	278.014441364256\\
243	270.725951432116\\
244	265.84818375836\\
245	273.575558865976\\
246	265.843545493986\\
247	260.809207537589\\
248	262.169072392346\\
249	259.541668522653\\
250	264.382464329194\\
251	254.785628117324\\
252	262.835190998783\\
253	261.143103280945\\
254	261.678918290097\\
255	256.043411432057\\
256	260.429945382733\\
};
\addlegendentry{Scatterer power}

\addplot [color=orange]
  table[row sep=crcr]{%
0	158828.655125224\\
300	158828.655125224\\
};
\addlegendentry{Average power of all scatterers}

\addplot [color=black, only marks, mark size=4.0pt, mark=o, mark options={solid, black}]
  table[row sep=crcr]{%
126	1567961.42840303\\
127	23221213.0950726\\
128	4865888.48976607\\
};
\addlegendentry{Candidate scatterers}

\addplot [color=red, only marks, mark size=7.5pt, mark=o, mark options={solid, red}]
  table[row sep=crcr]
                \caption{Scatterer power, \\ sf = 5.\label{subfig:sf5_corrRA_power}}
            \end{subfigure}
             &
            \begin{subfigure}{0.33\linewidth}
                \centering
                \resizebox{\linewidth}{!}{% This file was created by matlab2tikz.
%
%The latest updates can be retrieved from
%  http://www.mathworks.com/matlabcentral/fileexchange/22022-matlab2tikz-matlab2tikz
%where you can also make suggestions and rate matlab2tikz.
%
\definecolor{mycolor1}{rgb}{0.00000,0.44700,0.74100}%
%
\begin{tikzpicture}

\begin{axis}[%
width=6.028in,
height=4.754in,
at={(1.011in,0.642in)},
scale only axis,
xmin=0,
xmax=256,
xlabel style={font=\fontsize{25}{20}\selectfont\color{black}, yshift = -10},
xlabel={Range Bin},
ymin=0,
ymax=25000000,
ylabel style={font=\fontsize{25}{20}\selectfont\color{black}, yshift=10pt},
ylabel={Power},
axis background/.style={fill=white},
tick label style={font=\fontsize{20}{11}\selectfont\color{black}},
xtick distance = 50,
ytick distance= 2e6,
yticklabel={\ifdim\tick pt=0pt\else\pgfmathprintnumber{\tick}\fi}, 
scaled y ticks=base 10:-6,
legend style={legend cell align=left, align=left, draw=white!15!black, font=\fontsize{12}{11}\selectfont\color{black}}
]
\addplot [color=mycolor1, mark=asterisk, mark options={solid, mycolor1}]
  table[row sep=crcr]{%
1	260.757090596185\\
2	261.349448266049\\
3	258.844308641979\\
4	260.99983616491\\
5	261.116115558909\\
6	269.196452711933\\
7	260.682193999795\\
8	265.471868197095\\
9	266.477079167473\\
10	269.232963922838\\
11	270.138464929005\\
12	273.043327361471\\
13	273.62564006987\\
14	271.454418528639\\
15	274.3904109416\\
16	265.005913372826\\
17	275.639398903274\\
18	275.37204559559\\
19	283.867254514651\\
20	277.277950830027\\
21	287.384264710496\\
22	305.495212060368\\
23	289.44774297217\\
24	296.354344666488\\
25	296.638641274841\\
26	298.557913899803\\
27	312.424493674491\\
28	314.80503284847\\
29	319.714817046923\\
30	311.437671023239\\
31	313.879886187225\\
32	324.263258015104\\
33	333.784946542075\\
34	331.685893596448\\
35	349.164151373045\\
36	344.6435903982\\
37	354.942594612574\\
38	371.193024662765\\
39	377.371153339826\\
40	392.017816661179\\
41	385.536437438418\\
42	409.673280656308\\
43	422.209555724421\\
44	444.675193605828\\
45	454.432321135569\\
46	484.643775551396\\
47	504.451341148726\\
48	535.363082639264\\
49	572.071889434024\\
50	629.279327276237\\
51	701.308690678024\\
52	801.060800191796\\
53	947.102083174944\\
54	1170.73479266727\\
55	1607.02942933746\\
56	2595.55981309361\\
57	5812.86960576437\\
58	40487.3221677515\\
59	234662.60643436\\
60	16833.6285145892\\
61	4758.92392697897\\
62	3271.14525107973\\
63	3630.14568751605\\
64	7223.30318137183\\
65	76580.8108950504\\
66	213496.653023221\\
67	10148.190287012\\
68	4495.3497625272\\
69	3546.12800755943\\
70	4098.38020015848\\
71	8125.86219433519\\
72	126892.724550999\\
73	167250.020353373\\
74	9548.47811677503\\
75	4980.71467107675\\
76	4318.33386493922\\
77	5190.87048619526\\
78	10346.5620214141\\
79	182348.437135178\\
80	108122.579606752\\
81	8609.16567638802\\
82	4360.65851422563\\
83	3888.12324901019\\
84	5216.15943569194\\
85	14003.8628301505\\
86	227292.38173834\\
87	63274.6606646977\\
88	8201.67532897642\\
89	4147.86112014355\\
90	3483.13936838907\\
91	4882.03074468776\\
92	22649.03280625\\
93	244616.968724519\\
94	37573.9015011674\\
95	10541.5161574813\\
96	7094.75508933476\\
97	6689.33290583644\\
98	9235.0911245946\\
99	51536.1795732728\\
100	239639.456491946\\
101	21056.1015168124\\
102	12359.261566368\\
103	12948.1582508674\\
104	18627.4332582689\\
105	50401.4629805547\\
106	492414.995962025\\
107	722555.790212155\\
108	42965.7265527986\\
109	18963.2819331697\\
110	13311.5311164789\\
111	11921.0556504423\\
112	14186.432709524\\
113	161339.956565044\\
114	207474.526112217\\
115	32250.1735407387\\
116	26601.9336512659\\
117	28100.2828981279\\
118	33496.491222651\\
119	47460.4820517101\\
120	242881.500866766\\
121	113595.258541256\\
122	67922.9767501688\\
123	101375.484833191\\
124	175724.16231007\\
125	381962.197550843\\
126	1567886.22087659\\
127	23225975.3227313\\
128	4867247.19796107\\
129	624310.85052159\\
130	238975.098769047\\
131	125608.799294704\\
132	78090.4619349091\\
133	73018.8693531322\\
134	271074.146369045\\
135	65882.5404193608\\
136	35525.7194041419\\
137	27657.6671044679\\
138	24587.4723789782\\
139	26971.1195555857\\
140	93342.5206466693\\
141	280414.579199937\\
142	15249.1463940073\\
143	11661.2594790902\\
144	12743.3589792486\\
145	18353.576908794\\
146	54236.1111496537\\
147	559055.020673585\\
148	675705.880270755\\
149	46874.188812739\\
150	20348.0267881837\\
151	13616.2338193232\\
152	11723.911314611\\
153	14819.5513495774\\
154	166181.518183832\\
155	129254.288287589\\
156	12857.1380900099\\
157	8430.18179664529\\
158	7945.97175861454\\
159	9639.56336298348\\
160	18204.0485913735\\
161	221285.667916626\\
162	66139.6893443756\\
163	6313.52246622126\\
164	3568.26402514372\\
165	3694.14761048191\\
166	5811.51887875028\\
167	21221.8089970754\\
168	249993.262788285\\
169	37320.4904955616\\
170	6679.95817225113\\
171	3826.64403605272\\
172	3642.44554922554\\
173	6174.66338766456\\
174	39707.6934224181\\
175	244029.642915695\\
176	19122.757190502\\
177	6542.09172884734\\
178	4781.16967629281\\
179	5147.87783102795\\
180	8975.55892262907\\
181	79715.8622930272\\
182	213292.406214932\\
183	9185.52391495871\\
184	4380.04015306461\\
185	3748.55640973686\\
186	4597.55027155746\\
187	9191.68584402259\\
188	128933.104580069\\
189	162758.12802332\\
190	7561.38022797859\\
191	3471.68110990261\\
192	2897.2416018114\\
193	3737.36637371086\\
194	8599.90672828641\\
195	174707.94076613\\
196	107383.013167702\\
197	8104.43294227265\\
198	3217.87968532673\\
199	1901.12701656705\\
200	1331.03929132751\\
201	1027.99400712673\\
202	849.0161722243\\
203	750.237099974916\\
204	661.652776163157\\
205	605.086109408137\\
206	563.913893540252\\
207	531.67619600442\\
208	505.258278111065\\
209	479.494330188539\\
210	451.569531216314\\
211	443.698752243901\\
212	422.153400559185\\
213	406.375520103256\\
214	399.500362473823\\
215	399.441866692462\\
216	371.646522447958\\
217	372.725875540429\\
218	363.155705021535\\
219	348.298016776292\\
220	347.376026579026\\
221	336.307096456843\\
222	334.952486711036\\
223	328.401028502431\\
224	326.106363118331\\
225	319.905973550356\\
226	316.802587711091\\
227	312.82705969179\\
228	302.834591083687\\
229	299.488978776006\\
230	299.71382840282\\
231	296.923606461388\\
232	291.777170220202\\
233	299.586958187809\\
234	279.73348394927\\
235	287.318775687667\\
236	274.951389735924\\
237	282.884456518634\\
238	284.503755622997\\
239	280.142152545547\\
240	280.73720127584\\
241	271.453368629526\\
242	275.304240428404\\
243	271.207453570855\\
244	275.410068139259\\
245	273.594484812166\\
246	265.60622881618\\
247	262.967106050081\\
248	255.950228236889\\
249	264.682395020898\\
250	257.875867486201\\
251	269.323143719703\\
252	264.00100589459\\
253	264.238928945338\\
254	270.479568204838\\
255	262.719731607442\\
256	261.114946353416\\
};
\addlegendentry{Scatterer power}

\addplot [color=orange]
  table[row sep=crcr]{%
0	158852.856776151\\
300	158852.856776151\\
};
\addlegendentry{Average power of all scatterers}

\addplot [color=black, only marks, mark size=4.0pt, mark=o, mark options={solid, black}]
  table[row sep=crcr]{%
127	23225975.3227313\\
128	4867247.19796107\\
};
\addlegendentry{Candidate scatterers}

\addplot [color=red, only marks, mark size=7.5pt, mark=o, mark options={solid, red}]
  table[row sep=crcr]
                \caption{Scatterer power, \\ sf = 10.\label{subfig:sf10_corrRA_power}}
            \end{subfigure}
            \\
            \begin{subfigure}{0.33\linewidth}
                \centering
                \resizebox{\linewidth}{!}{% This file was created by matlab2tikz.
%
%The latest updates can be retrieved from
%  http://www.mathworks.com/matlabcentral/fileexchange/22022-matlab2tikz-matlab2tikz
%where you can also make suggestions and rate matlab2tikz.
%
\definecolor{mycolor1}{rgb}{0.00000,0.44700,0.74100}%
%
\begin{tikzpicture}

\begin{axis}[%
width=6.028in,
height=4.754in,
at={(1.011in,0.642in)},
scale only axis,
xmin=0,
xmax=256,
xlabel style={font=\fontsize{25}{20}\selectfont\color{black}, yshift = -10},
xlabel={Range Bin},
ymin=0,
ymax=0.2e4,
ylabel style={font=\fontsize{25}{20}\selectfont\color{black}, yshift=10pt},
ylabel={Amplitude Variance},
axis background/.style={fill=white},
tick label style={font=\fontsize{20}{11}\selectfont\color{black}},
xtick distance = 50,
ytick distance=2e2,
yticklabel={\ifdim\tick pt=0pt\else\pgfmathprintnumber{\tick}\fi}, 
scaled y ticks=base 10:-3,
legend style={legend cell align=left, align=left, draw=white!15!black, font=\fontsize{12}{11}\selectfont\color{black}}
]
\addplot [color=mycolor1, mark=asterisk, mark options={solid, mycolor1}]
  table[row sep=crcr]{%
1	1.97641752728258\\
2	1.92465227300996\\
3	2.01677704509271\\
4	2.04191645746993\\
5	1.94225082190169\\
6	2.01340616184179\\
7	2.0498819968885\\
8	1.89390380794565\\
9	1.94558067311954\\
10	1.9738912641259\\
11	1.97815209557455\\
12	1.97765550535603\\
13	1.85477630216396\\
14	2.02682069107973\\
15	1.96681105494079\\
16	1.99148497258703\\
17	2.01334294463219\\
18	2.16414266828998\\
19	2.10926076584767\\
20	2.04270512857579\\
21	2.1062884974288\\
22	2.00197380417894\\
23	2.15335722770074\\
24	2.09194089867042\\
25	2.17067854957384\\
26	2.17439148839012\\
27	2.32936321824661\\
28	2.20351118337395\\
29	2.27184887585204\\
30	2.32045352698206\\
31	2.2734449009604\\
32	2.37413249812751\\
33	2.47114749271323\\
34	2.44420676141444\\
35	2.49410318069689\\
36	2.48407937269401\\
37	2.52154760459016\\
38	2.67217490170516\\
39	2.56567973935654\\
40	2.72566130734291\\
41	2.92457187363807\\
42	2.86181306860463\\
43	3.01231344262538\\
44	3.00120220162575\\
45	3.08730511064516\\
46	3.29371975849508\\
47	3.53902246528163\\
48	3.4805154468051\\
49	3.91260312672885\\
50	4.21116735278162\\
51	4.51777637355335\\
52	4.85098649787763\\
53	5.38896909788765\\
54	6.62160640496387\\
55	8.88114828102086\\
56	15.0328074271429\\
57	40.3894912027225\\
58	468.238901021587\\
59	214.361354395342\\
60	120.7083644614\\
61	23.8850249177575\\
62	15.5814068705923\\
63	18.8778587760787\\
64	47.2427111672822\\
65	922.992756956494\\
66	624.612817970742\\
67	43.8204408048218\\
68	15.6424933541002\\
69	11.2040410297304\\
70	14.5803642443105\\
71	40.4097040482963\\
72	1165.66569043027\\
73	1081.20855613316\\
74	61.4698614101452\\
75	30.8603785764323\\
76	26.2124117526896\\
77	32.0760792005733\\
78	61.5334623213138\\
79	807.293838495445\\
80	1052.82750542741\\
81	50.7255245158404\\
82	24.9470089071189\\
83	24.8438502426379\\
84	35.2319381388203\\
85	89.1374965190646\\
86	411.555867253461\\
87	704.992645857298\\
88	70.467631166485\\
89	36.6140385268033\\
90	27.7970889086978\\
91	31.6965506631869\\
92	212.534247161509\\
93	168.421628315681\\
94	354.373505821634\\
95	74.4828548027723\\
96	46.6212966380576\\
97	41.3574951447182\\
98	53.935873819254\\
99	554.357157281377\\
100	271.601776861522\\
101	62.0957010124361\\
102	23.9958367734203\\
103	29.0126097983513\\
104	60.7418515456905\\
105	306.356273013474\\
106	5033.90960386235\\
107	10054.1871417464\\
108	486.097392226081\\
109	169.550270745128\\
110	77.4275209658147\\
111	42.4190457455589\\
112	35.1938461568103\\
113	1432.00823613622\\
114	1674.04632755325\\
115	171.961263202437\\
116	129.53783183329\\
117	130.91892354561\\
118	145.453542530959\\
119	180.876297483178\\
120	219.451202792274\\
121	631.883526085352\\
122	257.810535898264\\
123	421.721720021388\\
124	762.838761958395\\
125	1763.12531843661\\
126	10111.405455666\\
127	22345.0066611279\\
128	52011.7647347216\\
129	4430.86407432905\\
130	1519.55047812194\\
131	759.102951483715\\
132	440.351614183598\\
133	324.7986815083\\
134	84.8314598957466\\
135	400.365377446733\\
136	189.053170100435\\
137	152.708080441579\\
138	139.409637230254\\
139	153.694465178801\\
140	929.742219854818\\
141	973.672324678347\\
142	59.9314118609366\\
143	48.3748757341999\\
144	61.7878941572395\\
145	103.701695106787\\
146	393.097164519305\\
147	5775.65373034833\\
148	9601.68670300098\\
149	649.039783919637\\
150	252.970857246982\\
151	141.872631538722\\
152	98.7881243278488\\
153	87.0940768110391\\
154	773.754422674886\\
155	1034.64787343086\\
156	75.517933945033\\
157	47.2879367915503\\
158	44.5818743596568\\
159	51.9677344641759\\
160	86.3828987200951\\
161	370.579319456516\\
162	714.649096756247\\
163	43.5487738908741\\
164	27.2400656313167\\
165	30.5237193905561\\
166	41.5159230151348\\
167	170.036917917945\\
168	208.782451073201\\
169	393.300358822864\\
170	53.4433592895013\\
171	26.9425974136289\\
172	24.0670789563057\\
173	44.6198062573816\\
174	472.714440857346\\
175	222.161289611203\\
176	111.997687125469\\
177	27.4077619397583\\
178	19.9316739887525\\
179	21.8126794647209\\
180	49.175767354918\\
181	902.943734236173\\
182	629.276012550376\\
183	44.6772345072197\\
184	20.0064610727325\\
185	17.7115312060378\\
186	24.3887066522907\\
187	57.8808797396921\\
188	1168.48627722548\\
189	1101.50013606091\\
190	44.607986951784\\
191	17.778839306522\\
192	14.419621855298\\
193	18.6140776129948\\
194	39.8258180988013\\
195	781.669690799136\\
196	1053.61171225668\\
197	54.3134778648951\\
198	21.2273861412736\\
199	12.338339775384\\
200	9.34376282908235\\
201	7.80346075683523\\
202	6.81334105947163\\
203	6.24064800483896\\
204	5.8659231375596\\
205	5.58557850649065\\
206	5.12657196364757\\
207	5.08166797801978\\
208	4.70636215240297\\
209	4.54853190730584\\
210	4.41540385344589\\
211	4.18864023153467\\
212	3.91853555209817\\
213	3.71019847552223\\
214	3.60408374384803\\
215	3.39265405090082\\
216	3.45860219899869\\
217	3.38889267641868\\
218	3.01989195028836\\
219	3.2101389452585\\
220	3.05903839224465\\
221	2.95670547563513\\
222	2.93399370699177\\
223	2.64152086924243\\
224	2.64217908307083\\
225	2.66645906767761\\
226	2.53221919707173\\
227	2.40833556938761\\
228	2.46421365109454\\
229	2.4418761597358\\
230	2.29550474712252\\
231	2.40742174479694\\
232	2.28352905783689\\
233	2.22887744941043\\
234	2.31856115137575\\
235	2.2185953512023\\
236	2.10017789302739\\
237	2.24851579102314\\
238	2.19883255388\\
239	2.09996481189242\\
240	2.08556782553957\\
241	2.08808529685929\\
242	2.09941001356448\\
243	2.06914867290014\\
244	2.11055724659788\\
245	1.87923505470268\\
246	2.11176824076071\\
247	1.98957947910103\\
248	1.95295126765837\\
249	1.95384463384001\\
250	2.05365725074466\\
251	1.88307557763016\\
252	1.98151346774925\\
253	1.95868557883717\\
254	2.04765361386176\\
255	1.89262634767499\\
256	2.00533858902783\\
};
\addlegendentry{Scatterer variance}

\addplot [color=black, only marks, mark size=4.0pt, mark=o, mark options={solid, black}]
  table[row sep=crcr]{%
59	214.361354395342\\
66	624.612817970742\\
73	1081.20855613316\\
79	807.293838495445\\
86	411.555867253461\\
93	168.421628315681\\
100	271.601776861522\\
106	5033.90960386235\\
107	10054.1871417464\\
113	1432.00823613622\\
114	1674.04632755325\\
120	219.451202792274\\
124	762.838761958395\\
125	1763.12531843661\\
126	10111.405455666\\
127	22345.0066611279\\
128	52011.7647347216\\
129	4430.86407432905\\
130	1519.55047812194\\
134	84.8314598957466\\
141	973.672324678347\\
147	5775.65373034833\\
148	9601.68670300098\\
154	773.754422674886\\
161	370.579319456516\\
168	208.782451073201\\
175	222.161289611203\\
182	629.276012550376\\
189	1101.50013606091\\
195	781.669690799136\\
};
\addlegendentry{Candidate scatterers}

\addplot [color=red]
  table[row sep=crcr]{%
0	84.8314598957466\\
300	84.8314598957466\\
};
\addlegendentry{Minimum variance}

\addplot [color=red, only marks, mark size=4.0pt, mark=*, mark options={solid, fill=red, red}]
  table[row sep=crcr]
                \caption{Scatterer amplitude variance, \\ sf = 1.\label{subfig:sf1_corrRA_var}}
            \end{subfigure}
             &
            \begin{subfigure}{0.33\linewidth}
                \centering
                \resizebox{\linewidth}{!}{% This file was created by matlab2tikz.
%
%The latest updates can be retrieved from
%  http://www.mathworks.com/matlabcentral/fileexchange/22022-matlab2tikz-matlab2tikz
%where you can also make suggestions and rate matlab2tikz.
%
\definecolor{mycolor1}{rgb}{0.00000,0.44700,0.74100}%
%
\begin{tikzpicture}

\begin{axis}[%
width=6.028in,
height=4.754in,
at={(1.011in,0.642in)},
scale only axis,
xmin=0,
xmax=256,
xlabel style={font=\fontsize{25}{20}\selectfont\color{black}, yshift = -10},
xlabel={Range Bin},
ymin=0,
ymax=1.5e4,
ylabel style={font=\fontsize{25}{20}\selectfont\color{black}, yshift=10pt},
ylabel={Amplitude Variance},
axis background/.style={fill=white},
tick label style={font=\fontsize{20}{11}\selectfont\color{black}},
xtick distance = 50,
ytick distance=2e3,
yticklabel={\ifdim\tick pt=0pt\else\pgfmathprintnumber{\tick}\fi}, 
scaled y ticks=base 10:-3,
legend style={legend cell align=left, align=left, draw=white!15!black, font=\fontsize{12}{11}\selectfont\color{black}}
]
\addplot [color=mycolor1, mark=asterisk, mark options={solid, mycolor1}]
  table[row sep=crcr]{%
1	2.08321120363209\\
2	1.921904220233\\
3	1.92674362838173\\
4	1.93310144062469\\
5	1.94669622820027\\
6	1.94115478170521\\
7	1.96152333731778\\
8	1.94910798661964\\
9	1.93293267162336\\
10	1.9370473039196\\
11	2.04531502880971\\
12	2.0606074290485\\
13	2.05879951018943\\
14	1.95354714810257\\
15	2.04713092781947\\
16	2.07259003245576\\
17	2.18431716078799\\
18	2.16139285444331\\
19	1.99396514031343\\
20	2.02973960572453\\
21	2.14614518580977\\
22	2.06806558423006\\
23	2.14220786459038\\
24	2.10473044070364\\
25	2.19948889686861\\
26	2.05653994475873\\
27	2.11939291246629\\
28	2.14611488050265\\
29	2.24321711627365\\
30	2.22880779286643\\
31	2.34071037680191\\
32	2.35206267420149\\
33	2.63461240823655\\
34	2.53076965136433\\
35	2.5465674475122\\
36	2.44144910314457\\
37	2.67448674994679\\
38	2.68415813541077\\
39	2.71117842178543\\
40	2.79559045507104\\
41	2.72976214134926\\
42	2.9639386978755\\
43	2.9310724812868\\
44	2.98099508162445\\
45	3.17148203971716\\
46	3.34720133263892\\
47	3.44439859826344\\
48	3.61635200102043\\
49	3.7759460725445\\
50	4.24222884610444\\
51	4.41171653170368\\
52	4.86593602438878\\
53	5.59652722717545\\
54	6.83990514628735\\
55	8.7606167327905\\
56	15.1603291346533\\
57	39.857714591742\\
58	468.414312043591\\
59	214.15175500521\\
60	120.440382079514\\
61	23.8930464799152\\
62	15.3780020307775\\
63	18.8290993610706\\
64	46.6595174702634\\
65	922.568143445766\\
66	625.032365962269\\
67	44.05704330227\\
68	15.8644269339805\\
69	11.1725332061883\\
70	14.774710426899\\
71	40.5568358714716\\
72	1166.51692115366\\
73	1085.9702957141\\
74	62.0786532552127\\
75	30.9275788643156\\
76	26.7034698019049\\
77	32.6570858563442\\
78	60.6024460813725\\
79	806.370147932657\\
80	1055.5193335565\\
81	51.2091939438542\\
82	24.8237285196772\\
83	25.1471914630608\\
84	35.0129877460332\\
85	89.9765017262348\\
86	411.117168196505\\
87	704.521170856365\\
88	69.8777390161609\\
89	37.0737945028782\\
90	28.5321859248087\\
91	31.9567938559395\\
92	214.738533972753\\
93	167.318855751845\\
94	351.194143609148\\
95	74.2033347684922\\
96	47.1964462660784\\
97	41.1179678766444\\
98	53.9346322088505\\
99	555.235116840062\\
100	271.105771199784\\
101	62.8204179279888\\
102	23.7060761281031\\
103	29.1567725004453\\
104	60.8001855915124\\
105	307.0127762513\\
106	5034.48602161453\\
107	10047.2654511405\\
108	485.507187625322\\
109	170.32593734737\\
110	77.8493602408992\\
111	42.1948403667143\\
112	34.543033491961\\
113	1434.85050890689\\
114	1677.30095682208\\
115	171.046746001318\\
116	129.608089351639\\
117	130.119737254404\\
118	146.254578180326\\
119	181.054746006636\\
120	219.673286895483\\
121	632.649378056554\\
122	258.319822006803\\
123	419.48237887866\\
124	762.206044343748\\
125	1765.17471336436\\
126	10105.9322692203\\
127	22343.0539690425\\
128	52002.0409624597\\
129	4440.50813594317\\
130	1517.72141953784\\
131	760.764860530975\\
132	439.952566550395\\
133	325.073950651625\\
134	84.8916248026265\\
135	401.030102879254\\
136	188.4339020801\\
137	151.291225738727\\
138	138.967998223346\\
139	154.532662152589\\
140	931.236137173979\\
141	974.693206409318\\
142	60.0389147242318\\
143	48.0782970824718\\
144	62.4614636684976\\
145	103.841355927459\\
146	394.625533895844\\
147	5778.93574663015\\
148	9605.30431010017\\
149	647.922368646472\\
150	251.903487235542\\
151	141.207447634804\\
152	98.4228440450755\\
153	87.2835365460573\\
154	774.741761291363\\
155	1035.16198124819\\
156	75.0949669730951\\
157	47.1659715713286\\
158	43.9615843216195\\
159	52.9246787795588\\
160	85.1766827907494\\
161	369.370655894115\\
162	712.832183815632\\
163	43.6009739684981\\
164	27.8774282565255\\
165	30.7438787833207\\
166	41.6640871276399\\
167	169.245363794576\\
168	208.886236902711\\
169	391.750397461381\\
170	51.884188748898\\
171	26.9271548887131\\
172	23.77651623404\\
173	44.8021266942446\\
174	475.191624895443\\
175	220.283677369054\\
176	111.704438402747\\
177	27.1069952933976\\
178	20.1970746318427\\
179	21.6880541826496\\
180	49.2016026434613\\
181	901.98008777202\\
182	626.601268010464\\
183	44.1751687210038\\
184	20.2924146303747\\
185	17.4698168063274\\
186	24.5808853810647\\
187	58.9598628629286\\
188	1167.94979218178\\
189	1101.1811446464\\
190	44.5268793670615\\
191	17.9525560171142\\
192	14.4164991884608\\
193	18.7429427213321\\
194	39.8908113734624\\
195	782.795027117167\\
196	1053.17281965714\\
197	54.4759865138814\\
198	21.023955173951\\
199	12.463373161777\\
200	9.46280492938718\\
201	7.56221499198981\\
202	6.73711559281714\\
203	6.30383970438473\\
204	5.91551964608533\\
205	5.63258441170031\\
206	5.21877238604558\\
207	5.03659259613409\\
208	4.80044923921998\\
209	4.36802497130052\\
210	4.15476409241554\\
211	4.17955811497718\\
212	4.06456431004751\\
213	3.88415712535343\\
214	3.47659324750515\\
215	3.50194995122602\\
216	3.41268553750496\\
217	3.29023866229171\\
218	3.19060276195639\\
219	2.93870295146985\\
220	2.95272004449474\\
221	2.90255975777407\\
222	2.74807661107966\\
223	2.52835561147921\\
224	2.59918542173538\\
225	2.57146911719801\\
226	2.53875574874579\\
227	2.53371037121738\\
228	2.46559630817839\\
229	2.40884871847049\\
230	2.33789960933137\\
231	2.35200337423874\\
232	2.38004200076115\\
233	2.28926007940636\\
234	2.4326449579174\\
235	2.17387728846157\\
236	2.23416275002014\\
237	2.27313892893194\\
238	2.23653480673665\\
239	2.11207522728423\\
240	2.15124355169376\\
241	2.16217914832361\\
242	2.19569842630966\\
243	2.01961128525521\\
244	1.97298210453101\\
245	2.04204682136496\\
246	1.96305099932611\\
247	1.97122502172868\\
248	2.03333164065216\\
249	1.93804657326272\\
250	2.00670065443835\\
251	2.0374039342196\\
252	1.98702064338366\\
253	1.91936262117433\\
254	1.98105547011769\\
255	1.83687871903398\\
256	1.91547878858092\\
};
\addlegendentry{Scatterer variance}

\addplot [color=black, only marks, mark size=4.0pt, mark=o, mark options={solid, black}]
  table[row sep=crcr]{%
126	10105.9322692203\\
127	22343.0539690425\\
128	52002.0409624597\\
};
\addlegendentry{Candidate scatterers}

\addplot [color=red]
  table[row sep=crcr]{%
0	10105.9322692203\\
300	10105.9322692203\\
};
\addlegendentry{Minimum variance}

\addplot [color=red, only marks, mark size=4.0pt, mark=*, mark options={solid, fill=red, red}]
  table[row sep=crcr]
                \caption{Scatterer amplitude variance, \\ sf = 5.\label{subfig:sf5_corrRA_var}}
            \end{subfigure}
             &
            \begin{subfigure}{0.33\linewidth}
                \centering
                \resizebox{\linewidth}{!}{% This file was created by matlab2tikz.
%
%The latest updates can be retrieved from
%  http://www.mathworks.com/matlabcentral/fileexchange/22022-matlab2tikz-matlab2tikz
%where you can also make suggestions and rate matlab2tikz.
%
\definecolor{mycolor1}{rgb}{0.00000,0.44700,0.74100}%
%
\begin{tikzpicture}

\begin{axis}[%
width=6.028in,
height=4.754in,
at={(1.011in,0.642in)},
scale only axis,
xmin=0,
xmax=256,
xlabel style={font=\fontsize{25}{20}\selectfont\color{black}, yshift = -10},
xlabel={Range Bin},
ymin=0,
ymax=60000,
ylabel style={font=\fontsize{25}{20}\selectfont\color{black}, yshift=10pt},
ylabel={Amplitude Variance},
axis background/.style={fill=white},
tick label style={font=\fontsize{20}{11}\selectfont\color{black}},
xtick distance = 50,
ytick distance= 1e4,
yticklabel={\ifdim\tick pt=0pt\else\pgfmathprintnumber{\tick}\fi}, 
scaled y ticks=base 10:-3,
legend style={legend cell align=left, align=left, draw=white!15!black, font=\fontsize{12}{11}\selectfont\color{black}}
]
\addplot [color=mycolor1, mark=asterisk, mark options={solid, mycolor1}]
  table[row sep=crcr]{%
1	1.9133938710127\\
2	2.01072279916104\\
3	1.90058203394593\\
4	1.88378981609597\\
5	1.8478207975461\\
6	1.98347699479647\\
7	1.89872916806568\\
8	2.03189519032328\\
9	1.90112975728777\\
10	1.99553746416861\\
11	2.01374019624089\\
12	2.09836204539017\\
13	2.06227827424652\\
14	2.12185823456891\\
15	2.03859918049033\\
16	1.845245661314\\
17	2.07270440638032\\
18	2.01524253699117\\
19	2.0946155593745\\
20	2.03171287688091\\
21	2.09126071499639\\
22	2.12998451928258\\
23	2.03378926311033\\
24	2.27347384225681\\
25	2.20193768710672\\
26	2.02862284736567\\
27	2.21275528819339\\
28	2.28233776318313\\
29	2.29938376177454\\
30	2.36477223889563\\
31	2.35258179963599\\
32	2.30429158369687\\
33	2.35206064405741\\
34	2.35396837623126\\
35	2.41452565723408\\
36	2.41191386124048\\
37	2.42026157639091\\
38	2.62413472857586\\
39	2.56850577159116\\
40	2.7946491472182\\
41	2.6732532485586\\
42	2.76041694086011\\
43	3.00609946037035\\
44	3.18145501265013\\
45	3.21020407897488\\
46	3.3340302767689\\
47	3.54600972364066\\
48	3.50260219857389\\
49	3.8766086418855\\
50	4.19441994998233\\
51	4.4412405727336\\
52	5.06053174897812\\
53	5.40929588964922\\
54	6.5415279965314\\
55	9.14507844442924\\
56	15.0519461424467\\
57	39.9147620601247\\
58	467.555867660639\\
59	213.055564504016\\
60	120.565987442262\\
61	23.5718626160254\\
62	15.6267439669318\\
63	18.7429164340291\\
64	46.0204225796209\\
65	924.630081299313\\
66	624.3791768422\\
67	44.835406841558\\
68	15.3314608898529\\
69	11.0121316087715\\
70	14.6402359597468\\
71	40.3845778003644\\
72	1164.09800572068\\
73	1083.04946223052\\
74	61.9421232638481\\
75	30.8218200734104\\
76	26.4822330997694\\
77	32.4934070180211\\
78	60.8045501780425\\
79	807.964378162526\\
80	1052.93803156586\\
81	50.7639952163396\\
82	24.9614931399461\\
83	25.3665089785796\\
84	34.6822763542242\\
85	90.0302388764038\\
86	412.616432925964\\
87	709.521197682858\\
88	70.4704627923113\\
89	37.7515979202163\\
90	28.909715258636\\
91	32.7161513779526\\
92	213.995963367863\\
93	167.17466855772\\
94	352.750811963452\\
95	74.5612361698514\\
96	46.6921514192879\\
97	40.7292592015906\\
98	53.7753783289214\\
99	553.334724747479\\
100	272.745825837723\\
101	62.9702337993028\\
102	23.6562353616831\\
103	28.6386488078339\\
104	61.2646037443697\\
105	304.544103144069\\
106	5040.58819343192\\
107	10058.0935558378\\
108	486.270143992054\\
109	170.483535882326\\
110	77.7995851300929\\
111	42.1083765818408\\
112	34.3500883096186\\
113	1434.22091686599\\
114	1676.56860212765\\
115	172.914575371846\\
116	128.776437906702\\
117	129.978754750311\\
118	146.578983651699\\
119	181.035216301624\\
120	219.314394090354\\
121	634.630971049835\\
122	257.430724866297\\
123	421.054403103927\\
124	764.792884049211\\
125	1761.90288790021\\
126	10109.7199969574\\
127	22334.9321747989\\
128	52020.5564987292\\
129	4434.86746154475\\
130	1518.69413082832\\
131	759.196729938088\\
132	438.0368572698\\
133	326.523224926599\\
134	85.0656652066872\\
135	402.127154859699\\
136	187.316390913439\\
137	151.091735157512\\
138	138.515622479074\\
139	153.523359476704\\
140	931.876021703375\\
141	975.805936191809\\
142	59.3884464659696\\
143	48.2931581680193\\
144	62.1061653540574\\
145	102.953642132513\\
146	392.659099155085\\
147	5774.47944807882\\
148	9599.34787354486\\
149	647.706979933297\\
150	252.189930592577\\
151	142.584663834502\\
152	97.9649883304548\\
153	87.1012140646438\\
154	775.731748358594\\
155	1033.87137971855\\
156	74.8302638798941\\
157	47.324611827748\\
158	44.1542444330301\\
159	52.8498245801726\\
160	86.4869278695037\\
161	368.873145231315\\
162	712.003460485302\\
163	43.6867962734471\\
164	28.3921536341544\\
165	30.9206672758996\\
166	41.7469115041917\\
167	168.710784335025\\
168	206.586661871223\\
169	392.726507342833\\
170	52.9343450967726\\
171	27.0736242705292\\
172	23.4418308675402\\
173	44.7178399190357\\
174	473.405425090333\\
175	220.754276926185\\
176	111.989135912725\\
177	27.3791278506749\\
178	20.1353224521916\\
179	21.8887907736741\\
180	49.2998953783466\\
181	901.90304645969\\
182	628.408400634152\\
183	44.1234430768824\\
184	20.669674341607\\
185	17.716028790995\\
186	24.4548947124522\\
187	58.5183771065465\\
188	1168.76912884724\\
189	1101.25976889245\\
190	44.365095390368\\
191	17.8942269004586\\
192	14.4401480055255\\
193	18.2876278489029\\
194	39.8088557316537\\
195	779.678167552672\\
196	1052.28402448725\\
197	54.1173029789816\\
198	20.8858804842876\\
199	12.4580109915738\\
200	9.43720634815922\\
201	7.45820324957887\\
202	6.7409856328254\\
203	6.24786378954438\\
204	5.78374877049891\\
205	5.22341441451239\\
206	5.36648882210141\\
207	5.08768963190121\\
208	4.7541276152746\\
209	4.54650366302474\\
210	4.34928636340311\\
211	4.27544104616072\\
212	3.96149868708858\\
213	3.79984734386163\\
214	3.75335376397874\\
215	3.55382808024156\\
216	3.33906780783109\\
217	3.2182660095367\\
218	3.083163957648\\
219	2.99906362788988\\
220	2.87185963531737\\
221	2.85588003683741\\
222	2.70608576513909\\
223	2.92929818451152\\
224	2.79802652959837\\
225	2.61584594152127\\
226	2.56609915258513\\
227	2.48602160283973\\
228	2.52296739363419\\
229	2.47179727610708\\
230	2.42983196707561\\
231	2.33309469061028\\
232	2.3201545764392\\
233	2.36510361966071\\
234	2.29592941574246\\
235	2.06841894427994\\
236	2.20352276065941\\
237	2.18560663998336\\
238	2.14855993431592\\
239	2.15290004936119\\
240	2.12088813629976\\
241	2.06865221157781\\
242	2.14894753386742\\
243	2.07351400343677\\
244	2.07905497313625\\
245	2.04323682660534\\
246	2.15851964442678\\
247	1.93360921061516\\
248	2.0024913022139\\
249	2.05876418202431\\
250	2.03315703674219\\
251	2.0884965340862\\
252	1.92358563275556\\
253	1.96525319547271\\
254	2.0880168262253\\
255	1.9932171662393\\
256	1.90410475031383\\
};
\addlegendentry{Scatterer variance}

\addplot [color=black, only marks, mark size=4.0pt, mark=o, mark options={solid, black}]
  table[row sep=crcr]{%
127	22334.9321747989\\
128	52020.5564987292\\
};
\addlegendentry{Candidate scatterers}

\addplot [color=red]
  table[row sep=crcr]{%
0	22334.9321747989\\
300	22334.9321747989\\
};
\addlegendentry{Minimum variance}

\addplot [color=red, only marks, mark size=4.0pt, mark=*, mark options={solid, fill=red, red}]
  table[row sep=crcr]
                \caption{Scatterer amplitude variance, \\ sf = 10.\label{subfig:sf10_corrRA_var}}
            \end{subfigure}
        \end{tabular}
        \caption{\gls{ds} selection and autofocused \gls{isar} images for different scaling factors using Simple Correlation \gls{ra} \gls{hrr} profiles.\label{fig:sf_corrRA}}
    \end{minipage}
    \end{figure}

    % Grid of the HRRP and ISAR images SCRA
    \begin{figure}[H]
    \vspace*{\baselineskip}
    \centering
    \begin{minipage}{0.98\linewidth}
        \begin{tabular}{@{}ccc@{}}
            \begin{subfigure}{0.33\linewidth}
                \centering
                \resizebox{\linewidth}{!}{% This file was created by matlab2tikz.
%
%The latest updates can be retrieved from
%  http://www.mathworks.com/matlabcentral/fileexchange/22022-matlab2tikz-matlab2tikz
%where you can also make suggestions and rate matlab2tikz.
%
\begin{tikzpicture}
\begin{axis}[%
width=5.554in,
height=4.754in,
at={(0.932in,0.642in)},
scale only axis,
point meta min=-40,
point meta max=0,
axis on top,
xmin=0,
xmax=29.6578125,
xlabel style={font=\fontsize{25}{14}\selectfont\color{black}, yshift=-10pt},
xlabel={Range (m)},
ymin=-88.0034,
ymax=86.6175,
ylabel style={font=\fontsize{25}{14}\selectfont\color{black}},
ylabel={Doppler frquency (Hz)},
axis background/.style={fill=white},
tick label style={font=\fontsize{20}{11}\selectfont\color{black}},
xtick distance= 4,             % Set the spacing between x-axis ticks
ytick distance = 10,
colormap/jet,
colorbar
]
\addplot [forget plot] graphics [xmin=0, xmax=29.6578125, ymin=-88.0034, ymax=86.6175] {Figures/09Appendix/ScalingFactor/Measured/CorrRA/ISAR/HayAF_SCRA_Measured_ISAR_SF1.png};
\end{axis}

\end{tikzpicture}%}
                \caption{Autofocused \gls{isar} image, \\ sf = 1.\label{subfig:sf1_corrRA_isar}}
            \end{subfigure}
            &
            \begin{subfigure}{0.33\linewidth}
                \centering
                \resizebox{\linewidth}{!}{% This file was created by matlab2tikz.
%
%The latest updates can be retrieved from
%  http://www.mathworks.com/matlabcentral/fileexchange/22022-matlab2tikz-matlab2tikz
%where you can also make suggestions and rate matlab2tikz.
%
\begin{tikzpicture}
\begin{axis}[%
width=5.554in,
height=4.754in,
at={(0.932in,0.642in)},
scale only axis,
point meta min=-40,
point meta max=0,
axis on top,
xmin=0,
xmax=29.6578125,
xlabel style={font=\fontsize{25}{14}\selectfont\color{black}, yshift=-10pt},
xlabel={Range (m)},
ymin=-88.0034,
ymax=86.6175,
ylabel style={font=\fontsize{25}{14}\selectfont\color{black}},
ylabel={Doppler frquency (Hz)},
axis background/.style={fill=white},
tick label style={font=\fontsize{20}{11}\selectfont\color{black}},
xtick distance= 4,             % Set the spacing between x-axis ticks
ytick distance = 10,
colormap/jet,
colorbar
]
\addplot [forget plot] graphics [xmin=0, xmax=29.6578125, ymin=-88.0034, ymax=86.6175] {Figures/09Appendix/ScalingFactor/Measured/CorrRA/ISAR/HayAF_SCRA_Measured_ISAR_sf5.png};
\end{axis}

\end{tikzpicture}%}
                \caption{Autofocused \gls{isar} image, \\ sf = 5.\label{subfig:sf5_corrRA_isar}}
            \end{subfigure}
             &
            \begin{subfigure}{0.33\linewidth}
                \centering
                \resizebox{\linewidth}{!}{% This file was created by matlab2tikz.
%
%The latest updates can be retrieved from
%  http://www.mathworks.com/matlabcentral/fileexchange/22022-matlab2tikz-matlab2tikz
%where you can also make suggestions and rate matlab2tikz.
%
\begin{tikzpicture}
\begin{axis}[%
width=5.554in,
height=4.754in,
at={(0.932in,0.642in)},
scale only axis,
point meta min=-40,
point meta max=0,
axis on top,
xmin=0,
xmax=29.6578125,
xlabel style={font=\fontsize{25}{14}\selectfont\color{black}, yshift=-10pt},
xlabel={Range (m)},
ymin=-88.0034,
ymax=86.6175,
ylabel style={font=\fontsize{25}{14}\selectfont\color{black}},
ylabel={Doppler frquency (Hz)},
axis background/.style={fill=white},
tick label style={font=\fontsize{20}{11}\selectfont\color{black}},
xtick distance= 4,             % Set the spacing between x-axis ticks
ytick distance = 10,
colormap/jet,
colorbar
]
\addplot [forget plot] graphics [xmin=0, xmax=29.6578125, ymin=-88.0034, ymax=86.6175] {Figures/09Appendix/ScalingFactor/Measured/CorrRA/ISAR/HayAF_SCRA_Measured_ISAR_sf10.png};
\end{axis}

\end{tikzpicture}%}
                \caption{Autofocused \gls{isar} image, \\ sf = 10.\label{subfig:sf10_corrRA_isar}}
            \end{subfigure}
            \\
            \begin{subfigure}{0.33\linewidth}
                \centering
                \resizebox{\linewidth}{!}{% This file was created by matlab2tikz.
%
%The latest updates can be retrieved from
%  http://www.mathworks.com/matlabcentral/fileexchange/22022-matlab2tikz-matlab2tikz
%where you can also make suggestions and rate matlab2tikz.
%
\definecolor{mycolor1}{rgb}{0.00000,0.44700,0.74100}%
%
\begin{tikzpicture}

\begin{axis}[%
width=6.028in,
height=4.754in,
at={(1.011in,0.642in)},
scale only axis,
xmin=0,
xmax=256,
xlabel style={font=\fontsize{25}{20}\selectfont\color{black}, yshift = -10},
xlabel={Range Bin Number},
ymin=0,
ymax=25000000,
ylabel style={font=\fontsize{25}{20}\selectfont\color{black}, yshift=10pt},
ylabel={Power},
axis background/.style={fill=white},
tick label style={font=\fontsize{20}{11}\selectfont\color{black}},
xtick distance = 50,
yticklabel={\ifdim\tick pt=0pt\else\pgfmathprintnumber{\tick}\fi}, 
scaled y ticks=base 10:-6,
legend style={legend cell align=left, align = left, draw=white!15!black, font=\fontsize{12}{11}\selectfont\color{black}}
]
\addplot [color=mycolor1, mark=asterisk, mark options={solid, mycolor1}]
  table[row sep=crcr]{%
1	260.459583212987\\
2	255.853316684552\\
3	253.849369048871\\
4	262.902589895056\\
5	269.395431769183\\
6	257.942118684153\\
7	264.777673790391\\
8	267.345173590508\\
9	267.442550084895\\
10	274.763167201476\\
11	269.657038331506\\
12	273.52865970639\\
13	270.838241173277\\
14	270.848862745988\\
15	268.994507249835\\
16	278.092807479352\\
17	275.678440062448\\
18	280.764024183879\\
19	280.264596895383\\
20	280.844834122902\\
21	284.609662384031\\
22	289.349950042811\\
23	289.63453778992\\
24	297.216247874669\\
25	304.044956124828\\
26	299.728539743741\\
27	304.570239574428\\
28	312.586614746517\\
29	312.578394352122\\
30	313.03569904371\\
31	318.676695752314\\
32	334.429012992535\\
33	326.05568970326\\
34	329.794039972701\\
35	347.471923711035\\
36	357.356896491005\\
37	359.280067262786\\
38	370.172276306485\\
39	367.252690733492\\
40	384.036353269705\\
41	403.514451969125\\
42	409.987699077831\\
43	422.515008557612\\
44	437.076708129586\\
45	455.336146360901\\
46	474.987020352395\\
47	506.63892953046\\
48	524.698547335733\\
49	585.497580378801\\
50	627.110711153952\\
51	693.524309995736\\
52	784.287092174285\\
53	935.526405704739\\
54	1173.71124130302\\
55	1600.67007465833\\
56	2583.29049019285\\
57	5800.48900106346\\
58	40442.0449329261\\
59	234600.510606336\\
60	16827.8819816654\\
61	4770.1478970912\\
62	3240.29772132079\\
63	3634.40328263001\\
64	7268.77499224631\\
65	76531.2093170108\\
66	213667.240436973\\
67	10052.7936091226\\
68	4507.02792533953\\
69	3559.31518627866\\
70	4106.8388665825\\
71	8159.23968924983\\
72	127074.927266568\\
73	167160.607608639\\
74	9606.39694763152\\
75	4990.0602038816\\
76	4292.92605082353\\
77	5229.47347252321\\
78	10356.742292667\\
79	182309.818486979\\
80	108097.426019492\\
81	8627.78437379697\\
82	4318.19931958007\\
83	3854.49429303449\\
84	5199.41183792736\\
85	14022.0503602094\\
86	227432.741146917\\
87	63225.9569214638\\
88	8170.50934399543\\
89	4150.73895156004\\
90	3475.38764420246\\
91	4876.64326862932\\
92	22725.5309440892\\
93	244399.775686268\\
94	37495.2781672515\\
95	10573.4177170359\\
96	7100.60019107358\\
97	6753.16861846795\\
98	9242.40740232017\\
99	51530.6884280494\\
100	239715.038608182\\
101	21021.5521803837\\
102	12367.4161492386\\
103	12919.2292195377\\
104	18560.6270392406\\
105	50352.3007433402\\
106	491915.041237475\\
107	722546.673736452\\
108	42839.0513423732\\
109	18946.8184049635\\
110	13294.4378465799\\
111	11888.3265603265\\
112	14228.1063884036\\
113	161392.80534233\\
114	207528.804911758\\
115	32227.6875301607\\
116	26620.7506624016\\
117	28153.2588950145\\
118	33451.3666831277\\
119	47362.6527416718\\
120	243035.473131084\\
121	113402.734843283\\
122	67680.325702816\\
123	101247.331523687\\
124	175595.773473503\\
125	381658.413684424\\
126	1567657.49064522\\
127	23225002.3847125\\
128	4866592.85612614\\
129	624148.878839017\\
130	238968.489337471\\
131	125467.57654359\\
132	78185.2090712392\\
133	73030.3601940519\\
134	271320.771835579\\
135	66016.2557543734\\
136	35644.2191386429\\
137	27660.6982729765\\
138	24525.1417075442\\
139	27005.715637359\\
140	93273.0398557962\\
141	280507.640693694\\
142	15266.9861055596\\
143	11660.2329098454\\
144	12774.0609258221\\
145	18308.3665983194\\
146	54196.6326667642\\
147	559399.5054032\\
148	675605.960268304\\
149	46832.2981207944\\
150	20414.8653773866\\
151	13596.8722164049\\
152	11704.4608140175\\
153	14834.5483454693\\
154	165878.818809875\\
155	129155.452493415\\
156	12845.7472234616\\
157	8425.93245283305\\
158	7924.50240121107\\
159	9630.82839402855\\
160	18167.8674919882\\
161	221464.952930091\\
162	66296.2165083529\\
163	6282.75073094234\\
164	3550.82893427832\\
165	3693.84173148895\\
166	5793.15980013346\\
167	21194.8549443591\\
168	249988.987910509\\
169	37348.5151127678\\
170	6707.77320205252\\
171	3834.62559434609\\
172	3666.58727613934\\
173	6179.43754018277\\
174	39621.937991042\\
175	244027.580562077\\
176	19175.2805279218\\
177	6528.63349151219\\
178	4772.4713207705\\
179	5128.36765511789\\
180	9037.87893829267\\
181	79769.6559031287\\
182	213548.500351052\\
183	9189.08795174902\\
184	4363.70260986501\\
185	3712.07025077464\\
186	4602.33188561278\\
187	9220.50288249593\\
188	129112.974710105\\
189	162501.742729066\\
190	7574.41472023649\\
191	3473.96314063804\\
192	2883.8804389084\\
193	3771.98710657173\\
194	8603.85021353972\\
195	174816.050956961\\
196	107424.122922777\\
197	8158.46008299097\\
198	3188.81296756292\\
199	1874.3706597991\\
200	1319.41764201677\\
201	1034.95594151075\\
202	869.247784578133\\
203	742.863481776727\\
204	658.01083274331\\
205	610.719153796497\\
206	558.676138785167\\
207	527.727140126541\\
208	493.143467884876\\
209	472.767945849341\\
210	450.923013010194\\
211	429.591125867796\\
212	412.554276876183\\
213	404.874610605797\\
214	387.775528276528\\
215	391.405099058641\\
216	365.717226442089\\
217	364.330352475694\\
218	356.484671896599\\
219	358.687392545226\\
220	345.13318693796\\
221	338.830857316187\\
222	344.279776728209\\
223	335.852598946057\\
224	321.208276027795\\
225	320.221419869568\\
226	316.106449388835\\
227	306.342738798994\\
228	301.826156405654\\
229	300.26529353038\\
230	290.634496667669\\
231	295.148021859695\\
232	294.361094581251\\
233	291.018501216427\\
234	290.531983794715\\
235	293.620323539453\\
236	282.415873342337\\
237	281.56353725174\\
238	276.975587713\\
239	285.539867461153\\
240	274.106466494687\\
241	273.037195880877\\
242	282.942195034159\\
243	271.428841093763\\
244	265.861359523173\\
245	264.640464669466\\
246	269.787791563485\\
247	269.628633305419\\
248	259.668277498303\\
249	270.659371561747\\
250	265.10876919209\\
251	260.083470473997\\
252	261.220926812129\\
253	265.88563906247\\
254	257.030182737016\\
255	266.64222467541\\
256	259.75335716483\\
};
\addlegendentry{Scatterer power}

\addplot [color=orange]
  table[row sep=crcr]{%
0	158841.769294874\\
300	158841.769294874\\
};
\addlegendentry{Average power of all scatterers}

\addplot [color=black, only marks, mark size=4.0pt, mark=o, mark options={solid, black}]
  table[row sep=crcr]{%
59	234600.510606336\\
66	213667.240436973\\
73	167160.607608639\\
79	182309.818486979\\
86	227432.741146917\\
93	244399.775686268\\
100	239715.038608182\\
106	491915.041237475\\
107	722546.673736452\\
113	161392.80534233\\
114	207528.804911758\\
120	243035.473131084\\
124	175595.773473503\\
125	381658.413684424\\
126	1567657.49064522\\
127	23225002.3847125\\
128	4866592.85612614\\
129	624148.878839017\\
130	238968.489337471\\
134	271320.771835579\\
141	280507.640693694\\
147	559399.5054032\\
148	675605.960268304\\
154	165878.818809875\\
161	221464.952930091\\
168	249988.987910509\\
175	244027.580562077\\
182	213548.500351052\\
189	162501.742729066\\
195	174816.050956961\\
};
\addlegendentry{Candidate scatterers}

\addplot [color=red, only marks, mark size=7.5pt, mark=o, mark options={solid, red}]
  table[row sep=crcr]
                \caption{Scatterer power, \\ sf = 1.\label{subfig:sf1_corrRA_power}}
            \end{subfigure}
             &
            \begin{subfigure}{0.33\linewidth}
                \centering
                \resizebox{\linewidth}{!}{% This file was created by matlab2tikz.
%
%The latest updates can be retrieved from
%  http://www.mathworks.com/matlabcentral/fileexchange/22022-matlab2tikz-matlab2tikz
%where you can also make suggestions and rate matlab2tikz.
%
\definecolor{mycolor1}{rgb}{0.00000,0.44700,0.74100}%
%
\begin{tikzpicture}

\begin{axis}[%
width=6.028in,
height=4.754in,
at={(1.011in,0.642in)},
scale only axis,
xmin=0,
xmax=100,
xlabel style={font=\color{white!15!black}},
xlabel={Range Bin},
ymin=0,
ymax=450000000,
ylabel style={font=\color{white!15!black}},
ylabel={Power},
axis background/.style={fill=white},
title style={font=\bfseries},
title={Power of Scatterers},
legend style={legend cell align=left, align=left, draw=white!15!black}
]
\addplot [color=mycolor1, mark=asterisk, mark options={solid, mycolor1}]
  table[row sep=crcr]{%
1	861234.796275547\\
2	461974.086744226\\
3	212662.339773961\\
4	246092.798143225\\
5	577332.97113036\\
6	729796.050856513\\
7	285852.065768088\\
8	155361.175530255\\
9	136416.432899122\\
10	302947.017025261\\
11	380000.674236901\\
12	179786.527703499\\
13	90055.7061709671\\
14	134545.234798792\\
15	398767.920619562\\
16	294854.48094652\\
17	122705.065492897\\
18	158683.861429191\\
19	408318.951663835\\
20	743554.316483603\\
21	347829.42936103\\
22	213310.002614624\\
23	167223.526227361\\
24	220103.740306779\\
25	394848.395645618\\
26	204227.984290759\\
27	219002.288935564\\
28	190089.498926234\\
29	626846.49285405\\
30	1444945.06603732\\
31	3650263.95712096\\
32	11938621.9610267\\
33	44349992.8398528\\
34	58349363.7747511\\
35	110256789.005583\\
36	86307739.1322255\\
37	21485572.091045\\
38	23974523.3237539\\
39	20142931.4452402\\
40	10755159.5336766\\
41	11884190.0771573\\
42	14600825.1484434\\
43	128804982.972008\\
44	422629109.972909\\
45	154510209.337695\\
46	13638038.8013476\\
47	10474902.5022089\\
48	8795290.35070148\\
49	6149892.63456373\\
50	4530666.55097104\\
51	6218830.29546932\\
52	6627613.6289192\\
53	7702172.06942866\\
54	5670379.25198555\\
55	6029111.47245567\\
56	11146504.1938552\\
57	40973089.3177811\\
58	234197171.492459\\
59	261043993.014117\\
60	94038567.4996435\\
61	81074726.8044155\\
62	21215941.4722665\\
63	16843830.1409221\\
64	37411098.7126583\\
65	33057887.9061113\\
66	15025385.2684402\\
67	7620191.15082092\\
68	27270568.1058131\\
69	28803870.4548784\\
70	16286339.9782475\\
71	7778188.25426342\\
72	4043469.61373697\\
73	2202210.07194016\\
74	884615.743242843\\
75	636807.413944679\\
76	601554.468385522\\
77	547068.603506823\\
78	893601.681393332\\
79	506093.525588289\\
80	247064.291213952\\
81	187083.5266876\\
82	661330.807635752\\
83	688854.084561109\\
84	247028.312680993\\
85	173529.477201299\\
86	277910.276689611\\
87	440341.293104714\\
88	365167.084965798\\
89	162743.493973739\\
90	114180.84957851\\
91	243478.842305658\\
92	443347.004440988\\
93	394435.59230442\\
94	207066.130743798\\
95	204547.634516275\\
96	432162.614055562\\
};
\addlegendentry{Scatterer power}

\addplot [color=green]
  table[row sep=crcr]{%
0	112498936.835131\\
100	112498936.835131\\
};
\addlegendentry{Average power of all scatterers}

\addplot [color=black, only marks, mark size=4.0pt, mark=o, mark options={solid, black}]
  table[row sep=crcr]{%
43	128804982.972008\\
44	422629109.972909\\
45	154510209.337695\\
58	234197171.492459\\
59	261043993.014117\\
};
\addlegendentry{Candidate scatterers}

\addplot [color=red, only marks, mark size=7.5pt, mark=o, mark options={solid, red}]
  table[row sep=crcr]{%
43	128804982.972008\\
};
\addlegendentry{Dominant scatterer}

\end{axis}

\begin{axis}[%
width=7.778in,
height=5.833in,
at={(0in,0in)},
scale only axis,
xmin=0,
xmax=1,
ymin=0,
ymax=1,
axis line style={draw=none},
ticks=none,
axis x line*=bottom,
axis y line*=left
]
\end{axis}
\end{tikzpicture}%}
                \caption{Scatterer power, \\ sf = 5.\label{subfig:sf5_corrRA_power}}
            \end{subfigure}
             &
            \begin{subfigure}{0.33\linewidth}
                \centering
                \resizebox{\linewidth}{!}{% This file was created by matlab2tikz.
%
%The latest updates can be retrieved from
%  http://www.mathworks.com/matlabcentral/fileexchange/22022-matlab2tikz-matlab2tikz
%where you can also make suggestions and rate matlab2tikz.
%
\definecolor{mycolor1}{rgb}{0.00000,0.44700,0.74100}%
%
\begin{tikzpicture}

\begin{axis}[%
width=6.028in,
height=4.754in,
at={(1.011in,0.642in)},
scale only axis,
xmin=0,
xmax=100,
xlabel style={font=\color{white!15!black}},
xlabel={Range Bin},
ymin=0,
ymax=450000000,
ylabel style={font=\color{white!15!black}},
ylabel={Power},
axis background/.style={fill=white},
title style={font=\bfseries},
title={Power of Scatterers},
legend style={legend cell align=left, align=left, draw=white!15!black}
]
\addplot [color=mycolor1, mark=asterisk, mark options={solid, mycolor1}]
  table[row sep=crcr]{%
1	861234.796275547\\
2	461974.086744226\\
3	212662.339773961\\
4	246092.798143225\\
5	577332.97113036\\
6	729796.050856513\\
7	285852.065768088\\
8	155361.175530255\\
9	136416.432899122\\
10	302947.017025261\\
11	380000.674236901\\
12	179786.527703499\\
13	90055.7061709671\\
14	134545.234798792\\
15	398767.920619562\\
16	294854.48094652\\
17	122705.065492897\\
18	158683.861429191\\
19	408318.951663835\\
20	743554.316483603\\
21	347829.42936103\\
22	213310.002614624\\
23	167223.526227361\\
24	220103.740306779\\
25	394848.395645618\\
26	204227.984290759\\
27	219002.288935564\\
28	190089.498926234\\
29	626846.49285405\\
30	1444945.06603732\\
31	3650263.95712096\\
32	11938621.9610267\\
33	44349992.8398528\\
34	58349363.7747511\\
35	110256789.005583\\
36	86307739.1322255\\
37	21485572.091045\\
38	23974523.3237539\\
39	20142931.4452402\\
40	10755159.5336766\\
41	11884190.0771573\\
42	14600825.1484434\\
43	128804982.972008\\
44	422629109.972909\\
45	154510209.337695\\
46	13638038.8013476\\
47	10474902.5022089\\
48	8795290.35070148\\
49	6149892.63456373\\
50	4530666.55097104\\
51	6218830.29546932\\
52	6627613.6289192\\
53	7702172.06942866\\
54	5670379.25198555\\
55	6029111.47245567\\
56	11146504.1938552\\
57	40973089.3177811\\
58	234197171.492459\\
59	261043993.014117\\
60	94038567.4996435\\
61	81074726.8044155\\
62	21215941.4722665\\
63	16843830.1409221\\
64	37411098.7126583\\
65	33057887.9061113\\
66	15025385.2684402\\
67	7620191.15082092\\
68	27270568.1058131\\
69	28803870.4548784\\
70	16286339.9782475\\
71	7778188.25426342\\
72	4043469.61373697\\
73	2202210.07194016\\
74	884615.743242843\\
75	636807.413944679\\
76	601554.468385522\\
77	547068.603506823\\
78	893601.681393332\\
79	506093.525588289\\
80	247064.291213952\\
81	187083.5266876\\
82	661330.807635752\\
83	688854.084561109\\
84	247028.312680993\\
85	173529.477201299\\
86	277910.276689611\\
87	440341.293104714\\
88	365167.084965798\\
89	162743.493973739\\
90	114180.84957851\\
91	243478.842305658\\
92	443347.004440988\\
93	394435.59230442\\
94	207066.130743798\\
95	204547.634516275\\
96	432162.614055562\\
};
\addlegendentry{Scatterer power}

\addplot [color=green]
  table[row sep=crcr]{%
0	224997873.670263\\
100	224997873.670263\\
};
\addlegendentry{Average power of all scatterers}

\addplot [color=black, only marks, mark size=4.0pt, mark=o, mark options={solid, black}]
  table[row sep=crcr]{%
44	422629109.972909\\
58	234197171.492459\\
59	261043993.014117\\
};
\addlegendentry{Candidate scatterers}

\addplot [color=red, only marks, mark size=7.5pt, mark=o, mark options={solid, red}]
  table[row sep=crcr]{%
59	261043993.014117\\
};
\addlegendentry{Dominant scatterer}

\end{axis}

\begin{axis}[%
width=7.778in,
height=5.833in,
at={(0in,0in)},
scale only axis,
xmin=0,
xmax=1,
ymin=0,
ymax=1,
axis line style={draw=none},
ticks=none,
axis x line*=bottom,
axis y line*=left
]
\end{axis}
\end{tikzpicture}%}
                \caption{Scatterer power, \\ sf = 10.\label{subfig:sf10_corrRA_power}}
            \end{subfigure}
            \\
            \begin{subfigure}{0.33\linewidth}
                \centering
                \resizebox{\linewidth}{!}{% This file was created by matlab2tikz.
%
%The latest updates can be retrieved from
%  http://www.mathworks.com/matlabcentral/fileexchange/22022-matlab2tikz-matlab2tikz
%where you can also make suggestions and rate matlab2tikz.
%
\definecolor{mycolor1}{rgb}{0.00000,0.44700,0.74100}%
%
\begin{tikzpicture}

\begin{axis}[%
width=6.028in,
height=4.754in,
at={(1.011in,0.642in)},
scale only axis,
xmin=0,
xmax=100,
xlabel style={font=\color{white!15!black}},
xlabel={Range Bin},
ymin=0,
ymax=600000,
ylabel style={font=\color{white!15!black}},
ylabel={Amplitude Variance},
axis background/.style={fill=white},
title style={font=\bfseries},
title={Variance of Scatterers},
legend style={legend cell align=left, align=left, draw=white!15!black}
]
\addplot [color=mycolor1, mark=asterisk, mark options={solid, mycolor1}]
  table[row sep=crcr]{%
1	618.190133000413\\
2	641.073920671971\\
3	422.512775831874\\
4	342.530955124323\\
5	599.580876370041\\
6	931.691548806083\\
7	318.988608633882\\
8	236.80437716339\\
9	226.890255620622\\
10	458.171638136456\\
11	623.032438322094\\
12	326.547409786833\\
13	139.479408861878\\
14	153.571301611978\\
15	678.719610354443\\
16	378.028680888595\\
17	189.398142347566\\
18	293.700230537079\\
19	554.221100593161\\
20	808.444274291457\\
21	349.063629392178\\
22	403.543508145038\\
23	280.659159028239\\
24	350.836609029496\\
25	530.791168403754\\
26	270.867627639896\\
27	304.583711331055\\
28	288.806039398011\\
29	822.394108640439\\
30	2081.14893406934\\
31	5151.7589946324\\
32	18517.0209170654\\
33	65738.0448879609\\
34	93801.7638179143\\
35	115120.284135233\\
36	132173.939959572\\
37	38979.7480025487\\
38	45384.7243745594\\
39	27545.8723435685\\
40	14002.8572815111\\
41	16791.0802118644\\
42	25616.1698661406\\
43	113014.084207155\\
44	559848.864323783\\
45	158956.746947985\\
46	18660.3491477445\\
47	15733.5639697085\\
48	15935.6393678781\\
49	10455.8448372515\\
50	6860.38138484137\\
51	8742.19751476971\\
52	10483.0500009998\\
53	8437.94448607969\\
54	6774.29154241655\\
55	4393.63159660685\\
56	22951.874734016\\
57	76314.7795342918\\
58	245103.049430155\\
59	263297.977796135\\
60	239078.449962494\\
61	118714.253713222\\
62	24167.4056846046\\
63	29135.9504513176\\
64	35073.5310516745\\
65	24709.0471342535\\
66	15387.1300995051\\
67	4931.16417857924\\
68	34501.3446638806\\
69	25034.5995762561\\
70	27940.1314849251\\
71	12879.3147703385\\
72	7147.35232362722\\
73	4905.16196740392\\
74	1441.95002848645\\
75	932.394512340197\\
76	731.15729365425\\
77	973.987937406392\\
78	918.527094219897\\
79	594.998168271597\\
80	332.45976398315\\
81	265.163867084616\\
82	871.290685340808\\
83	859.811634275831\\
84	460.003653873436\\
85	386.602575956552\\
86	579.524956338482\\
87	781.800675304264\\
88	331.458314538084\\
89	271.699490941217\\
90	187.246670074608\\
91	300.976299842283\\
92	558.945620806556\\
93	407.26351676982\\
94	349.446013181976\\
95	320.22791369331\\
96	405.061423527288\\
};
\addlegendentry{Scatterer variance}

\addplot [color=black, only marks, mark size=4.0pt, mark=o, mark options={solid, black}]
  table[row sep=crcr]{%
33	65738.0448879609\\
34	93801.7638179143\\
35	115120.284135233\\
36	132173.939959572\\
38	45384.7243745594\\
43	113014.084207155\\
44	559848.864323783\\
45	158956.746947985\\
57	76314.7795342918\\
58	245103.049430155\\
59	263297.977796135\\
60	239078.449962494\\
61	118714.253713222\\
64	35073.5310516745\\
65	24709.0471342535\\
68	34501.3446638806\\
69	25034.5995762561\\
};
\addlegendentry{Candidate scatterers}

\addplot [color=red]
  table[row sep=crcr]{%
0	24709.0471342535\\
100	24709.0471342535\\
};
\addlegendentry{Minimum variance}

\addplot [color=red, only marks, mark size=4.0pt, mark=*, mark options={solid, fill=red, red}]
  table[row sep=crcr]{%
65	24709.0471342535\\
};
\addlegendentry{Dominant Scatterer}

\end{axis}

\begin{axis}[%
width=7.778in,
height=5.833in,
at={(0in,0in)},
scale only axis,
xmin=0,
xmax=1,
ymin=0,
ymax=1,
axis line style={draw=none},
ticks=none,
axis x line*=bottom,
axis y line*=left
]
\end{axis}
\end{tikzpicture}%}
                \caption{Scatterer amplitude variance, \\ sf = 1.\label{subfig:sf1_corrRA_var}}
            \end{subfigure}
             &
            \begin{subfigure}{0.33\linewidth}
                \centering
                \resizebox{\linewidth}{!}{% This file was created by matlab2tikz.
%
%The latest updates can be retrieved from
%  http://www.mathworks.com/matlabcentral/fileexchange/22022-matlab2tikz-matlab2tikz
%where you can also make suggestions and rate matlab2tikz.
%
\definecolor{mycolor1}{rgb}{0.00000,0.44700,0.74100}%
%
\begin{tikzpicture}

\begin{axis}[%
width=6.028in,
height=4.754in,
at={(1.011in,0.642in)},
scale only axis,
xmin=0,
xmax=100,
xlabel style={font=\color{white!15!black}},
xlabel={Range Bin},
ymin=0,
ymax=450000,
ylabel style={font=\color{white!15!black}},
ylabel={Amplitude Variance},
axis background/.style={fill=white},
title style={font=\bfseries},
title={Variance of Scatterers},
legend style={legend cell align=left, align=left, draw=white!15!black}
]
\addplot [color=mycolor1, mark=asterisk, mark options={solid, mycolor1}]
  table[row sep=crcr]{%
1	647.217984313041\\
2	563.100348876029\\
3	334.169339564685\\
4	380.725281584323\\
5	726.897722232442\\
6	849.058714157362\\
7	414.94803021772\\
8	236.004573680903\\
9	210.473449592015\\
10	488.636968725818\\
11	546.69001097757\\
12	287.356302235054\\
13	136.444016868391\\
14	251.745116732208\\
15	632.506144136254\\
16	434.815070048365\\
17	234.704074034683\\
18	280.607902377895\\
19	630.669747232136\\
20	860.768838400319\\
21	393.366881683836\\
22	399.240488738232\\
23	312.268341730517\\
24	360.315753592086\\
25	553.325041082902\\
26	340.638673436448\\
27	284.577914057137\\
28	286.976311067377\\
29	848.156695123514\\
30	2227.10271942391\\
31	7722.42896008108\\
32	21905.1518671198\\
33	64952.4541138596\\
34	94712.840244417\\
35	137074.721698901\\
36	144481.85618763\\
37	46818.8727397121\\
38	48369.4569567958\\
39	27360.0575282297\\
40	15519.7379197337\\
41	17297.7991252704\\
42	21415.8122868852\\
43	208642.940334747\\
44	448239.967068445\\
45	280396.467962111\\
46	24159.245676775\\
47	16025.3417256519\\
48	16395.1738007991\\
49	10617.5882928678\\
50	7992.92874011267\\
51	8591.81353013173\\
52	11706.5970623024\\
53	8064.07854091832\\
54	6599.19138865516\\
55	5603.31009808496\\
56	24929.1156949641\\
57	91388.6123871712\\
58	300762.7622818\\
59	275776.620655968\\
60	181490.462107662\\
61	106245.733622603\\
62	39161.9950864037\\
63	29490.6569020502\\
64	37585.2160411195\\
65	21922.5433356364\\
66	20528.7313915394\\
67	12017.6039950945\\
68	32538.5823087421\\
69	26872.8231195526\\
70	29710.2921247632\\
71	15630.5276582884\\
72	7609.18175407986\\
73	5176.12546005825\\
74	1906.60784267829\\
75	870.141203936639\\
76	917.089842018173\\
77	985.998698394865\\
78	999.733399745399\\
79	512.698003491124\\
80	346.293291979048\\
81	327.789481908754\\
82	942.764204564517\\
83	875.321214924181\\
84	437.660272970471\\
85	362.249842269553\\
86	596.938546884841\\
87	698.584533821333\\
88	467.391944515639\\
89	261.784906999898\\
90	194.299314334139\\
91	366.717511897959\\
92	478.089664815118\\
93	440.658162689373\\
94	336.011624348919\\
95	323.98566389372\\
96	554.935081389164\\
};
\addlegendentry{Scatterer variance}

\addplot [color=black, only marks, mark size=4.0pt, mark=o, mark options={solid, black}]
  table[row sep=crcr]{%
43	208642.940334747\\
44	448239.967068445\\
45	280396.467962111\\
58	300762.7622818\\
59	275776.620655968\\
};
\addlegendentry{Candidate scatterers}

\addplot [color=red]
  table[row sep=crcr]{%
0	208642.940334747\\
100	208642.940334747\\
};
\addlegendentry{Minimum variance}

\addplot [color=red, only marks, mark size=4.0pt, mark=*, mark options={solid, fill=red, red}]
  table[row sep=crcr]
                \caption{Scatterer amplitude variance, \\ sf = 5.\label{subfig:sf5_corrRA_var}}
            \end{subfigure}
             &
            \begin{subfigure}{0.33\linewidth}
                \centering
                \resizebox{\linewidth}{!}{% This file was created by matlab2tikz.
%
%The latest updates can be retrieved from
%  http://www.mathworks.com/matlabcentral/fileexchange/22022-matlab2tikz-matlab2tikz
%where you can also make suggestions and rate matlab2tikz.
%
\definecolor{mycolor1}{rgb}{0.00000,0.44700,0.74100}%
%
\begin{tikzpicture}

\begin{axis}[%
width=6.028in,
height=4.754in,
at={(1.011in,0.642in)},
scale only axis,
xmin=0,
xmax=100,
xlabel style={font=\color{white!15!black}},
xlabel={Range Bin},
ymin=0,
ymax=450000,
ylabel style={font=\color{white!15!black}},
ylabel={Amplitude Variance},
axis background/.style={fill=white},
title style={font=\bfseries},
title={Variance of Scatterers},
legend style={legend cell align=left, align=left, draw=white!15!black}
]
\addplot [color=mycolor1, mark=asterisk, mark options={solid, mycolor1}]
  table[row sep=crcr]{%
1	647.217984313041\\
2	563.100348876029\\
3	334.169339564685\\
4	380.725281584323\\
5	726.897722232442\\
6	849.058714157362\\
7	414.94803021772\\
8	236.004573680903\\
9	210.473449592015\\
10	488.636968725818\\
11	546.69001097757\\
12	287.356302235054\\
13	136.444016868391\\
14	251.745116732208\\
15	632.506144136254\\
16	434.815070048365\\
17	234.704074034683\\
18	280.607902377895\\
19	630.669747232136\\
20	860.768838400319\\
21	393.366881683836\\
22	399.240488738232\\
23	312.268341730517\\
24	360.315753592086\\
25	553.325041082902\\
26	340.638673436448\\
27	284.577914057137\\
28	286.976311067377\\
29	848.156695123514\\
30	2227.10271942391\\
31	7722.42896008108\\
32	21905.1518671198\\
33	64952.4541138596\\
34	94712.840244417\\
35	137074.721698901\\
36	144481.85618763\\
37	46818.8727397121\\
38	48369.4569567958\\
39	27360.0575282297\\
40	15519.7379197337\\
41	17297.7991252704\\
42	21415.8122868852\\
43	208642.940334747\\
44	448239.967068445\\
45	280396.467962111\\
46	24159.245676775\\
47	16025.3417256519\\
48	16395.1738007991\\
49	10617.5882928678\\
50	7992.92874011267\\
51	8591.81353013173\\
52	11706.5970623024\\
53	8064.07854091832\\
54	6599.19138865516\\
55	5603.31009808496\\
56	24929.1156949641\\
57	91388.6123871712\\
58	300762.7622818\\
59	275776.620655968\\
60	181490.462107662\\
61	106245.733622603\\
62	39161.9950864037\\
63	29490.6569020502\\
64	37585.2160411195\\
65	21922.5433356364\\
66	20528.7313915394\\
67	12017.6039950945\\
68	32538.5823087421\\
69	26872.8231195526\\
70	29710.2921247632\\
71	15630.5276582884\\
72	7609.18175407986\\
73	5176.12546005825\\
74	1906.60784267829\\
75	870.141203936639\\
76	917.089842018173\\
77	985.998698394865\\
78	999.733399745399\\
79	512.698003491124\\
80	346.293291979048\\
81	327.789481908754\\
82	942.764204564517\\
83	875.321214924181\\
84	437.660272970471\\
85	362.249842269553\\
86	596.938546884841\\
87	698.584533821333\\
88	467.391944515639\\
89	261.784906999898\\
90	194.299314334139\\
91	366.717511897959\\
92	478.089664815118\\
93	440.658162689373\\
94	336.011624348919\\
95	323.98566389372\\
96	554.935081389164\\
};
\addlegendentry{Scatterer variance}

\addplot [color=black, only marks, mark size=4.0pt, mark=o, mark options={solid, black}]
  table[row sep=crcr]{%
44	448239.967068445\\
58	300762.7622818\\
59	275776.620655968\\
};
\addlegendentry{Candidate scatterers}

\addplot [color=red]
  table[row sep=crcr]{%
0	275776.620655968\\
100	275776.620655968\\
};
\addlegendentry{Minimum variance}

\addplot [color=red, only marks, mark size=4.0pt, mark=*, mark options={solid, fill=red, red}]
  table[row sep=crcr]{%
59	275776.620655968\\
};
\addlegendentry{Dominant Scatterer}

\end{axis}

\begin{axis}[%
width=7.778in,
height=5.833in,
at={(0in,0in)},
scale only axis,
xmin=0,
xmax=1,
ymin=0,
ymax=1,
axis line style={draw=none},
ticks=none,
axis x line*=bottom,
axis y line*=left
]
\end{axis}
\end{tikzpicture}%}
                \caption{Scatterer amplitude variance, \\ sf = 10.\label{subfig:sf10_corrRA_var}}
            \end{subfigure}
        \end{tabular}
        \caption{\gls{ds} selection and autofocused \gls{isar} images for different scaling factors using Correlation \gls{ra} measured data \gls{hrr} profiles.\label{fig:sf_corrRA}}
    \end{minipage}
    \end{figure}

    %***************************************************************************************%   
    \subsection{Using Haywood Range-aligned Profiles}
    The Haywood range-aligned \gls{isar} image, as shown in \autoref{fig:sf_hayRA_isar}, is included for comparison to the autofocused \gls{isar} images, \autoref{fig:sf_hayRA}(a)-(c). Additionally, the \gls{isar} image generated when the object had no translational motion, as shown in \autoref{subfig:sf_focused_isar}, is a reference for the expected appearance of a focused image.
    
    \begin{figure}[H]
        \centering
        \begin{minipage}{0.4\linewidth}
            \begin{subfigure}{\linewidth}
                \centering
                \resizebox{\linewidth}{!}{% This file was created by matlab2tikz.
%
%The latest updates can be retrieved from
%  http://www.mathworks.com/matlabcentral/fileexchange/22022-matlab2tikz-matlab2tikz
%where you can also make suggestions and rate matlab2tikz.
%
\begin{tikzpicture}
\begin{axis}[%
width=5.554in,
height=4.754in,
at={(0.932in,0.642in)},
scale only axis,
point meta min=-40,
point meta max=0,
axis on top,
xmin=-0.0732421875,
xmax=37.4267578125,
xlabel style={font=\fontsize{25}{14}\selectfont\color{black}, yshift=-10pt},
xlabel={Range (m)},
ymin=-32.2265625,
ymax=30.2734375,
ylabel style={font=\fontsize{25}{14}\selectfont\color{black}},
ylabel={Doppler frquency (Hz)},
axis background/.style={fill=white},
tick label style={font=\fontsize{20}{11}\selectfont\color{black}},
xtick distance= 4,             % Set the spacing between x-axis ticks
ytick distance = 10,
colormap/jet,
colorbar
]
\addplot [forget plot] graphics [xmin=-0.0732421875, xmax=37.4267578125, ymin=-32.2265625, ymax=30.2734375] {Figures/04AlgoV&V/Simulation/03HayRA/HayRA_Sim_ISAR_1mps_6deg.png};
\end{axis}

\end{tikzpicture}%
}
            \end{subfigure}
            \caption{Range-aligned image.\label{subfig:sf_corrRA_isar}}   
        \end{minipage}
        \hspace{1cm}
        \begin{minipage}{0.4\linewidth}
            \begin{subfigure}{\linewidth}
                \centering
                \resizebox{\linewidth}{!}{% This file was created by matlab2tikz.
%
%The latest updates can be retrieved from
%  http://www.mathworks.com/matlabcentral/fileexchange/22022-matlab2tikz-matlab2tikz
%where you can also make suggestions and rate matlab2tikz.
%
\begin{tikzpicture}

    \begin{axis}[%
    width=5.554in,
    height=4.754in,
    at={(0.932in,0.642in)},
    scale only axis,
    point meta min=-40,
    point meta max=0,
    axis on top,
    xmin=-0.0732421875,
    xmax=37.4267578125,
    xlabel style={font=\color{white!15!black}},
    xlabel={Range (m)},
    ymin=-32.2265625,
    ymax=30.2734375,
    ylabel style={font=\color{white!15!black}},
    ylabel={Doppler frquency (Hz)},
    axis background/.style={fill=white},
    colormap/jet,
    colorbar
    ]
    \addplot [forget plot] graphics [xmin=-0.0732421875, xmax=37.4267578125, ymin=-32.2265625, ymax=30.2734375] {Sim_ISAR_0mps_6deg.png};
    \end{axis}
    
    \begin{axis}[%
    width=7.778in,
    height=5.833in,
    at={(0in,0in)},
    scale only axis,
    point meta min=0,
    point meta max=1,
    xmin=0,
    xmax=1,
    ymin=0,
    ymax=1,
    axis line style={draw=none},
    ticks=none,
    axis x line*=bottom,
    axis y line*=left
    ]
    \end{axis}
    \end{tikzpicture}%}
            \end{subfigure} 
            \caption{Reference focused image.\label{subfig:sf_focused_isar}}
        \end{minipage}
    \end{figure}

    \autoref{fig:sf_hayRA} shows autofocused \gls{isar} images and plots demonstrating the fulfillment of both selection criteria for various \gls{sf} values. Similar to the observations as described in \autoref{subsec:sf_corrRA}, can be made for the figures in \autoref{fig:sf_hayRA}; as the \gls{sf} increases \gls{ds} scatterers with higher power are selected. As previously discussed in \autoref{subsec:sf_corrRA}, the \gls{isar} images in this subsection are also affected by Doppler spreading caused by sidelobes.
    
    Comparing \autoref{fig:sf_hayRA}(a) to (b) and (c) shows that when a higher power \gls{ds} is selected, the \gls{isar} image becomes more focused, indicated by each scatterer's power residing mostly in one place. This reaffirms the conclusion drawn in \autoref{subsec:sf_corrRA}; introducing a \gls{sf} leads to a more focused image. Unlike the previous section, where scaling factors of 5 and 10 produced different \gls{ds} selections, in this case, they both lead to the same \gls{ds} choice. This consequently results in identical focused images in \autoref{fig:sf_hayRA}(a) and (b). Thus, the effectiveness and necessity of a scaling factor (SF) depend on the dataset; however, both the previous section and this section show that a scaling factor of 10 is suitable.

    Notably, for a scaling factor of 5, the spectral spreading observed in \autoref{subfig:sf5_corrRA_isar} does not appear in \autoref{fig:sf_hayRA}(b). This further reinforces that the image defocusing observed in the previous section when using a \gls{sf} of 5, is a consequence of the characteristics of the Simple Correlation dataset.
    
    The image focus can be quantitatively measured using the \gls{ic} value calculated as in \autoref{eq:image_contrast}. The values are \textbf{46.14}, \textbf{50.03}, and \textbf{50.03} for \autoref{fig:sf_hayRA}(a)-(c), respectively. These values indicate that both (b) and (c) have the highest \gls{ic} value, confirming the previous analysis that the image is most focused for scaling factors of 5 and 10. In summary, comparing \autoref{fig:sf_corrRA}(a)-(c) and their respective \gls{ic} values shows that introducing a \gls{sf} can result in a more focused \gls{isar} image. Specifically, for these Haywood Correlation range-aligned profiles, a \gls{sf} of 5 and 10 produces the most focused image.
    
    % Grid of the HRRP and ISAR images HaywoodRA
    \begin{figure}[H]
    \centering
    \begin{minipage}{0.98\linewidth}
        \begin{tabular}{@{}ccc@{}}
            \begin{subfigure}{0.33\linewidth}
                \centering
                \resizebox{\linewidth}{!}{% This file was created by matlab2tikz.
%
%The latest updates can be retrieved from
%  http://www.mathworks.com/matlabcentral/fileexchange/22022-matlab2tikz-matlab2tikz
%where you can also make suggestions and rate matlab2tikz.
%
\begin{tikzpicture}
\begin{axis}[%
width=5.554in,
height=4.754in,
at={(0.932in,0.642in)},
scale only axis,
point meta min=-40,
point meta max=0,
axis on top,
xmin=-0.0732421875,
xmax=37.4267578125,
xlabel style={font=\fontsize{25}{14}\selectfont\color{black}, yshift=-10pt},
xlabel={Range (m)},
ymin=-32.2265625,
ymax=30.2734375,
ylabel style={font=\fontsize{25}{14}\selectfont\color{black}},
ylabel={Doppler frquency (Hz)},
axis background/.style={fill=white},
tick label style={font=\fontsize{20}{11}\selectfont\color{black}},
xtick distance= 4,             % Set the spacing between x-axis ticks
ytick distance = 10,
colormap/jet,
colorbar
]
\addplot [forget plot] graphics [xmin=-0.0732421875, xmax=37.4267578125, ymin=-32.2265625, ymax=30.2734375] {Figures/09Appendix/ScalingFactor/HayRA/ISAR/HayAF_hayRA_Sim_ISAR_sf1.png};
\end{axis}

\end{tikzpicture}%}
                \caption{Autofocused \gls{isar} image, \\ sf = 1.\label{subfig:sf1_hayRA_isar}}
            \end{subfigure}
            &
            \begin{subfigure}{0.33\linewidth}
                \centering
                \resizebox{\linewidth}{!}{% This file was created by matlab2tikz.
%
%The latest updates can be retrieved from
%  http://www.mathworks.com/matlabcentral/fileexchange/22022-matlab2tikz-matlab2tikz
%where you can also make suggestions and rate matlab2tikz.
%
\begin{tikzpicture}
\begin{axis}[%
width=5.554in,
height=4.754in,
at={(0.932in,0.642in)},
scale only axis,
point meta min=-40,
point meta max=0,
axis on top,
xmin=-0.0732421875,
xmax=37.4267578125,
xlabel style={font=\fontsize{25}{14}\selectfont\color{black}, yshift=-10pt},
xlabel={Range (m)},
ymin=-32.2265625,
ymax=30.2734375,
ylabel style={font=\fontsize{25}{14}\selectfont\color{black}},
ylabel={Doppler frquency (Hz)},
axis background/.style={fill=white},
tick label style={font=\fontsize{20}{11}\selectfont\color{black}},
xtick distance= 4,             % Set the spacing between x-axis ticks
ytick distance = 10,
colormap/jet,
colorbar
]
\addplot [forget plot] graphics [xmin=-0.0732421875, xmax=37.4267578125, ymin=-32.2265625, ymax=30.2734375] {Figures/09Appendix/ScalingFactor/HayRA/ISAR/HayAF_hayRA_Sim_ISAR_sf5.png};
\end{axis}

\end{tikzpicture}%}
                \caption{Autofocused \gls{isar} image, \\ sf = 5.\label{subfig:sf5_hayRA_isar}}
            \end{subfigure}
             &
            \begin{subfigure}{0.33\linewidth}
                \centering
                \resizebox{\linewidth}{!}{% This file was created by matlab2tikz.
%
%The latest updates can be retrieved from
%  http://www.mathworks.com/matlabcentral/fileexchange/22022-matlab2tikz-matlab2tikz
%where you can also make suggestions and rate matlab2tikz.
%
\begin{tikzpicture}
\begin{axis}[%
width=5.554in,
height=4.754in,
at={(0.932in,0.642in)},
scale only axis,
point meta min=-40,
point meta max=0,
axis on top,
xmin=-0.0732421875,
xmax=37.4267578125,
xlabel style={font=\fontsize{25}{14}\selectfont\color{black}, yshift=-10pt},
xlabel={Range (m)},
ymin=-32.2265625,
ymax=30.2734375,
ylabel style={font=\fontsize{25}{14}\selectfont\color{black}},
ylabel={Doppler frquency (Hz)},
axis background/.style={fill=white},
tick label style={font=\fontsize{20}{11}\selectfont\color{black}},
xtick distance= 4,             % Set the spacing between x-axis ticks
ytick distance = 10,
colormap/jet,
colorbar
]
\addplot [forget plot] graphics [xmin=-0.0732421875, xmax=37.4267578125, ymin=-32.2265625, ymax=30.2734375] {Figures/09Appendix/ScalingFactor/HayRA/ISAR/HayAF_hayRA_Sim_ISAR_sf10.png};
\end{axis}

\end{tikzpicture}%}
                \caption{Autofocused \gls{isar} image, \\ sf = 10.\label{subfig:sf10_hayRA_isar}}
            \end{subfigure}
            \\
            \begin{subfigure}{0.33\linewidth}
                \centering
                \resizebox{\linewidth}{!}{% This file was created by matlab2tikz.
%
%The latest updates can be retrieved from
%  http://www.mathworks.com/matlabcentral/fileexchange/22022-matlab2tikz-matlab2tikz
%where you can also make suggestions and rate matlab2tikz.
%
\definecolor{mycolor1}{rgb}{0.00000,0.44700,0.74100}%
%
\begin{tikzpicture}

\begin{axis}[%
width=6.028in,
height=4.754in,
at={(1.011in,0.642in)},
scale only axis,
xmin=0,
xmax=256,
xlabel style={font=\fontsize{25}{20}\selectfont\color{black}, yshift = -10},
xlabel={Range Bin},
ymin=0,
ymax=0.1e7,
ylabel style={font=\fontsize{25}{20}\selectfont\color{black}, yshift=10pt},
ylabel={Power},
axis background/.style={fill=white},
tick label style={font=\fontsize{20}{11}\selectfont\color{black}},
xtick distance = 50,
ytick distance= 0.01e7,
yticklabel={\ifdim\tick pt=0pt\else\pgfmathprintnumber{\tick}\fi}, 
scaled y ticks=base 10:-6,
legend style={legend cell align=left, align = left, draw=white!15!black, font=\fontsize{12}{11}\selectfont\color{black}}
]
\addplot [color=mycolor1, mark=asterisk, mark options={solid, mycolor1}]
  table[row sep=crcr]{%
1	113.804819296221\\
2	119.790508454642\\
3	110.840531570274\\
4	120.850302357011\\
5	111.837384170332\\
6	116.667449877817\\
7	117.692077066363\\
8	120.256868860117\\
9	115.241558602183\\
10	118.30285153176\\
11	114.470137544551\\
12	121.432407236675\\
13	121.337250764455\\
14	114.439293759528\\
15	119.600398281531\\
16	118.657326663683\\
17	116.099604422439\\
18	119.825488936544\\
19	120.366515894014\\
20	122.677458684777\\
21	123.521175881764\\
22	119.243092700349\\
23	124.076018099531\\
24	125.804630162094\\
25	119.821843772234\\
26	122.447591776688\\
27	125.617229248134\\
28	130.457768503276\\
29	129.230592458908\\
30	130.073223193218\\
31	127.391272106831\\
32	136.364273625649\\
33	136.839604540093\\
34	135.840624484261\\
35	136.334013826011\\
36	142.29516362192\\
37	143.367729727864\\
38	144.312528197113\\
39	145.713410166678\\
40	151.872597381043\\
41	151.256909386449\\
42	152.490062152633\\
43	153.158019185266\\
44	154.443803023559\\
45	158.038007632275\\
46	165.477012125201\\
47	167.420516130403\\
48	169.698428696131\\
49	183.1972144082\\
50	197.056046303189\\
51	194.084790884547\\
52	213.156947014836\\
53	237.382771630557\\
54	259.040409614016\\
55	315.782725738232\\
56	445.035684238851\\
57	837.595253625467\\
58	3639.66916717166\\
59	305414.439334397\\
60	3835.43468066153\\
61	1887.86413875201\\
62	1742.56196307664\\
63	2414.06129850586\\
64	5308.66800400896\\
65	30906.6459653039\\
66	252715.132555661\\
67	14841.7787976771\\
68	6952.85515168846\\
69	5730.10241204839\\
70	7039.63510628107\\
71	14369.0293429266\\
72	100879.701771474\\
73	159371.271237325\\
74	17829.1554263361\\
75	8763.48111322606\\
76	7295.60634239695\\
77	9012.19812166504\\
78	18955.188749971\\
79	202979.862611798\\
80	68474.8525123593\\
81	10404.4399212362\\
82	4922.09676691648\\
83	3879.70294812571\\
84	4751.08824495969\\
85	10601.1037491353\\
86	287904.250570282\\
87	17187.4170468497\\
88	3506.45561692701\\
89	1728.77803224861\\
90	1292.71234685062\\
91	1181.24564815292\\
92	1465.51808500763\\
93	316942.238697319\\
94	1699.17595993759\\
95	1386.61179679004\\
96	1599.0508076363\\
97	2097.00063056775\\
98	3474.78710638171\\
99	11781.2957777295\\
100	300030.301222004\\
101	7963.83473289811\\
102	5538.69722784045\\
103	7390.65233919251\\
104	13949.8209511962\\
105	42799.1918747095\\
106	634387.580501098\\
107	542309.016875558\\
108	53217.7841546764\\
109	22243.3008695406\\
110	13947.4069372449\\
111	11711.1788276429\\
112	15897.2034063322\\
113	114260.165900619\\
114	164864.174606625\\
115	34643.1297338301\\
116	23436.7229513031\\
117	21929.4660498748\\
118	25240.3767842959\\
119	39049.1226027816\\
120	282606.694264803\\
121	54395.2896052709\\
122	30860.2311051468\\
123	41907.9753146184\\
124	69562.3173507848\\
125	143844.152005639\\
126	487085.30903027\\
127	30041238.2526662\\
128	820374.313320368\\
129	166965.968291204\\
130	67654.2816482138\\
131	35487.3847005029\\
132	21383.9254430813\\
133	14441.0848233335\\
134	308613.662690684\\
135	6557.85017715512\\
136	4395.53946595492\\
137	2939.37168520857\\
138	1860.34032242672\\
139	1503.8375621428\\
140	9632.22720322441\\
141	328680.609022961\\
142	20648.956434482\\
143	11876.763471802\\
144	11602.0768910459\\
145	17341.9490023339\\
146	51511.5001717174\\
147	702806.071232772\\
148	464083.278952348\\
149	55586.1740926012\\
150	25416.6743885364\\
151	17585.6235277664\\
152	16781.8344351078\\
153	25461.7723569636\\
154	177586.029270219\\
155	97332.3478692499\\
156	16162.0744047654\\
157	9238.59738885476\\
158	8158.50779689371\\
159	9723.13463618985\\
160	18794.1338869738\\
161	279041.234785391\\
162	24993.7804767813\\
163	4277.72060033848\\
164	2203.12618370688\\
165	1856.99085780678\\
166	2175.50626922384\\
167	4199.74173904722\\
168	328371.37127647\\
169	1477.78549243884\\
170	411.996093462925\\
171	398.861374192699\\
172	501.911812110507\\
173	799.772092867879\\
174	2753.10742682493\\
175	316447.535004739\\
176	4112.08241972925\\
177	2331.73319920106\\
178	2306.35613236482\\
179	3180.3766823674\\
180	6483.91254359034\\
181	32829.8494872609\\
182	250461.883577277\\
183	13740.4943085509\\
184	6326.9664842519\\
185	5428.00175032638\\
186	7092.84176261911\\
187	14978.010621753\\
188	102741.007893307\\
189	152709.324936418\\
190	15586.3902747824\\
191	6859.26643013147\\
192	5434.12197680116\\
193	6874.7753389585\\
194	16015.8789880838\\
195	190877.361663101\\
196	68685.1233758713\\
197	10289.0275223684\\
198	4135.60433066042\\
199	2315.11687589113\\
200	1510.32550688942\\
201	1103.52460918996\\
202	855.66362378199\\
203	701.621215197102\\
204	596.296685376475\\
205	509.877504974154\\
206	454.900061399629\\
207	398.938091933802\\
208	364.556346451259\\
209	338.564096901927\\
210	306.336191693764\\
211	294.916759271909\\
212	268.388418260672\\
213	253.256636188222\\
214	244.319941040509\\
215	235.445540647214\\
216	224.436877509467\\
217	213.445869995075\\
218	205.16380977634\\
219	201.622341164429\\
220	187.738353848786\\
221	187.537461010913\\
222	175.440863843598\\
223	167.00095080967\\
224	172.211728199617\\
225	165.141793221037\\
226	163.914153617217\\
227	154.496906026136\\
228	156.926946592658\\
229	149.086651108539\\
230	152.412417652061\\
231	150.73074673795\\
232	143.945843325761\\
233	144.781277838153\\
234	136.154191448897\\
235	144.111912858972\\
236	136.988146150938\\
237	126.815718815859\\
238	132.039610418778\\
239	135.391687473132\\
240	133.447223052532\\
241	129.602878429017\\
242	131.041100462298\\
243	131.157773828424\\
244	125.9457912643\\
245	123.471417291737\\
246	123.15984841204\\
247	121.847767769105\\
248	123.019718402894\\
249	121.302763782579\\
250	117.113748394336\\
251	116.765907343351\\
252	117.004625797412\\
253	121.474661416875\\
254	117.34725068376\\
255	121.093635775499\\
256	119.459452248273\\
};
\addlegendentry{Scatterer power}

\addplot [color=orange]
  table[row sep=crcr]{%
0	158820.254743555\\
300	158820.254743555\\
};
\addlegendentry{Average power of all scatterers}

\addplot [color=black, only marks, mark size=4.0pt, mark=o, mark options={solid, black}]
  table[row sep=crcr]{%
59	305414.439334397\\
66	252715.132555661\\
73	159371.271237325\\
79	202979.862611798\\
86	287904.250570282\\
93	316942.238697319\\
100	300030.301222004\\
106	634387.580501098\\
107	542309.016875558\\
114	164864.174606625\\
120	282606.694264803\\
126	487085.30903027\\
127	30041238.2526662\\
128	820374.313320368\\
129	166965.968291204\\
134	308613.662690684\\
141	328680.609022961\\
147	702806.071232772\\
148	464083.278952348\\
154	177586.029270219\\
161	279041.234785391\\
168	328371.37127647\\
175	316447.535004739\\
182	250461.883577277\\
195	190877.361663101\\
};
\addlegendentry{Candidate scatterers}

\addplot [color=red, only marks, mark size=7.5pt, mark=o, mark options={solid, red}]
  table[row sep=crcr]
                \caption{Scatterer power, \\ sf = 1.\label{subfig:sf1_hayRA_power}}
            \end{subfigure}
             &
            \begin{subfigure}{0.33\linewidth}
                \centering
                \resizebox{\linewidth}{!}{% This file was created by matlab2tikz.
%
%The latest updates can be retrieved from
%  http://www.mathworks.com/matlabcentral/fileexchange/22022-matlab2tikz-matlab2tikz
%where you can also make suggestions and rate matlab2tikz.
%
\definecolor{mycolor1}{rgb}{0.00000,0.44700,0.74100}%
%
\begin{tikzpicture}

\begin{axis}[%
width=6.028in,
height=4.754in,
at={(1.011in,0.642in)},
scale only axis,
xmin=0,
xmax=256,
xlabel style={font=\fontsize{25}{20}\selectfont\color{black}, yshift = -10},
xlabel={Range Bin},
ymin=0,
ymax=3.2e7,
ylabel style={font=\fontsize{25}{20}\selectfont\color{black}, yshift=10pt},
ylabel={Power},
axis background/.style={fill=white},
tick label style={font=\fontsize{20}{11}\selectfont\color{black}},
xtick distance = 50,
ytick distance= 0.4e7,
yticklabel={\ifdim\tick pt=0pt\else\pgfmathprintnumber{\tick}\fi}, 
scaled y ticks=base 10:-6,
legend style={legend cell align=left, align = left, draw=white!15!black, font=\fontsize{12}{11}\selectfont\color{black}}
]
\addplot [color=mycolor1, mark=asterisk, mark options={solid, mycolor1}]
  table[row sep=crcr]{%
1	112.384180867256\\
2	109.646095672531\\
3	118.96014837706\\
4	117.476585339924\\
5	116.328459445872\\
6	119.590812685594\\
7	115.893499756131\\
8	115.566974288439\\
9	120.11042149688\\
10	116.355721037063\\
11	115.52723137554\\
12	117.94166842883\\
13	115.743761364082\\
14	118.61534285481\\
15	124.24091287285\\
16	120.410533623534\\
17	124.940065597057\\
18	120.78582215986\\
19	121.993867593179\\
20	121.485471708295\\
21	127.551187722895\\
22	115.485669033559\\
23	117.569018600096\\
24	121.363107117523\\
25	126.525377415312\\
26	123.464988317691\\
27	127.112626706793\\
28	129.656645781009\\
29	130.932548117804\\
30	127.732507141047\\
31	131.567497457568\\
32	129.628875568085\\
33	134.443311148419\\
34	138.531643165652\\
35	134.86556301557\\
36	140.156486458027\\
37	139.729812235289\\
38	142.853315029299\\
39	141.526222775008\\
40	145.522964425312\\
41	146.376966726069\\
42	149.739066958189\\
43	154.583843178154\\
44	154.279999529174\\
45	168.654049882331\\
46	160.328037871685\\
47	164.450155133889\\
48	172.082183620454\\
49	179.072454331936\\
50	193.609241056442\\
51	190.296272064597\\
52	210.205265241294\\
53	230.31368738078\\
54	254.806481217974\\
55	316.94876192211\\
56	443.13850384637\\
57	847.975002088884\\
58	3628.14033741503\\
59	305445.010924619\\
60	3834.60931201002\\
61	1871.12469087038\\
62	1738.03508754856\\
63	2402.28097941144\\
64	5353.36562056612\\
65	30925.7694131865\\
66	253066.286025928\\
67	14850.3169313156\\
68	6942.51213220488\\
69	5735.70593541\\
70	7030.7265972469\\
71	14337.0747264869\\
72	100830.016310087\\
73	159114.02971476\\
74	17852.3519903482\\
75	8766.6612194845\\
76	7271.11710204421\\
77	9062.74299842152\\
78	18928.0073055731\\
79	202975.136832428\\
80	68558.1097093962\\
81	10362.036699241\\
82	4963.52348990839\\
83	3900.98097330893\\
84	4733.90776210354\\
85	10552.3853866207\\
86	287849.907325369\\
87	17166.4172261406\\
88	3511.34730983991\\
89	1756.61983300659\\
90	1298.01750991083\\
91	1181.74230347154\\
92	1463.13209647588\\
93	316728.640188459\\
94	1689.35614773872\\
95	1407.86534971747\\
96	1610.30135494865\\
97	2095.26857224835\\
98	3495.39874616453\\
99	11737.3995678897\\
100	299881.719064479\\
101	7984.95649770175\\
102	5542.17861704606\\
103	7367.56080751046\\
104	13950.1336445028\\
105	42809.058713082\\
106	634366.041388168\\
107	542642.429332531\\
108	53133.1581384267\\
109	22228.7621872914\\
110	13983.9895918118\\
111	11717.2551907\\
112	15894.0003185278\\
113	114505.783834162\\
114	164589.521651497\\
115	34628.7018304537\\
116	23378.7488790959\\
117	21958.5186248066\\
118	25208.3901562399\\
119	39023.8803477676\\
120	282472.182899936\\
121	54348.6836591464\\
122	30848.8664511733\\
123	41892.7821321158\\
124	69523.3119804823\\
125	144160.210166654\\
126	487293.049193245\\
127	30038156.1274032\\
128	820818.254370371\\
129	167045.221894336\\
130	67741.003031599\\
131	35543.1528845318\\
132	21446.8526525029\\
133	14458.7436291741\\
134	308538.701948681\\
135	6604.58908160894\\
136	4412.32104494618\\
137	2925.83795992775\\
138	1850.04604804664\\
139	1490.18416139027\\
140	9601.54305993904\\
141	328652.377180025\\
142	20632.140493422\\
143	11910.8375883418\\
144	11573.1616076769\\
145	17369.3463392586\\
146	51452.3450508591\\
147	703555.333721765\\
148	463837.122025396\\
149	55635.746220012\\
150	25381.3831717177\\
151	17718.6346689952\\
152	16693.8534562988\\
153	25509.9766937737\\
154	177628.057631192\\
155	97240.4964219228\\
156	16171.8562338259\\
157	9188.56631208727\\
158	8141.22804213182\\
159	9699.84224915145\\
160	18795.3139812171\\
161	279413.344031632\\
162	25067.3685734415\\
163	4264.51097093514\\
164	2177.01504036963\\
165	1863.63188085631\\
166	2197.49994757637\\
167	4173.82151075018\\
168	328548.84917336\\
169	1476.73384533416\\
170	428.906427155345\\
171	399.760442006739\\
172	502.630536388609\\
173	799.126375972359\\
174	2767.58052524102\\
175	316703.854307559\\
176	4108.89203439033\\
177	2350.47225768184\\
178	2316.98798260378\\
179	3168.06036309996\\
180	6473.21604448099\\
181	32832.6433290396\\
182	250255.177791691\\
183	13668.6597455487\\
184	6265.39472308797\\
185	5409.68985306578\\
186	7104.6703499689\\
187	14972.9344600149\\
188	102876.579796335\\
189	152654.6532748\\
190	15532.7878583694\\
191	6812.74858759881\\
192	5393.56879407977\\
193	6915.07045832242\\
194	16085.249150074\\
195	191074.845226518\\
196	68655.1750950112\\
197	10244.1404019371\\
198	4142.80664206729\\
199	2297.54803506376\\
200	1513.84140276204\\
201	1103.1890305133\\
202	862.214896062761\\
203	701.33473886157\\
204	577.00115025757\\
205	498.481008375462\\
206	455.670337385459\\
207	409.580790920133\\
208	359.990902998113\\
209	339.342570688449\\
210	306.271701578066\\
211	289.731229798503\\
212	276.642848260872\\
213	255.156993688365\\
214	246.438974722412\\
215	236.961185525295\\
216	221.854195518649\\
217	212.795504602092\\
218	207.168251611579\\
219	190.324765342878\\
220	194.507279535255\\
221	182.89985207742\\
222	178.433869909364\\
223	177.923440254269\\
224	169.780156982847\\
225	159.06128273129\\
226	167.709984442818\\
227	162.991874229182\\
228	154.319784967706\\
229	155.395636252728\\
230	150.892587383896\\
231	151.441806680492\\
232	145.677632209489\\
233	138.151265116066\\
234	137.214904253155\\
235	138.340987568253\\
236	140.420608799394\\
237	134.653294669229\\
238	130.610315394055\\
239	133.388655426571\\
240	131.683167243577\\
241	127.53276628616\\
242	126.269548970122\\
243	126.62562808135\\
244	124.025205847785\\
245	121.935196285008\\
246	121.857655471215\\
247	121.174269499379\\
248	121.610796596689\\
249	113.781845816813\\
250	120.245190118711\\
251	117.493212799768\\
252	118.452834977736\\
253	121.915713819732\\
254	118.805291823395\\
255	116.304276418008\\
256	122.638451441016\\
};
\addlegendentry{Scatterer power}

\addplot [color=orange]
  table[row sep=crcr]{%
0	158815.395682837\\
300	158815.395682837\\
};
\addlegendentry{Average power of all scatterers}

\addplot [color=black, only marks, mark size=4.0pt, mark=o, mark options={solid, black}]
  table[row sep=crcr]{%
127	30038156.1274032\\
128	820818.254370371\\
};
\addlegendentry{Candidate scatterers}

\addplot [color=red, only marks, mark size=7.5pt, mark=o, mark options={solid, red}]
  table[row sep=crcr]
                \caption{Scatterer power, \\ sf = 5.\label{subfig:sf5_hayRA_power}}
            \end{subfigure}
             &
            \begin{subfigure}{0.33\linewidth}
                \centering
                \resizebox{\linewidth}{!}{% This file was created by matlab2tikz.
%
%The latest updates can be retrieved from
%  http://www.mathworks.com/matlabcentral/fileexchange/22022-matlab2tikz-matlab2tikz
%where you can also make suggestions and rate matlab2tikz.
%
\definecolor{mycolor1}{rgb}{0.00000,0.44700,0.74100}%
%
\begin{tikzpicture}

\begin{axis}[%
width=6.028in,
height=4.754in,
at={(1.011in,0.642in)},
scale only axis,
xmin=0,
xmax=256,
xlabel style={font=\fontsize{25}{20}\selectfont\color{black}, yshift = -10},
xlabel={Range Bin},
ymin=0,
ymax=3.2e7,
ylabel style={font=\fontsize{25}{20}\selectfont\color{black}, yshift=10pt},
ylabel={Power},
axis background/.style={fill=white},
tick label style={font=\fontsize{20}{11}\selectfont\color{black}},
xtick distance = 50,
ytick distance= 0.4e7,
yticklabel={\ifdim\tick pt=0pt\else\pgfmathprintnumber{\tick}\fi}, 
scaled y ticks=base 10:-6,
legend style={legend cell align=left, align = left, draw=white!15!black, font=\fontsize{12}{11}\selectfont\color{black}}
]
\addplot [color=mycolor1, mark=asterisk, mark options={solid, mycolor1}]
  table[row sep=crcr]{%
1	117.499215819905\\
2	109.452183396871\\
3	116.688807001582\\
4	117.850464362541\\
5	111.648687805284\\
6	113.604475119668\\
7	116.89936081232\\
8	112.690077268105\\
9	111.075540310801\\
10	112.626083944696\\
11	122.155029691455\\
12	118.57564539035\\
13	115.874608497764\\
14	115.481127421211\\
15	121.407792929361\\
16	116.676218867179\\
17	118.995159853473\\
18	116.745552432055\\
19	119.200447071831\\
20	118.119429686806\\
21	123.558177776983\\
22	120.475622928487\\
23	125.031134475474\\
24	123.787702857413\\
25	127.609861416255\\
26	123.848750930398\\
27	127.148293087938\\
28	129.672826478489\\
29	125.753649892965\\
30	126.670050410357\\
31	132.070124953308\\
32	133.532678567188\\
33	130.301306342073\\
34	132.740120429045\\
35	135.798494630415\\
36	138.312075910382\\
37	137.323263816713\\
38	137.529470786644\\
39	140.292716925932\\
40	141.561962726047\\
41	147.680230729074\\
42	150.296611213075\\
43	149.45896573526\\
44	158.060919893177\\
45	155.150402074804\\
46	164.779929702071\\
47	164.076507276055\\
48	177.601596762954\\
49	178.371014120795\\
50	192.089164076873\\
51	197.300049625019\\
52	209.648568745839\\
53	230.126759403158\\
54	258.620063031844\\
55	317.238418951354\\
56	444.29150414864\\
57	850.732032493171\\
58	3642.92163637339\\
59	305528.483662733\\
60	3836.80932353521\\
61	1865.00339814322\\
62	1734.36871314262\\
63	2396.8581714461\\
64	5316.74640446383\\
65	30851.7842760342\\
66	252692.538479883\\
67	14882.4974555992\\
68	6964.11389330921\\
69	5745.78387453363\\
70	7069.26058024175\\
71	14302.5148092283\\
72	100995.49791186\\
73	159065.058150147\\
74	17878.0840864278\\
75	8740.39091206631\\
76	7322.41271969411\\
77	8979.60619434493\\
78	18981.1130560332\\
79	203038.795143038\\
80	68490.2021470035\\
81	10449.7579457136\\
82	4919.38943507189\\
83	3895.54653411463\\
84	4745.32172152746\\
85	10582.0138883319\\
86	287848.458803882\\
87	17161.1092641971\\
88	3528.06938212158\\
89	1743.87001674925\\
90	1297.1323661588\\
91	1171.15372228377\\
92	1474.74174833109\\
93	316945.323516938\\
94	1667.02886930197\\
95	1395.00025252891\\
96	1601.80521571335\\
97	2117.14621268327\\
98	3494.07654035355\\
99	11757.2793159782\\
100	299703.043571489\\
101	8021.28618095248\\
102	5547.17418177414\\
103	7346.30475748707\\
104	13925.7150920751\\
105	42774.9971755195\\
106	634191.858030485\\
107	542725.508797422\\
108	53189.657773488\\
109	22205.4980180911\\
110	13952.9140203014\\
111	11711.7383183755\\
112	15908.5793827527\\
113	114305.059158511\\
114	164819.262348459\\
115	34646.8738334023\\
116	23352.9536107725\\
117	21969.0179489317\\
118	25249.6581753933\\
119	39045.7581301666\\
120	282553.577988441\\
121	54315.9662217526\\
122	30865.6559808542\\
123	42030.4435031708\\
124	69316.3902927898\\
125	143987.536155066\\
126	487462.6191564\\
127	30044821.4895024\\
128	820508.097139868\\
129	166993.631193098\\
130	67631.778573945\\
131	35546.3659865803\\
132	21387.833031899\\
133	14464.4957485799\\
134	308689.047368547\\
135	6552.98463098176\\
136	4406.32693333636\\
137	2933.40353261806\\
138	1858.28751214872\\
139	1487.79478239764\\
140	9639.3602458365\\
141	328795.666417714\\
142	20693.3876157122\\
143	11856.9401764144\\
144	11591.4423430495\\
145	17358.1528038433\\
146	51489.2637074647\\
147	703889.556500803\\
148	464178.18762035\\
149	55582.0802836749\\
150	25298.9370635604\\
151	17614.2351060367\\
152	16733.0877378934\\
153	25476.2595797351\\
154	177709.188075241\\
155	97359.8303451963\\
156	16161.3336682724\\
157	9212.80698000786\\
158	8106.76275843477\\
159	9728.67964190724\\
160	18855.3726587178\\
161	279081.335681176\\
162	25075.6791338693\\
163	4269.19642663549\\
164	2201.1544063865\\
165	1885.04161903422\\
166	2172.78774456214\\
167	4140.96067788563\\
168	328472.143913369\\
169	1466.01199477531\\
170	415.464577589578\\
171	400.583574038274\\
172	499.486535808379\\
173	802.351320210922\\
174	2753.31161239505\\
175	316581.898061586\\
176	4127.98651126506\\
177	2326.26517023398\\
178	2315.77682058677\\
179	3211.27539507015\\
180	6481.83205943924\\
181	32772.1233458468\\
182	250361.92617757\\
183	13698.6264637817\\
184	6319.36119011521\\
185	5430.48160143407\\
186	7055.82202741167\\
187	15004.7682934324\\
188	102653.472716972\\
189	152849.767866669\\
190	15641.6390392279\\
191	6816.40037892868\\
192	5392.59255641435\\
193	6898.73229688697\\
194	16035.3906622739\\
195	190932.717606222\\
196	68750.5079281721\\
197	10261.5279603538\\
198	4103.61819214632\\
199	2282.42741049261\\
200	1512.46939961437\\
201	1099.08365981174\\
202	843.160548117673\\
203	695.110023559796\\
204	588.489853803761\\
205	499.504006047961\\
206	441.425679296879\\
207	403.181563729502\\
208	361.034158023724\\
209	334.176322066068\\
210	305.367869745129\\
211	291.966102331308\\
212	273.328124511529\\
213	252.128259981682\\
214	235.048666756314\\
215	233.53238219235\\
216	224.88828956926\\
217	219.02087552917\\
218	208.30023713305\\
219	205.250531066147\\
220	188.318382445142\\
221	186.408888790137\\
222	182.776227678814\\
223	172.846608309904\\
224	171.620255136064\\
225	171.64518977118\\
226	161.830870456624\\
227	163.378144208676\\
228	153.530646388173\\
229	152.089999639877\\
230	151.148615921096\\
231	148.450959360795\\
232	147.832554714483\\
233	146.326164404022\\
234	142.403990975542\\
235	132.92290516021\\
236	140.812563240552\\
237	129.735220704638\\
238	130.890076131805\\
239	132.316486854219\\
240	130.083350357331\\
241	123.557101513838\\
242	124.776092734947\\
243	123.824139335093\\
244	123.094346167518\\
245	125.083018175668\\
246	118.774410097392\\
247	121.051792571916\\
248	124.230213884176\\
249	119.254134058232\\
250	117.320406878689\\
251	116.105240567513\\
252	122.541843637373\\
253	118.937956518744\\
254	121.825462571252\\
255	112.138472521564\\
256	120.084763410552\\
};
\addlegendentry{Scatterer power}

\addplot [color=orange]
  table[row sep=crcr]{%
0	158840.909659185\\
300	158840.909659185\\
};
\addlegendentry{Average power of all scatterers}

\addplot [color=black, only marks, mark size=4.0pt, mark=o, mark options={solid, black}]
  table[row sep=crcr]{%
127	30044821.4895024\\
};
\addlegendentry{Candidate scatterers}

\addplot [color=red, only marks, mark size=7.5pt, mark=o, mark options={solid, red}]
  table[row sep=crcr]
                \caption{Scatterer power, \\ sf = 10.\label{subfig:sf10_hayRA_power}}
            \end{subfigure}
            \\
            \begin{subfigure}{0.33\linewidth}
                \centering
                \resizebox{\linewidth}{!}{% This file was created by matlab2tikz.
%
%The latest updates can be retrieved from
%  http://www.mathworks.com/matlabcentral/fileexchange/22022-matlab2tikz-matlab2tikz
%where you can also make suggestions and rate matlab2tikz.
%
\definecolor{mycolor1}{rgb}{0.00000,0.44700,0.74100}%
%
\begin{tikzpicture}

\begin{axis}[%
width=6.028in,
height=4.754in,
at={(1.011in,0.642in)},
scale only axis,
xmin=0,
xmax=256,
xlabel style={font=\fontsize{25}{20}\selectfont\color{black}, yshift = -10},
xlabel={Range Bin},
ymin=0,
ymax=1e2,
ylabel style={font=\fontsize{25}{20}\selectfont\color{black}, yshift=10pt},
ylabel={Amplitude Variance},
axis background/.style={fill=white},
tick label style={font=\fontsize{20}{11}\selectfont\color{black}},
xtick distance = 50,
ytick distance=0.1e2,
yticklabel={\ifdim\tick pt=0pt\else\pgfmathprintnumber{\tick}\fi}, 
scaled y ticks=base 10:-2,
legend style={legend cell align=left, align=left, draw=white!15!black, font=\fontsize{12}{11}\selectfont\color{black}}
]
\addplot [color=mycolor1, mark=asterisk, mark options={solid, mycolor1}]
  table[row sep=crcr]{%
1	0.449619922853435\\
2	0.481110785976974\\
3	0.385394582911888\\
4	0.379666552553458\\
5	0.409137420881376\\
6	0.437153815190172\\
7	0.451722841745874\\
8	0.42602965752468\\
9	0.372666094541466\\
10	0.411367673016778\\
11	0.430143783510877\\
12	0.42624657960784\\
13	0.38993925559292\\
14	0.383004227922503\\
15	0.3983798544833\\
16	0.412055287064594\\
17	0.460823434503937\\
18	0.405372972701677\\
19	0.426686714779478\\
20	0.444604211656875\\
21	0.390156165174725\\
22	0.416454326168287\\
23	0.473823486082934\\
24	0.423155625567495\\
25	0.359442924141811\\
26	0.408148717308329\\
27	0.436196996735716\\
28	0.457527849812631\\
29	0.493164531943168\\
30	0.4469718886934\\
31	0.498470064054803\\
32	0.421460540660801\\
33	0.474241261149456\\
34	0.501165471523015\\
35	0.490142797619704\\
36	0.451629821773701\\
37	0.517326732429481\\
38	0.496816369229517\\
39	0.494858729731704\\
40	0.453288881053193\\
41	0.545233181106095\\
42	0.513337113706376\\
43	0.524276964499169\\
44	0.540426061001006\\
45	0.524764301806155\\
46	0.681080143810131\\
47	0.721483456157288\\
48	0.685737468711359\\
49	0.725535057450704\\
50	0.917634929995549\\
51	0.935412511731842\\
52	1.00968373258261\\
53	1.3371977453319\\
54	1.55093060631098\\
55	2.07900006000172\\
56	2.59371141609888\\
57	4.34009335781153\\
58	22.2623982103353\\
59	5.02598138851483\\
60	8.98257816944832\\
61	1.97521584496127\\
62	1.35122321577438\\
63	1.43195637319804\\
64	1.57798290400767\\
65	16.1157231206346\\
66	4.04831883121557\\
67	15.7578589366932\\
68	10.7865983854892\\
69	7.19162067523872\\
70	5.41874617901396\\
71	4.72055705643279\\
72	25.8876477583487\\
73	15.1467852528186\\
74	4.42662787010076\\
75	6.56648115642971\\
76	7.72074253852262\\
77	7.96800363267145\\
78	7.77319139735101\\
79	42.9161102360422\\
80	49.8102637907423\\
81	9.24928130808422\\
82	5.38499798420808\\
83	4.10082504835115\\
84	3.89942451188348\\
85	3.26832081592265\\
86	28.6425000011606\\
87	61.5898921636853\\
88	23.0839016563524\\
89	13.8446498841393\\
90	11.4684271686152\\
91	10.4923241071854\\
92	13.579767936329\\
93	8.29065919673217\\
94	10.3194676505929\\
95	9.28615592893561\\
96	12.6811969194835\\
97	18.3260344414289\\
98	28.7084659382146\\
99	67.8457668357607\\
100	47.7296392455518\\
101	29.9583009581226\\
102	34.6038769208263\\
103	63.2540229695598\\
104	134.414328172497\\
105	418.550314426209\\
106	6985.46833556888\\
107	4754.09564890965\\
108	475.979446230676\\
109	175.466683584198\\
110	91.9304647247226\\
111	55.6747908821353\\
112	37.9442065511878\\
113	25.6274809553571\\
114	24.9915513493056\\
115	15.3510449600068\\
116	13.9415290971257\\
117	13.8244610980751\\
118	15.6332815716521\\
119	19.2938819817886\\
120	9.22328990971146\\
121	6.85461687526847\\
122	6.66813370039337\\
123	12.3658222498621\\
124	23.0254791104528\\
125	49.0258049506294\\
126	154.637358638669\\
127	81.5629459485985\\
128	537.938441285318\\
129	101.893597595448\\
130	43.8554260955342\\
131	26.012677698694\\
132	18.5409733943694\\
133	14.1920745439885\\
134	17.2157633751598\\
135	7.74641020656389\\
136	9.55989824343519\\
137	10.1859409949487\\
138	9.905036903813\\
139	12.6856471311161\\
140	21.733814628614\\
141	25.4705946898793\\
142	35.6825565728771\\
143	46.9975009213978\\
144	58.3639759598234\\
145	113.344430945573\\
146	434.318951589842\\
147	8141.88214590772\\
148	4058.65723616665\\
149	465.134267281806\\
150	163.591401069376\\
151	80.3426958868306\\
152	50.087777743054\\
153	36.5246447993899\\
154	34.5935703428752\\
155	32.0979602914974\\
156	22.9003252451846\\
157	19.0417962134469\\
158	16.1564276410868\\
159	13.1353377982017\\
160	10.6077622757934\\
161	7.76391680557549\\
162	6.90777956951867\\
163	5.84000652370785\\
164	6.27512150575237\\
165	6.92638792966441\\
166	8.09725895506342\\
167	9.85442020398789\\
168	8.92521051531793\\
169	7.31647961530973\\
170	2.40145104819385\\
171	2.01819382807564\\
172	2.50001238163445\\
173	2.86681199781792\\
174	2.61156282799139\\
175	3.53090501344398\\
176	5.09028000391029\\
177	5.56857144109646\\
178	6.04809600037607\\
179	6.39954701023353\\
180	6.36450671712389\\
181	6.98232507810898\\
182	4.93142666360249\\
183	3.80610259435725\\
184	2.96336324103898\\
185	2.61151877935065\\
186	2.51742582562832\\
187	3.09045826766393\\
188	7.00288026993538\\
189	7.3694073462095\\
190	4.21849964624268\\
191	4.547525179902\\
192	4.28171617264383\\
193	3.92420859699435\\
194	3.44247671821396\\
195	4.07237383513797\\
196	5.18678789046567\\
197	2.56151231480738\\
198	2.04979935383964\\
199	1.89539831511395\\
200	1.84911928648864\\
201	1.67719923480703\\
202	1.71398572068706\\
203	1.61334347308954\\
204	1.59368188416651\\
205	1.58853269356666\\
206	1.59600425846128\\
207	1.45152966721433\\
208	1.41432071235696\\
209	1.46306959346201\\
210	1.35874386450497\\
211	1.36792989144283\\
212	1.37119969757979\\
213	1.25368417462743\\
214	1.24295801051789\\
215	1.16669569994282\\
216	1.14614710407645\\
217	1.07684966493004\\
218	0.997390566512458\\
219	1.00985705627041\\
220	0.819027299891869\\
221	0.86509921948288\\
222	0.86492442148911\\
223	0.770777527116273\\
224	0.871400360876503\\
225	0.821604714184382\\
226	0.728175460809683\\
227	0.753318291786417\\
228	0.710690435709549\\
229	0.694825585504109\\
230	0.718962051414611\\
231	0.621956444066163\\
232	0.650248307744268\\
233	0.625581062895177\\
234	0.594967338491463\\
235	0.608506036117426\\
236	0.563719189946638\\
237	0.494100590336812\\
238	0.512058651115563\\
239	0.538309136808476\\
240	0.533466610256056\\
241	0.541904978164673\\
242	0.536779534405504\\
243	0.518025772304129\\
244	0.48688984201892\\
245	0.464807208514926\\
246	0.450699898490248\\
247	0.50965116136375\\
248	0.501275346102917\\
249	0.431269513309751\\
250	0.414208307329412\\
251	0.456817336060311\\
252	0.430450708596033\\
253	0.467613972652527\\
254	0.432318110454631\\
255	0.461472397969142\\
256	0.475966847299479\\
};
\addlegendentry{Scatterer variance}

\addplot [color=black, only marks, mark size=4.0pt, mark=o, mark options={solid, black}]
  table[row sep=crcr]{%
59	5.02598138851483\\
66	4.04831883121557\\
73	15.1467852528186\\
79	42.9161102360422\\
86	28.6425000011606\\
93	8.29065919673217\\
100	47.7296392455518\\
106	6985.46833556888\\
107	4754.09564890965\\
114	24.9915513493056\\
120	9.22328990971146\\
126	154.637358638669\\
127	81.5629459485985\\
128	537.938441285318\\
129	101.893597595448\\
134	17.2157633751598\\
141	25.4705946898793\\
147	8141.88214590772\\
148	4058.65723616665\\
154	34.5935703428752\\
161	7.76391680557549\\
168	8.92521051531793\\
175	3.53090501344398\\
182	4.93142666360249\\
195	4.07237383513797\\
};
\addlegendentry{Candidate scatterers}

\addplot [color=red]
  table[row sep=crcr]{%
0	3.53090501344398\\
300	3.53090501344398\\
};
\addlegendentry{Minimum variance}

\addplot [color=red, only marks, mark size=4.0pt, mark=*, mark options={solid, fill=red, red}]
  table[row sep=crcr]
                \caption{Scatterer amplitude variance, \\ sf = 1.\label{subfig:sf1_hayRA_var}}
            \end{subfigure}
             &
            \begin{subfigure}{0.33\linewidth}
                \centering
                \resizebox{\linewidth}{!}{% This file was created by matlab2tikz.
%
%The latest updates can be retrieved from
%  http://www.mathworks.com/matlabcentral/fileexchange/22022-matlab2tikz-matlab2tikz
%where you can also make suggestions and rate matlab2tikz.
%
\definecolor{mycolor1}{rgb}{0.00000,0.44700,0.74100}%
%
\begin{tikzpicture}

\begin{axis}[%
width=6.028in,
height=4.754in,
at={(1.011in,0.642in)},
scale only axis,
xmin=0,
xmax=256,
xlabel style={font=\fontsize{25}{20}\selectfont\color{black}, yshift = -10},
xlabel={Range Bin},
ymin=0,
ymax=2e2,
ylabel style={font=\fontsize{25}{20}\selectfont\color{black}, yshift=10pt},
ylabel={Amplitude Variance},
axis background/.style={fill=white},
tick label style={font=\fontsize{20}{11}\selectfont\color{black}},
xtick distance = 50,
ytick distance=0.2e2,
yticklabel={\ifdim\tick pt=0pt\else\pgfmathprintnumber{\tick}\fi}, 
scaled y ticks=base 10:-2,
legend style={legend cell align=left, align=left, draw=white!15!black, font=\fontsize{12}{11}\selectfont\color{black}}
]
\addplot [color=mycolor1, mark=asterisk, mark options={solid, mycolor1}]
  table[row sep=crcr]{%
1	0.44900931055432\\
2	0.436041456036243\\
3	0.465632272017068\\
4	0.396985808322752\\
5	0.456199788347306\\
6	0.408288679744467\\
7	0.447974820519696\\
8	0.410898887403355\\
9	0.413056968454751\\
10	0.42684181226682\\
11	0.445667234957617\\
12	0.476602239060736\\
13	0.450809661607939\\
14	0.390895296003263\\
15	0.397183211516423\\
16	0.420502262257264\\
17	0.452156522995739\\
18	0.353188987973899\\
19	0.369577131883001\\
20	0.385937298222227\\
21	0.415852762885712\\
22	0.383618820598718\\
23	0.454358515530079\\
24	0.404221027707768\\
25	0.449983994212423\\
26	0.403049348744987\\
27	0.423198682371225\\
28	0.465633820209481\\
29	0.451387895083736\\
30	0.427344585754942\\
31	0.419813197827116\\
32	0.430755280891373\\
33	0.440873463265558\\
34	0.518656483618094\\
35	0.476044292085455\\
36	0.450415774614515\\
37	0.452052048436982\\
38	0.499422576352221\\
39	0.484150785933179\\
40	0.582645920224029\\
41	0.539940057797545\\
42	0.537031254564031\\
43	0.546219523230876\\
44	0.55125301256819\\
45	0.584764710809205\\
46	0.650996384389922\\
47	0.630443667696604\\
48	0.738776783504571\\
49	0.856888412604821\\
50	0.900892585531352\\
51	0.881617342150204\\
52	1.08579395726925\\
53	1.27779912133498\\
54	1.48059909322243\\
55	2.09587357857007\\
56	2.65623559156292\\
57	4.48957541823802\\
58	21.9735060100956\\
59	4.9774809573459\\
60	9.2357480292317\\
61	1.88366211646477\\
62	1.39641701731969\\
63	1.48304119663555\\
64	1.48965944238381\\
65	16.1647166926563\\
66	4.03881225389144\\
67	15.8763327956806\\
68	10.6744996580139\\
69	7.54403874780402\\
70	5.24478380564569\\
71	4.46376220306432\\
72	25.9090617217363\\
73	15.1434992206028\\
74	4.69197932495333\\
75	6.64407099410066\\
76	7.36747809445858\\
77	8.13329663348981\\
78	7.7731687052711\\
79	42.8120010950467\\
80	49.7553512258146\\
81	9.14019897945562\\
82	5.37604252425413\\
83	4.12537115184818\\
84	3.95353413864267\\
85	3.41698775948459\\
86	29.2820014438772\\
87	61.3659091268964\\
88	23.3668267700285\\
89	14.1724537843404\\
90	11.4661705594384\\
91	10.6862861306953\\
92	13.3140352091037\\
93	8.32146647094473\\
94	10.0499935846847\\
95	9.47942927253125\\
96	12.2577520289763\\
97	18.1989923871519\\
98	28.9333494831153\\
99	68.0771066154795\\
100	47.6729954804013\\
101	29.9084096946238\\
102	35.1684031174008\\
103	63.3145047454115\\
104	135.284444349318\\
105	419.182723633642\\
106	6988.66213785547\\
107	4757.18702920067\\
108	475.815019968028\\
109	176.503337435434\\
110	91.9937203204833\\
111	55.6637177790953\\
112	37.8817096782203\\
113	26.2443873072798\\
114	24.2326158395039\\
115	15.3898538981368\\
116	13.9545186552552\\
117	13.7891888337596\\
118	15.1327985770145\\
119	19.2494559813735\\
120	9.27499864770849\\
121	6.65295205465672\\
122	6.98432754256794\\
123	12.5835258458294\\
124	23.2229462019141\\
125	49.4635883577314\\
126	155.045941530134\\
127	81.5895866496835\\
128	541.585797095946\\
129	102.470146478246\\
130	44.2119211898594\\
131	26.2681923989068\\
132	18.5576654014258\\
133	13.850780703139\\
134	16.8126002554989\\
135	8.01778437179774\\
136	9.61082657491525\\
137	10.4283540250606\\
138	9.95611841165008\\
139	12.8475581205412\\
140	22.0081794968488\\
141	25.358810651843\\
142	35.219020042154\\
143	46.599639721497\\
144	58.2831946631383\\
145	114.174159930919\\
146	434.013938327232\\
147	8159.88094469442\\
148	4055.76914848899\\
149	466.072846041826\\
150	163.275712822888\\
151	81.5942786566586\\
152	49.7485695453791\\
153	36.3408514465474\\
154	34.6677078542963\\
155	32.2229466128939\\
156	22.6400999509155\\
157	19.2288793555626\\
158	16.4430778799657\\
159	13.2318882118187\\
160	10.6245382284014\\
161	7.98870070309908\\
162	7.29332544527541\\
163	5.96874481471709\\
164	6.31010308205933\\
165	6.78657693980213\\
166	8.32290491894724\\
167	9.85437508059463\\
168	9.13256989074342\\
169	7.25430159689636\\
170	2.36443883756402\\
171	1.85757964641246\\
172	2.37564200324939\\
173	2.72715046638453\\
174	2.72627482844733\\
175	3.42320658270298\\
176	5.04838023547733\\
177	5.43867838293783\\
178	6.21922189202121\\
179	6.38832900074643\\
180	6.47737429991645\\
181	6.96334310487449\\
182	4.94126393383455\\
183	3.89107234889467\\
184	2.94606309678934\\
185	2.68589474184827\\
186	2.5733757590112\\
187	3.10567185890695\\
188	7.19555040291783\\
189	7.24580673962364\\
190	4.31380662954442\\
191	4.51582010341658\\
192	4.39111166345996\\
193	3.96273078944899\\
194	3.46741578477716\\
195	4.26406526359741\\
196	5.21700206990731\\
197	2.55914082466888\\
198	2.13734397800902\\
199	1.9469969032401\\
200	1.79198219572247\\
201	1.72136094801451\\
202	1.83982769161674\\
203	1.62127479973329\\
204	1.61940441100783\\
205	1.55243449924224\\
206	1.55362047467632\\
207	1.62166390001514\\
208	1.46757199594179\\
209	1.47151623370016\\
210	1.37091118063324\\
211	1.29257584337156\\
212	1.24943820568138\\
213	1.18774775429587\\
214	1.1136701672023\\
215	1.12595403258517\\
216	1.13370847528038\\
217	1.0239812541552\\
218	1.08912265285577\\
219	0.9305586094237\\
220	0.955068913011187\\
221	0.920490695683886\\
222	0.926621939322207\\
223	0.883352181768476\\
224	0.735594901556826\\
225	0.752130080692056\\
226	0.803490429853109\\
227	0.779338673759717\\
228	0.713022876016991\\
229	0.673442613665185\\
230	0.698685124855805\\
231	0.662740158275463\\
232	0.606159782472205\\
233	0.559796130085763\\
234	0.557615820283377\\
235	0.561906871284714\\
236	0.587046903815644\\
237	0.581964797155498\\
238	0.522851608278627\\
239	0.548299375962017\\
240	0.550194933831393\\
241	0.486195106687525\\
242	0.459665016370333\\
243	0.592541806137089\\
244	0.492640799283466\\
245	0.52393705259441\\
246	0.492996516900786\\
247	0.451886204166434\\
248	0.469489216293621\\
249	0.412531700959713\\
250	0.492834357886516\\
251	0.463276365631743\\
252	0.394728603249829\\
253	0.455049155547126\\
254	0.423704437316671\\
255	0.442084093692906\\
256	0.440526362310134\\
};
\addlegendentry{Scatterer variance}

\addplot [color=black, only marks, mark size=4.0pt, mark=o, mark options={solid, black}]
  table[row sep=crcr]{%
127	81.5895866496835\\
128	541.585797095946\\
};
\addlegendentry{Candidate scatterers}

\addplot [color=red]
  table[row sep=crcr]{%
0	81.5895866496835\\
300	81.5895866496835\\
};
\addlegendentry{Minimum variance}

\addplot [color=red, only marks, mark size=4.0pt, mark=*, mark options={solid, fill=red, red}]
  table[row sep=crcr]
                \caption{Scatterer amplitude variance, \\ sf = 5.\label{subfig:sf5_hayRA_var}}
            \end{subfigure}
             &
            \begin{subfigure}{0.33\linewidth}
                \centering
                \resizebox{\linewidth}{!}{% This file was created by matlab2tikz.
%
%The latest updates can be retrieved from
%  http://www.mathworks.com/matlabcentral/fileexchange/22022-matlab2tikz-matlab2tikz
%where you can also make suggestions and rate matlab2tikz.
%
\definecolor{mycolor1}{rgb}{0.00000,0.44700,0.74100}%
%
\begin{tikzpicture}

\begin{axis}[%
width=6.028in,
height=4.754in,
at={(1.011in,0.642in)},
scale only axis,
xmin=0,
xmax=256,
xlabel style={font=\fontsize{25}{20}\selectfont\color{black}, yshift = -10},
xlabel={Range Bin},
ymin=0,
ymax=9e3,
ylabel style={font=\fontsize{25}{20}\selectfont\color{black}, yshift=10pt},
ylabel={Amplitude Variance},
axis background/.style={fill=white},
tick label style={font=\fontsize{20}{11}\selectfont\color{black}},
xtick distance = 50,
ytick distance=1e3,
yticklabel={\ifdim\tick pt=0pt\else\pgfmathprintnumber{\tick}\fi}, 
scaled y ticks=base 10:-2,
legend style={legend cell align=left, align=left, draw=white!15!black, font=\fontsize{12}{11}\selectfont\color{black}}
]
\addplot [color=mycolor1, mark=asterisk, mark options={solid, mycolor1}]
  table[row sep=crcr]{%
1	0.455985462679299\\
2	0.381056645991907\\
3	0.420581882654458\\
4	0.44166597317531\\
5	0.435541938455742\\
6	0.373725208222093\\
7	0.422305368008936\\
8	0.38301228069887\\
9	0.415773283872783\\
10	0.39232193407367\\
11	0.458784192619873\\
12	0.443406531824308\\
13	0.381885880462338\\
14	0.380015837307499\\
15	0.418231409143131\\
16	0.412525086031274\\
17	0.425182080575272\\
18	0.470986430357996\\
19	0.440228683206715\\
20	0.38646328803057\\
21	0.447455564577623\\
22	0.444288115283693\\
23	0.423001455217548\\
24	0.419542594024307\\
25	0.451302626814493\\
26	0.39124586465566\\
27	0.404941859393083\\
28	0.470643047778305\\
29	0.430928737921285\\
30	0.455625946290659\\
31	0.432725896127507\\
32	0.471562241434301\\
33	0.460304274134263\\
34	0.400714488283232\\
35	0.438105051148075\\
36	0.427339003815915\\
37	0.474430526650319\\
38	0.462326026480975\\
39	0.422974672263266\\
40	0.533470932824327\\
41	0.542281148430601\\
42	0.544835818680682\\
43	0.489760978636219\\
44	0.617238661072358\\
45	0.644865011538935\\
46	0.709039704109636\\
47	0.68392101552619\\
48	0.786012980220423\\
49	0.836879933216266\\
50	0.819385188470993\\
51	0.955781060064419\\
52	1.17141376270871\\
53	1.28819816598056\\
54	1.55512637259975\\
55	2.16167814635204\\
56	2.68661326376782\\
57	4.46644443801499\\
58	22.0993651845424\\
59	5.25950193040588\\
60	9.17960531887785\\
61	1.79402116837256\\
62	1.37870555050168\\
63	1.44390132144379\\
64	1.39420853142286\\
65	15.8266285637654\\
66	4.14139326968357\\
67	15.9492208593619\\
68	10.6263123474289\\
69	7.44162151726017\\
70	5.47493839787214\\
71	4.45944131032319\\
72	25.9710338383818\\
73	14.8433982862096\\
74	4.57607267706007\\
75	6.66294813589298\\
76	7.79261355120579\\
77	7.99653465666674\\
78	7.64693354390838\\
79	42.8880780913441\\
80	49.8426693416952\\
81	9.17227132024363\\
82	5.43026013662168\\
83	4.21461333848318\\
84	3.98569464744449\\
85	3.20617394971219\\
86	29.3830229791602\\
87	61.1462649179744\\
88	23.3983182888545\\
89	14.2291217209046\\
90	11.3726454765982\\
91	10.3916543494212\\
92	13.4148497733812\\
93	8.06895424658861\\
94	9.70961956382287\\
95	9.37542055672721\\
96	12.2292074524896\\
97	18.4497581432663\\
98	29.140900609961\\
99	68.0794512884535\\
100	47.874720385712\\
101	30.7643361455385\\
102	35.1907045354007\\
103	62.7997877216203\\
104	134.818415097663\\
105	418.416116194853\\
106	6986.86186079429\\
107	4760.89760293411\\
108	474.525107279455\\
109	175.446860274804\\
110	91.8138116298956\\
111	55.821041105619\\
112	37.8136311065773\\
113	25.7523864804623\\
114	25.1758685117181\\
115	14.9424186504098\\
116	13.84023630976\\
117	14.0691047158676\\
118	15.3333787851471\\
119	19.5190963029825\\
120	8.92302567152681\\
121	6.7098195916361\\
122	6.83043224576022\\
123	12.4362093686044\\
124	22.3298524658741\\
125	48.7145056062823\\
126	153.719644433449\\
127	81.4882570210211\\
128	540.348376971546\\
129	102.059557220002\\
130	44.1090888629418\\
131	26.0873137293295\\
132	18.3416239395119\\
133	14.2452654970091\\
134	17.0512514626988\\
135	8.09092375224339\\
136	9.4666299867652\\
137	10.2836024833313\\
138	9.97957048679012\\
139	12.7133092375229\\
140	22.0494878094681\\
141	25.1816520759217\\
142	35.2599854496361\\
143	46.4175321764276\\
144	57.9849245929363\\
145	113.917064668035\\
146	433.37176299299\\
147	8156.9353484282\\
148	4058.14557272704\\
149	466.129386879217\\
150	162.394642000735\\
151	80.3357593340501\\
152	49.4280845950567\\
153	36.0425159227757\\
154	34.3475211154418\\
155	32.1767570825923\\
156	22.6029315844126\\
157	19.4414970072999\\
158	16.3124333519496\\
159	12.8804140919076\\
160	10.7638223587425\\
161	7.75856640229135\\
162	7.06142902418062\\
163	5.6141018607497\\
164	6.05505700478032\\
165	7.03644665433374\\
166	8.17203364013984\\
167	9.65946004602889\\
168	8.82489333743635\\
169	7.35108684017615\\
170	2.37521443473917\\
171	1.97181515175502\\
172	2.40671212434151\\
173	2.8091477761205\\
174	2.69510068666162\\
175	3.55671159441462\\
176	5.30236418380031\\
177	5.59462462611498\\
178	6.07056477923819\\
179	6.4048513117819\\
180	6.55919635361345\\
181	7.19184236286139\\
182	4.99560079653304\\
183	3.75963642147869\\
184	2.85327937162033\\
185	2.57323774338777\\
186	2.64948028891727\\
187	3.18571110701486\\
188	7.13943952020963\\
189	7.22741264170573\\
190	4.15998050319195\\
191	4.60843331929661\\
192	4.67001612767906\\
193	3.95611137961407\\
194	3.28442728013512\\
195	3.99801618426175\\
196	5.0848198416868\\
197	2.35314792036286\\
198	2.08726313706425\\
199	2.00148836128271\\
200	1.80426811789392\\
201	1.78126664657046\\
202	1.8648884514783\\
203	1.62767469270041\\
204	1.57002983629958\\
205	1.60926593368071\\
206	1.54007348157328\\
207	1.50864268192935\\
208	1.44371056221378\\
209	1.54003892804563\\
210	1.34822883288583\\
211	1.42903252906402\\
212	1.26094905484008\\
213	1.23092787329929\\
214	1.1425378733755\\
215	1.10597592216104\\
216	1.19566824465143\\
217	1.06582990418016\\
218	1.0911786469299\\
219	0.99029615795318\\
220	0.955693297860266\\
221	0.889239900991612\\
222	0.937424416715297\\
223	0.836844530419934\\
224	0.801458747665042\\
225	0.832349310616861\\
226	0.71354146459004\\
227	0.743269503459431\\
228	0.701985572077921\\
229	0.704186618953203\\
230	0.661539112034295\\
231	0.638422176750625\\
232	0.67962230086306\\
233	0.660687746987823\\
234	0.593728733432219\\
235	0.572484126166595\\
236	0.609787701031356\\
237	0.547521080328256\\
238	0.541929885854691\\
239	0.580195056661067\\
240	0.470780155192148\\
241	0.479844063322537\\
242	0.53483899088898\\
243	0.482887435761468\\
244	0.543344914694401\\
245	0.433562171364674\\
246	0.471855624735492\\
247	0.471037583888976\\
248	0.469879050462652\\
249	0.463433948497225\\
250	0.500360882088223\\
251	0.482331396857543\\
252	0.479758328396254\\
253	0.43894233042245\\
254	0.424244521374617\\
255	0.396884662777234\\
256	0.384260160862576\\
};
\addlegendentry{Scatterer variance}

\addplot [color=black, only marks, mark size=4.0pt, mark=o, mark options={solid, black}]
  table[row sep=crcr]{%
127	81.4882570210211\\
};
\addlegendentry{Candidate scatterers}

\addplot [color=red]
  table[row sep=crcr]{%
0	81.4882570210211\\
300	81.4882570210211\\
};
\addlegendentry{Minimum variance}

\addplot [color=red, only marks, mark size=4.0pt, mark=*, mark options={solid, fill=red, red}]
  table[row sep=crcr]
                \caption{Scatterer amplitude variance, \\ sf = 10.\label{subfig:sf10_hayRA_var}}
            \end{subfigure}
        \end{tabular}
        \caption{\gls{ds} selection and autofocused \gls{isar} images for different scaling factors using Haywood \gls{ra} \gls{hrr} profiles.\label{fig:sf_hayRA}}
    \end{minipage}
    \end{figure}

    % Grid of the HRRP and ISAR images HaywoodRA
    \begin{figure}[H]
    \centering
    \begin{minipage}{0.98\linewidth}
        \begin{tabular}{@{}ccc@{}}
            \begin{subfigure}{0.33\linewidth}
                \centering
                \resizebox{\linewidth}{!}{% This file was created by matlab2tikz.
%
%The latest updates can be retrieved from
%  http://www.mathworks.com/matlabcentral/fileexchange/22022-matlab2tikz-matlab2tikz
%where you can also make suggestions and rate matlab2tikz.
%
\begin{tikzpicture}
\begin{axis}[%
width=5.554in,
height=4.754in,
at={(0.932in,0.642in)},
scale only axis,
point meta min=-40,
point meta max=0,
axis on top,
xmin=0,
xmax=29.6578125,
xlabel style={font=\fontsize{25}{14}\selectfont\color{black}, yshift=-10pt},
xlabel={Range (m)},
ymin=-88.0034,
ymax=86.6175,
ylabel style={font=\fontsize{25}{14}\selectfont\color{black}},
ylabel={Doppler frquency (Hz)},
axis background/.style={fill=white},
tick label style={font=\fontsize{20}{11}\selectfont\color{black}},
xtick distance= 4,             % Set the spacing between x-axis ticks
ytick distance = 10,
colormap/jet,
colorbar
]
\addplot [forget plot] graphics [xmin=0, xmax=29.6578125, ymin=-88.0034, ymax=86.6175] {Figures/09Appendix/ScalingFactor/Measured/HayRA/ISAR/HayAF_HayRA_Measured_ISAR_SF1.png};
\end{axis}

\end{tikzpicture}%}
                \caption{Autofocused \gls{isar} image, \\ sf = 1.\label{subfig:sf1_hayRA_isar}}
            \end{subfigure}
            &
            \begin{subfigure}{0.33\linewidth}
                \centering
                \resizebox{\linewidth}{!}{% This file was created by matlab2tikz.
%
%The latest updates can be retrieved from
%  http://www.mathworks.com/matlabcentral/fileexchange/22022-matlab2tikz-matlab2tikz
%where you can also make suggestions and rate matlab2tikz.
%
\begin{tikzpicture}
\begin{axis}[%
width=5.554in,
height=4.754in,
at={(0.932in,0.642in)},
scale only axis,
point meta min=-40,
point meta max=0,
axis on top,
xmin=0,
xmax=29.6578125,
xlabel style={font=\fontsize{25}{14}\selectfont\color{black}, yshift=-10pt},
xlabel={Range (m)},
ymin=-88.0034,
ymax=86.6175,
ylabel style={font=\fontsize{25}{14}\selectfont\color{black}},
ylabel={Doppler frquency (Hz)},
axis background/.style={fill=white},
tick label style={font=\fontsize{20}{11}\selectfont\color{black}},
xtick distance= 4,             % Set the spacing between x-axis ticks
ytick distance = 10,
colormap/jet,
colorbar
]
\addplot [forget plot] graphics [xmin=0, xmax=29.6578125, ymin=-88.0034, ymax=86.6175] {Figures/09Appendix/ScalingFactor/Measured/HayRA/ISAR/HayAF_HayRA_Measured_ISAR_SF5.png};
\end{axis}

\end{tikzpicture}%}
                \caption{Autofocused \gls{isar} image, \\ sf = 5.\label{subfig:sf5_hayRA_isar}}
            \end{subfigure}
             &
            \begin{subfigure}{0.33\linewidth}
                \centering
                \resizebox{\linewidth}{!}{% This file was created by matlab2tikz.
%
%The latest updates can be retrieved from
%  http://www.mathworks.com/matlabcentral/fileexchange/22022-matlab2tikz-matlab2tikz
%where you can also make suggestions and rate matlab2tikz.
%
\begin{tikzpicture}
\begin{axis}[%
width=5.554in,
height=4.754in,
at={(0.932in,0.642in)},
scale only axis,
point meta min=-40,
point meta max=0,
axis on top,
xmin=0,
xmax=29.6578125,
xlabel style={font=\fontsize{25}{14}\selectfont\color{black}, yshift=-10pt},
xlabel={Range (m)},
ymin=-88.0034,
ymax=86.6175,
ylabel style={font=\fontsize{25}{14}\selectfont\color{black}},
ylabel={Doppler frquency (Hz)},
axis background/.style={fill=white},
tick label style={font=\fontsize{20}{11}\selectfont\color{black}},
xtick distance= 4,             % Set the spacing between x-axis ticks
ytick distance = 10,
colormap/jet,
colorbar
]
\addplot [forget plot] graphics [xmin=0, xmax=29.6578125, ymin=-88.0034, ymax=86.6175] {Figures/09Appendix/ScalingFactor/Measured/HayRA/ISAR/HayAF_HayRA_Measured_ISAR_SF10.png};
\end{axis}

\end{tikzpicture}%}
                \caption{Autofocused \gls{isar} image, \\ sf = 10.\label{subfig:sf10_hayRA_isar}}
            \end{subfigure}
            \\
            \begin{subfigure}{0.33\linewidth}
                \centering
                \resizebox{\linewidth}{!}{% This file was created by matlab2tikz.
%
%The latest updates can be retrieved from
%  http://www.mathworks.com/matlabcentral/fileexchange/22022-matlab2tikz-matlab2tikz
%where you can also make suggestions and rate matlab2tikz.
%
\definecolor{mycolor1}{rgb}{0.00000,0.44700,0.74100}%
%
\begin{tikzpicture}

\begin{axis}[%
width=6.028in,
height=4.754in,
at={(1.011in,0.642in)},
scale only axis,
xmin=0,
xmax=100,
xlabel style={font=\color{white!15!black}},
xlabel={Range Bin},
ymin=0,
ymax=500000000,
ylabel style={font=\color{white!15!black}},
ylabel={Power},
axis background/.style={fill=white},
title style={font=\bfseries},
title={Power of Scatterers},
legend style={legend cell align=left, align=left, draw=white!15!black}
]
\addplot [color=mycolor1, mark=asterisk, mark options={solid, mycolor1}]
  table[row sep=crcr]{%
1	899314.404068411\\
2	438057.505857389\\
3	244428.562207619\\
4	233002.309673522\\
5	565720.811386125\\
6	783684.466779614\\
7	254091.009627657\\
8	156368.189566121\\
9	148146.705653507\\
10	281413.751084289\\
11	383730.418945787\\
12	196530.975831858\\
13	89292.8417771284\\
14	115635.208698957\\
15	417252.21204137\\
16	302600.369524329\\
17	111318.901917494\\
18	164197.23344413\\
19	377897.043803359\\
20	785895.998801982\\
21	338583.472493692\\
22	225086.314760605\\
23	156838.839306662\\
24	193956.17961893\\
25	424105.359706039\\
26	181231.173237443\\
27	238092.339293839\\
28	179399.676812502\\
29	593732.662732542\\
30	1385343.20503296\\
31	3125053.10558304\\
32	11191889.3670391\\
33	46850606.3972826\\
34	50205923.7854111\\
35	113080735.273198\\
36	93240889.5574682\\
37	16660873.281384\\
38	25477272.1580532\\
39	20516586.4616072\\
40	10224035.8783329\\
41	12557490.1417185\\
42	15383349.1433136\\
43	113983311.955226\\
44	454318358.861187\\
45	138190424.323182\\
46	12438136.4192108\\
47	10397235.8574198\\
48	8952391.01364979\\
49	6206733.60226623\\
50	3900633.43848734\\
51	6498252.21427081\\
52	6341825.12954421\\
53	8409185.91235206\\
54	5495141.4442484\\
55	5467902.16557234\\
56	10349241.7922091\\
57	37937733.3997456\\
58	225088690.251356\\
59	261923567.957029\\
60	98477547.6624893\\
61	92500543.2872047\\
62	17690963.3767161\\
63	17757995.1709214\\
64	36154686.3218921\\
65	33876953.5995125\\
66	15931631.3216887\\
67	5186249.94500254\\
68	28881508.0192419\\
69	29405710.5292967\\
70	16288840.2066237\\
71	7098685.17002216\\
72	3737461.24139774\\
73	2317474.38181273\\
74	804519.810824472\\
75	643352.180399639\\
76	565613.886998036\\
77	526432.540866271\\
78	901382.029084835\\
79	528543.154192613\\
80	232375.137388086\\
81	165277.951355617\\
82	670305.015011577\\
83	699917.707440526\\
84	246156.041849285\\
85	180491.797831531\\
86	279886.117170204\\
87	433195.538582795\\
88	378509.94589131\\
89	159924.414152285\\
90	111443.309688548\\
91	222919.518797925\\
92	469580.646791455\\
93	371708.880322223\\
94	228900.902057002\\
95	192218.772375071\\
96	382263.270597065\\
};
\addlegendentry{Scatterer power}

\addplot [color=green]
  table[row sep=crcr]{%
0	22499787.3670263\\
100	22499787.3670263\\
};
\addlegendentry{Average power of all scatterers}

\addplot [color=black, only marks, mark size=4.0pt, mark=o, mark options={solid, black}]
  table[row sep=crcr]{%
33	46850606.3972826\\
34	50205923.7854111\\
35	113080735.273198\\
36	93240889.5574682\\
38	25477272.1580532\\
43	113983311.955226\\
44	454318358.861187\\
45	138190424.323182\\
57	37937733.3997456\\
58	225088690.251356\\
59	261923567.957029\\
60	98477547.6624893\\
61	92500543.2872047\\
64	36154686.3218921\\
65	33876953.5995125\\
68	28881508.0192419\\
69	29405710.5292967\\
};
\addlegendentry{Candidate scatterers}

\addplot [color=red, only marks, mark size=7.5pt, mark=o, mark options={solid, red}]
  table[row sep=crcr]
                \caption{Scatterer power, \\ sf = 1.\label{subfig:sf1_hayRA_power}}
            \end{subfigure}
             &
            \begin{subfigure}{0.33\linewidth}
                \centering
                \resizebox{\linewidth}{!}{% This file was created by matlab2tikz.
%
%The latest updates can be retrieved from
%  http://www.mathworks.com/matlabcentral/fileexchange/22022-matlab2tikz-matlab2tikz
%where you can also make suggestions and rate matlab2tikz.
%
\definecolor{mycolor1}{rgb}{0.00000,0.44700,0.74100}%
%
\begin{tikzpicture}

\begin{axis}[%
width=6.028in,
height=4.754in,
at={(1.011in,0.642in)},
scale only axis,
xmin=0,
xmax=100,
xlabel style={font=\color{white!15!black}},
xlabel={Range Bin},
ymin=0,
ymax=500000000,
ylabel style={font=\color{white!15!black}},
ylabel={Power},
axis background/.style={fill=white},
title style={font=\bfseries},
title={Power of Scatterers},
legend style={legend cell align=left, align=left, draw=white!15!black}
]
\addplot [color=mycolor1, mark=asterisk, mark options={solid, mycolor1}]
  table[row sep=crcr]{%
1	899314.404068411\\
2	438057.505857389\\
3	244428.562207619\\
4	233002.309673522\\
5	565720.811386125\\
6	783684.466779614\\
7	254091.009627657\\
8	156368.189566121\\
9	148146.705653507\\
10	281413.751084289\\
11	383730.418945787\\
12	196530.975831858\\
13	89292.8417771284\\
14	115635.208698957\\
15	417252.21204137\\
16	302600.369524329\\
17	111318.901917494\\
18	164197.23344413\\
19	377897.043803359\\
20	785895.998801982\\
21	338583.472493692\\
22	225086.314760605\\
23	156838.839306662\\
24	193956.17961893\\
25	424105.359706039\\
26	181231.173237443\\
27	238092.339293839\\
28	179399.676812502\\
29	593732.662732542\\
30	1385343.20503296\\
31	3125053.10558304\\
32	11191889.3670391\\
33	46850606.3972826\\
34	50205923.7854111\\
35	113080735.273198\\
36	93240889.5574682\\
37	16660873.281384\\
38	25477272.1580532\\
39	20516586.4616072\\
40	10224035.8783329\\
41	12557490.1417185\\
42	15383349.1433136\\
43	113983311.955226\\
44	454318358.861187\\
45	138190424.323182\\
46	12438136.4192108\\
47	10397235.8574198\\
48	8952391.01364979\\
49	6206733.60226623\\
50	3900633.43848734\\
51	6498252.21427081\\
52	6341825.12954421\\
53	8409185.91235206\\
54	5495141.4442484\\
55	5467902.16557234\\
56	10349241.7922091\\
57	37937733.3997456\\
58	225088690.251356\\
59	261923567.957029\\
60	98477547.6624893\\
61	92500543.2872047\\
62	17690963.3767161\\
63	17757995.1709214\\
64	36154686.3218921\\
65	33876953.5995125\\
66	15931631.3216887\\
67	5186249.94500254\\
68	28881508.0192419\\
69	29405710.5292967\\
70	16288840.2066237\\
71	7098685.17002216\\
72	3737461.24139774\\
73	2317474.38181273\\
74	804519.810824472\\
75	643352.180399639\\
76	565613.886998036\\
77	526432.540866271\\
78	901382.029084835\\
79	528543.154192613\\
80	232375.137388086\\
81	165277.951355617\\
82	670305.015011577\\
83	699917.707440526\\
84	246156.041849285\\
85	180491.797831531\\
86	279886.117170204\\
87	433195.538582795\\
88	378509.94589131\\
89	159924.414152285\\
90	111443.309688548\\
91	222919.518797925\\
92	469580.646791455\\
93	371708.880322223\\
94	228900.902057002\\
95	192218.772375071\\
96	382263.270597065\\
};
\addlegendentry{Scatterer power}

\addplot [color=green]
  table[row sep=crcr]{%
0	112498936.835131\\
100	112498936.835131\\
};
\addlegendentry{Average power of all scatterers}

\addplot [color=black, only marks, mark size=4.0pt, mark=o, mark options={solid, black}]
  table[row sep=crcr]{%
35	113080735.273198\\
43	113983311.955226\\
44	454318358.861187\\
45	138190424.323182\\
58	225088690.251356\\
59	261923567.957029\\
};
\addlegendentry{Candidate scatterers}

\addplot [color=red, only marks, mark size=7.5pt, mark=o, mark options={solid, red}]
  table[row sep=crcr]
                \caption{Scatterer power, \\ sf = 5.\label{subfig:sf5_hayRA_power}}
            \end{subfigure}
             &
            \begin{subfigure}{0.33\linewidth}
                \centering
                \resizebox{\linewidth}{!}{% This file was created by matlab2tikz.
%
%The latest updates can be retrieved from
%  http://www.mathworks.com/matlabcentral/fileexchange/22022-matlab2tikz-matlab2tikz
%where you can also make suggestions and rate matlab2tikz.
%
\definecolor{mycolor1}{rgb}{0.00000,0.44700,0.74100}%
%
\begin{tikzpicture}

\begin{axis}[%
width=6.028in,
height=4.754in,
at={(1.011in,0.642in)},
scale only axis,
xmin=0,
xmax=100,
xlabel style={font=\color{white!15!black}},
xlabel={Range Bin},
ymin=0,
ymax=500000000,
ylabel style={font=\color{white!15!black}},
ylabel={Power},
axis background/.style={fill=white},
title style={font=\bfseries},
title={Power of Scatterers},
legend style={legend cell align=left, align=left, draw=white!15!black}
]
\addplot [color=mycolor1, mark=asterisk, mark options={solid, mycolor1}]
  table[row sep=crcr]{%
1	899314.404068411\\
2	438057.505857389\\
3	244428.562207619\\
4	233002.309673522\\
5	565720.811386125\\
6	783684.466779614\\
7	254091.009627657\\
8	156368.189566121\\
9	148146.705653507\\
10	281413.751084289\\
11	383730.418945787\\
12	196530.975831858\\
13	89292.8417771284\\
14	115635.208698957\\
15	417252.21204137\\
16	302600.369524329\\
17	111318.901917494\\
18	164197.23344413\\
19	377897.043803359\\
20	785895.998801982\\
21	338583.472493692\\
22	225086.314760605\\
23	156838.839306662\\
24	193956.17961893\\
25	424105.359706039\\
26	181231.173237443\\
27	238092.339293839\\
28	179399.676812502\\
29	593732.662732542\\
30	1385343.20503296\\
31	3125053.10558304\\
32	11191889.3670391\\
33	46850606.3972826\\
34	50205923.7854111\\
35	113080735.273198\\
36	93240889.5574682\\
37	16660873.281384\\
38	25477272.1580532\\
39	20516586.4616072\\
40	10224035.8783329\\
41	12557490.1417185\\
42	15383349.1433136\\
43	113983311.955226\\
44	454318358.861187\\
45	138190424.323182\\
46	12438136.4192108\\
47	10397235.8574198\\
48	8952391.01364979\\
49	6206733.60226623\\
50	3900633.43848734\\
51	6498252.21427081\\
52	6341825.12954421\\
53	8409185.91235206\\
54	5495141.4442484\\
55	5467902.16557234\\
56	10349241.7922091\\
57	37937733.3997456\\
58	225088690.251356\\
59	261923567.957029\\
60	98477547.6624893\\
61	92500543.2872047\\
62	17690963.3767161\\
63	17757995.1709214\\
64	36154686.3218921\\
65	33876953.5995125\\
66	15931631.3216887\\
67	5186249.94500254\\
68	28881508.0192419\\
69	29405710.5292967\\
70	16288840.2066237\\
71	7098685.17002216\\
72	3737461.24139774\\
73	2317474.38181273\\
74	804519.810824472\\
75	643352.180399639\\
76	565613.886998036\\
77	526432.540866271\\
78	901382.029084835\\
79	528543.154192613\\
80	232375.137388086\\
81	165277.951355617\\
82	670305.015011577\\
83	699917.707440526\\
84	246156.041849285\\
85	180491.797831531\\
86	279886.117170204\\
87	433195.538582795\\
88	378509.94589131\\
89	159924.414152285\\
90	111443.309688548\\
91	222919.518797925\\
92	469580.646791455\\
93	371708.880322223\\
94	228900.902057002\\
95	192218.772375071\\
96	382263.270597065\\
};
\addlegendentry{Scatterer power}

\addplot [color=green]
  table[row sep=crcr]{%
0	224997873.670263\\
100	224997873.670263\\
};
\addlegendentry{Average power of all scatterers}

\addplot [color=black, only marks, mark size=4.0pt, mark=o, mark options={solid, black}]
  table[row sep=crcr]{%
44	454318358.861187\\
58	225088690.251356\\
59	261923567.957029\\
};
\addlegendentry{Candidate scatterers}

\addplot [color=red, only marks, mark size=7.5pt, mark=o, mark options={solid, red}]
  table[row sep=crcr]
                \caption{Scatterer power, \\ sf = 10.\label{subfig:sf10_hayRA_power}}
            \end{subfigure}
            \\
            \begin{subfigure}{0.33\linewidth}
                \centering
                \resizebox{\linewidth}{!}{% This file was created by matlab2tikz.
%
%The latest updates can be retrieved from
%  http://www.mathworks.com/matlabcentral/fileexchange/22022-matlab2tikz-matlab2tikz
%where you can also make suggestions and rate matlab2tikz.
%
\definecolor{mycolor1}{rgb}{0.00000,0.44700,0.74100}%
%
\begin{tikzpicture}

\begin{axis}[%
width=6.028in,
height=4.754in,
at={(1.011in,0.642in)},
scale only axis,
xmin=0,
xmax=100,
xlabel style={font=\color{white!15!black}},
xlabel={Range Bin},
ymin=0,
ymax=600000,
ylabel style={font=\color{white!15!black}},
ylabel={Amplitude Variance},
axis background/.style={fill=white},
title style={font=\bfseries},
title={Variance of Scatterers},
legend style={legend cell align=left, align=left, draw=white!15!black}
]
\addplot [color=mycolor1, mark=asterisk, mark options={solid, mycolor1}]
  table[row sep=crcr]{%
1	618.190133000413\\
2	641.073920671971\\
3	422.512775831874\\
4	342.530955124323\\
5	599.580876370041\\
6	931.691548806083\\
7	318.988608633882\\
8	236.80437716339\\
9	226.890255620622\\
10	458.171638136456\\
11	623.032438322094\\
12	326.547409786833\\
13	139.479408861878\\
14	153.571301611978\\
15	678.719610354443\\
16	378.028680888595\\
17	189.398142347566\\
18	293.700230537079\\
19	554.221100593161\\
20	808.444274291457\\
21	349.063629392178\\
22	403.543508145038\\
23	280.659159028239\\
24	350.836609029496\\
25	530.791168403754\\
26	270.867627639896\\
27	304.583711331055\\
28	288.806039398011\\
29	822.394108640439\\
30	2081.14893406934\\
31	5151.7589946324\\
32	18517.0209170654\\
33	65738.0448879609\\
34	93801.7638179143\\
35	115120.284135233\\
36	132173.939959572\\
37	38979.7480025487\\
38	45384.7243745594\\
39	27545.8723435685\\
40	14002.8572815111\\
41	16791.0802118644\\
42	25616.1698661406\\
43	113014.084207155\\
44	559848.864323783\\
45	158956.746947985\\
46	18660.3491477445\\
47	15733.5639697085\\
48	15935.6393678781\\
49	10455.8448372515\\
50	6860.38138484137\\
51	8742.19751476971\\
52	10483.0500009998\\
53	8437.94448607969\\
54	6774.29154241655\\
55	4393.63159660685\\
56	22951.874734016\\
57	76314.7795342918\\
58	245103.049430155\\
59	263297.977796135\\
60	239078.449962494\\
61	118714.253713222\\
62	24167.4056846046\\
63	29135.9504513176\\
64	35073.5310516745\\
65	24709.0471342535\\
66	15387.1300995051\\
67	4931.16417857924\\
68	34501.3446638806\\
69	25034.5995762561\\
70	27940.1314849251\\
71	12879.3147703385\\
72	7147.35232362722\\
73	4905.16196740392\\
74	1441.95002848645\\
75	932.394512340197\\
76	731.15729365425\\
77	973.987937406392\\
78	918.527094219897\\
79	594.998168271597\\
80	332.45976398315\\
81	265.163867084616\\
82	871.290685340808\\
83	859.811634275831\\
84	460.003653873436\\
85	386.602575956552\\
86	579.524956338482\\
87	781.800675304264\\
88	331.458314538084\\
89	271.699490941217\\
90	187.246670074608\\
91	300.976299842283\\
92	558.945620806556\\
93	407.26351676982\\
94	349.446013181976\\
95	320.22791369331\\
96	405.061423527288\\
};
\addlegendentry{Scatterer variance}

\addplot [color=black, only marks, mark size=4.0pt, mark=o, mark options={solid, black}]
  table[row sep=crcr]{%
33	65738.0448879609\\
34	93801.7638179143\\
35	115120.284135233\\
36	132173.939959572\\
38	45384.7243745594\\
43	113014.084207155\\
44	559848.864323783\\
45	158956.746947985\\
57	76314.7795342918\\
58	245103.049430155\\
59	263297.977796135\\
60	239078.449962494\\
61	118714.253713222\\
64	35073.5310516745\\
65	24709.0471342535\\
68	34501.3446638806\\
69	25034.5995762561\\
};
\addlegendentry{Candidate scatterers}

\addplot [color=red]
  table[row sep=crcr]{%
0	24709.0471342535\\
100	24709.0471342535\\
};
\addlegendentry{Minimum variance}

\addplot [color=red, only marks, mark size=4.0pt, mark=*, mark options={solid, fill=red, red}]
  table[row sep=crcr]
                \caption{Scatterer amplitude variance, \\ sf = 1.\label{subfig:sf1_hayRA_var}}
            \end{subfigure}
             &
            \begin{subfigure}{0.33\linewidth}
                \centering
                \resizebox{\linewidth}{!}{% This file was created by matlab2tikz.
%
%The latest updates can be retrieved from
%  http://www.mathworks.com/matlabcentral/fileexchange/22022-matlab2tikz-matlab2tikz
%where you can also make suggestions and rate matlab2tikz.
%
\definecolor{mycolor1}{rgb}{0.00000,0.44700,0.74100}%
%
\begin{tikzpicture}

\begin{axis}[%
width=6.028in,
height=4.754in,
at={(1.011in,0.642in)},
scale only axis,
xmin=0,
xmax=100,
xlabel style={font=\color{white!15!black}},
xlabel={Range Bin},
ymin=0,
ymax=600000,
ylabel style={font=\color{white!15!black}},
ylabel={Amplitude Variance},
axis background/.style={fill=white},
title style={font=\bfseries},
title={Variance of Scatterers},
legend style={legend cell align=left, align=left, draw=white!15!black}
]
\addplot [color=mycolor1, mark=asterisk, mark options={solid, mycolor1}]
  table[row sep=crcr]{%
1	618.190133000413\\
2	641.073920671971\\
3	422.512775831874\\
4	342.530955124323\\
5	599.580876370041\\
6	931.691548806083\\
7	318.988608633882\\
8	236.80437716339\\
9	226.890255620622\\
10	458.171638136456\\
11	623.032438322094\\
12	326.547409786833\\
13	139.479408861878\\
14	153.571301611978\\
15	678.719610354443\\
16	378.028680888595\\
17	189.398142347566\\
18	293.700230537079\\
19	554.221100593161\\
20	808.444274291457\\
21	349.063629392178\\
22	403.543508145038\\
23	280.659159028239\\
24	350.836609029496\\
25	530.791168403754\\
26	270.867627639896\\
27	304.583711331055\\
28	288.806039398011\\
29	822.394108640439\\
30	2081.14893406934\\
31	5151.7589946324\\
32	18517.0209170654\\
33	65738.0448879609\\
34	93801.7638179143\\
35	115120.284135233\\
36	132173.939959572\\
37	38979.7480025487\\
38	45384.7243745594\\
39	27545.8723435685\\
40	14002.8572815111\\
41	16791.0802118644\\
42	25616.1698661406\\
43	113014.084207155\\
44	559848.864323783\\
45	158956.746947985\\
46	18660.3491477445\\
47	15733.5639697085\\
48	15935.6393678781\\
49	10455.8448372515\\
50	6860.38138484137\\
51	8742.19751476971\\
52	10483.0500009998\\
53	8437.94448607969\\
54	6774.29154241655\\
55	4393.63159660685\\
56	22951.874734016\\
57	76314.7795342918\\
58	245103.049430155\\
59	263297.977796135\\
60	239078.449962494\\
61	118714.253713222\\
62	24167.4056846046\\
63	29135.9504513176\\
64	35073.5310516745\\
65	24709.0471342535\\
66	15387.1300995051\\
67	4931.16417857924\\
68	34501.3446638806\\
69	25034.5995762561\\
70	27940.1314849251\\
71	12879.3147703385\\
72	7147.35232362722\\
73	4905.16196740392\\
74	1441.95002848645\\
75	932.394512340197\\
76	731.15729365425\\
77	973.987937406392\\
78	918.527094219897\\
79	594.998168271597\\
80	332.45976398315\\
81	265.163867084616\\
82	871.290685340808\\
83	859.811634275831\\
84	460.003653873436\\
85	386.602575956552\\
86	579.524956338482\\
87	781.800675304264\\
88	331.458314538084\\
89	271.699490941217\\
90	187.246670074608\\
91	300.976299842283\\
92	558.945620806556\\
93	407.26351676982\\
94	349.446013181976\\
95	320.22791369331\\
96	405.061423527288\\
};
\addlegendentry{Scatterer variance}

\addplot [color=black, only marks, mark size=4.0pt, mark=o, mark options={solid, black}]
  table[row sep=crcr]{%
35	115120.284135233\\
43	113014.084207155\\
44	559848.864323783\\
45	158956.746947985\\
58	245103.049430155\\
59	263297.977796135\\
};
\addlegendentry{Candidate scatterers}

\addplot [color=red]
  table[row sep=crcr]{%
0	113014.084207155\\
100	113014.084207155\\
};
\addlegendentry{Minimum variance}

\addplot [color=red, only marks, mark size=4.0pt, mark=*, mark options={solid, fill=red, red}]
  table[row sep=crcr]
                \caption{Scatterer amplitude variance, \\ sf = 5.\label{subfig:sf5_hayRA_var}}
            \end{subfigure}
             &
            \begin{subfigure}{0.33\linewidth}
                \centering
                \resizebox{\linewidth}{!}{% This file was created by matlab2tikz.
%
%The latest updates can be retrieved from
%  http://www.mathworks.com/matlabcentral/fileexchange/22022-matlab2tikz-matlab2tikz
%where you can also make suggestions and rate matlab2tikz.
%
\definecolor{mycolor1}{rgb}{0.00000,0.44700,0.74100}%
%
\begin{tikzpicture}

\begin{axis}[%
width=6.028in,
height=4.754in,
at={(1.011in,0.642in)},
scale only axis,
xmin=0,
xmax=100,
xlabel style={font=\color{white!15!black}},
xlabel={Range Bin},
ymin=0,
ymax=600000,
ylabel style={font=\color{white!15!black}},
ylabel={Amplitude Variance},
axis background/.style={fill=white},
legend style={legend cell align=left, align=left, draw=white!15!black}
]
\addplot [color=mycolor1, mark=asterisk, mark options={solid, mycolor1}]
  table[row sep=crcr]{%
1	618.190133000413\\
2	641.073920671971\\
3	422.512775831874\\
4	342.530955124323\\
5	599.580876370041\\
6	931.691548806083\\
7	318.988608633882\\
8	236.80437716339\\
9	226.890255620622\\
10	458.171638136456\\
11	623.032438322094\\
12	326.547409786833\\
13	139.479408861878\\
14	153.571301611978\\
15	678.719610354443\\
16	378.028680888595\\
17	189.398142347566\\
18	293.700230537079\\
19	554.221100593161\\
20	808.444274291457\\
21	349.063629392178\\
22	403.543508145038\\
23	280.659159028239\\
24	350.836609029496\\
25	530.791168403754\\
26	270.867627639896\\
27	304.583711331055\\
28	288.806039398011\\
29	822.394108640439\\
30	2081.14893406934\\
31	5151.7589946324\\
32	18517.0209170654\\
33	65738.0448879609\\
34	93801.7638179143\\
35	115120.284135233\\
36	132173.939959572\\
37	38979.7480025487\\
38	45384.7243745594\\
39	27545.8723435685\\
40	14002.8572815111\\
41	16791.0802118644\\
42	25616.1698661406\\
43	113014.084207155\\
44	559848.864323783\\
45	158956.746947985\\
46	18660.3491477445\\
47	15733.5639697085\\
48	15935.6393678781\\
49	10455.8448372515\\
50	6860.38138484137\\
51	8742.19751476971\\
52	10483.0500009998\\
53	8437.94448607969\\
54	6774.29154241655\\
55	4393.63159660685\\
56	22951.874734016\\
57	76314.7795342918\\
58	245103.049430155\\
59	263297.977796135\\
60	239078.449962494\\
61	118714.253713222\\
62	24167.4056846046\\
63	29135.9504513176\\
64	35073.5310516745\\
65	24709.0471342535\\
66	15387.1300995051\\
67	4931.16417857924\\
68	34501.3446638806\\
69	25034.5995762561\\
70	27940.1314849251\\
71	12879.3147703385\\
72	7147.35232362722\\
73	4905.16196740392\\
74	1441.95002848645\\
75	932.394512340197\\
76	731.15729365425\\
77	973.987937406392\\
78	918.527094219897\\
79	594.998168271597\\
80	332.45976398315\\
81	265.163867084616\\
82	871.290685340808\\
83	859.811634275831\\
84	460.003653873436\\
85	386.602575956552\\
86	579.524956338482\\
87	781.800675304264\\
88	331.458314538084\\
89	271.699490941217\\
90	187.246670074608\\
91	300.976299842283\\
92	558.945620806556\\
93	407.26351676982\\
94	349.446013181976\\
95	320.22791369331\\
96	405.061423527288\\
};
\addlegendentry{Scatterer variance}

\addplot [color=black, only marks, mark size=4.0pt, mark=o, mark options={solid, black}]
  table[row sep=crcr]{%
44	559848.864323783\\
58	245103.049430155\\
59	263297.977796135\\
};
\addlegendentry{Candidate scatterers}

\addplot [color=red]
  table[row sep=crcr]{%
0	245103.049430155\\
100	245103.049430155\\
};
\addlegendentry{Minimum variance}

\addplot [color=red, only marks, mark size=4.0pt, mark=*, mark options={solid, fill=red, red}]
  table[row sep=crcr]
                \caption{Scatterer amplitude variance, \\ sf = 10.\label{subfig:sf10_hayRA_var}}
            \end{subfigure}
        \end{tabular}
        \caption{\gls{ds} selection and autofocused \gls{isar} images for different scaling factors using Haywood \gls{ra} measured data \gls{hrr} profiles.\label{fig:sf_hayRA}}
    \end{minipage}
    \end{figure}

    

%%%%%%%%%%%%%%%%%%%%%%%%%%%%%%%%%%%%%%%%%%%%%%%%%%%%%%%%%%%%%%%%%%%%%%%%%%%%%%%%%%%%%%%%%%%%%%%%%%%%
\section{Number of Dominant Scatterers Selected \label{apndxA:num_DS_effect}}






    
%%%%%%%%%%%%%%%%%%%%%%%%%%%%%%%%%%%%%%%%%%%%%%%%%%%%%%%%%%%%%%%%%%%%%%%%%%%%%%%%%%%%%%%%%%%%%%%%%%%%  
\section{Measured Data Frames for Algorithm Validation} \label{apndxA:verification_frames}
    The \gls{isar} image frames and \gls{hrr} profiles in \autoref{fig:measured_data_frames} were generated from a single measured data set. As discussed in the \autoref{sec:theory_cptwl}, the size of the \gls{cptwl} affects the quality of the image. In this report the \gls{cptwl} is fixed at 128 which is the set value for all of the frames. The frames considered in this section were chosen from the larger subset as they illustrate the necessity for this selection process. 
    \begin{figure}[H]
        \begin{minipage}{0.9\linewidth}
            \begin{figure}
                \begin{tabular}{@{}cccc@{}}
                    \begin{subfigure}{0.25\linewidth}
                        \centering
                        \resizebox{\linewidth}{!}{% This file was created by matlab2tikz.
%
%The latest updates can be retrieved from
%  http://www.mathworks.com/matlabcentral/fileexchange/22022-matlab2tikz-matlab2tikz
%where you can also make suggestions and rate matlab2tikz.
%
\begin{tikzpicture}

\begin{axis}[%
width=5.554in,
height=4.754in,
at={(0.932in,0.642in)},
scale only axis,
    point meta min=-3.40262853181023,
point meta max=70.6498190084972,
axis on top,
xmin=-0.15609375,
xmax=29.81390625,
xlabel style={font=\color{white!15!black}},
xlabel={Range (m)},
y dir=reverse,
ymin=0.5,
ymax=128.5,
ylabel style={font=\color{white!15!black}},
ylabel={Profile Number},
axis background/.style={fill=white},
colormap/jet,
colorbar
]
\addplot [forget plot] graphics [xmin=-0.15609375, xmax=29.81390625, ymin=0.5, ymax=128.5] {Figures/09Appendix/MeasuredDataFrames/Measured_HRRP_2464.png};
\end{axis}
\end{tikzpicture}%}
                        \caption{Frame 1: unaligned HRR profiles. \label{subfig:measured_data_frames_HRRP_2464}}
                    \end{subfigure}
                    &
                    \begin{subfigure}{0.25\linewidth}
                        \centering
                        \resizebox{\linewidth}{!}{% This file was created by matlab2tikz.
%
%The latest updates can be retrieved from
%  http://www.mathworks.com/matlabcentral/fileexchange/22022-matlab2tikz-matlab2tikz
%where you can also make suggestions and rate matlab2tikz.
%
\begin{tikzpicture}

\begin{axis}[%
width=5.554in,
height=4.754in,
at={(0.932in,0.642in)},
scale only axis,
    point meta min=-3.40262853181023,
point meta max=70.6498190084972,
axis on top,
xmin=-0.15609375,
xmax=29.81390625,
xlabel style={font=\color{white!15!black}},
xlabel={Range (m)},
y dir=reverse,
ymin=0.5,
ymax=128.5,
ylabel style={font=\color{white!15!black}},
ylabel={Profile Number},
axis background/.style={fill=white},
colormap/jet,
colorbar
]
\addplot [forget plot] graphics [xmin=-0.15609375, xmax=29.81390625, ymin=0.5, ymax=128.5] {Figures/09Appendix/MeasuredDataFrames/Measured_HRRP_2970.png};
\end{axis}
\end{tikzpicture}%}
                        \caption{Frame 2: unaligned HRR profiles.}
                    \end{subfigure}
                    &
                    \begin{subfigure}{0.25\linewidth}
                        \centering
                        \resizebox{\linewidth}{!}{% This file was created by matlab2tikz.
%
%The latest updates can be retrieved from
%  http://www.mathworks.com/matlabcentral/fileexchange/22022-matlab2tikz-matlab2tikz
%where you can also make suggestions and rate matlab2tikz.
%
\begin{tikzpicture}

\begin{axis}[%
width=5.554in,
height=4.754in,
at={(0.932in,0.642in)},
scale only axis,
    point meta min=-3.40262853181023,
point meta max=70.6498190084972,
axis on top,
xmin=0,
xmax=128,
xlabel style={font=\color{white!15!black}},
xlabel={Range (m)},
y dir=reverse,
ymin=0,
ymax=96,
ylabel style={font=\color{white!15!black}},
ylabel={Profile Number},
axis background/.style={fill=white},
colormap/jet,
colorbar
]
\addplot [forget plot] graphics [xmin=0, xmax=128, ymin=0, ymax=96] {Figures/09Appendix/MeasuredDataFrames/Measured_HRRP_3827.png};
\end{axis}
\end{tikzpicture}%
}
                        \caption{Frame 3: unaligned HRR profiles.}
                    \end{subfigure}
                    &
                    \begin{subfigure}{0.25\linewidth}
                        \centering
                        \resizebox{\linewidth}{!}{
% This file was created by matlab2tikz.
%
%The latest updates can be retrieved from
%  http://www.mathworks.com/matlabcentral/fileexchange/22022-matlab2tikz-matlab2tikz
%where you can also make suggestions and rate matlab2tikz.
%
\begin{tikzpicture}

\begin{axis}[%
width=5.554in,
height=4.754in,
at={(0.932in,0.642in)},
scale only axis,
    point meta min=-3.40262853181023,
point meta max=70.6498190084972,
axis on top,
xmin=0,
xmax=128,
xlabel style={font=\color{white!15!black}},
xlabel={Range (m)},
y dir=reverse,
ymin=0,
ymax=96,
ylabel style={font=\color{white!15!black}},
ylabel={Profile Number},
axis background/.style={fill=white},
colormap/jet,
colorbar
]
\addplot [forget plot] graphics [xmin=0, xmax=128, ymin=0, ymax=96] {Figures/09Appendix/MeasuredDataFrames/Measured_HRRP_4189.png};
\end{axis}
\end{tikzpicture}%
}
                        \caption{Frame 4: unaligned HRR profiles. \label{subfig:measured_data_frames_HRRP_4189}}
                    \end{subfigure}
                    \\
                    \begin{subfigure}{0.25\linewidth}
                        \centering
                        \resizebox{\linewidth}{!}{% This file was created by matlab2tikz.
%
%The latest updates can be retrieved from
%  http://www.mathworks.com/matlabcentral/fileexchange/22022-matlab2tikz-matlab2tikz
%where you can also make suggestions and rate matlab2tikz.
%
\begin{tikzpicture}
\begin{axis}[%
width=5.554in,
height=4.754in,
at={(0.932in,0.642in)},
scale only axis,
point meta min=-35,
point meta max=0,
axis on top,
xmin=-0.15609375,
xmax=29.81390625,
xlabel style={font=\color{white!15!black}},
xlabel={Range (m)},
ymin=-88.690863104058,
ymax=87.3049831146128,
ylabel style={font=\color{white!15!black}},
ylabel={Doppler frequency (Hz)},
axis background/.style={fill=white},
colormap/jet,
colorbar
]
\addplot [forget plot] graphics [xmin=-0.15609375, xmax=29.81390625, ymin=-88.690863104058, ymax=87.3049831146128] {Figures/09Appendix/MeasuredDataFrames/Measured_ISAR_2464.png};
\end{axis}

\end{tikzpicture}%}
                        \caption{Frame 1: unfocused ISAR image.}
                    \end{subfigure}
                    &
                    \begin{subfigure}{0.25\linewidth}
                        \centering
                        \resizebox{\linewidth}{!}{% This file was created by matlab2tikz.
%
%The latest updates can be retrieved from
%  http://www.mathworks.com/matlabcentral/fileexchange/22022-matlab2tikz-matlab2tikz
%where you can also make suggestions and rate matlab2tikz.
%
\begin{tikzpicture}
\begin{axis}[%
width=5.554in,
height=4.754in,
at={(0.932in,0.642in)},
scale only axis,
point meta min=-35,
point meta max=0,
axis on top,
xmin=-0.15609375,
xmax=29.81390625,
xlabel style={font=\color{white!15!black}},
xlabel={Range (m)},
ymin=-88.690863104058,
ymax=87.3049831146128,
ylabel style={font=\color{white!15!black}},
ylabel={Doppler frequency (Hz)},
axis background/.style={fill=white},
colormap/jet,
colorbar
]
\addplot [forget plot] graphics [xmin=-0.15609375, xmax=29.81390625, ymin=-88.690863104058, ymax=87.3049831146128] {Figures/09Appendix/MeasuredDataFrames/Measured_ISAR_2970.png};
\end{axis}

\end{tikzpicture}%}
                        \caption{Frame 2: unfocused ISAR image.}
                    \end{subfigure}
                    &
                    \begin{subfigure}{0.25\linewidth}
                        \centering
                        \resizebox{\linewidth}{!}{% This file was created by matlab2tikz.
%
%The latest updates can be retrieved from
%  http://www.mathworks.com/matlabcentral/fileexchange/22022-matlab2tikz-matlab2tikz
%where you can also make suggestions and rate matlab2tikz.
%
\begin{tikzpicture}
\begin{axis}[%
width=5.554in,
height=4.754in,
at={(0.932in,0.642in)},
scale only axis,
point meta min=-35,
point meta max=0,
axis on top,
xmin=-0.15609375,
xmax=29.81390625,
xlabel style={font=\color{white!15!black}},
xlabel={Range (m)},
ymin=-88.690863104058,
ymax=87.3049831146128,
ylabel style={font=\color{white!15!black}},
ylabel={Doppler frequency (Hz)},
axis background/.style={fill=white},
colormap/jet,
colorbar
]
\addplot [forget plot] graphics [xmin=-0.15609375, xmax=29.81390625, ymin=-88.690863104058, ymax=87.3049831146128] {Figures/09Appendix/MeasuredDataFrames/Measured_ISAR_3827.png};
\end{axis}

\end{tikzpicture}%}
                         \caption{Frame 3: unfocused ISAR image.}
                    \end{subfigure}
                    &
                    \begin{subfigure}{0.25\linewidth}
                        \centering
                        \resizebox{\linewidth}{!}{% This file was created by matlab2tikz.
%
%The latest updates can be retrieved from
%  http://www.mathworks.com/matlabcentral/fileexchange/22022-matlab2tikz-matlab2tikz
%where you can also make suggestions and rate matlab2tikz.
%
\begin{tikzpicture}
\begin{axis}[%
width=5.554in,
height=4.754in,
at={(0.932in,0.642in)},
scale only axis,
point meta min=-35,
point meta max=0,
axis on top,
xmin=-0.15609375,
xmax=29.81390625,
xlabel style={font=\color{white!15!black}},
xlabel={Range (m)},
ymin=-88.690863104058,
ymax=87.3049831146128,
ylabel style={font=\color{white!15!black}},
ylabel={Doppler frequency (Hz)},
axis background/.style={fill=white},
colormap/jet,
colorbar
]
\addplot [forget plot] graphics [xmin=-0.15609375, xmax=29.81390625, ymin=-88.690863104058, ymax=87.3049831146128] {Figures/09Appendix/MeasuredDataFrames/Measured_ISAR_4189.png};
\end{axis}

\end{tikzpicture}%}
                         \caption{Frame 4: unfocused ISAR image.}
                    \end{subfigure}
                \end{tabular}
                \caption{Different \gls{isar} image frames produced from one measured data set. \label{fig:measured_data_frames}}
            \end{figure}
        \end{minipage}
    \end{figure}


    Each frame provides a look of the same object over a different frame in time. Some of these are not suitable for verifying the \gls{ra} and \gls{af} processing algorithms considered in this report because of characteristics mentioned in \autoref{subsec:theory_isar}. The profiles in \autoref{subfig:measured_data_frames_HRRP_2464} and \autoref{subfig:measured_data_frames_HRRP_4189} have at least one clear \gls{ds} and are within the frame which makes them the ideal choice for verifying \gls{ra} algorithms. \autoref{fig:measured_data_frames_RA} shows the \gls{hrr} profiles of these two frames after applying Simple Correlation \gls{ra}, \autoref{alg:corr_RA}.
    \begin{figure} [H]
        \vspace*{\baselineskip}
        \centering
        \begin{subfigure}{0.4\textwidth}
            \centering
            \resizebox{\linewidth}{!}{% This file was created by matlab2tikz.
%
%The latest updates can be retrieved from
%  http://www.mathworks.com/matlabcentral/fileexchange/22022-matlab2tikz-matlab2tikz
%where you can also make suggestions and rate matlab2tikz.
%
\begin{tikzpicture}

\begin{axis}[%
width=5.554in,
height=4.754in,
at={(0.932in,0.642in)},
scale only axis,
    point meta min=-3.40262853181023,
point meta max=70.6498190084972,
axis on top,
xmin=0,
xmax=128,
xlabel style={font=\color{white!15!black}},
xlabel={Range (m)},
y dir=reverse,
ymin=0,
ymax=96,
ylabel style={font=\color{white!15!black}},
ylabel={Profile Number},
axis background/.style={fill=white},
colormap/jet,
colorbar
]
\addplot [forget plot] graphics [xmin=0, xmax=128, ymin=0, ymax=96] {Figures/09Appendix/MeasuredDataFrames/Measured_SCRA_HRRP_2464.png};

% Labels
\draw[->, black, line width=1.5pt] (35, 50) -- (57.75, 10);
\draw[->, black, line width=1.5pt] (35, 50) -- (56.5, 50);
\draw[->, black, line width=1.5pt] (35, 50) -- (57.75, 88);
\node[left] at (36, 50) {\tikz[baseline] \node[fill=white,inner sep=2pt] {Scatterer A};};

% Labels
\node[left] at (36, 50) {\tikz[baseline] \node[fill=white,inner sep=2pt] {Scatterer A};};
\draw[black, line width=1.5pt] (58 - 1, 96) rectangle (58 + 1, 0);

\end{axis}
\end{tikzpicture}%}
        \end{subfigure}
        \begin{subfigure}{0.4\textwidth}
            \centering
            \resizebox{\linewidth}{!}{% This file was created by matlab2tikz.
%
%The latest updates can be retrieved from
%  http://www.mathworks.com/matlabcentral/fileexchange/22022-matlab2tikz-matlab2tikz
%where you can also make suggestions and rate matlab2tikz.
%
\begin{tikzpicture}

\begin{axis}[%
width=5.554in,
height=4.754in,
at={(0.932in,0.642in)},
scale only axis,
    point meta min=-3.40262853181023,
point meta max=70.6498190084972,
axis on top,
xmin=0,
xmax=128,
xlabel style={font=\color{white!15!black}},
xlabel={Range (m)},
y dir=reverse,
ymin=0,
ymax=96,
ylabel style={font=\color{white!15!black}},
ylabel={Profile Number},
axis background/.style={fill=white},
colormap/jet,
colorbar
]
\addplot [forget plot] graphics [xmin=0, xmax=128, ymin=0, ymax=96] {Figures/09Appendix/MeasuredDataFrames/Measured_SCRA_HRRP_4189.png};

% Labels
\draw[->, black, line width=1.5pt] (35, 50) -- (60, 50);
\node[left] at (36, 50) {\tikz[baseline] \node[fill=white,inner sep=2pt] {Scatterer A};};
\draw[black, line width=1.5pt] (61.5 - 1, 96) rectangle (61.5 + 1, 0);

\end{axis}
\end{tikzpicture}%} 
        \end{subfigure}
        \caption{Frame 1 (left) and 4 (right): range-aligned \gls{hrr} profiles.} \label{fig:measured_data_frames_RA}
    \end{figure}
     As discussed in \autoref{subsec:theory_isar}, not all frames of \gls{isar} data can be focused due to the object's motion, the \gls{cptwl}, and other factors. It then follows that some measured data frames will not be perfectly range-aligned or autofocused. Comparing the profile frames in \autoref{fig:measured_data_frames_RA}, the profiles in Frame 1 are better aligned than those in Frame 4. Furthermore, in Frame 1, a distinct \gls{ds} exists at range bin 14. For validation testing, it is desirable to choose the best case data because the goal is to demonstrate that the algorithms work on measured data. The limitations of the algorithms were investigated and discussed through verification testing where non-ideal data is used. For these reasons, frame 1 was used for validation testing in \autoref{ch:algorithmV&V}. 


% ----------------------------------------------------
\ifstandalone
\bibliography{../Bibliography/References.bib}
\printnoidxglossary[type=\acronymtype,nonumberlist]
\fi
\end{document}
% ----------------------------------------------------
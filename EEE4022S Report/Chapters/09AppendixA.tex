% ----------------------------------------------------
% Appendix A
% ----------------------------------------------------
\documentclass[class=report,11pt,crop=false]{standalone}
% Page geometry
\usepackage[a4paper,margin=20mm,top=25mm,bottom=25mm]{geometry}

% Font choice
\usepackage{lmodern}

\usepackage{lipsum}

% Use IEEE bibliography style
\bibliographystyle{IEEEtran}

% Line spacing
\usepackage{setspace}
\setstretch{1.2}

% Ensure UTF8 encoding
\usepackage[utf8]{inputenc}

% Language standard (not too important)
\usepackage[english]{babel}

% Skip a line in between paragraphs
\usepackage{parskip}

% For the creation of dummy text
\usepackage{blindtext}

% Math
\usepackage{amsmath}

% Header & Footer stuff
\usepackage{fancyhdr}
\pagestyle{fancy}
\fancyhead{}
\fancyhead[R]{\nouppercase{\rightmark}}
\fancyfoot{}
\fancyfoot[C]{\thepage}
\renewcommand{\headrulewidth}{0.0pt}
\renewcommand{\footrulewidth}{0.0pt}
\setlength{\headheight}{13.6pt}

% Epigraphs
\usepackage{epigraph}
\setlength\epigraphrule{0pt}
\setlength{\epigraphwidth}{0.65\textwidth}

% Colour
\usepackage{color}
\usepackage[usenames,dvipsnames]{xcolor}

% Hyperlinks & References
\usepackage{hyperref}
\definecolor{linkColour}{RGB}{77,71,179}
\definecolor{urlColour}{RGB}{255, 179, 102}

\hypersetup{
    colorlinks=true,
    linkcolor=linkColour,
    filecolor=linkColour,
    urlcolor=urlColour,
    citecolor=linkColour,
}
\urlstyle{same}

% Automatically correct front-side quotes
\usepackage[autostyle=false, style=ukenglish]{csquotes}
\MakeOuterQuote{"}

% Graphics
\usepackage{graphicx}
\graphicspath{{Figures/}{../Figures/}}
\usepackage{makecell}
\usepackage{transparent}
\usepackage{pgfplots}
\pgfplotsset{compat=newest}
%% the following commands are needed for some matlab2tikz features
\usetikzlibrary{plotmarks}
\usetikzlibrary{arrows.meta}
\usepgfplotslibrary{patchplots}
\usepackage{float}

% Tables
\usepackage{multirow} 
\usepackage{colortbl}

% SI units
\usepackage{siunitx}

% Microtype goodness
\usepackage{microtype}

% Listings
\usepackage[T1]{fontenc}
\usepackage{listings}
\usepackage[scaled=0.8]{DejaVuSansMono}

% Custom colours for listings
\definecolor{backgroundColour}{RGB}{250,250,250}
\definecolor{commentColour}{RGB}{73, 175, 102}
\definecolor{identifierColour}{RGB}{196, 19, 66}
\definecolor{stringColour}{RGB}{252, 156, 30}
\definecolor{keywordColour}{RGB}{50, 38, 224}
\definecolor{lineNumbersColour}{RGB}{127,127,127}
\lstset{
  language=Matlab,
  captionpos=b,
  aboveskip=15pt,belowskip=10pt,
  backgroundcolor=\color{backgroundColour},
  basicstyle=\ttfamily,%\footnotesize,        % the size of the fonts that are used for the code
  breakatwhitespace=false,         % sets if automatic breaks should only happen at whitespace
  breaklines=true,                 % sets automatic line breaking
  postbreak=\mbox{\textcolor{red}{$\hookrightarrow$}\space},
  commentstyle=\color{commentColour},    % comment style
  identifierstyle=\color{identifierColour},
  stringstyle=\color{stringColour},
   keywordstyle=\color{keywordColour},       % keyword style
  %escapeinside={\%*}{*)},          % if you want to add LaTeX within your code
  extendedchars=true,              % lets you use non-ASCII characters; for 8-bits encodings only, does not work with UTF-8
  frame=single,	                   % adds a frame around the code
  keepspaces=true,                 % keeps spaces in text, useful for keeping indentation of code (possibly needs columns=flexible)
  morekeywords={*,...},            % if you want to add more keywords to the set
  numbers=left,                    % where to put the line-numbers; possible values are (none, left, right)
  numbersep=5pt,                   % how far the line-numbers are from the code
  numberstyle=\tiny\color{lineNumbersColour}, % the style that is used for the line-numbers
  rulecolor=\color{black},         % if not set, the frame-color may be changed on line-breaks within not-black text (e.g. comments (green here))
  showspaces=false,                % show spaces everywhere adding particular underscores; it overrides 'showstringspaces'
  showstringspaces=false,          % underline spaces within strings only
  showtabs=false,                  % show tabs within strings adding particular underscores
  stepnumber=1,                    % the step between two line-numbers. If it's 1, each line will be numbered
  tabsize=2,	                   % sets default tabsize to 2 spaces
  %title=\lstname                   % show the filename of files included with \lstinputlisting; also try caption instead of title
}

% Caption stuff
\usepackage[hypcap=true, justification=centering]{caption}
\usepackage{subcaption}

% Glossary package
% \usepackage[acronym]{glossaries}
\usepackage{glossaries-extra}
\setabbreviationstyle[acronym]{long-short}

% For Proofs & Theorems
\usepackage{amsthm}

% Maths symbols
\usepackage{amssymb}
\usepackage{mathrsfs}
\usepackage{mathtools}

% For algorithms
\usepackage[]{algorithm2e}

% Spacing stuff
\setlength{\abovecaptionskip}{5pt plus 3pt minus 2pt}
\setlength{\belowcaptionskip}{5pt plus 3pt minus 2pt}
\setlength{\textfloatsep}{10pt plus 3pt minus 2pt}
\setlength{\intextsep}{15pt plus 3pt minus 2pt}

% For aligning footnotes at bottom of page, instead of hugging text
\usepackage[bottom]{footmisc}

% Add LoF, Bib, etc. to ToC
\usepackage[nottoc]{tocbibind}

% SI
\usepackage{siunitx}

% For removing some whitespace in Chapter headings etc
\usepackage{etoolbox}
\makeatletter
\patchcmd{\@makechapterhead}{\vspace*{50\p@}}{\vspace*{-10pt}}{}{}%
\patchcmd{\@makeschapterhead}{\vspace*{50\p@}}{\vspace*{-10pt}}{}{}%
\makeatother

% Wrap figure
\usepackage{wrapfig}
\makenoidxglossaries

\newacronym{af}{AF}{Autofocus}
\newacronym{cli}{CLI}{Command-line Interface}
\newacronym{cpi}{CPI}{Coherent Processing Interval}
\newacronym{cptwl}{CPTWL}{Coherent Processing Time Window Length}
\newacronym{cw}{CW}{Continuous Waveform}
\newacronym{ds}{DS}{Dominant Scatterer}
\newacronym{dsa}{DSA}{Dominant Scatterer Algorithm}
\newacronym{sdsaf}{SDSAF}{Single Dominant Scatterer Autofocus}
\newacronym{fft}{FFT}{Fast Fourier Transform}
\newacronym{fmcw}{FMCW}{Frequency Modulated Continuous Waveform} % Not sure
\newacronym{hrr}{HRR}{High Resolution Range}
\newacronym{hrrp}{HRRP}{High Resolution Range Profile}
\newacronym{ic}{IC}{Image Contrast}
\newacronym{isar}{ISAR}{Inverse Synthetic Aperture Radar}
\newacronym{jtf}{JTF}{Joint Time-Frequency}
\newacronym{pri}{PRI}{Pulse Repetition Interval}
\newacronym{prf}{PRF}{Pulse Repetition Frequency}
\newacronym{qlp}{QLP}{Quick-look Processor}
\newacronym{ra}{RA}{Range Alignment}
\newacronym{rlos}{RLOS}{Radar Line of Sight}
\newacronym{rmc}{RMC}{Rotational Motion Compensation}
\newacronym{sfw}{SFW}{Stepped Frequency Waveform}
\newacronym{sf}{SF}{Scaling Factor for Haywood Autofocus}
\newacronym{sar}{SAR}{Synthetic Aperture Radar}
\newacronym{snr}{SNR}{Signal-to-Noise Ratio}
\newacronym{sir}{SIR}{Signal-to-Interference Ratio}
\newacronym{tmc}{TMC}{Translational Motion Compensation}


\begin{document}
% ----------------------------------------------------
\chapter{Appendix A \label{apndxA}}
% ----------------------------------------------------
\section{Measured Data Frames for Algorithm Verification} \label{apndxA:verification_frames}
The \gls{isar} image frames and \gls{hrr} profiles in \autoref{fig:measured_data_frames} were generated from a single measured data set. As discussed in the \autoref{sec:theory_cptwl}, the size of the \gls{cptwl} affects the quality of the image, in this report this value is assumed to be fixed at 128 which is the set value for all of the frames. These frames were chosen from the larger subset as they illustrate the necessity for this selection process. 

Each frame provides a look of the same object over a different frame in time. Some of these are not suitable for verifying the \gls{ra} and \gls{af} processing algorithms considered in this report. The profiles in \autoref{fig:measured_data_frames_HRRP_2464} and \autoref{fig:measured_data_frames_HRRP_4189} are clear and are all neatly within the frame which makes them the ideal choice for verifying \gls{ra} algorithms. Looking at the corresponding \gls{isar} images for Frame 1 and 4, Frame 1 has a more Doppler spread. This makes it more suitable for verifying \gls{af} algorithms.
    \begin{figure}[h]
        \begin{minipage}{0.9\linewidth}
            \begin{figure}
                \begin{tabular}{@{}cccc@{}}
                    \begin{subfigure}{0.25\linewidth}
                        \centering
                        \resizebox{\linewidth}{!}{% This file was created by matlab2tikz.
%
%The latest updates can be retrieved from
%  http://www.mathworks.com/matlabcentral/fileexchange/22022-matlab2tikz-matlab2tikz
%where you can also make suggestions and rate matlab2tikz.
%
\begin{tikzpicture}

\begin{axis}[%
width=5.554in,
height=4.754in,
at={(0.932in,0.642in)},
scale only axis,
    point meta min=-3.40262853181023,
point meta max=70.6498190084972,
axis on top,
xmin=-0.15609375,
xmax=29.81390625,
xlabel style={font=\color{white!15!black}},
xlabel={Range (m)},
y dir=reverse,
ymin=0.5,
ymax=128.5,
ylabel style={font=\color{white!15!black}},
ylabel={Profile Number},
axis background/.style={fill=white},
colormap/jet,
colorbar
]
\addplot [forget plot] graphics [xmin=-0.15609375, xmax=29.81390625, ymin=0.5, ymax=128.5] {Figures/09Appendix/MeasuredDataFrames/Measured_HRRP_2464.png};
\end{axis}
\end{tikzpicture}%}
                        \caption{Frame 1: unaligned HRR profiles.}
                    \end{subfigure}
                    &
                    \begin{subfigure}{0.25\linewidth}
                        \centering
                        \resizebox{\linewidth}{!}{% This file was created by matlab2tikz.
%
%The latest updates can be retrieved from
%  http://www.mathworks.com/matlabcentral/fileexchange/22022-matlab2tikz-matlab2tikz
%where you can also make suggestions and rate matlab2tikz.
%
\begin{tikzpicture}

\begin{axis}[%
width=5.554in,
height=4.754in,
at={(0.932in,0.642in)},
scale only axis,
    point meta min=-3.40262853181023,
point meta max=70.6498190084972,
axis on top,
xmin=-0.15609375,
xmax=29.81390625,
xlabel style={font=\color{white!15!black}},
xlabel={Range (m)},
y dir=reverse,
ymin=0.5,
ymax=128.5,
ylabel style={font=\color{white!15!black}},
ylabel={Profile Number},
axis background/.style={fill=white},
colormap/jet,
colorbar
]
\addplot [forget plot] graphics [xmin=-0.15609375, xmax=29.81390625, ymin=0.5, ymax=128.5] {Figures/09Appendix/MeasuredDataFrames/Measured_HRRP_2970.png};
\end{axis}
\end{tikzpicture}%}
                        \caption{Frame 2: unaligned HRR profiles.}
                    \end{subfigure}
                    &
                    \begin{subfigure}{0.25\linewidth}
                        \centering
                        \resizebox{\linewidth}{!}{% This file was created by matlab2tikz.
%
%The latest updates can be retrieved from
%  http://www.mathworks.com/matlabcentral/fileexchange/22022-matlab2tikz-matlab2tikz
%where you can also make suggestions and rate matlab2tikz.
%
\begin{tikzpicture}

\begin{axis}[%
width=5.554in,
height=4.754in,
at={(0.932in,0.642in)},
scale only axis,
    point meta min=-3.40262853181023,
point meta max=70.6498190084972,
axis on top,
xmin=0,
xmax=128,
xlabel style={font=\color{white!15!black}},
xlabel={Range (m)},
y dir=reverse,
ymin=0,
ymax=96,
ylabel style={font=\color{white!15!black}},
ylabel={Profile Number},
axis background/.style={fill=white},
colormap/jet,
colorbar
]
\addplot [forget plot] graphics [xmin=0, xmax=128, ymin=0, ymax=96] {Figures/09Appendix/MeasuredDataFrames/Measured_HRRP_3827.png};
\end{axis}
\end{tikzpicture}%
}
                        \caption{Frame 3: unaligned HRR profiles.}
                    \end{subfigure}
                    &
                    \begin{subfigure}{0.25\linewidth}
                        \centering
                        \resizebox{\linewidth}{!}{
% This file was created by matlab2tikz.
%
%The latest updates can be retrieved from
%  http://www.mathworks.com/matlabcentral/fileexchange/22022-matlab2tikz-matlab2tikz
%where you can also make suggestions and rate matlab2tikz.
%
\begin{tikzpicture}

\begin{axis}[%
width=5.554in,
height=4.754in,
at={(0.932in,0.642in)},
scale only axis,
    point meta min=-3.40262853181023,
point meta max=70.6498190084972,
axis on top,
xmin=0,
xmax=128,
xlabel style={font=\color{white!15!black}},
xlabel={Range (m)},
y dir=reverse,
ymin=0,
ymax=96,
ylabel style={font=\color{white!15!black}},
ylabel={Profile Number},
axis background/.style={fill=white},
colormap/jet,
colorbar
]
\addplot [forget plot] graphics [xmin=0, xmax=128, ymin=0, ymax=96] {Figures/09Appendix/MeasuredDataFrames/Measured_HRRP_4189.png};
\end{axis}
\end{tikzpicture}%
}
                        \caption{Frame 4: unaligned HRR profiles.}
                    \end{subfigure}
                    \\
                    \begin{subfigure}{0.25\linewidth}
                        \centering
                        \resizebox{\linewidth}{!}{% This file was created by matlab2tikz.
%
%The latest updates can be retrieved from
%  http://www.mathworks.com/matlabcentral/fileexchange/22022-matlab2tikz-matlab2tikz
%where you can also make suggestions and rate matlab2tikz.
%
\begin{tikzpicture}
\begin{axis}[%
width=5.554in,
height=4.754in,
at={(0.932in,0.642in)},
scale only axis,
point meta min=-35,
point meta max=0,
axis on top,
xmin=-0.15609375,
xmax=29.81390625,
xlabel style={font=\color{white!15!black}},
xlabel={Range (m)},
ymin=-88.690863104058,
ymax=87.3049831146128,
ylabel style={font=\color{white!15!black}},
ylabel={Doppler frequency (Hz)},
axis background/.style={fill=white},
colormap/jet,
colorbar
]
\addplot [forget plot] graphics [xmin=-0.15609375, xmax=29.81390625, ymin=-88.690863104058, ymax=87.3049831146128] {Figures/09Appendix/MeasuredDataFrames/Measured_ISAR_2464.png};
\end{axis}

\end{tikzpicture}%}
                        \caption{Frame 1: Unfocused ISAR image.}
                    \end{subfigure}
                    &
                    \begin{subfigure}{0.25\linewidth}
                        \centering
                        \resizebox{\linewidth}{!}{% This file was created by matlab2tikz.
%
%The latest updates can be retrieved from
%  http://www.mathworks.com/matlabcentral/fileexchange/22022-matlab2tikz-matlab2tikz
%where you can also make suggestions and rate matlab2tikz.
%
\begin{tikzpicture}
\begin{axis}[%
width=5.554in,
height=4.754in,
at={(0.932in,0.642in)},
scale only axis,
point meta min=-35,
point meta max=0,
axis on top,
xmin=-0.15609375,
xmax=29.81390625,
xlabel style={font=\color{white!15!black}},
xlabel={Range (m)},
ymin=-88.690863104058,
ymax=87.3049831146128,
ylabel style={font=\color{white!15!black}},
ylabel={Doppler frequency (Hz)},
axis background/.style={fill=white},
colormap/jet,
colorbar
]
\addplot [forget plot] graphics [xmin=-0.15609375, xmax=29.81390625, ymin=-88.690863104058, ymax=87.3049831146128] {Figures/09Appendix/MeasuredDataFrames/Measured_ISAR_2970.png};
\end{axis}

\end{tikzpicture}%}
                        \caption{Frame 2: Unfocused ISAR image.}
                    \end{subfigure}
                    &
                    \begin{subfigure}{0.25\linewidth}
                        \centering
                        \resizebox{\linewidth}{!}{% This file was created by matlab2tikz.
%
%The latest updates can be retrieved from
%  http://www.mathworks.com/matlabcentral/fileexchange/22022-matlab2tikz-matlab2tikz
%where you can also make suggestions and rate matlab2tikz.
%
\begin{tikzpicture}
\begin{axis}[%
width=5.554in,
height=4.754in,
at={(0.932in,0.642in)},
scale only axis,
point meta min=-35,
point meta max=0,
axis on top,
xmin=-0.15609375,
xmax=29.81390625,
xlabel style={font=\color{white!15!black}},
xlabel={Range (m)},
ymin=-88.690863104058,
ymax=87.3049831146128,
ylabel style={font=\color{white!15!black}},
ylabel={Doppler frequency (Hz)},
axis background/.style={fill=white},
colormap/jet,
colorbar
]
\addplot [forget plot] graphics [xmin=-0.15609375, xmax=29.81390625, ymin=-88.690863104058, ymax=87.3049831146128] {Figures/09Appendix/MeasuredDataFrames/Measured_ISAR_3827.png};
\end{axis}

\end{tikzpicture}%}
                         \caption{Frame 3: Unfocused ISAR image.}
                    \end{subfigure}
                    &
                    \begin{subfigure}{0.25\linewidth}
                        \centering
                        \resizebox{\linewidth}{!}{% This file was created by matlab2tikz.
%
%The latest updates can be retrieved from
%  http://www.mathworks.com/matlabcentral/fileexchange/22022-matlab2tikz-matlab2tikz
%where you can also make suggestions and rate matlab2tikz.
%
\begin{tikzpicture}
\begin{axis}[%
width=5.554in,
height=4.754in,
at={(0.932in,0.642in)},
scale only axis,
point meta min=-35,
point meta max=0,
axis on top,
xmin=-0.15609375,
xmax=29.81390625,
xlabel style={font=\color{white!15!black}},
xlabel={Range (m)},
ymin=-88.690863104058,
ymax=87.3049831146128,
ylabel style={font=\color{white!15!black}},
ylabel={Doppler frequency (Hz)},
axis background/.style={fill=white},
colormap/jet,
colorbar
]
\addplot [forget plot] graphics [xmin=-0.15609375, xmax=29.81390625, ymin=-88.690863104058, ymax=87.3049831146128] {Figures/09Appendix/MeasuredDataFrames/Measured_ISAR_4189.png};
\end{axis}

\end{tikzpicture}%}
                         \caption{Frame 4: Unfocused ISAR image.}
                    \end{subfigure}
                \end{tabular}
                \caption{Different \gls{isar} image frames produced from one measured data set. \label{fig:measured_data_frames}}
            \end{figure}
        \end{minipage}
    \end{figure}

%%%%%%%%%%%%%%%%%%%%%%%%%%%%%%%%%%%%%%%%%%%%%%%%%%%%%%%%%%%%%%%%%%%%%%%%%%%%%%%%%%%%%%%%%%%%%%%%%%%%
\section{Scaling Factor in \gls{ds} Selection \label{apndxA:scale_factor_effect}}
In the Haywood \gls{af} algorithm, given in \autoref{alg:haywood_AF}, phase compensation is calculated based on the phase history of the \gls{ds}. A set of selection criteria is used to determine the \gls{ds}. Initially, candidate scatterers are chosen, and from this set, the scatterer with the minimum variance is selected as the \gls{ds}. The candidate scatters are the scatterers with a power greater than the average scatterer power (the threshold value). As discussed in \autoref{theory:noise}, signals can be frequency-modulated by external interference sources, which can affect the phase of the received signal. Therefore, noise filtering is essential to enhance the performance of this algorithm. An average power scaling factor (SF) was introduced to select higher power scatterers as candidates. This is advantageous because higher power scatterers tend to have a phase history that is less affected by noise.

% In the Haywood \gls{af} algorithm, \autoref{alg:haywood_AF}, the phase compensation is calculated based on the phase history of the \gls{ds}. A set of selection criteria is used to determine the \gls{ds}. First, candidate scatterers are chosen, from this the scatterer with the minimum variance is selected as the \gls{ds}. The candidate dominant scatterers are chosen as all scatterers with a power greater than the average power of all scatterers (the threshold value). A average power scaling factor was introduced to counteract one of the limitations of \gls{ds} algorithms: they assume the selected \gls{ds} is the center of rotation (the 0 Doppler point) which is not always the case. Introducing a scaling factor means that higher power scatterers are chosen as a candidates thus a higher power \gls{ds} will be chosen. This is beneficial because higher power scatters will have a phase history which is less contaminated by noise than a scatterer with less power as discussed in \autoref{theory:noise}. 

    \subsection{Effect of SF on ISAR Image}
    The effect of the SF on the selection of the \gls{ds} and subsequently the \gls{isar} image is investigated in this section.

    % Grid of the HRRP and ISAR images
    \begin{figure}[h]
    \begin{minipage}{0.98\linewidth}
        \begin{tabular}{@{}cccc@{}}
            \begin{subfigure}{0.25\linewidth}
                \centering
                \resizebox{\linewidth}{!}{% This file was created by matlab2tikz.
%
%The latest updates can be retrieved from
%  http://www.mathworks.com/matlabcentral/fileexchange/22022-matlab2tikz-matlab2tikz
%where you can also make suggestions and rate matlab2tikz.
%
\begin{tikzpicture}
\begin{axis}[%
width=5.554in,
height=4.754in,
at={(0.932in,0.642in)},
scale only axis,
point meta min=-40,
point meta max=0,
axis on top,
xmin=-0.0732421875,
xmax=37.4267578125,
xlabel style={font=\fontsize{25}{14}\selectfont\color{black}, yshift=-10pt},
xlabel={Range (m)},
ymin=-32.2265625,
ymax=30.2734375,
ylabel style={font=\fontsize{25}{14}\selectfont\color{black}},
ylabel={Doppler frquency (Hz)},
axis background/.style={fill=white},
tick label style={font=\fontsize{20}{11}\selectfont\color{black}},
xtick distance= 4,             % Set the spacing between x-axis ticks
ytick distance = 10,
colormap/jet,
colorbar
]
\addplot [forget plot] graphics [xmin=-0.0732421875, xmax=37.4267578125, ymin=-32.2265625, ymax=30.2734375] {Figures/09Appendix/ScalingFactor/CorrRA/HayAF_SCRA_Sim_ISAR_sf1.png};
\end{axis}

\end{tikzpicture}%}
                \caption{Autofocused \gls{isar} image, SF = 1.\label{subfig:sf1_corrRA_isar}}
            \end{subfigure}
            &
            \begin{subfigure}{0.25\linewidth}
                \centering
                \resizebox{\linewidth}{!}{% This file was created by matlab2tikz.
%
%The latest updates can be retrieved from
%  http://www.mathworks.com/matlabcentral/fileexchange/22022-matlab2tikz-matlab2tikz
%where you can also make suggestions and rate matlab2tikz.
%
\begin{tikzpicture}
\begin{axis}[%
width=5.554in,
height=4.754in,
at={(0.932in,0.642in)},
scale only axis,
point meta min=-40,
point meta max=0,
axis on top,
xmin=-0.0732421875,
xmax=37.4267578125,
xlabel style={font=\fontsize{25}{14}\selectfont\color{black}, yshift=-10pt},
xlabel={Range (m)},
ymin=-32.2265625,
ymax=30.2734375,
ylabel style={font=\fontsize{25}{14}\selectfont\color{black}},
ylabel={Doppler frquency (Hz)},
axis background/.style={fill=white},
tick label style={font=\fontsize{20}{11}\selectfont\color{black}},
xtick distance= 4,             % Set the spacing between x-axis ticks
ytick distance = 10,
colormap/jet,
colorbar
]
\addplot [forget plot] graphics [xmin=-0.0732421875, xmax=37.4267578125, ymin=-32.2265625, ymax=30.2734375] {Figures/09Appendix/ScalingFactor/CorrRA/HayAF_SCRA_Sim_ISAR_sf2.png};
\end{axis}

\end{tikzpicture}%}
                \caption{Autofocused \gls{isar} image, SF = 2.\label{subfig:sf2_corrRA_isar}}
            \end{subfigure}
            &
            \begin{subfigure}{0.25\linewidth}
                \centering
                \resizebox{\linewidth}{!}{% This file was created by matlab2tikz.
%
%The latest updates can be retrieved from
%  http://www.mathworks.com/matlabcentral/fileexchange/22022-matlab2tikz-matlab2tikz
%where you can also make suggestions and rate matlab2tikz.
%
\begin{tikzpicture}
\begin{axis}[%
width=5.554in,
height=4.754in,
at={(0.932in,0.642in)},
scale only axis,
point meta min=-40,
point meta max=0,
axis on top,
xmin=-0.0732421875,
xmax=37.4267578125,
xlabel style={font=\fontsize{25}{14}\selectfont\color{black}, yshift=-10pt},
xlabel={Range (m)},
ymin=-32.2265625,
ymax=30.2734375,
ylabel style={font=\fontsize{25}{14}\selectfont\color{black}},
ylabel={Doppler frquency (Hz)},
axis background/.style={fill=white},
tick label style={font=\fontsize{20}{11}\selectfont\color{black}},
xtick distance= 4,             % Set the spacing between x-axis ticks
ytick distance = 10,
colormap/jet,
colorbar
]
\addplot [forget plot] graphics [xmin=-0.0732421875, xmax=37.4267578125, ymin=-32.2265625, ymax=30.2734375] {Figures/09Appendix/ScalingFactor/CorrRA/HayAF_SCRA_Sim_ISAR_sf5.png};
\end{axis}

\end{tikzpicture}%}
                \caption{Autofocused \gls{isar} image, SF = 5.\label{subfig:sf5_corrRA_isar}}
            \end{subfigure}
             &
            \begin{subfigure}{0.25\linewidth}
                \centering
                \resizebox{\linewidth}{!}{% This file was created by matlab2tikz.
%
%The latest updates can be retrieved from
%  http://www.mathworks.com/matlabcentral/fileexchange/22022-matlab2tikz-matlab2tikz
%where you can also make suggestions and rate matlab2tikz.
%
\begin{tikzpicture}
\begin{axis}[%
width=5.554in,
height=4.754in,
at={(0.932in,0.642in)},
scale only axis,
point meta min=-40,
point meta max=0,
axis on top,
xmin=-0.0732421875,
xmax=37.4267578125,
xlabel style={font=\fontsize{25}{14}\selectfont\color{black}, yshift=-10pt},
xlabel={Range (m)},
ymin=-32.2265625,
ymax=30.2734375,
ylabel style={font=\fontsize{25}{14}\selectfont\color{black}},
ylabel={Doppler frquency (Hz)},
axis background/.style={fill=white},
tick label style={font=\fontsize{20}{11}\selectfont\color{black}},
xtick distance= 4,             % Set the spacing between x-axis ticks
ytick distance = 10,
colormap/jet,
colorbar
]
\addplot [forget plot] graphics [xmin=-0.0732421875, xmax=37.4267578125, ymin=-32.2265625, ymax=30.2734375] {Figures/09Appendix/ScalingFactor/CorrRA/ISAR/HayAF_SCRA_Sim_ISAR_sf10.png};
\end{axis}

\end{tikzpicture}%}
                \caption{Autofocused \gls{isar} image, SF = 10.\label{subfig:sf10_corrRA_isar}}
            \end{subfigure}
            \\
            \begin{subfigure}{0.25\linewidth}
                \centering
                \resizebox{\linewidth}{!}{% This file was created by matlab2tikz.
%
%The latest updates can be retrieved from
%  http://www.mathworks.com/matlabcentral/fileexchange/22022-matlab2tikz-matlab2tikz
%where you can also make suggestions and rate matlab2tikz.
%
\definecolor{mycolor1}{rgb}{0.00000,0.44700,0.74100}%
%
\begin{tikzpicture}

\begin{axis}[%
width=6.028in,
height=4.754in,
at={(1.011in,0.642in)},
scale only axis,
xmin=0,
xmax=256,
xlabel style={font=\fontsize{25}{20}\selectfont\color{black}, yshift = -10},
xlabel={Range Bin},
ymin=0,
ymax=0.1e7,
ylabel style={font=\fontsize{25}{20}\selectfont\color{black}, yshift=10pt},
ylabel={Power},
axis background/.style={fill=white},
tick label style={font=\fontsize{20}{11}\selectfont\color{black}},
xtick distance = 50,
ytick distance= 0.1e6,
yticklabel={\ifdim\tick pt=0pt\else\pgfmathprintnumber{\tick}\fi}, 
scaled y ticks=base 10:-6,
legend style={legend cell align=left, align = left, draw=white!15!black, font=\fontsize{12}{11}\selectfont\color{black}}
]
\addplot [color=mycolor1, mark=asterisk, mark options={solid, mycolor1}]
  table[row sep=crcr]{%
1	262.577633405177\\
2	262.223997203794\\
3	260.587746143257\\
4	262.165948835539\\
5	266.370740045406\\
6	269.751691129423\\
7	268.372577835589\\
8	264.328240273671\\
9	267.210920185527\\
10	267.159343588493\\
11	272.471964953293\\
12	271.733768437709\\
13	273.204252219756\\
14	273.401297413845\\
15	279.939014967128\\
16	271.317790363962\\
17	276.66694833236\\
18	287.453857022635\\
19	282.498378926959\\
20	282.424615908933\\
21	282.237995184674\\
22	287.182445609935\\
23	296.88246320039\\
24	293.376809736088\\
25	299.544038232535\\
26	307.945899499856\\
27	308.34916592009\\
28	302.46767851089\\
29	313.545932424688\\
30	319.693901409099\\
31	327.838402778644\\
32	325.412275000734\\
33	329.962225473231\\
34	339.780823710529\\
35	345.065837305361\\
36	351.620452666554\\
37	364.37120330017\\
38	373.063512866931\\
39	374.996429565808\\
40	388.481018840225\\
41	402.028509784367\\
42	410.253994002813\\
43	415.872942397272\\
44	433.048131010869\\
45	459.915080945587\\
46	473.409570465863\\
47	509.273698648161\\
48	534.084651023651\\
49	571.863528488545\\
50	632.011666961169\\
51	686.10388722041\\
52	800.909763589265\\
53	926.970635820219\\
54	1169.47290501226\\
55	1606.85543326294\\
56	2579.98910836732\\
57	5803.63039172138\\
58	40476.8926273612\\
59	234509.09387101\\
60	16812.5905675379\\
61	4748.44031343692\\
62	3258.56766572553\\
63	3609.19091675377\\
64	7271.89793873345\\
65	76462.8638749144\\
66	213763.19883202\\
67	10080.9121387595\\
68	4505.24246037311\\
69	3559.60602970705\\
70	4097.00870601672\\
71	8129.95852185237\\
72	127110.777437309\\
73	167004.96702928\\
74	9571.29347875923\\
75	4973.42213117302\\
76	4303.78311248336\\
77	5199.58233305912\\
78	10381.4623302359\\
79	182254.013181938\\
80	108107.786003549\\
81	8587.05940745672\\
82	4386.38218972955\\
83	3888.88544639908\\
84	5201.55122335099\\
85	13990.6637254485\\
86	227467.353622077\\
87	63195.4072386991\\
88	8193.22529846067\\
89	4118.72807301388\\
90	3448.91074307703\\
91	4833.85184580242\\
92	22694.9097035337\\
93	244827.045465842\\
94	37643.2398942935\\
95	10525.6936569466\\
96	7119.48875276094\\
97	6687.54158992564\\
98	9259.70452372373\\
99	51581.5898604238\\
100	239626.484632749\\
101	21081.8937461551\\
102	12390.466664605\\
103	12969.8628577967\\
104	18553.0224931073\\
105	50425.0459051908\\
106	492159.951355135\\
107	722466.661470341\\
108	42845.4679587927\\
109	18904.8966167844\\
110	13311.8882371492\\
111	11892.6969105579\\
112	14231.0511261405\\
113	161310.022127177\\
114	207383.060829013\\
115	32230.4864489048\\
116	26689.6559112108\\
117	28029.5540798509\\
118	33541.5756317694\\
119	47288.5364117497\\
120	243049.420993879\\
121	113356.462295404\\
122	67951.6373400222\\
123	101249.279715054\\
124	175647.487522582\\
125	382031.425633381\\
126	1567528.09945156\\
127	23222434.9618517\\
128	4866763.94966091\\
129	624247.624633293\\
130	239143.421720153\\
131	125461.785185967\\
132	78179.4257393306\\
133	73077.6537215823\\
134	271201.515747773\\
135	65931.0424830517\\
136	35672.0138791419\\
137	27654.2722732861\\
138	24583.9369850773\\
139	26999.5479715209\\
140	93379.6124259705\\
141	280300.088807601\\
142	15205.4776885098\\
143	11661.1310430251\\
144	12769.6894139469\\
145	18402.2829947879\\
146	54225.2198477609\\
147	558976.693699555\\
148	675665.481591376\\
149	46962.7983886691\\
150	20393.9719870215\\
151	13535.4930230347\\
152	11761.055490961\\
153	14768.0813924932\\
154	165989.325372009\\
155	129169.53691609\\
156	12868.1621171875\\
157	8411.35286185695\\
158	7933.21185777849\\
159	9588.44659244235\\
160	18160.9178834827\\
161	221597.321922832\\
162	66308.3324321176\\
163	6306.92369286591\\
164	3520.35951295018\\
165	3676.63483627014\\
166	5820.45767786015\\
167	21232.7602733964\\
168	249945.230252575\\
169	37403.2425101297\\
170	6704.96902736564\\
171	3832.53243857378\\
172	3700.63923155125\\
173	6150.75623269814\\
174	39604.4771244479\\
175	244069.97144358\\
176	19163.6650471279\\
177	6543.53768190825\\
178	4763.80584744868\\
179	5143.26651622724\\
180	8981.76443566913\\
181	79701.9991589925\\
182	213463.509099999\\
183	9205.80202458775\\
184	4369.7249562291\\
185	3773.22980782136\\
186	4606.80345066959\\
187	9211.49891750507\\
188	128870.792246111\\
189	162669.293467407\\
190	7576.16282016939\\
191	3496.35460457662\\
192	2909.90245100321\\
193	3774.65158520432\\
194	8582.92339736164\\
195	174796.753846002\\
196	107450.969942292\\
197	8153.14181718801\\
198	3237.84037236851\\
199	1875.56915152923\\
200	1335.84461730023\\
201	1033.91058588523\\
202	856.439259758838\\
203	762.653045285108\\
204	668.930743746777\\
205	625.901072207345\\
206	556.486706127849\\
207	530.503276413855\\
208	496.045662997232\\
209	476.96061148105\\
210	460.039553501425\\
211	440.427277040158\\
212	420.744070083467\\
213	406.26548994886\\
214	391.074160852518\\
215	379.015338937019\\
216	381.002373420812\\
217	370.718395878786\\
218	346.158491498091\\
219	357.849584994842\\
220	352.449619069298\\
221	340.339510167936\\
222	345.10518776573\\
223	327.335679051124\\
224	323.783159711637\\
225	317.325257908808\\
226	304.192683315749\\
227	295.199951649738\\
228	305.86238727585\\
229	296.591571815832\\
230	298.662236201926\\
231	298.492937094633\\
232	290.683540518951\\
233	290.366845415873\\
234	289.732973521913\\
235	290.309428094255\\
236	280.995516102322\\
237	278.531696332776\\
238	276.730525062284\\
239	274.924875519189\\
240	271.13797603103\\
241	270.958320234206\\
242	272.70345874414\\
243	276.476328776733\\
244	273.23394172863\\
245	264.938748656207\\
246	266.485536484425\\
247	265.766858599058\\
248	265.207047749107\\
249	258.427321219398\\
250	268.381198158333\\
251	261.430204439809\\
252	264.739520910823\\
253	262.567571330099\\
254	266.722422494786\\
255	261.47945048988\\
256	265.524771370158\\
};
\addlegendentry{Scatterer power}

\addplot [color=orange]
  table[row sep=crcr]{%
0	158835.729579474\\
300	158835.729579474\\
};
\addlegendentry{Average power of all scatterers}

\addplot [color=black, only marks, mark size=4.0pt, mark=o, mark options={solid, black}]
  table[row sep=crcr]{%
59	234509.09387101\\
66	213763.19883202\\
73	167004.96702928\\
79	182254.013181938\\
86	227467.353622077\\
93	244827.045465842\\
100	239626.484632749\\
106	492159.951355135\\
107	722466.661470341\\
113	161310.022127177\\
114	207383.060829013\\
120	243049.420993879\\
124	175647.487522582\\
125	382031.425633381\\
126	1567528.09945156\\
127	23222434.9618517\\
128	4866763.94966091\\
129	624247.624633293\\
130	239143.421720153\\
134	271201.515747773\\
141	280300.088807601\\
147	558976.693699555\\
148	675665.481591376\\
154	165989.325372009\\
161	221597.321922832\\
168	249945.230252575\\
175	244069.97144358\\
182	213463.509099999\\
189	162669.293467407\\
195	174796.753846002\\
};
\addlegendentry{Candidate scatterers}

\addplot [color=red, only marks, mark size=7.5pt, mark=o, mark options={solid, red}]
  table[row sep=crcr]
                \caption{.\label{subfig:hayRA_sim_hrrp}}
            \end{subfigure}
             &
            \begin{subfigure}{0.25\linewidth}
                \centering
                \resizebox{\linewidth}{!}{% This file was created by matlab2tikz.
%
%The latest updates can be retrieved from
%  http://www.mathworks.com/matlabcentral/fileexchange/22022-matlab2tikz-matlab2tikz
%where you can also make suggestions and rate matlab2tikz.
%
\definecolor{mycolor1}{rgb}{0.00000,0.44700,0.74100}%
%
\begin{tikzpicture}

\begin{axis}[%
width=6.028in,
height=4.754in,
at={(1.011in,0.642in)},
scale only axis,
xmin=0,
xmax=300,
xlabel style={font=\fontsize{25}{20}\selectfont\color{black}, yshift = -10},
xlabel={Power},
ymin=0,
ymax=25000000,
ylabel style={font=\fontsize{25}{20}\selectfont\color{black}},
ylabel={Range Bin},
axis background/.style={fill=white},
tick label style={font=\fontsize{20}{11}\selectfont\color{black}},
xtick distance = 50,
legend style={legend cell align=left, align=left, draw=white!15!black}
]
\addplot [color=mycolor1, mark=asterisk, mark options={solid, mycolor1}]
  table[row sep=crcr]{%
1	255.913903600941\\
2	260.657201360851\\
3	262.172435827598\\
4	250.921964659475\\
5	266.737454311012\\
6	265.073804347981\\
7	261.827725943363\\
8	266.120552728393\\
9	257.232194000252\\
10	270.133161039495\\
11	263.723328851549\\
12	274.594493071301\\
13	269.029772501486\\
14	279.818642444619\\
15	276.652379476398\\
16	282.03049012937\\
17	282.88136621913\\
18	279.540396361505\\
19	276.402887430598\\
20	283.112475915646\\
21	280.403397603717\\
22	289.604292124238\\
23	294.84338016525\\
24	297.975773654332\\
25	298.640009222281\\
26	298.584431085498\\
27	313.438497677396\\
28	314.185595405489\\
29	310.5177582279\\
30	317.601574396994\\
31	317.875967467536\\
32	321.629215790047\\
33	325.896873895269\\
34	343.600992051259\\
35	345.833831891833\\
36	356.474387736686\\
37	356.749307369527\\
38	357.673463771464\\
39	371.778375236922\\
40	389.393065488178\\
41	389.963787055605\\
42	400.207301393094\\
43	424.955184089638\\
44	436.142903322458\\
45	463.878116637577\\
46	479.398770017183\\
47	511.615974827302\\
48	529.176849857973\\
49	575.739620940422\\
50	635.299085652741\\
51	688.899922670585\\
52	798.976576409165\\
53	942.88021870543\\
54	1158.40954876987\\
55	1614.44756790782\\
56	2597.49326770122\\
57	5792.82273507302\\
58	40516.5208804401\\
59	234491.323573846\\
60	16859.8129681379\\
61	4747.86690406277\\
62	3268.57196940437\\
63	3624.62405683739\\
64	7242.73267655396\\
65	76582.1524706418\\
66	213660.566520541\\
67	10103.431605782\\
68	4524.42699477825\\
69	3561.64932376731\\
70	4118.67717520311\\
71	8096.78995355576\\
72	127134.424870654\\
73	167011.49072067\\
74	9623.01376444448\\
75	4993.06051673588\\
76	4280.40912497671\\
77	5212.44904136988\\
78	10335.302337844\\
79	182291.810249744\\
80	108071.096279117\\
81	8653.1024671522\\
82	4370.77817669893\\
83	3857.38201390396\\
84	5211.85716541651\\
85	14026.3470175849\\
86	227575.76062754\\
87	63296.8059347195\\
88	8189.27801296807\\
89	4150.90116507453\\
90	3468.94074403249\\
91	4905.09155972574\\
92	22652.0120684173\\
93	244651.059234743\\
94	37596.3023755014\\
95	10499.0305037036\\
96	7103.17409908961\\
97	6730.34011290116\\
98	9216.53814267745\\
99	51517.2057366214\\
100	239717.438005992\\
101	20984.9669751931\\
102	12428.9663690319\\
103	12960.5504276021\\
104	18577.2252018787\\
105	50291.3633750061\\
106	492345.285552265\\
107	722422.843009396\\
108	42964.3247353232\\
109	18931.7015186062\\
110	13337.0577146104\\
111	11918.3234222555\\
112	14209.4179452284\\
113	161365.855145732\\
114	207566.706843034\\
115	32193.0689645564\\
116	26566.1093775142\\
117	28036.2032370565\\
118	33375.9417000765\\
119	47403.8462210652\\
120	242906.610792707\\
121	113367.472306356\\
122	67943.1860420572\\
123	101476.968540918\\
124	175757.903113325\\
125	382144.551729205\\
126	1567806.36829179\\
127	23222830.6558939\\
128	4866233.18616336\\
129	624325.939865116\\
130	238994.776765402\\
131	125375.772325989\\
132	78149.4769256337\\
133	72958.5162935334\\
134	271304.013801624\\
135	65934.8061788983\\
136	35578.920119417\\
137	27659.5769934385\\
138	24572.3857411676\\
139	26945.9856955091\\
140	93396.2739728246\\
141	280271.53874598\\
142	15231.2485331395\\
143	11688.6607694689\\
144	12803.2756268234\\
145	18313.3094732058\\
146	54269.5400185932\\
147	558849.855314574\\
148	675616.999291431\\
149	46826.1003429453\\
150	20430.9915364383\\
151	13557.681631592\\
152	11679.9802867242\\
153	14847.5879211756\\
154	166107.929333448\\
155	129180.556987218\\
156	12834.4276479728\\
157	8382.55047551408\\
158	7962.05118515591\\
159	9612.75679012557\\
160	18210.4621977994\\
161	221234.442807453\\
162	66203.3858591753\\
163	6331.23691492933\\
164	3551.51054722895\\
165	3691.23877260298\\
166	5800.44490335582\\
167	21194.9967052716\\
168	250170.697345067\\
169	37354.7641938466\\
170	6657.50119487548\\
171	3825.59349666832\\
172	3690.9966466907\\
173	6136.77531549204\\
174	39588.6120186432\\
175	244083.479007594\\
176	19175.9031874486\\
177	6548.6104218156\\
178	4752.87048126753\\
179	5139.44688782652\\
180	9005.90932797861\\
181	79691.7539892643\\
182	213436.929423769\\
183	9179.13009803847\\
184	4380.03786852226\\
185	3751.43468914538\\
186	4630.08109974715\\
187	9218.48509461629\\
188	128815.624323147\\
189	162658.082617928\\
190	7549.46062565505\\
191	3488.18034122662\\
192	2902.93741380408\\
193	3783.43948452934\\
194	8604.44424360079\\
195	174985.709215168\\
196	107316.508738174\\
197	8130.45392758719\\
198	3211.22872341984\\
199	1869.77015987127\\
200	1328.15360017168\\
201	1036.00019751651\\
202	862.303415278445\\
203	737.247614403178\\
204	658.355180171797\\
205	603.106633495516\\
206	565.47645697892\\
207	538.068890346742\\
208	497.993886116829\\
209	474.618347184779\\
210	453.18918533489\\
211	435.283167106067\\
212	423.629160056077\\
213	407.734068314516\\
214	396.004246011281\\
215	389.703037528818\\
216	380.066411064617\\
217	364.477497169878\\
218	360.817356683635\\
219	338.508564990661\\
220	340.073057503001\\
221	334.613591183148\\
222	327.439812538694\\
223	326.527559352029\\
224	322.556127734252\\
225	323.73198393364\\
226	313.276703362812\\
227	304.832071052304\\
228	304.683277317955\\
229	298.651632074311\\
230	303.512913711326\\
231	299.258048382335\\
232	297.565442021007\\
233	285.286353284321\\
234	287.727828408757\\
235	287.718658238076\\
236	288.485121606531\\
237	274.712912254342\\
238	275.987965538583\\
239	266.445329343585\\
240	268.368402100186\\
241	265.223415186079\\
242	271.227053805237\\
243	264.607198911506\\
244	263.287876097651\\
245	265.753086305707\\
246	267.666706940957\\
247	267.345503753095\\
248	256.178579535151\\
249	255.998988660238\\
250	258.05980779006\\
251	263.611278757491\\
252	264.198981830757\\
253	267.031251696537\\
254	268.74677689552\\
255	263.141056527177\\
256	262.431210221851\\
};
\addlegendentry{Scatterer power}

\addplot [color=green]
  table[row sep=crcr]{%
0	158835.306841345\\
300	158835.306841345\\
};
\addlegendentry{Average power of all scatterers}

\addplot [color=black, only marks, mark size=4.0pt, mark=o, mark options={solid, black}]
  table[row sep=crcr]{%
106	492345.285552265\\
107	722422.843009396\\
125	382144.551729205\\
126	1567806.36829179\\
127	23222830.6558939\\
128	4866233.18616336\\
129	624325.939865116\\
147	558849.855314574\\
148	675616.999291431\\
};
\addlegendentry{Candidate scatterers}

\addplot [color=red, only marks, mark size=7.5pt, mark=o, mark options={solid, red}]
  table[row sep=crcr]
                \caption{.\label{subfig:hayRA_sim_hrrp}}
            \end{subfigure}
             &
            \begin{subfigure}{0.25\linewidth}
                \centering
                \resizebox{\linewidth}{!}{% This file was created by matlab2tikz.
%
%The latest updates can be retrieved from
%  http://www.mathworks.com/matlabcentral/fileexchange/22022-matlab2tikz-matlab2tikz
%where you can also make suggestions and rate matlab2tikz.
%
\definecolor{mycolor1}{rgb}{0.00000,0.44700,0.74100}%
%
\begin{tikzpicture}

\begin{axis}[%
width=6.028in,
height=4.754in,
at={(1.011in,0.642in)},
scale only axis,
xmin=0,
xmax=256,
xlabel style={font=\fontsize{25}{20}\selectfont\color{black}, yshift = -10},
xlabel={Range Bin},
ymin=0,
ymax=0.2e7,
ylabel style={font=\fontsize{25}{20}\selectfont\color{black}, yshift=10pt},
ylabel={Power},
axis background/.style={fill=white},
tick label style={font=\fontsize{20}{11}\selectfont\color{black}},
xtick distance = 50,
ytick distance= 0.2e6,
yticklabel={\ifdim\tick pt=0pt\else\pgfmathprintnumber{\tick}\fi}, 
scaled y ticks=base 10:-6,
legend style={legend cell align=left, align=left, draw=white!15!black, font=\fontsize{12}{11}\selectfont\color{black}}
]
\addplot [color=mycolor1, mark=asterisk, mark options={solid, mycolor1}]
  table[row sep=crcr]{%
1	267.590403019806\\
2	255.400968720464\\
3	264.999912312277\\
4	262.539753606501\\
5	261.818417045418\\
6	259.15658133819\\
7	264.817711251638\\
8	262.058780690642\\
9	262.840089084763\\
10	266.144147892013\\
11	267.54053374168\\
12	268.812123087204\\
13	266.485018620937\\
14	266.321566167241\\
15	270.424770217735\\
16	282.047638635658\\
17	286.386800530431\\
18	283.397256119929\\
19	277.299315120883\\
20	286.016840486917\\
21	290.983367844274\\
22	285.858487721665\\
23	293.132533075198\\
24	281.135394758485\\
25	298.253256486322\\
26	297.893202974783\\
27	303.787988123915\\
28	298.598345463695\\
29	317.925516046478\\
30	316.977325686178\\
31	320.098331654412\\
32	331.157223148045\\
33	345.514345303823\\
34	333.794692741485\\
35	338.767343531348\\
36	340.100850905873\\
37	365.176234348452\\
38	372.220240788854\\
39	371.123466580276\\
40	395.974354115365\\
41	394.878451441685\\
42	415.752888381649\\
43	426.594244085755\\
44	433.003089511907\\
45	453.923958176231\\
46	476.476159925392\\
47	503.550671601751\\
48	541.64219679519\\
49	567.793341857371\\
50	642.477554924576\\
51	692.489020420194\\
52	794.003581796397\\
53	928.187586114199\\
54	1173.74462908604\\
55	1605.5889473773\\
56	2585.31945511748\\
57	5820.11088797017\\
58	40512.0024360843\\
59	234637.484959815\\
60	16841.0078036831\\
61	4789.1001667563\\
62	3235.72345396025\\
63	3621.671124788\\
64	7239.74525142419\\
65	76426.8531261069\\
66	213431.257296552\\
67	10058.9398378074\\
68	4515.78881683889\\
69	3533.05011180871\\
70	4122.91917015136\\
71	8118.33604195623\\
72	127267.481225726\\
73	167263.308179538\\
74	9616.46683114758\\
75	4967.89555999995\\
76	4307.28710110246\\
77	5197.775364982\\
78	10335.3220378342\\
79	182284.419488649\\
80	108159.966044409\\
81	8621.12783580651\\
82	4345.11299173339\\
83	3870.44077563474\\
84	5222.66415707551\\
85	14004.491689984\\
86	227352.392122548\\
87	63160.7997367143\\
88	8154.67252546312\\
89	4139.70776610491\\
90	3438.24643259268\\
91	4860.24406921386\\
92	22698.2058656717\\
93	244742.461098225\\
94	37575.5521866883\\
95	10451.6123049568\\
96	7133.19454891044\\
97	6681.78274838573\\
98	9278.44061855741\\
99	51591.9260422492\\
100	239784.107740901\\
101	20964.6344712149\\
102	12396.3888609053\\
103	12941.7533465924\\
104	18625.6476907128\\
105	50418.5631415139\\
106	491958.821642342\\
107	722240.762895506\\
108	42885.4646331668\\
109	18959.5021376887\\
110	13278.4529976445\\
111	11908.5376459887\\
112	14192.120205667\\
113	161433.670830332\\
114	207629.120295204\\
115	32180.8378020127\\
116	26657.8956947516\\
117	27986.7983238225\\
118	33437.4799171163\\
119	47362.1644074165\\
120	242716.673323717\\
121	113430.417500742\\
122	67857.6266078146\\
123	101318.162526693\\
124	175545.020154429\\
125	381883.532548747\\
126	1567961.42840303\\
127	23221213.0950726\\
128	4865888.48976607\\
129	624544.880595286\\
130	238973.040538831\\
131	125587.29946538\\
132	78308.7898539806\\
133	73037.3974013629\\
134	271150.038317276\\
135	65995.0943438357\\
136	35703.7422212397\\
137	27633.7808084823\\
138	24599.5035856174\\
139	27019.9191448833\\
140	93374.2459183023\\
141	280533.979110995\\
142	15226.2813621478\\
143	11640.2633507886\\
144	12759.534727283\\
145	18334.5773873386\\
146	54271.533584491\\
147	558933.217235893\\
148	675903.495221173\\
149	46977.2541009707\\
150	20387.1812894993\\
151	13564.6968715487\\
152	11740.3807702939\\
153	14834.6826789188\\
154	166262.022447405\\
155	129036.06070688\\
156	12895.1728444693\\
157	8328.85836700699\\
158	7889.91099433692\\
159	9630.51999193638\\
160	18107.3298460024\\
161	221470.143124542\\
162	66275.4401033131\\
163	6299.37000443946\\
164	3539.31390591582\\
165	3685.73411017274\\
166	5833.57351459431\\
167	21241.5801918213\\
168	249982.579358422\\
169	37267.5206214353\\
170	6662.50837550657\\
171	3815.86398452993\\
172	3656.65401558687\\
173	6170.72388839779\\
174	39691.5323123567\\
175	244179.195121135\\
176	19182.3434080757\\
177	6519.73352353916\\
178	4790.89753149547\\
179	5131.21312940206\\
180	8976.73409770214\\
181	79722.5873118596\\
182	213262.97070559\\
183	9170.79786938848\\
184	4350.72299991406\\
185	3741.46915392263\\
186	4617.38028479086\\
187	9303.87082312299\\
188	128787.013352456\\
189	162816.2429628\\
190	7590.51648705448\\
191	3494.63609691419\\
192	2887.46816905788\\
193	3794.70083757787\\
194	8580.58562399651\\
195	174660.466107146\\
196	107476.021551741\\
197	8153.64261900702\\
198	3227.70448374539\\
199	1896.24067264913\\
200	1317.72083921525\\
201	1026.6086873185\\
202	855.916133552541\\
203	751.700182898775\\
204	664.170504494067\\
205	613.657513818042\\
206	559.85004035324\\
207	535.728893655505\\
208	504.963024589523\\
209	468.145632839188\\
210	451.131468064014\\
211	434.298406952323\\
212	424.340171971602\\
213	416.483041566607\\
214	393.8252565115\\
215	390.151872640734\\
216	376.590918924518\\
217	361.839384165296\\
218	357.693505138926\\
219	352.112203810827\\
220	347.181768094572\\
221	342.338390012586\\
222	329.359845621858\\
223	321.603250396006\\
224	320.483416025978\\
225	315.446100007911\\
226	310.357583630252\\
227	317.482771819105\\
228	298.762186832923\\
229	303.209356418795\\
230	302.292389452814\\
231	298.112076351372\\
232	293.947221471244\\
233	283.999343889006\\
234	298.431349883647\\
235	272.52168599285\\
236	280.402929072728\\
237	290.74558337388\\
238	276.068587248429\\
239	274.299548595029\\
240	274.127617228544\\
241	277.252237791542\\
242	278.014441364256\\
243	270.725951432116\\
244	265.84818375836\\
245	273.575558865976\\
246	265.843545493986\\
247	260.809207537589\\
248	262.169072392346\\
249	259.541668522653\\
250	264.382464329194\\
251	254.785628117324\\
252	262.835190998783\\
253	261.143103280945\\
254	261.678918290097\\
255	256.043411432057\\
256	260.429945382733\\
};
\addlegendentry{Scatterer power}

\addplot [color=orange]
  table[row sep=crcr]{%
0	158828.655125224\\
300	158828.655125224\\
};
\addlegendentry{Average power of all scatterers}

\addplot [color=black, only marks, mark size=4.0pt, mark=o, mark options={solid, black}]
  table[row sep=crcr]{%
126	1567961.42840303\\
127	23221213.0950726\\
128	4865888.48976607\\
};
\addlegendentry{Candidate scatterers}

\addplot [color=red, only marks, mark size=7.5pt, mark=o, mark options={solid, red}]
  table[row sep=crcr]
                \caption{.\label{subfig:hayRA_sim_hrrp}}
            \end{subfigure}
             &
            \begin{subfigure}{0.25\linewidth}
                \centering
                \resizebox{\linewidth}{!}{% This file was created by matlab2tikz.
%
%The latest updates can be retrieved from
%  http://www.mathworks.com/matlabcentral/fileexchange/22022-matlab2tikz-matlab2tikz
%where you can also make suggestions and rate matlab2tikz.
%
\definecolor{mycolor1}{rgb}{0.00000,0.44700,0.74100}%
%
\begin{tikzpicture}

\begin{axis}[%
width=6.028in,
height=4.754in,
at={(1.011in,0.642in)},
scale only axis,
xmin=0,
xmax=256,
xlabel style={font=\fontsize{25}{20}\selectfont\color{black}, yshift = -10},
xlabel={Range Bin},
ymin=0,
ymax=25000000,
ylabel style={font=\fontsize{25}{20}\selectfont\color{black}, yshift=10pt},
ylabel={Power},
axis background/.style={fill=white},
tick label style={font=\fontsize{20}{11}\selectfont\color{black}},
xtick distance = 50,
ytick distance= 2e6,
yticklabel={\ifdim\tick pt=0pt\else\pgfmathprintnumber{\tick}\fi}, 
scaled y ticks=base 10:-6,
legend style={legend cell align=left, align=left, draw=white!15!black, font=\fontsize{12}{11}\selectfont\color{black}}
]
\addplot [color=mycolor1, mark=asterisk, mark options={solid, mycolor1}]
  table[row sep=crcr]{%
1	260.757090596185\\
2	261.349448266049\\
3	258.844308641979\\
4	260.99983616491\\
5	261.116115558909\\
6	269.196452711933\\
7	260.682193999795\\
8	265.471868197095\\
9	266.477079167473\\
10	269.232963922838\\
11	270.138464929005\\
12	273.043327361471\\
13	273.62564006987\\
14	271.454418528639\\
15	274.3904109416\\
16	265.005913372826\\
17	275.639398903274\\
18	275.37204559559\\
19	283.867254514651\\
20	277.277950830027\\
21	287.384264710496\\
22	305.495212060368\\
23	289.44774297217\\
24	296.354344666488\\
25	296.638641274841\\
26	298.557913899803\\
27	312.424493674491\\
28	314.80503284847\\
29	319.714817046923\\
30	311.437671023239\\
31	313.879886187225\\
32	324.263258015104\\
33	333.784946542075\\
34	331.685893596448\\
35	349.164151373045\\
36	344.6435903982\\
37	354.942594612574\\
38	371.193024662765\\
39	377.371153339826\\
40	392.017816661179\\
41	385.536437438418\\
42	409.673280656308\\
43	422.209555724421\\
44	444.675193605828\\
45	454.432321135569\\
46	484.643775551396\\
47	504.451341148726\\
48	535.363082639264\\
49	572.071889434024\\
50	629.279327276237\\
51	701.308690678024\\
52	801.060800191796\\
53	947.102083174944\\
54	1170.73479266727\\
55	1607.02942933746\\
56	2595.55981309361\\
57	5812.86960576437\\
58	40487.3221677515\\
59	234662.60643436\\
60	16833.6285145892\\
61	4758.92392697897\\
62	3271.14525107973\\
63	3630.14568751605\\
64	7223.30318137183\\
65	76580.8108950504\\
66	213496.653023221\\
67	10148.190287012\\
68	4495.3497625272\\
69	3546.12800755943\\
70	4098.38020015848\\
71	8125.86219433519\\
72	126892.724550999\\
73	167250.020353373\\
74	9548.47811677503\\
75	4980.71467107675\\
76	4318.33386493922\\
77	5190.87048619526\\
78	10346.5620214141\\
79	182348.437135178\\
80	108122.579606752\\
81	8609.16567638802\\
82	4360.65851422563\\
83	3888.12324901019\\
84	5216.15943569194\\
85	14003.8628301505\\
86	227292.38173834\\
87	63274.6606646977\\
88	8201.67532897642\\
89	4147.86112014355\\
90	3483.13936838907\\
91	4882.03074468776\\
92	22649.03280625\\
93	244616.968724519\\
94	37573.9015011674\\
95	10541.5161574813\\
96	7094.75508933476\\
97	6689.33290583644\\
98	9235.0911245946\\
99	51536.1795732728\\
100	239639.456491946\\
101	21056.1015168124\\
102	12359.261566368\\
103	12948.1582508674\\
104	18627.4332582689\\
105	50401.4629805547\\
106	492414.995962025\\
107	722555.790212155\\
108	42965.7265527986\\
109	18963.2819331697\\
110	13311.5311164789\\
111	11921.0556504423\\
112	14186.432709524\\
113	161339.956565044\\
114	207474.526112217\\
115	32250.1735407387\\
116	26601.9336512659\\
117	28100.2828981279\\
118	33496.491222651\\
119	47460.4820517101\\
120	242881.500866766\\
121	113595.258541256\\
122	67922.9767501688\\
123	101375.484833191\\
124	175724.16231007\\
125	381962.197550843\\
126	1567886.22087659\\
127	23225975.3227313\\
128	4867247.19796107\\
129	624310.85052159\\
130	238975.098769047\\
131	125608.799294704\\
132	78090.4619349091\\
133	73018.8693531322\\
134	271074.146369045\\
135	65882.5404193608\\
136	35525.7194041419\\
137	27657.6671044679\\
138	24587.4723789782\\
139	26971.1195555857\\
140	93342.5206466693\\
141	280414.579199937\\
142	15249.1463940073\\
143	11661.2594790902\\
144	12743.3589792486\\
145	18353.576908794\\
146	54236.1111496537\\
147	559055.020673585\\
148	675705.880270755\\
149	46874.188812739\\
150	20348.0267881837\\
151	13616.2338193232\\
152	11723.911314611\\
153	14819.5513495774\\
154	166181.518183832\\
155	129254.288287589\\
156	12857.1380900099\\
157	8430.18179664529\\
158	7945.97175861454\\
159	9639.56336298348\\
160	18204.0485913735\\
161	221285.667916626\\
162	66139.6893443756\\
163	6313.52246622126\\
164	3568.26402514372\\
165	3694.14761048191\\
166	5811.51887875028\\
167	21221.8089970754\\
168	249993.262788285\\
169	37320.4904955616\\
170	6679.95817225113\\
171	3826.64403605272\\
172	3642.44554922554\\
173	6174.66338766456\\
174	39707.6934224181\\
175	244029.642915695\\
176	19122.757190502\\
177	6542.09172884734\\
178	4781.16967629281\\
179	5147.87783102795\\
180	8975.55892262907\\
181	79715.8622930272\\
182	213292.406214932\\
183	9185.52391495871\\
184	4380.04015306461\\
185	3748.55640973686\\
186	4597.55027155746\\
187	9191.68584402259\\
188	128933.104580069\\
189	162758.12802332\\
190	7561.38022797859\\
191	3471.68110990261\\
192	2897.2416018114\\
193	3737.36637371086\\
194	8599.90672828641\\
195	174707.94076613\\
196	107383.013167702\\
197	8104.43294227265\\
198	3217.87968532673\\
199	1901.12701656705\\
200	1331.03929132751\\
201	1027.99400712673\\
202	849.0161722243\\
203	750.237099974916\\
204	661.652776163157\\
205	605.086109408137\\
206	563.913893540252\\
207	531.67619600442\\
208	505.258278111065\\
209	479.494330188539\\
210	451.569531216314\\
211	443.698752243901\\
212	422.153400559185\\
213	406.375520103256\\
214	399.500362473823\\
215	399.441866692462\\
216	371.646522447958\\
217	372.725875540429\\
218	363.155705021535\\
219	348.298016776292\\
220	347.376026579026\\
221	336.307096456843\\
222	334.952486711036\\
223	328.401028502431\\
224	326.106363118331\\
225	319.905973550356\\
226	316.802587711091\\
227	312.82705969179\\
228	302.834591083687\\
229	299.488978776006\\
230	299.71382840282\\
231	296.923606461388\\
232	291.777170220202\\
233	299.586958187809\\
234	279.73348394927\\
235	287.318775687667\\
236	274.951389735924\\
237	282.884456518634\\
238	284.503755622997\\
239	280.142152545547\\
240	280.73720127584\\
241	271.453368629526\\
242	275.304240428404\\
243	271.207453570855\\
244	275.410068139259\\
245	273.594484812166\\
246	265.60622881618\\
247	262.967106050081\\
248	255.950228236889\\
249	264.682395020898\\
250	257.875867486201\\
251	269.323143719703\\
252	264.00100589459\\
253	264.238928945338\\
254	270.479568204838\\
255	262.719731607442\\
256	261.114946353416\\
};
\addlegendentry{Scatterer power}

\addplot [color=orange]
  table[row sep=crcr]{%
0	158852.856776151\\
300	158852.856776151\\
};
\addlegendentry{Average power of all scatterers}

\addplot [color=black, only marks, mark size=4.0pt, mark=o, mark options={solid, black}]
  table[row sep=crcr]{%
127	23225975.3227313\\
128	4867247.19796107\\
};
\addlegendentry{Candidate scatterers}

\addplot [color=red, only marks, mark size=7.5pt, mark=o, mark options={solid, red}]
  table[row sep=crcr]
                \caption{.\label{subfig:hayRA_sim_hrrp}}
            \end{subfigure}
        \end{tabular}
        \caption{ \label{fig:hayRA_sim}}
    \end{minipage}
    \end{figure}



% ----------------------------------------------------
\ifstandalone
\bibliography{../Bibliography/References.bib}
\printnoidxglossary[type=\acronymtype,nonumberlist]
\fi
\end{document}
% ----------------------------------------------------
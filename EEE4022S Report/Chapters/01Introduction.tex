% ----------------------------------------------------
% Introduction
% ----------------------------------------------------
\documentclass[class=report,11pt,crop=false]{standalone}
% Page geometry
\usepackage[a4paper,margin=20mm,top=25mm,bottom=25mm]{geometry}

% Font choice
\usepackage{lmodern}

\usepackage{lipsum}

% Use IEEE bibliography style
\bibliographystyle{IEEEtran}

% Line spacing
\usepackage{setspace}
\setstretch{1.2}

% Ensure UTF8 encoding
\usepackage[utf8]{inputenc}

% Language standard (not too important)
\usepackage[english]{babel}

% Skip a line in between paragraphs
\usepackage{parskip}

% For the creation of dummy text
\usepackage{blindtext}

% Math
\usepackage{amsmath}

% Header & Footer stuff
\usepackage{fancyhdr}
\pagestyle{fancy}
\fancyhead{}
\fancyhead[R]{\nouppercase{\rightmark}}
\fancyfoot{}
\fancyfoot[C]{\thepage}
\renewcommand{\headrulewidth}{0.0pt}
\renewcommand{\footrulewidth}{0.0pt}
\setlength{\headheight}{13.6pt}

% Epigraphs
\usepackage{epigraph}
\setlength\epigraphrule{0pt}
\setlength{\epigraphwidth}{0.65\textwidth}

% Colour
\usepackage{color}
\usepackage[usenames,dvipsnames]{xcolor}

% Hyperlinks & References
\usepackage{hyperref}
\definecolor{linkColour}{RGB}{77,71,179}
\definecolor{urlColour}{RGB}{255, 179, 102}

\hypersetup{
    colorlinks=true,
    linkcolor=linkColour,
    filecolor=linkColour,
    urlcolor=urlColour,
    citecolor=linkColour,
}
\urlstyle{same}

% Automatically correct front-side quotes
\usepackage[autostyle=false, style=ukenglish]{csquotes}
\MakeOuterQuote{"}

% Graphics
\usepackage{graphicx}
\graphicspath{{Figures/}{../Figures/}}
\usepackage{makecell}
\usepackage{transparent}
\usepackage{pgfplots}
\pgfplotsset{compat=newest}
%% the following commands are needed for some matlab2tikz features
\usetikzlibrary{plotmarks}
\usetikzlibrary{arrows.meta}
\usepgfplotslibrary{patchplots}

% SI units
\usepackage{siunitx}

% Microtype goodness
\usepackage{microtype}

% Listings
\usepackage[T1]{fontenc}
\usepackage{listings}
\usepackage[scaled=0.8]{DejaVuSansMono}

% Custom colours for listings
\definecolor{backgroundColour}{RGB}{250,250,250}
\definecolor{commentColour}{RGB}{73, 175, 102}
\definecolor{identifierColour}{RGB}{196, 19, 66}
\definecolor{stringColour}{RGB}{252, 156, 30}
\definecolor{keywordColour}{RGB}{50, 38, 224}
\definecolor{lineNumbersColour}{RGB}{127,127,127}
\lstset{
  language=Matlab,
  captionpos=b,
  aboveskip=15pt,belowskip=10pt,
  backgroundcolor=\color{backgroundColour},
  basicstyle=\ttfamily,%\footnotesize,        % the size of the fonts that are used for the code
  breakatwhitespace=false,         % sets if automatic breaks should only happen at whitespace
  breaklines=true,                 % sets automatic line breaking
  postbreak=\mbox{\textcolor{red}{$\hookrightarrow$}\space},
  commentstyle=\color{commentColour},    % comment style
  identifierstyle=\color{identifierColour},
  stringstyle=\color{stringColour},
   keywordstyle=\color{keywordColour},       % keyword style
  %escapeinside={\%*}{*)},          % if you want to add LaTeX within your code
  extendedchars=true,              % lets you use non-ASCII characters; for 8-bits encodings only, does not work with UTF-8
  frame=single,	                   % adds a frame around the code
  keepspaces=true,                 % keeps spaces in text, useful for keeping indentation of code (possibly needs columns=flexible)
  morekeywords={*,...},            % if you want to add more keywords to the set
  numbers=left,                    % where to put the line-numbers; possible values are (none, left, right)
  numbersep=5pt,                   % how far the line-numbers are from the code
  numberstyle=\tiny\color{lineNumbersColour}, % the style that is used for the line-numbers
  rulecolor=\color{black},         % if not set, the frame-color may be changed on line-breaks within not-black text (e.g. comments (green here))
  showspaces=false,                % show spaces everywhere adding particular underscores; it overrides 'showstringspaces'
  showstringspaces=false,          % underline spaces within strings only
  showtabs=false,                  % show tabs within strings adding particular underscores
  stepnumber=1,                    % the step between two line-numbers. If it's 1, each line will be numbered
  tabsize=2,	                   % sets default tabsize to 2 spaces
  %title=\lstname                   % show the filename of files included with \lstinputlisting; also try caption instead of title
}

% Caption stuff
\usepackage[hypcap=true, justification=centering]{caption}
\usepackage{subcaption}

% Glossary package
% \usepackage[acronym]{glossaries}
\usepackage{glossaries-extra}
\setabbreviationstyle[acronym]{long-short}

% For Proofs & Theorems
\usepackage{amsthm}

% Maths symbols
\usepackage{amssymb}
\usepackage{mathrsfs}
\usepackage{mathtools}

% For algorithms
\usepackage[]{algorithm2e}

% Spacing stuff
\setlength{\abovecaptionskip}{5pt plus 3pt minus 2pt}
\setlength{\belowcaptionskip}{5pt plus 3pt minus 2pt}
\setlength{\textfloatsep}{10pt plus 3pt minus 2pt}
\setlength{\intextsep}{15pt plus 3pt minus 2pt}

% For aligning footnotes at bottom of page, instead of hugging text
\usepackage[bottom]{footmisc}

% Add LoF, Bib, etc. to ToC
\usepackage[nottoc]{tocbibind}

% SI
\usepackage{siunitx}

% For removing some whitespace in Chapter headings etc
\usepackage{etoolbox}
\makeatletter
\patchcmd{\@makechapterhead}{\vspace*{50\p@}}{\vspace*{-10pt}}{}{}%
\patchcmd{\@makeschapterhead}{\vspace*{50\p@}}{\vspace*{-10pt}}{}{}%
\makeatother

% Wrap figure
\usepackage{wrapfig}
\makenoidxglossaries

\newacronym{af}{AF}{Autofocus}
\newacronym{cli}{CLI}{Command-line Interface}
\newacronym{cpi}{CPI}{Coherent Processing Interval}
\newacronym{cptwl}{CPTWL}{Coherent Processing Time Window Length}
\newacronym{cw}{CW}{Continuous Waveform}
\newacronym{ds}{DS}{Dominant Scatterer}
\newacronym{fft}{FFT}{Fast Fourier Transform}
\newacronym{fmcw}{FMCW}{Frequency Modulated Continuous Waveform} % Not sure
\newacronym{hrr}{HRR}{High Resolution Range}
\newacronym{hrrp}{HRRP}{High Resolution Range Profile}
\newacronym{ic}{IC}{Image Contrast}
\newacronym{isar}{ISAR}{Inverse Synthetic Aperture Radar}
\newacronym{jtf}{JTF}{Joint Time-Frequency}
\newacronym{pri}{PRI}{Pulse Repetition Interval}
\newacronym{prf}{PRF}{Pulse Repetition Frequency}
\newacronym{qlp}{QLP}{Quick-look Processor}
\newacronym{ra}{RA}{Range Alignment}
\newacronym{rlos}{RLOS}{Radar Line of Sight}
\newacronym{rmc}{RMC}{Rotational Motion Compensation}
\newacronym{sfw}{SFW}{Stepped Frequency Waveform}
\newacronym{sf}{SF}{Scaling Factor for Haywood Autofocus}
\newacronym{sar}{SAR}{Synthetic Aperture Radar}
\newacronym{snr}{SNR}{Signal-to-Noise Ratio}
\newacronym{sir}{SIR}{Signal-to-Interference Ratio}
\newacronym{tmc}{TMC}{Translational Motion Compensation}


\begin{document}
% ----------------------------------------------------
\chapter{Introduction}
% ----------------------------------------------------
This report details the development of a \gls{qlp} for \gls{isar} imaging, specifically in the context of sea vessels. In this report, a \gls{qlp} is an \gls{isar} image processing tool with a shorter processing time than the measurement time of the data it is processing. This is useful for validating field experimentation setups to ensure the collection of high-quality data. The project involved research into \gls{isar} imaging and the implementation of low computation \gls{isar} image processing algorithms. These algorithms were used in the design and implementation of a \gls{qlp}, which was validated using multiple measured datasets.

\section{Background}
Radar images contain rich information that can be used for imaging and classification of objects \cite{ISARtextbook_Martorella}. In the past, sea vessels could not be classified because of their non-cooperative \footnote{A term used to describe an object with unknown motion.} motion and low range resolution. However, the introduction of \gls{isar}, a radar imaging technique that does not require knowledge of the object's motion, and the development of radar systems with high range resolution capability has made imaging and classification of sea vessels possible. These advancements have had several applications in civilian wide-area persistent maritime surveillance \cite{quick-look_detection}. 

\gls{hrr} data can be used to produce a 2-D radar image of an object without a priori of the motion of the object. However, these images are often blurred due to both the 3-D rotation and translation motion of the object. Sea vessels often have both types of motion which change drastically over time. This variability makes it challenging to achieve the necessary 2-D rotational motion required to generate an image that resembles the object. To compensate for the effects of translation motion, \gls{ra} and \gls{af} algorithms can be used. Many \gls{ra} and \gls{af} algorithms with varying levels of computational-complexity exist to focus the image \cite{ISARtextbook_Martorella,ISARtextbook_Matlab}.

\gls{isar} imaging techniques are an active area of research and, as discussed in \cite{quick-look_detection}, \gls{hrr} and high \gls{snr} is required to investigate the performance of these techniques. When collecting data for \gls{isar} research, it is common practice to dedicate days to field work and only begin processing of the data in following months. \gls{isar} data may be obtained using pulsed wave radar systems. These pulsed wave forms have several parameters that can be changed and which affect the quality of the recorded data. Additionally, the systems are susceptible to noise and interference which further affects the quality of the data. However, until the data is processed, its quality cannot be assessed.

In other research areas such as detection of small sea vessels \cite{quick-look_detection} and airborne \gls{sar} \cite{quick-look_SAR}, low-complexity and robust \gls{qlp}s have been built to aid in assessing data quality. These systems provide real-time feedback of equipment setup during field work which can be used to confim correct operation of various systems. The low-complexity aspect removes the need for expensive, specialised, and high processing power machines to be used in the field.

%%%%%%%%%%%%%%%%%%%%%%%%%%%%%%%%%%%%%%%%%%%%%%%%%%%%%%%%%%%%%%%%%%%%%%%%%%%%%%%%%%%
\section{Problem Statement}
\gls{isar} image processing is a specialised technique that requires \gls{hrr} data to produce a 2-D image of an object. The object's translation motion complicates the imaging process and is compensated by using \gls{ra} and \gls{af} techniques. Considerable research has been conducted on these algorithms, and the progress of this research depends on the availability of high-quality data. However, the quality of \gls{isar} data can only be assessed in the months following data collection.

Whilst research has been conducted on different techniques, there is insufficient work in the literature on comparing the performance of different \gls{ra} and \gls{af} algorithms performed on the same dataset \cite{4022}. Additionally, advancements have been made on the development of computationally expensive approaches to improve image focus \cite{ISARtextbook_Martorella}, however little insight into the perfomance of low computation algorithms is available. 

The problem addressed in this project is producing highly focused \gls{isar} images of sea vessels using low computational \gls{ra} and \gls{af} algorithms. This is beneficial to researchers for use in assessing the quality of collected data  to correct problems with equipment used in the experimental setup.

%%%%%%%%%%%%%%%%%%%%%%%%%%%%%%%%%%%%%%%%%%%%%%%%%%%%%%%%%%%%%%%%%%%%%%%%%%%%%%%%%%%
\section{Objectives}\label{sec:intro_objectives}
The broad objective of this project was to design a \gls{qlp} for \gls{isar} imaging of sea vessel \gls{isar} data. The following list outlines the objectives
\begin{itemize}
    \item Select and implement low-computation \gls{tmc} techniques to focus the \gls{isar} images. Verify and validate the implementation of these techniques before use in the \gls{qlp} and improve robustness where possible.
    \item Investigate and compare the performance of different combinations of the implemented algorithms. Select the best performance \gls{ra} and \gls{af} algorithm for use in the \gls{qlp}. In the selection step, assess both the computational speed and quality of the focused image.
    \item Develop a \gls{qlp} that uses \gls{isar} data to produce an \gls{isar} movie and is suitable for use in the field.
\end{itemize}

%%%%%%%%%%%%%%%%%%%%%%%%%%%%%%%%%%%%%%%%%%%%%%%%%%%%%%%%%%%%%%%%%%%%%%%%%%%%%%%%%%%
\section{Technical Requirements}
Based on the objectives of the project discussed in \autoref{sec:intro_objectives}, the following technical requirements were defined
\begin{itemize}
    \item The \gls{qlp} should process the data in a shorter time than the measurement time of the data. 
    \item The \gls{qlp} should produce a more focused \gls{isar} image after \gls{tmc}.
    \item The \gls{qlp} should be user-friendly.
\end{itemize}
%%%%%%%%%%%%%%%%%%%%%%%%%%%%%%%%%%%%%%%%%%%%%%%%%%%%%%%%%%%%%%%%%%%%%%%%%%%%%%%%%%%
\section{Scope \& Limitations}
The time available to complete the project was approximately 13 weeks which prevented thorough research into all aspects of the problem. 

The scope of the project involved implementing two \gls{ra} and two \gls{af} low computation algorithms in \textsc{MATLAB}. Followed by verification and validation of these algorithms as well as improvements to the robustness and the processing time of each algorithm. These algorithms were candidate algorithms for the \gls{qlp} for \gls{isar} imaging of sea vessels. Each candidate algorithm was validated in terms of image quality and processing time. Thereafter, the best \gls{ra} and \gls{af} combination was selected for the \gls{qlp}. Next, the \gls{qlp} was applied to a few measured datasets of sea vessels.

There were several project limitations. First, only \gls{ra} and \gls{af} \gls{tmc} algorithms were considered and, no \gls{rmc} or other processes were investigated or implemented to attempt to focus the \gls{isar} image. No new radar data was collected during the project, rather measured datasets from the \gls{csir} were used. \textsc{MATLAB} was selected as the software platform for this project due to its efficient matrix operation features. The effect of \gls{cpi} on the focus of the images was not discussed or investigated, and the \gls{cpi} value was kept constant throughout the report. Finally, the scope of the project was limited to radar datasets of sea vessels measured using a \gls{sfw}.
%%%%%%%%%%%%%%%%%%%%%%%%%%%%%%%%%%%%%%%%%%%%%%%%%%%%%%%%%%%%%%%%%%%%%%%%%%%%%%%%%%%%%
\section{Original Contributions}
During the implementation of the algorithms considered in this report, various adjustments were made to enhance the robustness and quality of the \gls{isar} images generated. These adjustments include:
\begin{itemize}
    \item For \gls{sdsaf}, a technique for selecting a higher power \gls{ds} was introduced to improve the focus of the images.
    \item For \gls{mdsaf}, several modifications were made to \cite{yuan_AF}'s algorithm to improve the focus of the images. A noise filtering technique was introduced to ensure that the algorithm only selected scatterers within the object to focus the image. Additionally, \cite{yuan_AF} suggested using 6-18 scatterers, but it was determined that this range was not always feasible. Consequently, the highest number below eleven was chosen.  Finally, \cite{yuan_AF} did not specify how to select the reference profile used in focusing the image, a technique was introduced for systematically selecting the profile. 
\end{itemize}

Additionally, the literature survey revealed that while research on \gls{qlp}s has been conducted for other radar applications \cite{quick-look_detection,quick-look_SAR}, there is limited research pertaining to \gls{qlp}s for\gls{isar} imaging of sea vessels. This report presents the design and implementation of a low-complexity \gls{qlp} to address this research gap.

%%%%%%%%%%%%%%%%%%%%%%%%%%%%%%%%%%%%%%%%%%%%%%%%%%%%%%%%%%%%%%%%%%%%%%%%%%%%%%%%%%%%%
\section{Report Outline}
The report begins with a theoretical background to develop \gls{isar} imaging concepts in Chapter 2. Thereafter, a broad review of literature relevant to this project in Chapter 3. Chapter 4 then describes the verification and validation of the \gls{ra} and \gls{af} algorithms implemented in this report. These algorithms are considered and evaluated for use in the \gls{qlp} system design discussed in Chapter 5. Thereafter, the system is validated in Chapter 6. Finally, Chapter 7 provides the conclusions drawn and is followed by recommendations for future work. Appendices provide additional results for sections throughout the report.
% ----------------------------------------------------
\ifstandalone
\bibliography{../Bibliography/References.bib}
\printnoidxglossary[type=\acronymtype,nonumberlist]
\fi
\end{document}
% ----------------------------------------------------
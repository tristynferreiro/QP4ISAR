% ----------------------------------------------------
% ----------------------------------------------------
% Design
% ----------------------------------------------------
\documentclass[class=report,11pt,crop=false]{standalone}
% Page geometry
\usepackage[a4paper,margin=20mm,top=25mm,bottom=25mm]{geometry}

% Font choice
\usepackage{lmodern}

\usepackage{lipsum}

% Use IEEE bibliography style
\bibliographystyle{IEEEtran}

% Line spacing
\usepackage{setspace}
\setstretch{1.2}

% Ensure UTF8 encoding
\usepackage[utf8]{inputenc}

% Language standard (not too important)
\usepackage[english]{babel}

% Skip a line in between paragraphs
\usepackage{parskip}

% For the creation of dummy text
\usepackage{blindtext}

% Math
\usepackage{amsmath}

% Header & Footer stuff
\usepackage{fancyhdr}
\pagestyle{fancy}
\fancyhead{}
\fancyhead[R]{\nouppercase{\rightmark}}
\fancyfoot{}
\fancyfoot[C]{\thepage}
\renewcommand{\headrulewidth}{0.0pt}
\renewcommand{\footrulewidth}{0.0pt}
\setlength{\headheight}{13.6pt}

% Epigraphs
\usepackage{epigraph}
\setlength\epigraphrule{0pt}
\setlength{\epigraphwidth}{0.65\textwidth}

% Colour
\usepackage{color}
\usepackage[usenames,dvipsnames]{xcolor}

% Hyperlinks & References
\usepackage{hyperref}
\definecolor{linkColour}{RGB}{77,71,179}
\definecolor{urlColour}{RGB}{255, 179, 102}

\hypersetup{
    colorlinks=true,
    linkcolor=linkColour,
    filecolor=linkColour,
    urlcolor=urlColour,
    citecolor=linkColour,
}
\urlstyle{same}

% Automatically correct front-side quotes
\usepackage[autostyle=false, style=ukenglish]{csquotes}
\MakeOuterQuote{"}

% Graphics
\usepackage{graphicx}
\graphicspath{{Figures/}{../Figures/}}
\usepackage{makecell}
\usepackage{transparent}
\usepackage{pgfplots}
\pgfplotsset{compat=newest}
%% the following commands are needed for some matlab2tikz features
\usetikzlibrary{plotmarks}
\usetikzlibrary{arrows.meta}
\usepgfplotslibrary{patchplots}

% SI units
\usepackage{siunitx}

% Microtype goodness
\usepackage{microtype}

% Listings
\usepackage[T1]{fontenc}
\usepackage{listings}
\usepackage[scaled=0.8]{DejaVuSansMono}

% Custom colours for listings
\definecolor{backgroundColour}{RGB}{250,250,250}
\definecolor{commentColour}{RGB}{73, 175, 102}
\definecolor{identifierColour}{RGB}{196, 19, 66}
\definecolor{stringColour}{RGB}{252, 156, 30}
\definecolor{keywordColour}{RGB}{50, 38, 224}
\definecolor{lineNumbersColour}{RGB}{127,127,127}
\lstset{
  language=Matlab,
  captionpos=b,
  aboveskip=15pt,belowskip=10pt,
  backgroundcolor=\color{backgroundColour},
  basicstyle=\ttfamily,%\footnotesize,        % the size of the fonts that are used for the code
  breakatwhitespace=false,         % sets if automatic breaks should only happen at whitespace
  breaklines=true,                 % sets automatic line breaking
  postbreak=\mbox{\textcolor{red}{$\hookrightarrow$}\space},
  commentstyle=\color{commentColour},    % comment style
  identifierstyle=\color{identifierColour},
  stringstyle=\color{stringColour},
   keywordstyle=\color{keywordColour},       % keyword style
  %escapeinside={\%*}{*)},          % if you want to add LaTeX within your code
  extendedchars=true,              % lets you use non-ASCII characters; for 8-bits encodings only, does not work with UTF-8
  frame=single,	                   % adds a frame around the code
  keepspaces=true,                 % keeps spaces in text, useful for keeping indentation of code (possibly needs columns=flexible)
  morekeywords={*,...},            % if you want to add more keywords to the set
  numbers=left,                    % where to put the line-numbers; possible values are (none, left, right)
  numbersep=5pt,                   % how far the line-numbers are from the code
  numberstyle=\tiny\color{lineNumbersColour}, % the style that is used for the line-numbers
  rulecolor=\color{black},         % if not set, the frame-color may be changed on line-breaks within not-black text (e.g. comments (green here))
  showspaces=false,                % show spaces everywhere adding particular underscores; it overrides 'showstringspaces'
  showstringspaces=false,          % underline spaces within strings only
  showtabs=false,                  % show tabs within strings adding particular underscores
  stepnumber=1,                    % the step between two line-numbers. If it's 1, each line will be numbered
  tabsize=2,	                   % sets default tabsize to 2 spaces
  %title=\lstname                   % show the filename of files included with \lstinputlisting; also try caption instead of title
}

% Caption stuff
\usepackage[hypcap=true, justification=centering]{caption}
\usepackage{subcaption}

% Glossary package
% \usepackage[acronym]{glossaries}
\usepackage{glossaries-extra}
\setabbreviationstyle[acronym]{long-short}

% For Proofs & Theorems
\usepackage{amsthm}

% Maths symbols
\usepackage{amssymb}
\usepackage{mathrsfs}
\usepackage{mathtools}

% For algorithms
\usepackage[]{algorithm2e}

% Spacing stuff
\setlength{\abovecaptionskip}{5pt plus 3pt minus 2pt}
\setlength{\belowcaptionskip}{5pt plus 3pt minus 2pt}
\setlength{\textfloatsep}{10pt plus 3pt minus 2pt}
\setlength{\intextsep}{15pt plus 3pt minus 2pt}

% For aligning footnotes at bottom of page, instead of hugging text
\usepackage[bottom]{footmisc}

% Add LoF, Bib, etc. to ToC
\usepackage[nottoc]{tocbibind}

% SI
\usepackage{siunitx}

% For removing some whitespace in Chapter headings etc
\usepackage{etoolbox}
\makeatletter
\patchcmd{\@makechapterhead}{\vspace*{50\p@}}{\vspace*{-10pt}}{}{}%
\patchcmd{\@makeschapterhead}{\vspace*{50\p@}}{\vspace*{-10pt}}{}{}%
\makeatother

% Wrap figure
\usepackage{wrapfig}
\makenoidxglossaries

\newacronym{af}{AF}{Autofocus}
\newacronym{cli}{CLI}{Command-line Interface}
\newacronym{cpi}{CPI}{Coherent Processing Interval}
\newacronym{cptwl}{CPTWL}{Coherent Processing Time Window Length}
\newacronym{cw}{CW}{Continuous Waveform}
\newacronym{ds}{DS}{Dominant Scatterer}
\newacronym{fft}{FFT}{Fast Fourier Transform}
\newacronym{fmcw}{FMCW}{Frequency Modulated Continuous Waveform} % Not sure
\newacronym{hrr}{HRR}{High Resolution Range}
\newacronym{hrrp}{HRRP}{High Resolution Range Profile}
\newacronym{ic}{IC}{Image Contrast}
\newacronym{isar}{ISAR}{Inverse Synthetic Aperture Radar}
\newacronym{jtf}{JTF}{Joint Time-Frequency}
\newacronym{pri}{PRI}{Pulse Repetition Interval}
\newacronym{prf}{PRF}{Pulse Repetition Frequency}
\newacronym{qlp}{QLP}{Quick-look Processor}
\newacronym{ra}{RA}{Range Alignment}
\newacronym{rlos}{RLOS}{Radar Line of Sight}
\newacronym{rmc}{RMC}{Rotational Motion Compensation}
\newacronym{sfw}{SFW}{Stepped Frequency Waveform}
\newacronym{sf}{SF}{Scaling Factor for Haywood Autofocus}
\newacronym{sar}{SAR}{Synthetic Aperture Radar}
\newacronym{snr}{SNR}{Signal-to-Noise Ratio}
\newacronym{sir}{SIR}{Signal-to-Interference Ratio}
\newacronym{tmc}{TMC}{Translational Motion Compensation}


\begin{document}
\ifstandalone
\tableofcontents
\fi
% ----------------------------------------------------
\chapter{Algorithm Verification and Validation \label{ch:algorithmV&V}}
% ----------------------------------------------------
% Characteristics of the practical work section: implementation, simulation
% •	Practical work done was relevant. Quantity and quality of the work was suitable for the degree 
% •	Document how you captured experimental data. Include photos of experimental environment and equipment used. Record details of equipment configuration, calibration procedures. 
% •	Document sub-systems were tested and evaluate if technical specifications were met
% •	Document system testing and evaluate if technical specifications were met 

 % Characteristics of the results section:  analysis of measured data and interpretation of results
% •	Clear, logical, relevant, and accurate analysis of the data and interpretation of results 
% •	Important features of the data are highlighted to the reader. 
% •	The logical conclusions from the interpretations of the data are linked to the research objectives/questions of the work 
% •	Includes a discussion on how results in this work are similar or different to results presented in the literature. If similar, state reasons why its similar. If different, explain reasons for differences. Simply saying the results are similar or different is not sufficient. 
 
% Important elements when presenting graphs in your work
% •	Explain the axes of the graph 
% •	Identify the portion of the graphs that are important. Consider annotating your graphs to ease this process
% •	Interpret what the graph means in the context of your work

\emph{Chapter introduction: Clear motivations provided in selecting the approach used to validate and verify algorithms (why I used simulated and measured data)}
% •	Document simulations and verification tests to show accuracy of implemented code  
\textsc{MATLAB} by MathWorks \cite{mathworks}
\section{ISAR Simulator}
Simulated data is useful when verifying that the algorithms investigated and implemented in this report work as expected. The simulator is a \textsc{Matlab} script which was developed by \cite{matlab}. It uses a \gls{sfw} signal and user defined point scatterer x-y coordinates to produce \gls{hrrp}s. The configurable motion parameters are values for rotational motion (in degrees per second) and translation motion (in meters per second). The motion parameters are used to simulate range walk and range offset which the algorithms are meant to compensate for. A for loop is used to iterate over the \gls{cpi}, each iteration the scatterer array is moved by the set motion parameter amount. The output is the generated \gls{hrrp}s based on the motion input parameters for the configured scatterers.

    %***************************************************************************************%
    \subsection{Simulator Changes}
    The \href{https://github.com/tristynferreiro/QP4ISAR/blob/main/src/Simulator/Original%20Simulator%20Files/Version_3_object_translation_and_rotation_motion.m}{original} has been altered to better suit the data needed to verify the algorithms. \gls{ra} and \gls{af} algorithms use \gls{ds}s to correct the \gls{isar} image. The \href{https://github.com/tristynferreiro/QP4ISAR/blob/main/src/Simulator/Revised%20Simulator%20Files/Version_3_rev1_object_translation_and_rotation_motion.m}{new version} has scatterer amplitudes as a parameter to ensure that some scatterers can be configured to be stronger than others.
    
    In the \href{https://github.com/tristynferreiro/QP4ISAR/blob/main/src/Simulator/Final_Simulator.m}{final simulator} used for data generation in this report, more features were added. The amplitude values are set using a Gaussian-like distribution which aids in randomising the scatterer amplitudes. The \gls{ra} and \gls{af} steps have been added in to the simulator and a simple \gls{cli} has been implemented. The \gls{cli} allows users to configure the image processing pipeline by choosin the translational and rotational motion and which \gls{ra} and \gls{af} algorithms, if any, to apply to the data. 
    
    %***************************************************************************************%
    \subsection{Simulator set-up for testing} \label{subsec:suimulation_setup}
    The \href{https://github.com/tristynferreiro/QP4ISAR/tree/main/src/Simulator/Testing%20Setup}{testing code and configuration} is available for ease of repetition of these experiments. The scatterers were setup as illustrated in \autoref{fig:sim_scatterers}. The experiment motion input parameters for all subsections of this chapter are specified in \autoref{tab:simulation_setup}. The \gls{ra} and \gls{af} algorithms were selected in the \gls{cli} depending on which algorithm was being investigated in each section.
    
    \begin{table}[ht]
        \centering
        {\small
        \begin{tabular}{|c|c|c|}
            \hline
            & \textbf{Translational velocity (m/s)} & \textbf{Rotational velocity (deg/s)} \\
            \hline
            No translational motion (Focused image) & 0 & 6 \\
            \hline
            Unfocused image & 1 & 6 \\
            \hline
        \end{tabular}
        }
        \caption{Caption}
        \label{tab:simulation_setup}
    \end{table}

    \begin{figure}[ht]
        \centering
        \captionsetup{type=figure}
        \begin{minipage}{0.5\linewidth} % Adjust the width as needed
            \centering
            \resizebox{\linewidth}{!}{% This file was created by matlab2tikz.
%
%The latest updates can be retrieved from
%  http://www.mathworks.com/matlabcentral/fileexchange/22022-matlab2tikz-matlab2tikz
%where you can also make suggestions and rate matlab2tikz.
%
\definecolor{mycolor1}{rgb}{0.00000,0.44700,0.74100}%
%
\begin{tikzpicture}

\begin{axis}[%
width=6.028in,
height=4.754in,
at={(1.011in,0.642in)},
scale only axis,
xmin=-10,
xmax=10,
xlabel style={font=\fontsize{20}{20}\selectfont\color{black}, yshift = -10},
xlabel={Length (m)},
ymin=0,
ymax=3,
ylabel style={font=\fontsize{20}{20}\selectfont\color{black}},
ylabel={Height (m)},
axis background/.style={fill=white},
tick label style={font=\fontsize{15}{11}\selectfont\color{black}},
ytick distance = 0.5,
axis x line*=bottom,
axis y line*=left
]
\addplot[only marks, mark=o, mark options={}, mark size=1.5000pt, draw=mycolor1, forget plot] table[row sep=crcr]
            \caption{X-Y plot of scatterers used in the simulation.\label{fig:sim_scatterers}}
        \end{minipage}
    \end{figure}

    All simulation data used on which \gls{ra} and \gls{af} are performed throughout this section was generated for the object in \autoref{fig:sim_scatterers} with a translational motion of 1m/s and rotational motion of 6deg/s. The ideal result of the \gls{isar} images after range alignment and autofocus was simulated using a translational motion of 0m/s and a rotational motion of 6deg/s this is reffered to as the 'Reference focused \gls{isar} image'  or the 'Focused \gls{isar} image'.
% Focused image: 0 mps 6 degps
% Simulaiton: 1mps 6degps
%%%%%%%%%%%%%%%%%%%%%%%%%%%%%%%%%%%%%%%%%%%%%%%%%%%%%%%%%%%%%%%%%%%%%%%%%%%%%%%%%%%%%%%%%%%%%%%%%%%%
\section{Description of Measured Data Setup} \label{subsec:measuredData_setup}
To perform the algorithm verification, measured data containing already constructed \gls{HRR} profiles was used. A \textsc{MATLAB} \href{}{script} was developed to make verification testing simple, the full testing setup is available on \href{}{GitHub}. Several unfocused image frames \footnote{A frame is a single \gls{isar} image developed using a set amount of \gls{hrr} profiles} were considered, and the one which most clearly resembled the imaged object was chosen to verify all algorithms. All frames considered are available in the \autoref{apndxA:verification_frames} and the chosen frame is shown in \autoref{fig:corrRA_measured_frame}. The \gls{cptwl} is 128 \gls{hrr} profiles in length.
%%%%%%%%%%%%%%%%%%%%%%%%%%%%%%%%%%%%%%%%%%%%%%%%%%%%%%%%%%%%%%%%%%%%%%%%%%%%%%%%%%%%%%%%%%%%%%%%%%%%
\section{Correlation Range Alignment Algorithm}\label{subsec:corrRA}
% martorella ch 4.1.1. % Martorella pg 13
This is a simplistic \gls{ra} algorithm that uses cross-correlation to determine the number of integer range bins by which a profile is misaligned with respect to a reference profile.
%to determine the extent of misalignment between a profile and a reference profile in terms of the number of range bins.
The calculated number of bins is used to shift the misaligned range profile into alignment with the reference profile. An outline of the correlation \gls{ra} algorithm as described in Chapter 4.1.1. \cite{ISARtextbook_Martorella} is given in \autoref{alg:corr_RA}.
    %***************************************************************************************%
    \subsection{Pseudo Code and Implementation}
    % Corr RA Pseudo code
    \begin{figure}[ht]
      \vspace{0.5cm}
      \centering
      \captionsetup{type=figure}
      \begin{minipage}{.7\linewidth}
        \begin{algorithm}[h]
            \caption{Correlation \gls{ra} Algorithm.\label{alg:corr_RA}}
            
            \LinesNumbered % NUMBER THE LINES
            \DontPrintSemicolon
            \SetAlgoLined
            \SetKwInOut{Input}{input}\SetKwInOut{Output}{output}\SetKwInOut{Parameter}{parameter}
    
            \Input{All \gls{hrrp}s, matrix $hrrp_{all}$}
            \Output{Range-aligned \gls{hrrp}s, matrix $hrrp_{RA}$}
            \Parameter{Reference \gls{hrrp} number, $refIndex$}
            
            \BlankLine
            \Begin{
                $hrrp_{ref}\leftarrow hrrp_{all}[refIndex] $\;
                $correlation  \leftarrow $ auto-cross-correlate $hrrp_{ref}$\;
                $peakIndex_{ref} \leftarrow$ index of $max(correlation)$\;
                \For(){k in 1 to rows($hrrp_{all}$)}{
                    $correlation  \leftarrow $ cross-correlate $hrrp_{all}[k]$ and $hrrp_{ref}$ \;
                    $peakIndex \leftarrow$ index of $max(correlation)$\;
                    $shift \leftarrow peakIndex_{ref} - peakIndex $\;
                    $hrrp_{RA}[k] \leftarrow circularshift$ $hrrp_{all}[k]$ by $shift$\; 
                }
            }
          \vspace{0.5cm}
        \end{algorithm}
      \end{minipage}
    \end{figure}

    The Correlation Algorithm, \autoref{alg:corr_RA}, was implemented as a stand-alone function in \textsc{MATLAB}. \textsc{MATLAB} is designed to operate on matrices and arrays and so for-loops were omitted in the implementation. The function, \href{}{corrRA.m}, was validated and verified in this section.
    
    %***************************************************************************************%
    \subsection{Algorithm Verification}
    The goal of this section was to verify that \autoref{alg:corr_RA} was implemented correctly in \textsc{MATLAB}. The simulation setup described in \autoref{subsec:suimulation_setup} was used to generate \gls{hrr} profiles on which \autoref{alg:corr_RA} implementation was performed.
    
    % Grid of the HRRP 
    \begin{figure}[h]
        \begin{minipage}{0.60\linewidth}
            \begin{figure}
                \begin{tabular}{@{}cc@{}}
                    \begin{subfigure}{0.5\linewidth}
                        \centering
                        \resizebox{\linewidth}{!}{\input{Figures/04AlgoV&V/Simulation/01WithMotion/Sim_HRRP_1mps_6deg}}
                        \caption{Unaligned profiles.}\label{subfig:corrRA_sim_hrrp_1mps}
                    \end{subfigure}
                    &
                    \begin{subfigure}{0.5\linewidth}
                        \centering
                        \resizebox{\linewidth}{!}{% This file was created by matlab2tikz.
%
%The latest updates can be retrieved from
%  http://www.mathworks.com/matlabcentral/fileexchange/22022-matlab2tikz-matlab2tikz
%where you can also make suggestions and rate matlab2tikz.
%
\begin{tikzpicture}

\begin{axis}[%
width=5.554in,
height=4.754in,
at={(0.932in,0.642in)},
scale only axis,
point meta min=-35,
point meta max=0,
axis on top,
xmin=-0.0732421875,
xmax=37.4267578125,
xlabel style={font=\color{white!15!black}},
xlabel={Range (m)},
y dir=reverse,
ymin=0.5,
ymax=32.5,
ylabel style={font=\color{white!15!black}},
ylabel={Profile Number},
axis background/.style={fill=white},
title style={font=\bfseries},
title={Range-aligned HRR Profiles},
colormap/jet,
colorbar
]
\addplot [forget plot] graphics [xmin=-0.0732421875, xmax=37.4267578125, ymin=0.5, ymax=32.5] {SCRA_Sim_HRRP_1mps_6deg-1.png};
\end{axis}

\begin{axis}[%
width=7.778in,
height=5.833in,
at={(0in,0in)},
scale only axis,
point meta min=0,
point meta max=1,
xmin=0,
xmax=1,
ymin=0,
ymax=1,
axis line style={draw=none},
ticks=none,
axis x line*=bottom,
axis y line*=left
]
\end{axis}
\end{tikzpicture}%}
                        \caption{Range-aligned profiles.}\label{subfig:corrRA_sim_hrrp}
                    \end{subfigure}
                \end{tabular}
                \caption{Simulated \gls{hrr} profiles before and after Correlation \gls{ra}. \label{fig:corrRA_sim}}
            \end{figure}
        \end{minipage}
        \hfill
        \begin{minipage}{0.30\linewidth}
            \begin{figure}
                \begin{tabular}{@{}c@{}}
                    \begin{subfigure}{\linewidth}
                        \centering
                        \resizebox{\linewidth}{!}{% This file was created by matlab2tikz.
%
%The latest updates can be retrieved from
%  http://www.mathworks.com/matlabcentral/fileexchange/22022-matlab2tikz-matlab2tikz
%where you can also make suggestions and rate matlab2tikz.
%
\begin{tikzpicture}

    \begin{axis}[%
    width=5.554in,
    height=4.754in,
    at={(0.932in,0.642in)},
    scale only axis,
    point meta min=-35,
    point meta max=0,
    axis on top,
    xmin=-0.0732421875,
    xmax=37.4267578125,
    xlabel style={font=\color{white!15!black}},
    xlabel={Range (m)},
    y dir=reverse,
    ymin=0.5,
    ymax=32.5,
    ylabel style={font=\color{white!15!black}},
    ylabel={Profile Number},
    axis background/.style={fill=white},
    colormap/jet,
    colorbar
    ]
    \addplot [forget plot] graphics [xmin=-0.0732421875, xmax=37.4267578125, ymin=0.5, ymax=32.5] {Sim_HRRP_0mps_6deg.png};
    \end{axis}
    \end{tikzpicture}%}
                        \caption{Focused profiles.\label{subfig:corrRA_sim_hrrp_0mps}}
                    \end{subfigure}
                \end{tabular}
                \caption{Simulated object with no translation motion.} \label{fig:sim_0mps}
            \end{figure}
        \end{minipage}
    \end{figure}
    
    Recall from \autoref{theory:RA} that an object's motion causes range migration of the object between range profiles. \autoref{subfig:corrRA_sim_hrrp_1mps} shows that Scatterer A migrates between profiles as expected, the vertical line further illustrates that the profiles are not aligned. Correlation \gls{ra} was applied to the unaligned profiles in an attempt to align the profiles to achieve straight line scatterers similar to \autoref{subfig:corrRA_sim_hrrp_0mps}.

    Using line 7 of \autoref{alg:corr_RA}, the peak location (index) of each profile's cross-correlation with respect to profile 1 was calculated. This value was then used to calculate the number of bins each profile was shifted from profile 1 as in line 8 of \autoref{alg:corr_RA}. \autoref{fig:corrRA_sim_shifts} shows the calculated bin shifts for each profile in \autoref{fig:corrRA_sim_shifts}.

    \begin{figure}
        \centering
        \resizebox{0.45\linewidth}{!}{% This file was created by matlab2tikz.
%
%The latest updates can be retrieved from
%  http://www.mathworks.com/matlabcentral/fileexchange/22022-matlab2tikz-matlab2tikz
%where you can also make suggestions and rate matlab2tikz.
%
\definecolor{mycolor1}{rgb}{0.00000,0.44700,0.74100}%
%
\begin{tikzpicture}

\begin{axis}[%
width=6.028in,
height=4.754in,
at={(1.011in,0.642in)},
scale only axis,
xmin=0,
xmax=35,
xlabel style={font=\fontsize{20}{20}\selectfont\color{black}, yshift = -10},
xlabel={Profile Number},
ymin=-4,
ymax=0,
ylabel style={font=\fontsize{20}{20}\selectfont\color{black}},
ylabel={Number of bin shifts},
axis background/.style={fill=white},
tick label style={font=\fontsize{15}{11}\selectfont\color{black}},
xtick distance= 4            % Set the spacing between x-axis ticks
]
\addplot [color=mycolor1, forget plot]
  table[row sep=crcr]
        \caption{Staircase curve of range bin shifts per \gls{hrrp} for Correlation \gls{ra} on simulated profiles.\label{fig:corrRA_sim_shifts}}
    \end{figure}
    % \begin{wrapfigure}{l}{0.45\linewidth}
    %   \centering
    %   \vspace*{-\baselineskip}
    %   \resizebox{\linewidth}{!}{% This file was created by matlab2tikz.
%
%The latest updates can be retrieved from
%  http://www.mathworks.com/matlabcentral/fileexchange/22022-matlab2tikz-matlab2tikz
%where you can also make suggestions and rate matlab2tikz.
%
\definecolor{mycolor1}{rgb}{0.00000,0.44700,0.74100}%
%
\begin{tikzpicture}

\begin{axis}[%
width=6.028in,
height=4.754in,
at={(1.011in,0.642in)},
scale only axis,
xmin=0,
xmax=35,
xlabel style={font=\fontsize{20}{20}\selectfont\color{black}, yshift = -10},
xlabel={Profile Number},
ymin=-4,
ymax=0,
ylabel style={font=\fontsize{20}{20}\selectfont\color{black}},
ylabel={Number of bin shifts},
axis background/.style={fill=white},
tick label style={font=\fontsize{15}{11}\selectfont\color{black}},
xtick distance= 4            % Set the spacing between x-axis ticks
]
\addplot [color=mycolor1, forget plot]
  table[row sep=crcr]
    %   \caption{Staircase curve of range bin shifts per \gls{hrrp} for Correlation \gls{ra} on simulated profiles.\label{fig:corrRA_sim_shifts}}
    %   \vspace*{-\baselineskip}
    % \end{wrapfigure} 
    
    \autoref{subfig:corrRA_sim_hrrp_1mps} shows that Scatterer A migrated by a fraction of a range bin between succeessive profiles. However, the 'steps' in \autoref{fig:corrRA_sim_shifts} show that each profile was always shifted by an integer number of bins. As explained before, the bin shifts were calculated as the difference between array indices which was an integer value. Therefore, profiles with fractional bin shifts were rather treated as a full bin shift as shown by the flat horizontal steps in \autoref{fig:corrRA_sim_shifts}. In terms of the algorithm, which expects all bin shifts to be integer numbers, \autoref{fig:corrRA_sim_shifts} is correct.

    Applying the implemented \autoref{alg:corr_RA} to the unaligned profiles in \autoref{subfig:corrRA_sim_hrrp_1mps} did not produce range-aligned profiles in \autoref{subfig:corrRA_sim_hrrp} that look like the profiles in \autoref{subfig:corrRA_sim_hrrp_0mps}. However, the range-aligned profiles do follow the staircase pattern of \autoref{fig:corrRA_sim_shifts}. Although, the profiles are not perfectly aligned as expected, they are correctly aligned in terms of the bin shift values in \autoref{alg:corr_RA}. Therefore, because the correct bin shifts were calculated and applied, \autoref{alg:corr_RA} was implemented correctly.

    %***************************************************************************************%
    \subsection{Algorithm Validation}
    The aim of validation was to test that Correlation \gls{ra} could align the \gls{hrr} profiles in the measured data frame, discussed in \autoref{subsec:measuredData_setup}.
    
    % Grid of the HRRP 
    \begin{figure}[h]
        \centering
        \begin{subfigure}{0.4\linewidth}
                \centering
                \resizebox{\linewidth}{!}{% This file was created by matlab2tikz.
%
%The latest updates can be retrieved from
%  http://www.mathworks.com/matlabcentral/fileexchange/22022-matlab2tikz-matlab2tikz
%where you can also make suggestions and rate matlab2tikz.
%
\begin{tikzpicture}
\begin{axis}[%
width=5.554in,
height=4.754in,
at={(0.932in,0.642in)},
scale only axis,
point meta min=-35,
point meta max=0,
axis on top,
xmin=0,
xmax=128,
xlabel style={font=\fontsize{25}{14}\selectfont\color{black}, yshift=-10pt},
xlabel={Range Bin Number},
y dir=reverse,
ymin=0,
ymax=96,
ylabel style={font=\fontsize{25}{14}\selectfont\color{black}, yshift=10pt},
ylabel={Profile Number},
axis background/.style={fill=white},
tick label style={font=\fontsize{20}{11}\selectfont\color{black}},
xtick distance= 20,             % Set the spacing between x-axis ticks
colormap/jet,
colorbar
]
\addplot [forget plot] graphics [xmin=0, xmax=128, ymin=0, ymax=96] {Figures/04AlgoV&V/Measured/01Original/Measured_HRRP_frame2464.png};

% Add a vertical black line at x = 50
\addplot [black, line width=2pt] coordinates {(59, 0) (59, 96)};

% Add a black arrow pointing at (60, 30)
\draw[->, black, line width=1.5pt] (35, 8) -- (57.5, 8);
\draw[->, black, line width=1.5pt] (35, 18) -- (56, 18);
\draw[->, black, line width=1.5pt] (30, 76) -- (50.5, 76);
% Add a label to the left of the arrow with a white box
\node[left] at (36, 8) {\tikz[baseline] \node[fill=white,inner sep=2pt] {Scatterer A};};
\node[left] at (36, 18) {\tikz[baseline] \node[fill=white,inner sep=2pt] {Scatterer A};};
\node[left] at (30, 76) {\tikz[baseline] \node[fill=white,inner sep=2pt] {Scatterer A};};

\end{axis}
\end{tikzpicture}%}
                \caption{Unaligned profiles.\label{subfig:corrRA_measured_hrrp_unaligned}}
        \end{subfigure}
        \hspace{1cm}
        \begin{subfigure}{0.4\linewidth}
                \centering
                \resizebox{\linewidth}{!}{\input{Figures/04AlgoV&V/Measured/02CorrRA/SCRA_Measured_HRRP}}
                \caption{Range-aligned profiles.\label{subfig:corrRA_measureed_hrrp}}
        \end{subfigure}
        \caption{\gls{hrr} profiles of measured data frame before and after Correlation \gls{ra}. \label{fig:corrRA_measured}}
    \end{figure}

    \autoref{subfig:corrRA_measured_hrrp_unaligned} shows that Scatterer A migrated between profiles. Comparing the position of Scatterer A, in all profiles, to the vertical black line in \autoref{subfig:corrRA_measured_hrrp_unaligned} illustrates that before \gls{ra}, the profiles were not aligned to profile 1.
    
    \begin{figure}
        \centering
        \resizebox{0.45\linewidth}{!}{% This file was created by matlab2tikz.
%
%The latest updates can be retrieved from
%  http://www.mathworks.com/matlabcentral/fileexchange/22022-matlab2tikz-matlab2tikz
%where you can also make suggestions and rate matlab2tikz.
%
\definecolor{mycolor1}{rgb}{0.00000,0.44700,0.74100}%
%
\begin{tikzpicture}

\begin{axis}[%
width=6.028in,
height=4.754in,
at={(1.011in,0.642in)},
scale only axis,
xmin=0,
xmax=128,
xlabel style={font=\fontsize{20}{20}\selectfont\color{black}, yshift=-10pt},
xlabel={Profile Number},
ymin=0,
ymax=7,
ylabel style={font=\fontsize{20}{20}\selectfont\color{black}, yshift=10pt},
ylabel={Number of bin shifts},
axis background/.style={fill=white},
xtick distance = 20,
tick label style={font=\fontsize{15}{11}\selectfont\color{black}},
ytick={0,1,2,3,4,5,6,7}, % Set the y-axis ticks explicitly
grid = both
]
\addplot [color=mycolor1,forget plot]
  table[row sep=crcr]{%
1	0\\
2	0\\
3	0\\
4	0\\
5	0\\
6	0\\
7	0\\
8	0\\
9	0\\
10	0\\
11	0\\
12	0\\
13	1\\
14	0\\
15	1\\
16	1\\
17	1\\
18	1\\
19	1\\
20	1\\
21	1\\
22	1\\
23	1\\
24	1\\
25	1\\
26	1\\
27	1\\
28	1\\
29	1\\
30	1\\
31	1\\
32	1\\
33	1\\
34	1\\
35	1\\
36	2\\
37	2\\
38	2\\
39	2\\
40	2\\
41	2\\
42	2\\
43	2\\
44	2\\
45	2\\
46	2\\
47	2\\
48	2\\
49	2\\
50	2\\
51	2\\
52	2\\
53	2\\
54	3\\
55	3\\
56	3\\
57	2\\
58	3\\
59	3\\
60	3\\
61	3\\
62	3\\
63	3\\
64	3\\
65	3\\
66	3\\
67	3\\
68	3\\
69	3\\
70	3\\
71	3\\
72	3\\
73	3\\
74	3\\
75	3\\
76	3\\
77	4\\
78	4\\
79	4\\
80	4\\
81	4\\
82	4\\
83	4\\
84	4\\
85	4\\
86	4\\
87	5\\
88	4\\
89	4\\
90	5\\
91	5\\
92	5\\
93	5\\
94	5\\
95	5\\
96	5\\
97	5\\
98	5\\
99	5\\
100	5\\
101	5\\
102	5\\
103	5\\
104	5\\
105	5\\
106	5\\
107	5\\
108	5\\
109	5\\
110	5\\
111	5\\
112	5\\
113	5\\
114	5\\
115	6\\
116	6\\
117	6\\
118	6\\
119	6\\
120	6\\
121	6\\
122	6\\
123	6\\
124	6\\
125	6\\
126	6\\
127	6\\
128	6\\
};
\end{axis}
\end{tikzpicture}%}
        \caption{Staircase curve of range bin shifts per \gls{hrrp} for Correlation \gls{ra} on measured profiles.}\label{fig:corrRA_Measured_shifts}
    \end{figure}
    % \begin{wrapfigure}{l}{0.45\linewidth}
    %   \centering
    %   \vspace*{-\baselineskip}
    %   \resizebox{\linewidth}{!}{% This file was created by matlab2tikz.
%
%The latest updates can be retrieved from
%  http://www.mathworks.com/matlabcentral/fileexchange/22022-matlab2tikz-matlab2tikz
%where you can also make suggestions and rate matlab2tikz.
%
\definecolor{mycolor1}{rgb}{0.00000,0.44700,0.74100}%
%
\begin{tikzpicture}

\begin{axis}[%
width=6.028in,
height=4.754in,
at={(1.011in,0.642in)},
scale only axis,
xmin=0,
xmax=128,
xlabel style={font=\fontsize{20}{20}\selectfont\color{black}, yshift=-10pt},
xlabel={Profile Number},
ymin=0,
ymax=7,
ylabel style={font=\fontsize{20}{20}\selectfont\color{black}, yshift=10pt},
ylabel={Number of bin shifts},
axis background/.style={fill=white},
xtick distance = 20,
tick label style={font=\fontsize{15}{11}\selectfont\color{black}},
ytick={0,1,2,3,4,5,6,7}, % Set the y-axis ticks explicitly
grid = both
]
\addplot [color=mycolor1,forget plot]
  table[row sep=crcr]{%
1	0\\
2	0\\
3	0\\
4	0\\
5	0\\
6	0\\
7	0\\
8	0\\
9	0\\
10	0\\
11	0\\
12	0\\
13	1\\
14	0\\
15	1\\
16	1\\
17	1\\
18	1\\
19	1\\
20	1\\
21	1\\
22	1\\
23	1\\
24	1\\
25	1\\
26	1\\
27	1\\
28	1\\
29	1\\
30	1\\
31	1\\
32	1\\
33	1\\
34	1\\
35	1\\
36	2\\
37	2\\
38	2\\
39	2\\
40	2\\
41	2\\
42	2\\
43	2\\
44	2\\
45	2\\
46	2\\
47	2\\
48	2\\
49	2\\
50	2\\
51	2\\
52	2\\
53	2\\
54	3\\
55	3\\
56	3\\
57	2\\
58	3\\
59	3\\
60	3\\
61	3\\
62	3\\
63	3\\
64	3\\
65	3\\
66	3\\
67	3\\
68	3\\
69	3\\
70	3\\
71	3\\
72	3\\
73	3\\
74	3\\
75	3\\
76	3\\
77	4\\
78	4\\
79	4\\
80	4\\
81	4\\
82	4\\
83	4\\
84	4\\
85	4\\
86	4\\
87	5\\
88	4\\
89	4\\
90	5\\
91	5\\
92	5\\
93	5\\
94	5\\
95	5\\
96	5\\
97	5\\
98	5\\
99	5\\
100	5\\
101	5\\
102	5\\
103	5\\
104	5\\
105	5\\
106	5\\
107	5\\
108	5\\
109	5\\
110	5\\
111	5\\
112	5\\
113	5\\
114	5\\
115	6\\
116	6\\
117	6\\
118	6\\
119	6\\
120	6\\
121	6\\
122	6\\
123	6\\
124	6\\
125	6\\
126	6\\
127	6\\
128	6\\
};
\end{axis}
\end{tikzpicture}%}
    %   \caption{Staircase curve of range bin shifts per \gls{hrrp} for Correlation \gls{ra} on measured profiles.}\label{fig:corrRA_Measured_shifts}
    %   \vspace*{-1.5\baselineskip}
    % \end{wrapfigure} 
    
    \autoref{fig:corrRA_Measured_shifts} shows the calculated bin shifts for all profiles, calculated with respect to profile 1, in \autoref{subfig:corrRA_measured_hrrp_unaligned}. These shifts were applied to the unaligned profiles, resulting in the range-aligned profiles in \autoref{subfig:corrRA_measured_hrrp}. \autoref{subfig:corrRA_measured_hrrp} shows that after range-alignment, Scatterer A does not migrate between profiles. All scatterers are also aligned with the vertical black line. Therefore, Correlation \gls{ra} was able to align the misaligned profiles and meets the requirements of a \gls{ra} algorithm.
    
%%%%%%%%%%%%%%%%%%%%%%%%%%%%%%%%%%%%%%%%%%%%%%%%%%%%%%%%%%%%%%%%%%%%%%%%%%%%%%%%%%%%%%%%%%%%%%%%%%%%%%%%%%%%%%%%%%%%%%%%%%%%%%%%%%%%%%%%%%%%%%%%%%%%%%%%%%%%%%%%%%%%%%%%%%%%%%%%
\section{Haywood's Range Alignment Algorithm}\label{subsec:HayRA}
This \gls{ra} algorithm accommodates fractional bin shifts, unlike the integer-only range bin shifts allowed in \autoref{alg:corr_RA}. This is achieved by linearising the calculated bin shifts and applying them to the profiles as a phase shift, $\phi$. An outline of the Haywood \gls{ra} algorithm, as described in \cite{haywood_RA_AF,zyweck}, is provided in \autoref{alg:haywood_RA}.
    \subsection{Pseudo code and Implementation}
    % Haywood RA Pseudo code
    \begin{figure}[ht]
      \vspace{0.5cm}
      \centering
      \captionsetup{type=figure}
      \begin{minipage}{.7\linewidth}
        \begin{algorithm}[H]
        \caption{Haywood \gls{ra} Algorithm.\label{alg:haywood_RA}}
    
        \LinesNumbered % NUMBER THE LINES
        \DontPrintSemicolon
        \SetAlgoLined
        \SetKwInOut{Input}{input}\SetKwInOut{Output}{output}\SetKwInOut{Parameter}{parameter}
    
        \Input{All \gls{hrrp}s, matrix $hrrp_{all}$}
        \Output{Range-aligned \gls{hrrp}s, matrix $hrrp_{RA}$}
        \Parameter{Reference \gls{hrrp} number, $refIndex$}
    
        \BlankLine
        \Begin{
            $hrrp_{ref}\leftarrow hrrp_{all}[refIndex] $\;
            \For(){$k$ in 1 to rows($hrrp_{all})$}{
                $correlation[k]  \leftarrow $ cross-correlate $hrrp_{all}[k]$ and $hrrp_{ref}$ \;
                $peakIndex[k] \leftarrow$ index of $\max(correlation[k] )$\;
            }
            $shifts \leftarrow$  linearise $peakIndex$\;
            $N \leftarrow$ length($shifts$)\;
            
            \For(){$k$ in 1 to rows($hrrp_{all}$)}{
                \For(){$n$ in 1 to $N$}{
                    $\phi[n] \leftarrow \exp(-j \cdot shifts[k] \cdot \frac{n}{N})$\;
                }
                $hrrp_{RA}[k] \leftarrow IFFT( \phi \cdot FFT(hrrp_{all}[k]) )$\;
            }
        }
        \vspace{0.5cm}
        \end{algorithm}
      \end{minipage}
    \end{figure}

    The \gls{ra} algorithm, \autoref{alg:haywood_RA}, was implemented as a stand-alone function in \textsc{MATLAB}. The implementation, \href{}{haywoodRA.m}, was validated and verified in this section.
    
    %***************************************************************************************%
    \subsection{Algorithm Verification}
    The aim of the testing in this section was to use simulated \gls{hrr} profiles to verify that \autoref{alg:haywood_RA} was implemented correctly.

    % Grid of the HRRP and ISAR images
    \begin{figure}[h]
    \begin{minipage}{0.6\linewidth}
        \begin{tabular}{@{}cc@{}}
            \begin{subfigure}{0.5\linewidth}
                \centering
                \resizebox{\linewidth}{!}{\input{Figures/04AlgoV&V/Simulation/01WithMotion/Sim_HRRP_1mps_6deg}}
                \caption{Unaligned profiles.\label{subfig:sim_hrrp_1mps}}
            \end{subfigure}
            &
            \begin{subfigure}{0.5\linewidth}
                \centering
                \resizebox{\linewidth}{!}{% This file was created by matlab2tikz.
%
%The latest updates can be retrieved from
%  http://www.mathworks.com/matlabcentral/fileexchange/22022-matlab2tikz-matlab2tikz
%where you can also make suggestions and rate matlab2tikz.
%
\begin{tikzpicture}
\begin{axis}[%
width=5.554in,
height=4.754in,
at={(0.932in,0.642in)},
scale only axis,
point meta min=-35,
point meta max=0,
axis on top,
xmin=0,
xmax=256,
xlabel style={font=\fontsize{25}{14}\selectfont\color{black}, yshift=-10pt},
xlabel={Range (m)},
y dir=reverse,
ymin=0.5,
ymax=32.5,
ylabel style={font=\fontsize{25}{14}\selectfont\color{black}, yshift=10pt},
ylabel={Profile Number},
axis background/.style={fill=white},
tick label style={font=\fontsize{20}{11}\selectfont\color{black}},
xtick distance= 50,             % Set the spacing between x-axis ticks
colormap/jet,
colorbar
]
\addplot [forget plot] graphics [xmin=0, xmax=256, ymin=0.5, ymax=32.5] {Figures/04AlgoV&V/Simulation/03HayRA/HayRA_Sim_HRRP_1mps_6deg.png};

% Add a vertical black line at x = 50
\addplot [black, line width=2pt] coordinates {(56, 0.5) (56, 32.5)};

% Add a black arrow pointing at (60, 30)
\draw[->, black, line width=1.5pt] (45, 31) -- (61, 31);
\draw[->, black, line width=1.5pt] (45, 20) -- (60, 20);
\draw[->, black, line width=1.5pt] (45, 11) -- (59, 11);
% Add a label to the left of the arrow with a white box
\node[left] at (50, 31) {\tikz[baseline] \node[fill=white,inner sep=2pt] {Scatterer A};};
\node[left] at (51, 20) {\tikz[baseline] \node[fill=white,inner sep=2pt] {Scatterer A};};
\node[left] at (52, 11) {\tikz[baseline] \node[fill=white,inner sep=2pt] {Scatterer A};};

\end{axis}
\end{tikzpicture}%}
                \caption{Range-aligned profiles.\label{subfig:hayRA_sim_hrrp}}
            \end{subfigure}
        \end{tabular}
        \caption{Simulated \gls{hrr} profiles before and after Haywood \gls{ra}. \label{subfig:hayRA_sim}}
    \end{minipage}
    \hfill
    \begin{minipage}{0.3\linewidth}
        \begin{tabular}{@{}c@{}}
            \begin{subfigure}{\linewidth}
                \centering
                \resizebox{\linewidth}{!}{% This file was created by matlab2tikz.
%
%The latest updates can be retrieved from
%  http://www.mathworks.com/matlabcentral/fileexchange/22022-matlab2tikz-matlab2tikz
%where you can also make suggestions and rate matlab2tikz.
%
\begin{tikzpicture}

    \begin{axis}[%
    width=5.554in,
    height=4.754in,
    at={(0.932in,0.642in)},
    scale only axis,
    point meta min=-35,
    point meta max=0,
    axis on top,
    xmin=-0.0732421875,
    xmax=37.4267578125,
    xlabel style={font=\color{white!15!black}},
    xlabel={Range (m)},
    y dir=reverse,
    ymin=0.5,
    ymax=32.5,
    ylabel style={font=\color{white!15!black}},
    ylabel={Profile Number},
    axis background/.style={fill=white},
    colormap/jet,
    colorbar
    ]
    \addplot [forget plot] graphics [xmin=-0.0732421875, xmax=37.4267578125, ymin=0.5, ymax=32.5] {Sim_HRRP_0mps_6deg.png};
    \end{axis}
    \end{tikzpicture}%}
                \caption{Reference aligned profiles.\label{subfig:hayRA_sim_hrrp_0mps}}
            
            \end{subfigure}
        \end{tabular}
        \caption{Simulated object with no translation motion. \label{fig:sim_0mps}}
    \end{minipage}
    \end{figure}

    The vertical black line in \autoref{subfig:hayRA_sim_hrrp} shows that the range profiles are not aligned. Haywood's \gls{ra} algorithm was applied to the unaligned profiles in an attempt to align them such that Scatterer A stays in the same range bin across all profiles. \autoref{subfig:hayRA_sim_hrrp_0mps} serves as a reference for what range-aligned profiles should look like.

    \begin{figure}
        \centering
        \resizebox{0.45\linewidth}{!}{% This file was created by matlab2tikz.
%
% The latest updates can be retrieved from
% http://www.mathworks.com/matlabcentral/fileexchange/22022-matlab2tikz-matlab2tikz
% where you can also make suggestions and rate matlab2tikz.
%
\definecolor{mycolor1}{rgb}{0.00000,0.44700,0.74100}%
\definecolor{mycolor2}{rgb}{0.85000,0.32500,0.09800}%
%
\begin{tikzpicture}

\begin{axis}[%
width=6.028in,
height=4.754in,
at={(1.011in,0.642in)},
scale only axis,
xmin=0,
xmax=35,
xlabel style={font=\fontsize{20}{20}\selectfont\color{black}, yshift = -10},
xlabel={Profile Number},
ymin=-4,
ymax=0,
ylabel style={font=\fontsize{20}{20}\selectfont\color{black}},
ylabel={Number of bin shifts},
axis background/.style={fill=white},
tick label style={font=\fontsize{15}{11}\selectfont\color{black}},
xtick distance= 4             % Set the spacing between x-axis ticks
]
\addplot [color=mycolor1, forget plot]
  table[row sep=crcr]{%
1	0\\
2	0\\
3	0\\
4	-1\\
5	-1\\
6	-1\\
7	-1\\
8	-1\\
9	-1\\
10	-1\\
11	-1\\
12	-1\\
13	-1\\
14	-2\\
15	-2\\
16	-2\\
17	-2\\
18	-2\\
19	-2\\
20	-2\\
21	-2\\
22	-2\\
23	-3\\
24	-3\\
25	-3\\
26	-3\\
27	-3\\
28	-3\\
29	-3\\
30	-3\\
31	-3\\
32	-4\\
};
\addplot [color=mycolor2, forget plot]
  table[row sep=crcr]{%
1	-0.181818181818181\\
2	-0.289039589442815\\
3	-0.396260997067448\\
4	-0.503482404692082\\
5	-0.610703812316715\\
6	-0.717925219941349\\
7	-0.825146627565982\\
8	-0.932368035190615\\
9	-1.03958944281525\\
10	-1.14681085043988\\
11	-1.25403225806452\\
12	-1.36125366568915\\
13	-1.46847507331378\\
14	-1.57569648093842\\
15	-1.68291788856305\\
16	-1.79013929618768\\
17	-1.89736070381232\\
18	-2.00458211143695\\
19	-2.11180351906158\\
20	-2.21902492668622\\
21	-2.32624633431085\\
22	-2.43346774193548\\
23	-2.54068914956012\\
24	-2.64791055718475\\
25	-2.75513196480938\\
26	-2.86235337243402\\
27	-2.96957478005865\\
28	-3.07679618768328\\
29	-3.18401759530792\\
30	-3.29123900293255\\
31	-3.39846041055718\\
32	-3.50568181818182\\
};
\end{axis}
\end{tikzpicture}%
}
        \caption{Linearised staircase curve of range bin shifts per \gls{hrrp}.\label{fig:hayRA_sim_shifts}}
    \end{figure}

    % \begin{wrapfigure}{l}{0.45\linewidth}
    %     \centering
    %     \vspace*{-\baselineskip}
    %     \resizebox{\linewidth}{!}{% This file was created by matlab2tikz.
%
% The latest updates can be retrieved from
% http://www.mathworks.com/matlabcentral/fileexchange/22022-matlab2tikz-matlab2tikz
% where you can also make suggestions and rate matlab2tikz.
%
\definecolor{mycolor1}{rgb}{0.00000,0.44700,0.74100}%
\definecolor{mycolor2}{rgb}{0.85000,0.32500,0.09800}%
%
\begin{tikzpicture}

\begin{axis}[%
width=6.028in,
height=4.754in,
at={(1.011in,0.642in)},
scale only axis,
xmin=0,
xmax=35,
xlabel style={font=\fontsize{20}{20}\selectfont\color{black}, yshift = -10},
xlabel={Profile Number},
ymin=-4,
ymax=0,
ylabel style={font=\fontsize{20}{20}\selectfont\color{black}},
ylabel={Number of bin shifts},
axis background/.style={fill=white},
tick label style={font=\fontsize{15}{11}\selectfont\color{black}},
xtick distance= 4             % Set the spacing between x-axis ticks
]
\addplot [color=mycolor1, forget plot]
  table[row sep=crcr]{%
1	0\\
2	0\\
3	0\\
4	-1\\
5	-1\\
6	-1\\
7	-1\\
8	-1\\
9	-1\\
10	-1\\
11	-1\\
12	-1\\
13	-1\\
14	-2\\
15	-2\\
16	-2\\
17	-2\\
18	-2\\
19	-2\\
20	-2\\
21	-2\\
22	-2\\
23	-3\\
24	-3\\
25	-3\\
26	-3\\
27	-3\\
28	-3\\
29	-3\\
30	-3\\
31	-3\\
32	-4\\
};
\addplot [color=mycolor2, forget plot]
  table[row sep=crcr]{%
1	-0.181818181818181\\
2	-0.289039589442815\\
3	-0.396260997067448\\
4	-0.503482404692082\\
5	-0.610703812316715\\
6	-0.717925219941349\\
7	-0.825146627565982\\
8	-0.932368035190615\\
9	-1.03958944281525\\
10	-1.14681085043988\\
11	-1.25403225806452\\
12	-1.36125366568915\\
13	-1.46847507331378\\
14	-1.57569648093842\\
15	-1.68291788856305\\
16	-1.79013929618768\\
17	-1.89736070381232\\
18	-2.00458211143695\\
19	-2.11180351906158\\
20	-2.21902492668622\\
21	-2.32624633431085\\
22	-2.43346774193548\\
23	-2.54068914956012\\
24	-2.64791055718475\\
25	-2.75513196480938\\
26	-2.86235337243402\\
27	-2.96957478005865\\
28	-3.07679618768328\\
29	-3.18401759530792\\
30	-3.29123900293255\\
31	-3.39846041055718\\
32	-3.50568181818182\\
};
\end{axis}
\end{tikzpicture}%
}
    %     \caption{Linearised staircase curve of range bin shifts per \gls{hrrp}.\label{fig:hayRA_sim_shifts}}
    %     \vspace*{-1.5\baselineskip}
    % \end{wrapfigure}
    
    The blue curve in \autoref{fig:hayRA_sim_shifts} shows the bin shifts calculated using lines 3 to 6 of \autoref{alg:haywood_RA} and the red line shows the linearised shifts as per line 7 of \autoref{alg:haywood_RA}. The red line shows that fractional bin shifts were calculated in the implementation. \autoref{subfig:hayRA_sim} shows the \gls{hrr} profiles after performing Haywood \gls{ra}.

    In \autoref{subfig:hayRA_sim} scatter A did not migrate between profiles and shows that the calculated bin shifts successfully range-aligned the profiles. Comparing the scatterer lines to the vertical black line further proves that the profiles are aligned after Haywood \gls{ra}. Therefore, Haywood \gls{ra} yielded the eexpected result which verifies that \autoref{alg:haywood_RA} was implemented correctly.
    
    %***************************************************************************************%
    \subsection{Algorithm Validation}
    The aim of this section was test weather Haywood \gls{ra} works on a measured data frame thereby validating the algorithm for use in this report.

    \begin{figure}[h]
        \centering
        \begin{subfigure}{0.4\linewidth}
                \centering
                \resizebox{\linewidth}{!}{% This file was created by matlab2tikz.
%
%The latest updates can be retrieved from
%  http://www.mathworks.com/matlabcentral/fileexchange/22022-matlab2tikz-matlab2tikz
%where you can also make suggestions and rate matlab2tikz.
%
\begin{tikzpicture}
\begin{axis}[%
width=5.554in,
height=4.754in,
at={(0.932in,0.642in)},
scale only axis,
point meta min=-35,
point meta max=0,
axis on top,
xmin=0,
xmax=128,
xlabel style={font=\fontsize{25}{14}\selectfont\color{black}, yshift=-10pt},
xlabel={Range Bin Number},
y dir=reverse,
ymin=0,
ymax=96,
ylabel style={font=\fontsize{25}{14}\selectfont\color{black}, yshift=10pt},
ylabel={Profile Number},
axis background/.style={fill=white},
tick label style={font=\fontsize{20}{11}\selectfont\color{black}},
xtick distance= 20,             % Set the spacing between x-axis ticks
colormap/jet,
colorbar
]
\addplot [forget plot] graphics [xmin=0, xmax=128, ymin=0, ymax=96] {Figures/04AlgoV&V/Measured/01Original/Measured_HRRP_frame2464.png};

% Add a vertical black line at x = 50
\addplot [black, line width=2pt] coordinates {(59, 0) (59, 96)};

% Add a black arrow pointing at (60, 30)
\draw[->, black, line width=1.5pt] (35, 8) -- (57.5, 8);
\draw[->, black, line width=1.5pt] (35, 18) -- (56, 18);
\draw[->, black, line width=1.5pt] (30, 76) -- (50.5, 76);
% Add a label to the left of the arrow with a white box
\node[left] at (36, 8) {\tikz[baseline] \node[fill=white,inner sep=2pt] {Scatterer A};};
\node[left] at (36, 18) {\tikz[baseline] \node[fill=white,inner sep=2pt] {Scatterer A};};
\node[left] at (30, 76) {\tikz[baseline] \node[fill=white,inner sep=2pt] {Scatterer A};};

\end{axis}
\end{tikzpicture}%}
                \caption{Unaligned profiles.\label{subfig:hayRA_measured_hrrp_unaligned}}
        \end{subfigure}
        \begin{subfigure}{0.4\linewidth}
                \centering
                \resizebox{\linewidth}{!}{\input{Figures/04AlgoV&V/Measured/03HayRA/HayRA_Measured_HRRP}}
                \caption{Range-aligned profiles.\label{subfig:hayRA_measureed_hrrp}}
        \end{subfigure}
        \caption{\gls{hrr} profiles of a measured data frame before and after Haywood \gls{ra}. \label{fig:hayRA_measured}}
    \end{figure}

    \autoref{subfig:hayRA_measured_hrrp_unaligned} shows that Scatterer A migrated between profiles and that before \gls{ra}, the profiles were not aligned to profile 1. \autoref{fig:hayRA_Measured_shifts} shows the calculated bin shifts for the unaligned profiles which were calculated with respect to profile 1. \autoref{subfig:hayRA_measureed_hrrp} shows that, after range-alignment, Scatterer A did not migrate between profiles and that all scatterers are aligned with the vertical black line. Therefore, Haywood \gls{ra} range-aligned the profiles and meets the requirements of a \gls{ra} algorithm.
    
    \begin{figure}
        \centering
        \resizebox{0.45\linewidth}{!}{% This file was created by matlab2tikz.
%
%The latest updates can be retrieved from
%  http://www.mathworks.com/matlabcentral/fileexchange/22022-matlab2tikz-matlab2tikz
%where you can also make suggestions and rate matlab2tikz.
%
\definecolor{mycolor1}{rgb}{0.00000,0.44700,0.74100}%
\definecolor{mycolor2}{rgb}{0.85000,0.32500,0.09800}%
%
\begin{tikzpicture}

\begin{axis}[%
width=6.028in,
height=4.754in,
at={(1.011in,0.642in)},
scale only axis,
xmin=0,
xmax=128,
xlabel style={font=\fontsize{25}{14}\selectfont\color{black}, yshift=-10pt},
xlabel={Profile Number},
ymin=-1,
ymax=7,
ylabel style={font=\fontsize{25}{14}\selectfont\color{black}, yshift=10pt},
ylabel={Number of bin shifts},
xtick distance = 20,
tick label style={font=\fontsize{15}{11}\selectfont\color{black}},
axis background/.style={fill=white}
]
\addplot [color=mycolor1, forget plot]
  table[row sep=crcr]{%
1	0\\
2	0\\
3	0\\
4	0\\
5	0\\
6	0\\
7	0\\
8	0\\
9	0\\
10	0\\
11	0\\
12	0\\
13	1\\
14	0\\
15	1\\
16	1\\
17	1\\
18	1\\
19	1\\
20	1\\
21	1\\
22	1\\
23	1\\
24	1\\
25	1\\
26	1\\
27	1\\
28	1\\
29	1\\
30	1\\
31	1\\
32	1\\
33	1\\
34	1\\
35	1\\
36	2\\
37	2\\
38	2\\
39	2\\
40	2\\
41	2\\
42	2\\
43	2\\
44	2\\
45	2\\
46	2\\
47	2\\
48	2\\
49	2\\
50	2\\
51	2\\
52	2\\
53	2\\
54	3\\
55	3\\
56	3\\
57	2\\
58	3\\
59	3\\
60	3\\
61	3\\
62	3\\
63	3\\
64	3\\
65	3\\
66	3\\
67	3\\
68	3\\
69	3\\
70	3\\
71	3\\
72	3\\
73	3\\
74	3\\
75	3\\
76	3\\
77	4\\
78	4\\
79	4\\
80	4\\
81	4\\
82	4\\
83	4\\
84	4\\
85	4\\
86	4\\
87	5\\
88	4\\
89	4\\
90	5\\
91	5\\
92	5\\
93	5\\
94	5\\
95	5\\
96	5\\
97	5\\
98	5\\
99	5\\
100	5\\
101	5\\
102	5\\
103	5\\
104	5\\
105	5\\
106	5\\
107	5\\
108	5\\
109	5\\
110	5\\
111	5\\
112	5\\
113	5\\
114	5\\
115	6\\
116	6\\
117	6\\
118	6\\
119	6\\
120	6\\
121	6\\
122	6\\
123	6\\
124	6\\
125	6\\
126	6\\
127	6\\
128	6\\
};
\addplot [color=mycolor2, forget plot]
  table[row sep=crcr]{%
1	-0.203125\\
2	-0.152189960629921\\
3	-0.101254921259843\\
4	-0.0503198818897639\\
5	0.000615157480314821\\
6	0.0515501968503936\\
7	0.102485236220472\\
8	0.153420275590551\\
9	0.20435531496063\\
10	0.255290354330709\\
11	0.306225393700787\\
12	0.357160433070866\\
13	0.408095472440945\\
14	0.459030511811023\\
15	0.509965551181102\\
16	0.560900590551181\\
17	0.61183562992126\\
18	0.662770669291338\\
19	0.713705708661417\\
20	0.764640748031496\\
21	0.815575787401575\\
22	0.866510826771653\\
23	0.917445866141732\\
24	0.968380905511811\\
25	1.01931594488189\\
26	1.07025098425197\\
27	1.12118602362205\\
28	1.17212106299213\\
29	1.2230561023622\\
30	1.27399114173228\\
31	1.32492618110236\\
32	1.37586122047244\\
33	1.42679625984252\\
34	1.4777312992126\\
35	1.52866633858268\\
36	1.57960137795276\\
37	1.63053641732283\\
38	1.68147145669291\\
39	1.73240649606299\\
40	1.78334153543307\\
41	1.83427657480315\\
42	1.88521161417323\\
43	1.93614665354331\\
44	1.98708169291339\\
45	2.03801673228346\\
46	2.08895177165354\\
47	2.13988681102362\\
48	2.1908218503937\\
49	2.24175688976378\\
50	2.29269192913386\\
51	2.34362696850394\\
52	2.39456200787402\\
53	2.44549704724409\\
54	2.49643208661417\\
55	2.54736712598425\\
56	2.59830216535433\\
57	2.64923720472441\\
58	2.70017224409449\\
59	2.75110728346457\\
60	2.80204232283465\\
61	2.85297736220472\\
62	2.9039124015748\\
63	2.95484744094488\\
64	3.00578248031496\\
65	3.05671751968504\\
66	3.10765255905512\\
67	3.1585875984252\\
68	3.20952263779528\\
69	3.26045767716535\\
70	3.31139271653543\\
71	3.36232775590551\\
72	3.41326279527559\\
73	3.46419783464567\\
74	3.51513287401575\\
75	3.56606791338583\\
76	3.61700295275591\\
77	3.66793799212598\\
78	3.71887303149606\\
79	3.76980807086614\\
80	3.82074311023622\\
81	3.8716781496063\\
82	3.92261318897638\\
83	3.97354822834646\\
84	4.02448326771654\\
85	4.07541830708661\\
86	4.12635334645669\\
87	4.17728838582677\\
88	4.22822342519685\\
89	4.27915846456693\\
90	4.33009350393701\\
91	4.38102854330709\\
92	4.43196358267717\\
93	4.48289862204724\\
94	4.53383366141732\\
95	4.5847687007874\\
96	4.63570374015748\\
97	4.68663877952756\\
98	4.73757381889764\\
99	4.78850885826772\\
100	4.8394438976378\\
101	4.89037893700787\\
102	4.94131397637795\\
103	4.99224901574803\\
104	5.04318405511811\\
105	5.09411909448819\\
106	5.14505413385827\\
107	5.19598917322835\\
108	5.24692421259843\\
109	5.2978592519685\\
110	5.34879429133858\\
111	5.39972933070866\\
112	5.45066437007874\\
113	5.50159940944882\\
114	5.5525344488189\\
115	5.60346948818898\\
116	5.65440452755906\\
117	5.70533956692913\\
118	5.75627460629921\\
119	5.80720964566929\\
120	5.85814468503937\\
121	5.90907972440945\\
122	5.96001476377953\\
123	6.01094980314961\\
124	6.06188484251969\\
125	6.11281988188976\\
126	6.16375492125984\\
127	6.21468996062992\\
128	6.265625\\
};
\end{axis}
\end{tikzpicture}%}
        \caption{Linearised staircase curve of range bin shifts per \gls{hrrp}.\label{fig:hayRA_Measured_shifts}}
    \end{figure}

%%%%%%%%%%%%%%%%%%%%%%%%%%%%%%%%%%%%%%%%%%%%%%%%%%%%%%%%%%%%%%%%%%%%%%%%%%%%%%%%%%%%%%%%%%%%%%%%%%%%
% Read Martorella pg 83 and onwards. Try and get the phase function plots
\section{Single Dominant Scatterer Autofocus Algorithm} \label{subsec:hayAF}

% Read the Haywood paper and zyweck paper
This is a type of \gls{dsa} that uses a single \gls{ds} to correct the phase errors introduced by \gls{ra}. It selects the \gls{ds} based on a set of selection criteria and uses the \gls{ds}'s phase to correct all other \gls{hrr} profiles. An outline of the Single Dominant Scatterer \gls{af} algorithm, as described by \cite{haywood_RA_AF}, is given in \autoref{alg:haywood_AF}.
    %***************************************************************************************%
    \subsection{Pseudo Code and Implementation}
    % Haywood AF Pseudo code
    \begin{figure}[ht]
      \vspace{0.5cm}
      \centering
      \captionsetup{type=figure}
      \begin{minipage}{.7\linewidth}
        \begin{algorithm}[H]
            \caption{Single Dominant Scatterer \gls{af} Algorithm.\label{alg:haywood_AF}}

            \LinesNumbered % NUMBER THE LINES
            \DontPrintSemicolon
            \SetAlgoLined
            \SetKwInOut{Input}{input}\SetKwInOut{Output}{output}\SetKwInOut{Parameter}{parameter}
    
            \Input{Range-aligned \gls{hrrp}s, matrix $hrrp_{RA}$}
            \Output{Autofocused \gls{hrrp}s, matrix $hrrp_{AF}$}
    
            \BlankLine
            \Begin{
                $scattererPower \leftarrow 0$\;
                \For(){$k$ in 1 to rows($hrrp_{RA}$)}{
                    \For(){$n$ in 1 to columns($hrrp_{RA}$)}{
                        $scattererPower \leftarrow scattererPower + |hrrp_{RA}[k][n]|^2 $\;
                    }
                    $allPower[k] \leftarrow scattererPower $\;
                }
                $avgPower \leftarrow mean(allPower)$\;
                $candidateIndices \leftarrow $ indices of $ allPower>avgPower$\;
                \For(){$i$ in 1 to length($candidateIndices$)}{
                    \For(){$k$ in 1 to rows($hrrp_{RA}$)}{
                        $amplitudes[k] \leftarrow |hrrp_{RA}[k][candidateIndices[i]]| $\;
                    }
                    $variance[i] \leftarrow variance(amplitudes) $\;
                }
                \gls{ds}$Index \leftarrow candidatesIndex[$index of $min(variance)]$\;
                \For(){$k$ in 1 to rows($hrrp_{RA}$)}{
                    $angle \leftarrow angle(hrrp_{RA}[k][\gls{ds}Index]) $\;
                    $phaseHistory[k] \leftarrow exp(-1i \cdot angle)$\;
                }
                \For(){$k$ in 1 to rows($hrrp_{RA}$)}{
                    \For(){$n$ in 1 to columns($hrrp_{RA}$)}{
                        $hrrp_{AF}[k][n] \leftarrow hrrp_{RA}[k][n] \cdot phaseHistory[k]$
                    }
                }
            }
          \vspace{0.5cm}
        \end{algorithm}
      \end{minipage}
    \end{figure}

    The \gls{sdsaf} algorithm, \autoref{alg:haywood_AF}, was implemented as a stand-alone \textsc{MATLAB} function. The implemented function, \href{}{haywoodAF.m}, was validated and verified in this section.
    
     %***************************************************************************************%
    \subsection{Algorithm Verification}
    The aim of this section is to use the \gls{sdsaf} to correct the phase errors in the range-aligned simulated \gls{hrr} profiles. Only Correlation range-aligned profiles were used. To clearly see the effects of applying \gls{sdsaf}, an \gls{isar} image of the range-aligned profiles was formed as shown in \autoref{fig:hayRA_sim}(a). \autoref{subfig:hayAF_SCRA_sim_isar_AF} shows the image formed from simulated \gls{hrr} profiles of the object with no translational motion and represents the ideal focused image that should be formed after \gls{ra} and\gls{af}.

    % Grid of the HRRP and ISAR images
    \begin{figure}[h]
        \centering
        \begin{subfigure}{0.3\linewidth}
            \resizebox{\linewidth}{!}{% This file was created by matlab2tikz.
%
%The latest updates can be retrieved from
%  http://www.mathworks.com/matlabcentral/fileexchange/22022-matlab2tikz-matlab2tikz
%where you can also make suggestions and rate matlab2tikz.
%
\begin{tikzpicture}
\begin{axis}[%
width=5.554in,
height=4.754in,
at={(0.932in,0.642in)},
scale only axis,
point meta min=-40,
point meta max=0,
axis on top,
xmin=-0.0732421875,
xmax=37.4267578125,
xlabel style={font=\fontsize{25}{14}\selectfont\color{black}, yshift=-10pt},
xlabel={Range (m)},
ymin=-32.2265625,
ymax=30.2734375,
ylabel style={font=\fontsize{25}{14}\selectfont\color{black}},
ylabel={Doppler frquency (Hz)},
axis background/.style={fill=white},
tick label style={font=\fontsize{20}{11}\selectfont\color{black}},
xtick distance= 4,             % Set the spacing between x-axis ticks
ytick distance = 10,
colormap/jet,
colorbar
]
\addplot [forget plot] graphics [xmin=-0.0732421875, xmax=37.4267578125, ymin=-32.2265625, ymax=30.2734375] {Figures/04AlgoV&V/Simulation/02CorrRA/SCRA_Sim_ISAR_1mps_6deg.png};

% Add the rectangles
\draw[white, thick] (7,0.5) -- (7,-5) -- (29,-1.5) -- (29,3.5) -- (7,0.5);
\draw[white,thick]  (7,-7)--(7,-12.5)--(29,-9)--(29,-3.5)--(7,-7);

% Label boxes
\draw[->, white, line width=1.5pt] (5, -3) -- (7, -3);
\node[left] at (5, -3) {\tikz[baseline] \node[fill=white,inner sep=2pt] {A};};

\draw[->, white, line width=1.5pt] (5, -10) -- (7, -10);
\node[left] at (5, -10) {\tikz[baseline] \node[fill=white,inner sep=2pt] {B};};

\end{axis}

\end{tikzpicture}%}
            \caption{Range-aligned image. \label{subfig:hayAF_SCRA_sim_isar_RA}}
        \end{subfigure}
        \begin{subfigure}{0.3\linewidth}
            \resizebox{\linewidth}{!}{\input{Figures/04AlgoV&V/Simulation/04HayAF/HayAF_SCRA_Sim_ISAR_1mps_6deg}}
            \caption{Autofocused \gls{isar} image. \label{subfig:hayAF_SCRA_sim_isar_AF}}
        \end{subfigure}
        \begin{subfigure}{0.3\linewidth}
            \resizebox{\linewidth}{!}{\input{Figures/04AlgoV&V/Simulation/00NoMotion/Sim_ISAR_0mps_6deg}}
            \caption{'Focused' image.\label{subfig:hayAF_sim_isar_0mps}}
        \end{subfigure}
        \caption{(a) Correlation range-aligned \gls{isar} image, (b) \gls{sdsaf} autofocused \gls{isar} image, (c) \gls{isar} image of simulated object with no translation motion. \label{fig:hayAF_sim}}
    \end{figure}
    
    As discussed in \autoref{subsec:theory_AF}, after \gls{ra}, \gls{af} is used to correct phase errors that cause Doppler spreading which defocuses the image. As shown by the scatterer co-ordinates in \autoref{fig:sim_scatterers} of \autoref{subsec:suimulation_setup}, the simulated object only had one thin hull which is reflected in \autoref{hayAF_sim_0mps}, the 'focused' image. However, in \autoref{subfig:hayAF_SCRA_sim_isar_RA} there are two similar sets of bright scatterers (labeled A and B) that look like the hull of the ship-like object. The duplication of the hull in \autoref{subfig:hayAF_SCRA_sim_isar_RA} was the result of Doppler spreading and shows that the image was still defocused after \gls{ra}.

    \gls{sdsaf} was applied to the Correlation range-aligned profiles shown in \autoref{subfig:hayRA_sim_hrrp}. For a scatterer to be chosen as the \gls{ds}, it needs to fulfill \autoref{alg:haywood_AF}'s selection criteria. \autoref{subfig:hayAF_SCRA_sim_power} shows the candidate scatterers selected when line 10 of \autoref{alg:haywood_AF} was implemented. All the candidate scatterers appear above the average power threshold (orange line) which illustrates that criterion 1 was implemented correctly. \autoref{subfig:hayAF_SCRA_sim_power} shows the \gls{ds} selection based on criterion 2 in line 17 of \autoref{alg:haywood_AF}. The solid red line shows the mininimum variance of all candidate scatterers and the red dot is the selected \gls{ds}. Comparing the variance of the selected \gls{ds} to all other candidate scatterers illustrates that the chosen \gls{ds} was the scatterer with minimum variance. Therefore the selection criteria was implemented correctly.

    \begin{figure}
        \centering
        \begin{subfigure}{0.45\linewidth}
            \resizebox{\linewidth}{!}{% This file was created by matlab2tikz.
%
%The latest updates can be retrieved from
%  http://www.mathworks.com/matlabcentral/fileexchange/22022-matlab2tikz-matlab2tikz
%where you can also make suggestions and rate matlab2tikz.
%
\definecolor{mycolor1}{rgb}{0.00000,0.44700,0.74100}%
%
\begin{tikzpicture}

\begin{axis}[%
width=6.028in,
height=4.754in,
at={(1.011in,0.642in)},
scale only axis,
xmin=0,
xmax=256,
xlabel style={font=\fontsize{25}{20}\selectfont\color{black}, yshift = -10},
xlabel={Range Bin Number},
ymin=0,
ymax=0.1e7,
ylabel style={font=\fontsize{25}{20}\selectfont\color{black}, yshift=10pt},
ylabel={Power},
axis background/.style={fill=white},
tick label style={font=\fontsize{20}{11}\selectfont\color{black}},
xtick distance = 50,
yticklabel={\ifdim\tick pt=0pt\else\pgfmathprintnumber{\tick}\fi}, 
legend style={legend cell align=left, align = left, draw=white!15!black, font=\fontsize{12}{11}\selectfont\color{black}}
]
\addplot [color=mycolor1, mark=asterisk, mark options={solid, mycolor1}]
  table[row sep=crcr]{%
1	260.459583212987\\
2	255.853316684552\\
3	253.849369048871\\
4	262.902589895056\\
5	269.395431769183\\
6	257.942118684153\\
7	264.777673790391\\
8	267.345173590508\\
9	267.442550084895\\
10	274.763167201476\\
11	269.657038331506\\
12	273.52865970639\\
13	270.838241173277\\
14	270.848862745988\\
15	268.994507249835\\
16	278.092807479352\\
17	275.678440062448\\
18	280.764024183879\\
19	280.264596895383\\
20	280.844834122902\\
21	284.609662384031\\
22	289.349950042811\\
23	289.63453778992\\
24	297.216247874669\\
25	304.044956124828\\
26	299.728539743741\\
27	304.570239574428\\
28	312.586614746517\\
29	312.578394352122\\
30	313.03569904371\\
31	318.676695752314\\
32	334.429012992535\\
33	326.05568970326\\
34	329.794039972701\\
35	347.471923711035\\
36	357.356896491005\\
37	359.280067262786\\
38	370.172276306485\\
39	367.252690733492\\
40	384.036353269705\\
41	403.514451969125\\
42	409.987699077831\\
43	422.515008557612\\
44	437.076708129586\\
45	455.336146360901\\
46	474.987020352395\\
47	506.63892953046\\
48	524.698547335733\\
49	585.497580378801\\
50	627.110711153952\\
51	693.524309995736\\
52	784.287092174285\\
53	935.526405704739\\
54	1173.71124130302\\
55	1600.67007465833\\
56	2583.29049019285\\
57	5800.48900106346\\
58	40442.0449329261\\
59	234600.510606336\\
60	16827.8819816654\\
61	4770.1478970912\\
62	3240.29772132079\\
63	3634.40328263001\\
64	7268.77499224631\\
65	76531.2093170108\\
66	213667.240436973\\
67	10052.7936091226\\
68	4507.02792533953\\
69	3559.31518627866\\
70	4106.8388665825\\
71	8159.23968924983\\
72	127074.927266568\\
73	167160.607608639\\
74	9606.39694763152\\
75	4990.0602038816\\
76	4292.92605082353\\
77	5229.47347252321\\
78	10356.742292667\\
79	182309.818486979\\
80	108097.426019492\\
81	8627.78437379697\\
82	4318.19931958007\\
83	3854.49429303449\\
84	5199.41183792736\\
85	14022.0503602094\\
86	227432.741146917\\
87	63225.9569214638\\
88	8170.50934399543\\
89	4150.73895156004\\
90	3475.38764420246\\
91	4876.64326862932\\
92	22725.5309440892\\
93	244399.775686268\\
94	37495.2781672515\\
95	10573.4177170359\\
96	7100.60019107358\\
97	6753.16861846795\\
98	9242.40740232017\\
99	51530.6884280494\\
100	239715.038608182\\
101	21021.5521803837\\
102	12367.4161492386\\
103	12919.2292195377\\
104	18560.6270392406\\
105	50352.3007433402\\
106	491915.041237475\\
107	722546.673736452\\
108	42839.0513423732\\
109	18946.8184049635\\
110	13294.4378465799\\
111	11888.3265603265\\
112	14228.1063884036\\
113	161392.80534233\\
114	207528.804911758\\
115	32227.6875301607\\
116	26620.7506624016\\
117	28153.2588950145\\
118	33451.3666831277\\
119	47362.6527416718\\
120	243035.473131084\\
121	113402.734843283\\
122	67680.325702816\\
123	101247.331523687\\
124	175595.773473503\\
125	381658.413684424\\
126	1567657.49064522\\
127	23225002.3847125\\
128	4866592.85612614\\
129	624148.878839017\\
130	238968.489337471\\
131	125467.57654359\\
132	78185.2090712392\\
133	73030.3601940519\\
134	271320.771835579\\
135	66016.2557543734\\
136	35644.2191386429\\
137	27660.6982729765\\
138	24525.1417075442\\
139	27005.715637359\\
140	93273.0398557962\\
141	280507.640693694\\
142	15266.9861055596\\
143	11660.2329098454\\
144	12774.0609258221\\
145	18308.3665983194\\
146	54196.6326667642\\
147	559399.5054032\\
148	675605.960268304\\
149	46832.2981207944\\
150	20414.8653773866\\
151	13596.8722164049\\
152	11704.4608140175\\
153	14834.5483454693\\
154	165878.818809875\\
155	129155.452493415\\
156	12845.7472234616\\
157	8425.93245283305\\
158	7924.50240121107\\
159	9630.82839402855\\
160	18167.8674919882\\
161	221464.952930091\\
162	66296.2165083529\\
163	6282.75073094234\\
164	3550.82893427832\\
165	3693.84173148895\\
166	5793.15980013346\\
167	21194.8549443591\\
168	249988.987910509\\
169	37348.5151127678\\
170	6707.77320205252\\
171	3834.62559434609\\
172	3666.58727613934\\
173	6179.43754018277\\
174	39621.937991042\\
175	244027.580562077\\
176	19175.2805279218\\
177	6528.63349151219\\
178	4772.4713207705\\
179	5128.36765511789\\
180	9037.87893829267\\
181	79769.6559031287\\
182	213548.500351052\\
183	9189.08795174902\\
184	4363.70260986501\\
185	3712.07025077464\\
186	4602.33188561278\\
187	9220.50288249593\\
188	129112.974710105\\
189	162501.742729066\\
190	7574.41472023649\\
191	3473.96314063804\\
192	2883.8804389084\\
193	3771.98710657173\\
194	8603.85021353972\\
195	174816.050956961\\
196	107424.122922777\\
197	8158.46008299097\\
198	3188.81296756292\\
199	1874.3706597991\\
200	1319.41764201677\\
201	1034.95594151075\\
202	869.247784578133\\
203	742.863481776727\\
204	658.01083274331\\
205	610.719153796497\\
206	558.676138785167\\
207	527.727140126541\\
208	493.143467884876\\
209	472.767945849341\\
210	450.923013010194\\
211	429.591125867796\\
212	412.554276876183\\
213	404.874610605797\\
214	387.775528276528\\
215	391.405099058641\\
216	365.717226442089\\
217	364.330352475694\\
218	356.484671896599\\
219	358.687392545226\\
220	345.13318693796\\
221	338.830857316187\\
222	344.279776728209\\
223	335.852598946057\\
224	321.208276027795\\
225	320.221419869568\\
226	316.106449388835\\
227	306.342738798994\\
228	301.826156405654\\
229	300.26529353038\\
230	290.634496667669\\
231	295.148021859695\\
232	294.361094581251\\
233	291.018501216427\\
234	290.531983794715\\
235	293.620323539453\\
236	282.415873342337\\
237	281.56353725174\\
238	276.975587713\\
239	285.539867461153\\
240	274.106466494687\\
241	273.037195880877\\
242	282.942195034159\\
243	271.428841093763\\
244	265.861359523173\\
245	264.640464669466\\
246	269.787791563485\\
247	269.628633305419\\
248	259.668277498303\\
249	270.659371561747\\
250	265.10876919209\\
251	260.083470473997\\
252	261.220926812129\\
253	265.88563906247\\
254	257.030182737016\\
255	266.64222467541\\
256	259.75335716483\\
};
\addlegendentry{Scatterer power}

\addplot [color=orange]
  table[row sep=crcr]{%
0	158841.769294874\\
300	158841.769294874\\
};
\addlegendentry{Average power of all scatterers}

\addplot [color=black, only marks, mark size=4.0pt, mark=o, mark options={solid, black}]
  table[row sep=crcr]{%
59	234600.510606336\\
66	213667.240436973\\
73	167160.607608639\\
79	182309.818486979\\
86	227432.741146917\\
93	244399.775686268\\
100	239715.038608182\\
106	491915.041237475\\
107	722546.673736452\\
113	161392.80534233\\
114	207528.804911758\\
120	243035.473131084\\
124	175595.773473503\\
125	381658.413684424\\
126	1567657.49064522\\
127	23225002.3847125\\
128	4866592.85612614\\
129	624148.878839017\\
130	238968.489337471\\
134	271320.771835579\\
141	280507.640693694\\
147	559399.5054032\\
148	675605.960268304\\
154	165878.818809875\\
161	221464.952930091\\
168	249988.987910509\\
175	244027.580562077\\
182	213548.500351052\\
189	162501.742729066\\
195	174816.050956961\\
};
\addlegendentry{Candidate scatterers}

\addplot [color=red, only marks, mark size=7.5pt, mark=o, mark options={solid, red}]
  table[row sep=crcr]
            \caption{Scatterer power.} \label{subfig:hayAF_SCRA_sim_power}
        \end{subfigure}
        \hspace{1cm}
        \begin{subfigure}{0.45\linewidth}
            \resizebox{\linewidth}{!}{% This file was created by matlab2tikz.
%
%The latest updates can be retrieved from
%  http://www.mathworks.com/matlabcentral/fileexchange/22022-matlab2tikz-matlab2tikz
%where you can also make suggestions and rate matlab2tikz.
%
\definecolor{mycolor1}{rgb}{0.00000,0.44700,0.74100}%
%
\begin{tikzpicture}

\begin{axis}[%
width=6.028in,
height=4.754in,
at={(1.011in,0.642in)},
scale only axis,
xmin=0,
xmax=256,
xlabel style={font=\fontsize{25}{20}\selectfont\color{black}, yshift = -10},
xlabel={Range Bin Number},
ymin=0,
ymax=0.2e4,
ylabel style={font=\fontsize{25}{20}\selectfont\color{black}, yshift=10pt},
ylabel={Amplitude Variance},
axis background/.style={fill=white},
tick label style={font=\fontsize{20}{11}\selectfont\color{black}},
xtick distance = 50,
ytick distance=2e2,
yticklabel={\ifdim\tick pt=0pt\else\pgfmathprintnumber{\tick}\fi}, 
scaled y ticks=base 10:-3,
legend style={legend cell align=left, align=left, draw=white!15!black, font=\fontsize{12}{11}\selectfont\color{black}},
grid=both,  % Enables major grids
minor tick num=10,  % Sets the number of minor grid lines between major lines
grid style={dashed,gray!50},  % Style for major grids
minor grid style={dotted,gray!50} % Style for minor grids
]
\addplot [color=mycolor1, mark=asterisk, mark options={solid, mycolor1}]
  table[row sep=crcr]{%
1	2.05737908304292\\
2	1.85575449052375\\
3	1.94419916622086\\
4	1.97731558212904\\
5	1.94093380004352\\
6	1.9358937287336\\
7	1.93953862201703\\
8	1.9771528727676\\
9	1.96834637986995\\
10	2.12660136918566\\
11	2.03673430039263\\
12	2.01268686658586\\
13	2.01698766709781\\
14	1.84279497260678\\
15	2.00164902278947\\
16	2.09460771999247\\
17	2.04257816341812\\
18	2.12812067993302\\
19	2.12138163189761\\
20	2.0625414274544\\
21	2.06221226740546\\
22	2.08858806772501\\
23	2.1085576416507\\
24	2.11695168130969\\
25	2.11884352293369\\
26	2.27845217791114\\
27	2.13188273293057\\
28	2.2958260078983\\
29	2.36145825292486\\
30	2.35296425246966\\
31	2.40675658285026\\
32	2.40474114323719\\
33	2.31045758149772\\
34	2.34457118447211\\
35	2.4153213714403\\
36	2.62643897627921\\
37	2.54331345642505\\
38	2.5631609465458\\
39	2.66570288041185\\
40	2.73083388981557\\
41	2.79542599543643\\
42	2.9032614798364\\
43	2.97833691890515\\
44	3.02154550379851\\
45	3.14889199955147\\
46	3.22320968920339\\
47	3.68837266262699\\
48	3.62315673508141\\
49	3.95134251699233\\
50	4.12410179466886\\
51	4.48042444893653\\
52	4.75538497619942\\
53	5.49179425478158\\
54	6.59567935877715\\
55	8.8620418620446\\
56	15.0805575299783\\
57	40.0918635305862\\
58	467.366777344471\\
59	212.860517690975\\
60	120.635250334277\\
61	23.8443247966236\\
62	15.32203908096\\
63	18.8754409707475\\
64	46.960564364429\\
65	921.409134695644\\
66	624.956756008045\\
67	44.3462878157575\\
68	15.8434588046976\\
69	11.0313705900366\\
70	14.7148791080368\\
71	41.2129123274094\\
72	1165.78339202722\\
73	1083.47517174755\\
74	61.9353118582872\\
75	30.6317777939068\\
76	26.2331153776291\\
77	32.0589292149724\\
78	61.208296977669\\
79	807.193952491206\\
80	1054.00562008634\\
81	51.3086401021939\\
82	25.0552367271964\\
83	24.645435876197\\
84	34.3880585361111\\
85	89.2616429763252\\
86	412.022171953476\\
87	707.385505854269\\
88	70.4909740208722\\
89	37.3708696489367\\
90	28.1872662770878\\
91	31.8181059014351\\
92	214.803956656843\\
93	167.035155026629\\
94	352.24109571606\\
95	74.9937503692081\\
96	47.4843272285187\\
97	41.2377883806217\\
98	53.1785040986701\\
99	554.177762709179\\
100	273.400370982818\\
101	62.4294210083709\\
102	24.0867058952232\\
103	29.0626937054166\\
104	60.8987997357927\\
105	306.705302279755\\
106	5031.43973780062\\
107	10058.7117990739\\
108	486.334706443482\\
109	171.388272475032\\
110	76.8102436538542\\
111	41.5496908073745\\
112	34.7807147450841\\
113	1432.7499803385\\
114	1675.52928762012\\
115	173.681500952215\\
116	129.429083547151\\
117	131.33153604024\\
118	146.04394046422\\
119	181.461838494977\\
120	219.699661675122\\
121	634.57507614067\\
122	257.072279734164\\
123	419.825796876926\\
124	763.967800150026\\
125	1759.99290585144\\
126	10117.5535871621\\
127	22350.9565937575\\
128	52005.592228381\\
129	4429.99410666212\\
130	1517.5166733067\\
131	760.384400265963\\
132	440.85626253539\\
133	326.167164976106\\
134	84.2369082187049\\
135	403.975442296336\\
136	189.002195910455\\
137	152.035304484769\\
138	139.704510429657\\
139	154.469781784144\\
140	928.719587239615\\
141	974.893785657169\\
142	60.5969765410862\\
143	48.8125082224575\\
144	62.2348167674982\\
145	102.870139666629\\
146	393.548183623953\\
147	5781.49829683031\\
148	9600.11309481653\\
149	646.908659456387\\
150	251.700343749591\\
151	142.601347529567\\
152	97.7909012341403\\
153	88.1273955882369\\
154	778.168464203035\\
155	1035.20520987721\\
156	74.4928058637182\\
157	47.7002388054471\\
158	44.7046391902101\\
159	52.6983762010651\\
160	87.1865984504365\\
161	369.965474095604\\
162	712.37171516924\\
163	43.5249895781135\\
164	27.6563750594264\\
165	30.7407061806614\\
166	41.4463707434985\\
167	169.057497144586\\
168	207.224769574435\\
169	393.275327237745\\
170	52.5608321701357\\
171	26.5607340504985\\
172	24.3550232952744\\
173	44.8069531340425\\
174	473.552490683966\\
175	219.636189360627\\
176	112.430916983961\\
177	27.2388390380248\\
178	20.1657757051255\\
179	21.6440464932856\\
180	49.6055785788647\\
181	904.160241436579\\
182	629.458968056839\\
183	44.4994350956159\\
184	20.2320153874046\\
185	17.2463729802717\\
186	24.5568335642233\\
187	57.9769881936061\\
188	1170.87795976595\\
189	1099.66918779979\\
190	44.4768291335985\\
191	17.8383389635699\\
192	14.2504589958134\\
193	18.6065341225249\\
194	39.8488773241925\\
195	781.857147753294\\
196	1052.95949881977\\
197	54.6145182595696\\
198	20.7228409375793\\
199	12.3911543835641\\
200	9.45644508502595\\
201	7.58770076056527\\
202	6.70432818996595\\
203	6.23214022376627\\
204	5.66638646341624\\
205	5.42401685490908\\
206	5.16946424795573\\
207	5.05526728453367\\
208	4.79270036541067\\
209	4.34485583684973\\
210	4.19585936898847\\
211	4.00205564088481\\
212	3.89409316323992\\
213	3.78251711220201\\
214	3.5544127265486\\
215	3.56839797454136\\
216	3.33975364055965\\
217	3.1982732388029\\
218	3.1706795359667\\
219	3.02504835985895\\
220	2.91675093067297\\
221	2.93859595723472\\
222	2.89170407386251\\
223	2.81003807626215\\
224	2.62368908556373\\
225	2.6936235341362\\
226	2.57290357536041\\
227	2.41203824206536\\
228	2.43818227906128\\
229	2.48520498880339\\
230	2.2298127822684\\
231	2.40054548431245\\
232	2.27833759145899\\
233	2.30355762862229\\
234	2.23527485425152\\
235	2.35724671848064\\
236	2.23265253940036\\
237	2.21914065307578\\
238	2.08545587217413\\
239	2.20622021835369\\
240	2.17663890274453\\
241	2.15226712266348\\
242	2.26672380614512\\
243	2.00511541234779\\
244	1.98795345957498\\
245	1.97291109691645\\
246	2.08455595601723\\
247	2.0180567846452\\
248	2.04170701359294\\
249	2.11865871826218\\
250	1.96113633416688\\
251	1.92609804526956\\
252	1.95535187999296\\
253	1.9345429363074\\
254	1.90359892233789\\
255	2.02345059762138\\
256	1.92279491529172\\
};
\addlegendentry{Scatterer variance}

\addplot [color=black, only marks, mark size=4.0pt, mark=o, mark options={solid, black}]
  table[row sep=crcr]{%
59	212.860517690975\\
66	624.956756008045\\
73	1083.47517174755\\
79	807.193952491206\\
86	412.022171953476\\
93	167.035155026629\\
100	273.400370982818\\
106	5031.43973780062\\
107	10058.7117990739\\
113	1432.7499803385\\
114	1675.52928762012\\
120	219.699661675122\\
124	763.967800150026\\
125	1759.99290585144\\
126	10117.5535871621\\
127	22350.9565937575\\
128	52005.592228381\\
129	4429.99410666212\\
130	1517.5166733067\\
134	84.2369082187049\\
141	974.893785657169\\
147	5781.49829683031\\
148	9600.11309481653\\
154	778.168464203035\\
161	369.965474095604\\
168	207.224769574435\\
175	219.636189360627\\
182	629.458968056839\\
189	1099.66918779979\\
195	781.857147753294\\
};
\addlegendentry{Candidate scatterers}

\addplot [color=red]
  table[row sep=crcr]{%
0	84.2369082187049\\
300	84.2369082187049\\
};
\addlegendentry{Minimum variance}

\addplot [color=red, only marks, mark size=4.0pt, mark=*, mark options={solid, fill=red, red}]
  table[row sep=crcr]
            \caption{Scatterer amplitude variance.} \label{subfig:hayAF_SCRA_sim_var}
        \end{subfigure}
        \caption{\gls{sdsaf} \gls{ds} selection plots for Simple Correlation \gls{ra} \gls{hrr} profiles.\label{subfig:hayAF_SCRA_sim_power&var}}
    \end{figure}

    The phase history of the \gls{ds} was calculated as the conjugate of the angle that the \gls{ds} moved through in each profile, for all profiles as in lines 18-21 of \autoref{alg:haywood_AF}. After applying the phase history to the \gls{ds}, its angle should be reduced to 0 for all profiles making it the zero-Doppler point in the autofocused image. \autoref{subfig:hayAF_SCRA_sim_power&var} shows the angle of the \gls{ds} before and after phase correction and the small values in \autoref{subfig:hayAF_SCRA_sim_var} illustrate that the angle is effectively reduced to zero. Therefore \autoref{alg:haywood_AF} has been implemented correctly.
    \begin{figure}
        \centering
        \begin{subfigure}{0.45\linewidth}
            \resizebox{\linewidth}{!}{% This file was created by matlab2tikz.
%
%The latest updates can be retrieved from
%  http://www.mathworks.com/matlabcentral/fileexchange/22022-matlab2tikz-matlab2tikz
%where you can also make suggestions and rate matlab2tikz.
%
\definecolor{mycolor1}{rgb}{0.00000,0.44700,0.74100}%
%
\begin{tikzpicture}

\begin{axis}[%
width=6.028in,
height=4.754in,
at={(1.011in,0.642in)},
scale only axis,
xmin=0,
xmax=32,
xlabel style={font=\fontsize{25}{20}\selectfont\color{black}, yshift = -10},
xlabel={Profile Number},
ymin=-4,
ymax=3,
ylabel style={font=\fontsize{25}{20}\selectfont\color{black}, yshift=10pt},
ylabel={Phase Angle},
axis background/.style={fill=white},
xtick distance = 4,
tick label style={font=\fontsize{20}{11}\selectfont\color{black}},
grid=both,  % Enables major grids
minor tick num=3,  % Sets the number of minor grid lines between major lines
grid style={dashed,gray!50},  % Style for major grids
minor grid style={dotted,gray!50} % Style for minor grids
]
\addplot [color=mycolor1, forget plot]
  table[row sep=crcr]{%
1	-0.713652998873585\\
2	-1.26328398961862\\
3	-1.79761267535875\\
4	2.11723648169395\\
5	1.52890140406838\\
6	0.933432252246175\\
7	0.343937565370561\\
8	-0.258904474711182\\
9	-0.829898857738967\\
10	-1.32815255562551\\
11	-1.89579871862167\\
12	-2.45425890388898\\
13	-2.90917521126634\\
14	0.918508750369201\\
15	0.36435881260357\\
16	-0.234756315479329\\
17	-0.811935144614113\\
18	-1.34967521347663\\
19	-1.85491320114706\\
20	-2.4440814742834\\
21	-3.02730416254395\\
22	2.81049908082331\\
23	0.376388812791332\\
24	-0.142087705005013\\
25	-0.738608192995369\\
26	-1.24225854609403\\
27	-1.76452604632288\\
28	-2.32564715390667\\
29	-2.85613646189082\\
30	2.81629102559214\\
31	2.31164459812681\\
32	-0.0567907978676315\\
};
\end{axis}
\end{tikzpicture}%}
            \caption{Phase angle before correction.} \label{subfig:hayAF_SCRA_sim_DSangle_before}
        \end{subfigure}
        \hspace{1cm}
        \begin{subfigure}{0.45\linewidth}
            \resizebox{\linewidth}{!}{% This file was created by matlab2tikz.
%
%The latest updates can be retrieved from
%  http://www.mathworks.com/matlabcentral/fileexchange/22022-matlab2tikz-matlab2tikz
%where you can also make suggestions and rate matlab2tikz.
%
\definecolor{mycolor1}{rgb}{0.00000,0.44700,0.74100}%
%
\begin{tikzpicture}

\begin{axis}[%
width=6.028in,
height=4.754in,
at={(1.011in,0.642in)},
scale only axis,
xmin=0,
xmax=32,
xlabel style={font=\fontsize{25}{20}\selectfont\color{black}, yshift = -10},
xlabel={Profile Number},
ymin=-4e-16,
ymax=4e-16,
ylabel style={font=\fontsize{25}{20}\selectfont\color{black}, yshift=10pt},
ylabel={Phase Angle},
axis background/.style={fill=white},
xtick distance = 4,
tick label style={font=\fontsize{20}{11}\selectfont\color{black}},
grid=both,  % Enables major grids
minor tick num=3,  % Sets the number of minor grid lines between major lines
grid style={dashed,gray!50},  % Style for major grids
minor grid style={dotted,gray!50} % Style for minor grids
]
\addplot [color=mycolor1, forget plot]
  table[row sep=crcr]{%
1	-6.71334535649059e-17\\
2	-7.1179279960309e-17\\
3	1.51061026746944e-16\\
4	-3.84883865715775e-16\\
5	-2.69238746269896e-17\\
6	0\\
7	0\\
8	0\\
9	0\\
10	-1.01863287932294e-16\\
11	1.1163470762614e-16\\
12	7.66721213224169e-17\\
13	0\\
14	0\\
15	-3.92685073215461e-17\\
16	-3.7377970799296e-17\\
17	0\\
18	-6.89164145169786e-17\\
19	0\\
20	-1.62970475826135e-16\\
21	-2.06577679366204e-17\\
22	9.18762137933272e-17\\
23	0\\
24	0\\
25	0\\
26	0\\
27	1.48459127252762e-16\\
28	3.10747426206038e-16\\
29	-1.71884769308618e-16\\
30	4.73423668707e-17\\
31	-9.96888474401872e-17\\
32	0\\
};
\end{axis}
\end{tikzpicture}%}
            \caption{Phase angle after correction} \label{subfig:hayAF_SCRA_sim_DSangle_after}
        \end{subfigure}
        \caption{\gls{ds} phase angle before and after \gls{sdsaf} phase correction.\label{subfig:hayAF_SCRA_sim_DSangle}}
    \end{figure}
    
    RA IC = 24.7384
    AF IC = 34.0975
    
    \gls{sdsaf} was applied to the Correlation range-aligned profiles, \autoref{subfig:hayRA_sim_hrrp}, and produced the autofocused image in \autoref{subfig:hayAF_SCRA_sim_isar_AF}. Unlike in \autoref{subfig:hayAF_SCRA_sim_isar_RA}, the image in \autoref{subfig:hayAF_SCRA_sim_isar_AF} shows a single base centered at zero-Doppler. This    
    
    %***************************************************************************************%
    \subsection{Algorithm Validation}
     An adjustment was made to this algorithm during implementation. As discussed in \autoref{theory:noise}, signals can be frequency-modulated by external interference sources, which can affect the phase of the received signal. These effects were seen in the initial \gls{isar} images hence a threshold \gls{sf} was introduced. It ensures that higher power scatterers are chosen as the \gls{ds} to reduce the effects of noise. The advantage of this is discussed using comparative results in \autoref{apndxA:scale_factor_effect}. The scaling factor was implemented as shown below.
       
    The best \gls{sf} for the simulated data was found to be 10 for both Haywood and Correlation data. Choosing a SF of 10 demonstartes the best case autofocused image where Haywood AF chooses the center of rotation \gls{ds} to use to phase correct all profiles. This will not always be the case in measured data however for the the purpose of verifying that the algorithm works as expected, the most ideal data is preferable and so the SF of 10 will be used in this section to verify the algorithm.
    
    \begin{itemize}
        \item Implementation Revision: Discuss adjustments made to algorithm and why: higher power candidate scatterers chosen when using measured data was necessary and improved results
        \item pseudo code update
        \item Discuss chosen Scatterers, calculated image contrast results and visual signs of focused image
        \item \textbf{Figures} thresholding of scatterers correct and dominant scatterer chosen, ISAR image before and after AF.
    \end{itemize}

    \gls{sdsaf} was applied to the Correlation range-aligned profiles, \autoref{subfig:hayRA_sim_hrrp}, and produced the autofocused image in \autoref{subfig:hayAF_SCRA_sim_isar_AF}. Unlike in \autoref{subfig:hayAF_SCRA_sim_isar_RA}, the image in \autoref{subfig:hayAF_SCRA_sim_isar_AF} shows a single base centered at zero-Doppler. This 
    
    Different scaling factors were tested for both the Correlation and Haywood range-aligned profiles to find the most suitable value for each. For correlation it was found to be x and for Haywood RA it was found to be x. As discussed in \autoref{apndxA:scale_factor_effect}, the SF will not always be ideal for all frames of data in the dataset. For the purpose of validation, the ideal case for the chosen frame was found inorder to validate that the algorithm produces a focused \gls{isar} image.

    In measureed data there is often more than one \gls{ds}. A limitation of the Haywood AF is that it uses only one \gls{ds} to phase corrects all profiles. This limits how well the image can be focused when using data where more than one \gls{ds} exists. %Yuan paper end of pg95 and beginning of 97
    The \gls{ds} selected is assumed to be the cor which is not always the case in measured data where more than one \gls{ds} is selected. his means that phase corrections will not be perfect and so the image does not exactly like the reference focused isar image. Using multiple dominant scatterers stands to improve the focus of the image.
    Recall from \autoref{theory:motion} that the center of rotation is the static point around which the object rotates and is the zero-Doppler point. A limitation of this \gls{dsa} is that it assumes that the selected \gls{ds} is the center of rotation, which is not always true. This means that phase corrections will not be perfect and so the image does not exactly like the reference focused isar image.

%%%%%%%%%%%%%%%%%%%%%%%%%%%%%%%%%%%%%%%%%%%%%%%%%%%%%%%%%%%%%%%%%%%%%%%%%%%%%%%%%%%%%%%%%%%%%%%%%%%%
\section{Multiple Dominant Scatterer Autofocus Algorithm}\label{subsec:yuanAF}
In measured data there often exists more than one \gls{ds}. The Multiple Dominant Scatterer \gls{af} algorithm is a type of \gls{dsa} that selects a set of dominant scatterers to use in phase correction. The average of all the phase differences of all dominant scatterers is used for the phase correction. Since the amplitude of each \gls{ds} is included in the calculated average, the phase of the stronger scatterers contribute more to the phase correction. Hence, this is a weighted multiple scatterer approach to \gls{af}.
An outline of the Yuan \gls{af} algorithm as described in \cite{yuan_AF} is given in \autoref{alg:yuan_AF}.
 % Read Yuan's paper

    %***************************************************************************************%
    \subsection{Pseudo Code and Implementation}
    % Yuan AF Pseudo code
    \begin{figure}[h]
      \vspace{0.5cm}
      \centering
      \captionsetup{type=figure}
      \begin{minipage}{.7\linewidth}
        \begin{algorithm}[H]
            \caption{Multiple Dominant Scatterer \gls{af} Algorithm.\label{alg:yuan_AF}}
    
            \DontPrintSemicolon
            \SetAlgoLined
            \SetKwInOut{Input}{input}\SetKwInOut{Output}{output}\SetKwInOut{Parameter}{parameter}
    
            \Input{\gls{ra} \gls{hrrp}s, matrix $hrrp_{RA}$}
            \Output{Autofocused \gls{hrrp}s, matrix $hrrp_{AF}$}
            \Parameter{Reference \gls{hrrp} number, $refIndex$}
    
            \BlankLine
            \Begin{
                \For(){$k$ in 1 to rows($hrrp_{all}$)}{
                    $variance[k] \leftarrow variance(| hrrp_{all}[k] |) $\;
                }
            }
          \vspace{0.5cm}
        \end{algorithm}
      \end{minipage}
    \end{figure}
    Note that the original algorithm suggests choosing a number of dominnat scatterers between 6 and 18, optimally 11, however it was found that not all \gls{hrr} profiles contain enough dominant scatterers to fulfill this. An check has been implemented that either uses 11 or otherwise the minimum number of dominant scatterers available. INCLUDE THE IF STATEMENT IN THE PSEUDO CODE.
    %***************************************************************************************%
    \subsection{Algorithm Verification}
    \begin{itemize}
        \item \textbf{IF this is not a good show of the algorithm, discuss wwhy: limitations of simulator}
        \item Confirm that dominant scatterer is chosen, calculated image contrast results and visual signs of focused image
        \item \textbf{Figures} thresholding of scatterers correct and dominant scatterer chosen, ISAR image before and after AF.
    \end{itemize}

    IC values: SCRA=38.42  and HayRA =50.03 
    
    %***************************************************************************************%
    \subsection{Algorithm Validation}
    \begin{itemize}
        \item Implementation Revision: Discuss adjustments made to algorithm and why: filtering out noisy scatterers
        \item pseudo code update
        \item Discuss chosen Scatterers, calculated image contrast results and visual signs of focused image
        \item \textbf{Figures} thresholding of scatterers correct and dominant scatterer chosen, ISAR image before and after AF.
    \end{itemize}
%%%%%%%%%%%%%%%%%%%%%%%%%%%%%%%%%%%%%%%%%%%%%%%%%%%%%%%%%%%%%%%%%%%%%%%%%%%%%%%%%%%%%%%%%%%%%%%%%%%%
\section{Optimisation of Implementation to reduce runtime \label{sec:V&V_optimisation}}
\begin{itemize}
    \item \textbf{Table} timing before and after optimisation - in APPENDIX
    \item Discuss time testing and code optimisation approaches
    \item Emphasise that the results before and after are still the same - not sure that proof is necessary.
\end{itemize}
\textsc{MATLAB} is designed to operate on arrays and matrices and has built-in functions that enable multithreaded processing of matrix operations. This eliminates the need for iterative loops required in other programming languages. The implementation of all algorithms used in this report leveraged these capabilities to achieve improved runtime performance.  Additionally, find() was replaced with \textsc{MATLAB} logical indexing \cite{matlab_logical_indexing} where possible. 

All optimisation changes and adjustments were carefully checked to ensure that the correct results were produced. The tabulated timing results can be found in \autoref{apndxA:optimisation_timing_full}, but the runtime improvement percentage for each algorithm is specified in \ref{tab:optimisation_timing}.

\begin{table}
    \centering
    \begin{tabular}{|l|c|}
        \hline
        \multicolumn{1}{|c|}{\textbf{Algorithm}} & \textbf{Runtime Improvement (s)} \\
        \hline
        Correlation \gls{ra}    & \\
        \hline
        Haywood \gls{ra}               & \\
        \hline
        Haywood \gls{af}               & \\
        \hline
        Yuan \gls{af}                  & \\
        \hline
    \end{tabular}
    \caption{Caption \label{tab:optimisation_timing}}
\end{table}



    %***************************************************************************************%
    \subsection{Correlation Range Alignment Algorithm}
    Cross-correlation can be performed in the time or frequency domain, \cite{ISARtextbook_Martorella} states that the frequency domain computation is faster. However, \textsc{Matlab} has a built-in two-dimensional time domain cross-correlation function that is optimised for \textsc{Matlab} use. This made the time domain calculation faster than performing multiple steps to do the frequency domain correlation in this \textsc{Matlab} specific implementation. Notably, the final script achieved a x reduction in runtime compared to the initial version.
    
    %***************************************************************************************%
    \subsection{Haywood Range Alignment Algorithm}
    The initial implementation was revised for further optimisation. This process involved introducing global parameters to reduce the frequency of function calls.  In the initial version, the correlation \gls{ra} algorithm was used to perform cross-correlation, while in the revised script, this operation was directly integrated into the code. This revision not only reduced runtime but also removed the dependence on another algorithm. Notably, the final script achieved a x reduction in runtime compared to the initial version.
    
    %***************************************************************************************%
    \subsection{Haywood Autofocus Algorithm}
    Further optimisation was achieved by introducing global parameters to reduce the frequency of function calls. While this adjustment required the code to be restructured, it still yielded the same results. Notably, the final script achieved a x reduction in runtime compared to the initial version.

    %***************************************************************************************%
    \subsection{Yuan Autofocus Algorithm}
    The \textsc{MATLAB} functions used to calculate values in the algorithm were optimized to use faster alternatives that still yielded correct results. For example, the mink() function, which returns the index of the smallest k values, was replaced by the the sort() function, and the returned array was then indexed. This change was made because sort() is multithreaded, and resulted in runtime improvements. Notably, the final script achieved a x reduction in runtime compared to the initial version.
    
%%%%%%%%%%%%%%%%%%%%%%%%%%%%%%%%%%%%%%%%%%%%%%%%%%%%%%%%%%%%%%%%%%%%%%%%%%%%%%%%%%%%%%%%%%%%%%%%%%%%
\section{Summary}
The integer bin shifting presented a limitation of \autoref{alg:corr_RA} which prevented perfect aligment of the range profiles in both simulation and when using measured data.

Because Corr Ra only allows integer bin shifts, and in some profiles the object only moves by a fractional bin amount, the range alignment is not perfect. This is a limitation of algorithm which Haywood RA overcomes.

LIn measured data there is often more than one \gls{ds}. A limitation of the Haywood AF is that it uses only one \gls{ds} to phase corrects all profiles and assumes that this \gls{ds} is the cor. This limits how well the image can be focused when using measure data, as shown in the previous section. Yuan AF improves on this by selecting multiple \gls{ds} to use in phase correction.

In measured data there is often more than one \gls{ds}. A limitation of the Haywood AF is that it uses only one \gls{ds} to phase corrects all profiles and assumes that this \gls{ds} is the cor. This limits how well the image can be focused when using measure data, as shown in the previous section. Yuan AF improves on this by selecting multiple \gls{ds} to use in phase correction.
% ----------------------------------------------------
\ifstandalone
\bibliography{../Bibliography/References.bib}
\printnoidxglossary[type=\acronymtype,nonumberlist]
\fi
\end{document}
% ----------------------------------------------------